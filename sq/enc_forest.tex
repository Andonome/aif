\section{Forest Encounters}
\index{Encounters!Forest}

\epigraph{Come now, my child, if we were planning to harm you, do you think we'd be lurking here beside the path in the very darkest part of the forest?}{Kenneth Patchen}

\renewcommand{\sqarea}{Forest}

\setcounter{enc}{0}

\begin{multicols}{2}

\sidequest{The Little Prince}\label{littleprince}

\startcontents[sq]

\sqminitoc

\noindent
An elf is assaulted by bandits.
If the characters help, they receive help in return later.
If they join the bandits and rob the elf, other elves will have their vengeance later.

\sqpart{Forest}% AREA
{The Elven Prince}% NAME
{The Woodspy Bandits are attacking a rich elf}% SUMMARY

\begin{boxtext}
  A voice in the distant forest cries out.
  \begin{quote}
    I'll summon griffins to pull your stomach out!  I'll enchant you to make you eat until your stomach explodes!
  \end{quote}

  A gruff voice laughs nearby.  ``If you could, you'd have done it already.  And that's some pretty jewels you got there.  Rich ladies like elf jewels.''

\end{boxtext}

\humansoldier[\npc{\T}{Eight Bandits}]

\elfprince

If saved, \gls{elfprince} will promise to repay the characters sometime, but immediately leaves.
If the elf dies, the party notice a magpie eyeing them (it will later relay the information to other elves).

\sqpart{Villages}% AREA
{\squash Karma}% NAME
{Three elves return to repay the characters for their previous actions}% SUMMARY

Play this encounter at the same time as the next Side Quest.
The prince's friends journey towards the sea, and they have heard of the characters.
If they aided \gls{elfprince} before, the troop will aid them in return.
If the characters harmed \gls{elfprince}, the elven troop will have heard of it from the bandits, or through a sentient bird.

The elves will not approach at first, but attempt to sneak and observe the party for a while.

\paragraph{If the elves want to hurt the PCs,}
they use their highest level spheres, hopefully while the PCs have already entered combat with someone.

\paragraph{If the elves want to help the PCs,}
they either aid in a battle, or offer a single favour.
The PCs must ask for the favour immediately, or they will never see the elves again.

\elf[\NPC{\M}{Sindon}{Lively}{Strokes blonde hair}{Experience}]

\paragraph{If the PCs ask about the bandits,}
Sindon can give them an approximate idea of where the immortal bandits are, as long as they can understand his references (he does not direct them according to which road and villages are near, but by river-currents and tree-types).
Understanding him requires Intelligence + Wyldcrafting, TN 10.

\elf[\NPC{\F}{Vanwe}{Dour}{Wipes eyes}{Experience}]

Vanwe was in love the \gls{elfprince}, and plans to either aid the characters well, or \emph{really} hurt them, depending upon their previous actions.

\elf[\NPC{\M}{Neuror}{Jovial}{Spits}{Tribe}]
\label{neuror}

\paragraph{If the PCs ask about \gls{lostcity},}
Neuror begins a repetitive rant about how it was all the humans fault for messing with nura magic, although he doesn't really know many details.

\stopcontents[sq]

\resumecontents[Villages]
\sidequest{\Glsentrytext{spiderqueen}'s Song}
\stopcontents[Villages]

\startcontents[sq]

\sqminitoc

\noindent
When elves become old, they get weird.
\Gls{spiderqueen} has left her people, and devoted her life to enchanting animals with song and fostering a kinship with them.
Currently, she has collected four pet chitincrawlers, but she's having trouble keeping them, because they require too much food.\footnote{See page \pageref{chitincrawler} for chitincrawlers.}
She has also begun Polymorphing her own body to look progressively more like a chitincrawler.

Over the course of these encounters, \gls{spiderqueen} becomes progressively more irritating to the party and the local area, until the players finally happen upon her lair.

Chitincrawlers don't operate during the Winter, so if you run into a Winter season, just miss these encounters until the world is warmer.

\sqpart{Forest}% AREA
{The Arachnid Double Cross}% NAME
{\Gls{spiderqueen} double-bluffs the party, attacking with illusory chitincrawlers, mixed in with real ones}% SUMMARY
\label{spiderqueenssong}

\paragraph{Background:}
\Gls{spiderqueen} wants her chitincrawlers to feed well (on the PCs, since they're here), but doesn't want them to get hurt.
She begins by casting an illusion of seven chitincrawlers, so the PCs will spot the fakery, and stop trying to resist so the real ones can eat them more easily.
To complete this illusion, she casts yet another illusion of a gnome sitting in a tree, so the PCs will feel convinced that nothing they see is worrying.

\begin{boxtext}

  You can always tell elven music by a sort of off-beat, where the beat goes wrong in a regular way.
  This one is soft and high-pitched, and interrupted by the sound of snapping twigs.
  More crackles come from in front.
  The setting Sun casts a red shimmer over the armoured bodies of a dozen man-sized chitinous, crawling creatures.
  The trees drop a small platoon of arachnids, and in a moment a hundred eyes are calculating how you taste.

\end{boxtext}

The PCs roll Wits + Vigilance to understand their environment.

\begin{rollchart}

  \textbf{TN} & \textbf{Result} \\\hline
  8 & The Sunset red on the chitin is too much, like the creatures don't look right.  You instantly spot that these are illusions. \\
  9 & On a nearby branch a little gnome sits, quietly giggling to himself, then looks shocked as you spot him. \\
  11 & The distant song seems to be coming from a single chitincrawler in the distance, \\
  12 & though she looks different from the rest. \\
  13 & Looking past the poor chitincrawler illusions in front of you, you notice that the rest are completely and definitively real. \\
  14 & The little gnome, however, is entirely fake. \\

\end{rollchart}

Ten chitincrawlers attack (3 real, 7 fake).

\paragraph{If any PC cannot tell the real chitincrawlers from the fakes,}
they will have real problems in combat as they will expend all their AP on the fake ones (which attack first).

\paragraph{If the kill a chitincrawler,}
the \gls{spiderqueen} `changes her tune', and sings another song to bring them back to her.

\paragraph{If the PCs get close to \gls{spiderqueen},}
she polymorphs into a bird and flies away.

\chitincrawler[\npc{\A\T}{3 Chitincrawlers}]

\sqpart{Villages}% AREA
{Sheep Stampede}% NAME
{\Glsentrytext{spiderqueen} summons sheep to be eaten by her chitincrawlers}% SUMMARY

\begin{boxtext}

  A nearby shepherd suddenly shouts out ``Hey!'', as he loses control of his flock.
  A distant song entices the sheep to run toward its discordant melody. It is as if the singer caused the music itself to decay.

\end{boxtext}

\paragraph{Background:}
\Gls{spiderqueen} has gathered more chitincrawlers, and she needs to feed them again, so they have laid out their webs, and await their dinner.

\begin{boxtext}

  The sheep go beyond sight, and into the distant trees, then the song stops, and they begin to cry out in a way you've never heard sheep cry before.  Half of them flee straight back out of the forest.

\end{boxtext}

\paragraph{If the PCs do nothing,}
the chitincrawlers feed for thirty minutes, then leave.

\paragraph{If the characters pursue,}
they encounters webs (Wits + Vigilance, TN 9 to spot), then see the chitincrawlers feasting on the sheep.
They will disengage and attack the characters if they take any Damage.

\paragraph{If the players mention specifically to look out for webs,}
their characters should spot them immediately.

As before, \gls{spiderqueen} waits in the distance, and flees at the first sign of trouble.

\chitincrawler[\npc{\A\T}{8 Chitincrawlers}]

\sqpart{Villages}% AREA
{Quiet Little Hamlet}% NAME
{An entire hamlet has been eaten by chitincrawlers}% SUMMARY

\widePic{Dyson_Logos/ruined_village}

\paragraph{Background:}
Most villages in the inner circle, closer to a town, feel safe.
If woodspies or chitincrawlers attack, they generally engage with earlier, walled towns, rather than those close to the centre of the circle of civilization.

Knowing this, \gls{spiderqueen} coaxed her babies past the dangerous walled towns, full of armoured archers, and brought them to a quiet little hamlet where they could feed safely, and lay lots of eggs.

The plan went perfectly, so she left them to sleep and mate, and plans to return when she has more little babies to bring home with her.

\begin{boxtext}

  The little hamlet rests quietly.
  The air is cool, but then a single cockerel lets off half a crow in the distance, and goes suddenly silent before he's finished.
  It's only then you really notice: the fields have no animals, and the farmhouse chimneys don't give out any smoke.

\end{boxtext}

However, inside each of the four farmhouses, rooms are filled wall-to-wall with webbing.  Each house contains the same thing:

\begin{enumerate}
  \item
  Dead villagers in webs.
  \item
  Half-dead villagers in webs, waiting to be eaten.
  \item
  Great sacks of chitincrawler eggs, ready to burst out and feed.
  \item
  One male and one female chitincrawler.
\end{enumerate}

\Gls{spiderqueen} herself has since moved away and left her creatures to multiply.

\paragraph{If the party want to leave}
then they can, without issue.

\paragraph{If the PCs make a lot of noise,}
all of the male chitincrawlers come out, and pursue.
The females remain with their eggs.

\chitincrawler[\npc{\A\T}{Male Chitincrawlers}]

\chitincrawler[\npc{\A\T}{Female Chitincrawlers}]

\sqpart{Forest}% AREA
{The Lone Ranger}% NAME
{A member of the \glsentrytext{guard} stalks \glsentrytext{spiderqueen}}% SUMMARY

\paragraph{Background:}
Gregory, a scout in the \gls{guard}, went out to track down \gls{spiderqueen}.
He has succeeded, but won't approach her alone.

\begin{boxtext}

  A man ahead, dressed in greens, stares at you, then slowly wanders forward.  He puts his finger to his mouth, indicating you need to be silent.

\end{boxtext}

Gregory approaches slowly, and explains his solo mission to track down \gls{spiderqueen}.
He confesses he feels scared, and asks the party to come with him (``but quietly.!.'').

\paragraph{If the party continue following the tracks,}
they encounter \gls{spiderqueen}'s lair.

\begin{boxtext}
  In the distance, you see trees covered in so much webbing it seems like a fortress of goo.
  No clear path presents itself, and shortly after the webs you can see five, then ten, arachnid silhouettes.
  A few move slowly down their trees.
\end{boxtext}

Having found the location, Gregory only wants to return to \gls{town} and give his report.

\paragraph{If the PCs fight,}
the chitincrawlers don't move out quickly -- they feel safer in their lair, but they may come out if fire begins to burn their home.

Twelve chitincrawlers in total remain in the fortress of webs, and two emerge each round.
\Gls{spiderqueen} then moves out to cast aggressive Polymorph spells, turning them into goats, birds, or other creatures.

\paragraph{If the PCs push for Gregory to join them,}
someone needs to roll Charisma + Combat, TN 10.
Whether or not they succeed, Gregory impresses on them that they have no duty to fight.

\humansoldier[\NPC{\M}{Gregory}{Suspicious}{Purses lips}{Tribe}]

\paragraph{If the party attempt to light the forest on fire,}
they will have a hard time.
The forest is muggy and damp at the best of times.
Even in the warmer seasons, lighting a fire requires an Intelligence + Wyldcrafting roll, TN 13.
No fire will spread fast, so the chitincrawlers will still have time to attack.

\sqpart{Town}% AREA
{\squash The Disappearing Fortress}% NAME
{The \gls{spiderqueen} has moved her fortress}% SUMMARY

While the next Side Quest plays out, drop the bad news on the PCs -- the \gls{guard} moved out en masse to defeat \gls{spiderqueen}, but by that time she had moved her home and brood elsewhere.

Nobody has any idea where she might build her new home, but they know the walls of civilization cannot take much more.

\sqpart{Forest}% AREA
{The Cunning Plan}% NAME
{Three gnomes have an elaborate plan for the party to kill \gls{spiderqueen}}% SUMMARY

\paragraph{Background:}
Three gnomes have been debating about how to approach the party about their plan.
\Gls{keras} thinks that it's best to honest, and just approach the party and ask if they would like to fight giant spiders.
However, Holly is the chief illusionist, and her nose is longer than \gls{keras}'s,\footnote{Gnomes consider this to be a very important point.} so she says there's no use talking to the party without testing if they really can fight chitincrawlers.
Greg, meanwhile, just wants both of them to stop fighting and make a decision.
He's been depressed ever since their village was eaten by chitincrawlers.

The final plan is to cast an illusion of a chitincrawler and see how the party react.
If they appear as skilled warriors, the gnomes approach and tell them the plan to defeat \gls{spiderqueen}.

\begin{boxtext}

  As you nip to the side to take a quick piss, a rustle above you shows that a giant arachnid has suddenly appeared, and looks down at you with dripping fangs.
  In the distance, high-pitched snickering can be heard.

\end{boxtext}

\paragraph{If the party flee,}
the encounter ends.

\paragraph{If the illusion of a chitincrawler has been vanquished,}
the three gnomes step forward.
Holly begins talking like she's some kind of trader.

\begin{speechtext}

  So you don't like the chitincrawlers?

  You really hate them?

  How much would it be worth to you to be rid of \gls{spiderqueen}, who guides them through the human villages?

  And you seem to be adventurers, in the employment of destroying monsters, is that so?

  And what if I told you that we could aid you pushing back against \gls{spiderqueen}?

\end{speechtext}

It's only after the characters emphatically agree that they do want to kill \gls{spiderqueen} that Holly informs them that she's feeling so generous that she's going to help them for free, and indeed has already laid plans.

If the players ask how the gnomes know exactly where she is, they explain they have triangulated her position through her periodic singing.
If they ask how the gnomes can be so certain that a half-kilometre tunnel, going somewhere the gnomes have never seen, can be so precisely dug, Gregory shows them his calculations.
An Intelligence + Academics roll at TN 11 shows that they are correct.

The players should be aware that if they jump out \emph{near} \gls{spiderqueen}, but not quite at her, they will be attacked instantly, with little hope of survival.
Their only hope is to break out of the earth, kill her in an instant, and hope the chitincrawlers flee once her spell has been broken.

\begin{speechtext}

  It's simple really.
  Anyone wandering close to that pit of spiders will be eaten by spiders.
  Any large army approaches, and she will flee, with no option to track here whereabouts.
  The only way to be rid of her is a fast, decisive attack.
  But she has herself covered there too -- not yesterday we spoke to an elf who had spoken to local birds, who informed us that even the tops of the trees there are covered in webs.
  Her mobile fortress is impregnable, and hungry, and they will feed again soon.

  However, with our compass and our calculations, we have found a different way.
  We know that she rests not a kilometre \emph{that} way, and so half a kilometre that way there is a tunnel which we have almost completed.
  Once done, it will open \emph{directly} beneath the very place \gls{spiderqueen} sits.

  You know what you need to do.

\end{speechtext}

If the characters agree to squeeze through the tunnel, dig the very last few feet, then burst out, then each one has to make a Speed + Athletics check at TN 7.
Success indicates that the character can spend 4 Initiative to climb out of the hole.
Failure indicates that the character will not be able to get out of the hole that round, and neither will anyone behind them.

\begin{boxtext}

  You look up at the wide eyes of \gls{spiderqueen}.
  She immediately starts climbing higher up the tree, as dozens of chitincrawlers all around race towards you.

\end{boxtext}

Once out, they can shoot at \gls{spiderqueen}, climb the tree, or otherwise attack her.
Two of the chitincrawlers will arrive to attack each round, but once \gls{spiderqueen} dies, any who have not yet come forward do not attack.

\keras

\gnome[\NPC{\F}{Holly}{Inquisitive}{Picks nose}{Tribe}]

\gnome[\NPC{\M}{Greg}{Creepy}{Scratches Adams apple}{Acquisition}]

\spiderqueen

\Gls{spiderqueen} has spent 4 MP to gain a spider-like body, with +3 Strength and DR 3.

\chitincrawler[\npc{\A\T}{Chitincrawlers}]

\paragraph{Success} means \gls{spiderqueen} has been killed or quelled.
If she's damaged and her chitinous children pushed back, she flees to seek new adventures elsewhere, and without killing random villagers.

\paragraph{Failure} occurs when the characters fail to damage \gls{spiderqueen} or her children before they flee.
Things get difficult here.
She attacks neighbouring villages twice, then gains her fifth level of Polymorph, and decides to become an air spirit for a while.
These two attacks play out as above, so she can only be stopped by a full-on assault at her lair, without the aid of a gnomish tunnel.

\stopcontents[sq]

\sidequest{Interruptions}\label{interruptions}

\startcontents[sq]

\sqminitoc

\noindent
The deep forest is no place to build relationships or get into prolonged battles -- it is a chaotic environment, where one never knows what the next day brings.
These disjointed Side Quests don't fit with anything in particular, but exist to provide little clues to other quests, or simple distractions.

\sqpart{Forest}% AREA
{\squash Broken Sword}% NAME
{One of the characters' weapons breaks}% SUMMARY

\begin{boxtext}

  Your sword plunges into the chitincrawler's face, but as you pull it out, the creature twists its body, and your sword shatters.
  You pull out the handle with a metal stump, and the next creature attacks.

\end{boxtext}

One of the characters' weapons shatters at just the wrong time.
The next time any character uses a m\^el\'ee weapon, that weapon shatters unless there is some reason it cannot shatter, such as being a magical item, or a weapon renowned for being of excellent quality.
If one weapon cannot shatter, move to the next which is used.

This encounter combines with the next Side Quest, so the players will most likely find a weapon shattering during combat.
This might happen when the character smashes their weapon into an enemy's, or perhaps when stabbing at an enemy so deeply the weapon embeds in a creature's hide, and then snaps off when the weapon is withdrawn.


\sqpart{Forest}% AREA
{The Curious Crawler}% NAME
{A chitincrawler tries to dig up a little gnome}% SUMMARY

A chitincrawler pulls up the earth beside a tree, as if trying to dig under it.
Shrewd characters might spot that under the tree a little gnome lives.
The side of the tree opens, revealing a very small staircase.
The chitincrawler has smelled the gnome cooking food, and has decided to stay up top and dig until he catches the little creature.

\begin{boxtext}

  A distant shuffling past some trees starts, then stops, then starts then stops.
  In the far distance, you see the dim silhouette of a chitincrawler scratching around the base of a tree, as if trying to dig something up.

\end{boxtext}

Wits + Crafts (TN 9) to understand that the tree leads to an underground home.

\chitincrawler

\keras

\paragraph{If rescued,}
\gls{keras} is delighted, and gifts the characters a scroll which -- once read aloud -- will cast an illusion of a chitincrawler.  It was studying chitincrawlers for his spells that got him into this mess.

\sqpart{Forest}% AREA
{\squash Weather}% NAME
{Seasonally appropriate weather strikes}% SUMMARY

Play this at the same time as the next Side Quest part.
The party are assaulted by weather.
Exactly how this plays out depends upon what type of season it is.
See \autoref{astronomy} for more details.

\begin{rollchart}

  Mild & Defer to the next Season, but lessen the effects and TN by 1. \\
  Stormy & A flash flood occurs.  The party must find a different route.  Intelligence + Wyldcrafting, TN 10.  Each margin of failure has the party lost for an additional day. \\
  Hot & The Sun beats down relentlessly today.  +2 Fatigue points. \\
  Cold & A sudden snowstorm comes, bringing cold and confusion.  The characters gain 2 Fatigue from the cold, and make an Intelligence + Wyldcrafting roll, TN 9.  Each margin of failure has them lost for an additional day.

\end{rollchart}

\sqpart{Forest}% AREA
{Random Traders}% NAME
{Three tradesmen are lost in the forest}% SUMMARY

\paragraph{Background:}
Aaron, carrying various flowers, and over a hundred eggs, started the day late, and knew that his cargo would be bad before reaching \gls{town}, so he convinced Jason (carrying uncured meats) and Steve (carting blood sausage) to go with him via an old road his grandfather told him about.
However, the road is completely overgrown, so the traders are now stuck in the woods, and lost.

\begin{boxtext}
  In the distance, you see a group of a dozen men trying to get their first wagon out of a muddy ditch.
  Two more wagons sit behind.
\end{boxtext}

\humantrader[\npc{\T\M}{Aaron, Jason, and Steve}]

\sqpart{Forest}% AREA
{The Elven Party}% NAME
{The party are told to dance, and dance they must}% SUMMARY

\begin{figure*}[b]
\begin{speechtext}

  They never attacked, but cleaned up the mess.  Alchemists back in those days had no legal restrictions, and many were originally tradesmen.  They opened portals to various other lands, and soon began trading goods with the strange creatures there.

  Nura came through at one point, invaded the city, and laid waste to it.  My grandfather came to save people, but upon seeing the complete ruin of the city, devoted himself to ridding the area of nura.  He died in that war, as so many other elves did.

  Men did not repopulate the area because of their grief, and their fear of undiscovered portals.
  Perhaps four in total were created before the assault began.
  If people populate the area, it will only be a matter of time before they find those now-buried portals, and try to make use of them.

  But this is a dark conversation, and we wandered to dance, and celebrate the changing season.

\end{speechtext}
\end{figure*}

Elves have better eyesight than most, so many of their feasts take place in the darkness, and involve games of hide and seek, or enchantment.

\begin{boxtext}

  Off-kilter music and half rhyming words, wander out from the forest, then gentle footsteps to the far right, and more in the distant left.

  The scent of fresh fruit, salads and salmon hit you.  There's a low-burning fire in the distance, looking enticing.

\end{boxtext}

The elves hear the characters, and quickly hide as a game.  Those at the farthest reaches of the gathering shout out that the game is on, and everyone between hides quickly.

\begin{boxtext}

  The noise of little feet darts around the silent forest, but nobody responds.
  A feast lies on the blanket.

\end{boxtext}

\paragraph{If the characters eat the food, nothing bad happens.}
It tastes great.
The game doesn't end until the characters settle down to eat or they find an elf.

The characters can roll Wits + Vigilance, TN 8 to see how quickly they find an elf, but there are two dozen, so it's only a matter of time before they see one.

Once the game is up, all the elves come out of hiding and laugh.  They dance, and sing, and feast.  However, the elves get a little too carried away, and eventually enchant the party to continue dancing all night.  The elven illusions make sure that the songs echo long past when the singers have gone for the night, and the characters just continue dancing.
 
\elf[\NPC{\F}{Aiw\"{e}}{Jester}{Looks upwards}{Tribe}]

Aiw\"{e} loves a laugh but never learnt when she's gone too far, and will fashion leaf-crowns for dancing characters, adorning them while they dance.

\elf[\NPC{\M}{Taurestel}{Pedagogue}{``For example\ldots''}{Experience}]

\elvenenchanter[\NPC{\F}{Erende}{Curt}{Raises Eyebrow}{Acquisition}]

\paragraph{If the characters ever ask about \gls{lostcity},}
and whether or not elves destroyed it, the elves present say that they were never there, but have heard the story from their elders (see below).

\paragraph{Once the party is in full swing,}
ask for a Wits + Academics roll, TN 12.
Each time the party fail, they dance for another scene and lose 3 Fatigue points.
They dance until they pass out or until someone succeeds in the roll.

\paragraph{But if they become violent,}
Erende responds with a \textit{Fast, Wide, Sleep} spell.
There are twenty elves, so the characters have little chance of victory.

\sqpart{Forest}% AREA
{Furry Traders}% NAME
{Three gnolls are here to trade}% SUMMARY

\paragraph{Background:}
The gnolls have caught four deer in a trap, eaten one, and cured the meat from the other three.  They are willing to trade.

\begin{boxtext}

  In the distance, hunched humanoids carrying spears and heavy loads on their backs stop suddenly.
  They eye you up, then come forward, with ears pricked up high.
  It's a group of gnolls, carrying large sacks.

\end{boxtext}

\gnollhunter[\npc{\T}{6 Gnolls}]

\sqpart{Forest}% AREA
{\N The Mouth of Hell}% NAME
{A thousand woodspies have gathered around a portal to Hell}% SUMMARY

Some encounters cannot be bested. The only thing for the party to do is take running away as the best possible victory.

\begin{boxtext}

  Pushing more foliage aside, you notice this area looks strange somehow, like the trees are made of wax.

\end{boxtext}

\paragraph{Background:}
An underground kingdom of nura have found a portal to Fenestra, and one might expect hobgoblins and nastier things to pour out and devour people.
However, the first few were caught by a woodspy, who bred, and started a family, and the same happened to the next few.
At this point, around 300 woodspies live in an entirely unnatural (and slightly inbred) alliance around a single hole in the ground, where their food comes from.

On the other side of the portal, the nura have begun using the mysterious hole to dispose of their unwanted.
Criminals, wounded and useless nura, or just those who annoy everyone, find themselves thrown into the hole.

\begin{boxtext}

  Just ahead of you, you see a pit lined with stones, each with expensive gems and covered in alchemical writing, carved into the rock.
  An intense heat emanates from the pit.

  Suddenly, the waxy parts of the trees start to move, revealing itself to be a three-limbed creature, shifting across the bark.
  Behind you is another, and another, and then an entire tree pulls itself apart, revealing another two dozen of the shape-shifting creatures.

\end{boxtext}

\paragraph{If the woodspies ever leave the area,}
the nura will have a safe portal to Fenestra, and they will raid the local area.
If the portal ever closes, the woodspies will become ravenous, and invade the local population.
The two stand in a tasty equilibrium, and the best the characters can do is flee.

The woodspies are fat and unconcerned with chasing the characters far.

\woodspy[\npc{\T\A}{300 Woodspies}]

\paragraph{Take a pen,}
and add this nura portal to your map of the area, wherever the PCs have wandered.

\paragraph{If the local nura rating ever reaches 7,}
the nura will raise an army big enough to break out, and drag any woodspies in the area below to become nura themselves.
Raise the local Nura Ratking by 1.

\nurawoodspy[\npc{\T\N}{Nura Woodspies}]

\stopcontents[sq]

\end{multicols}

\stopcontents[Forest]

