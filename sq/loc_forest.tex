\section{Forest Locations}

\begin{multicols}{2}

\setcounter{list}{0}

\subsection{The Necromancer's Lair}\label{necromancers_lair}
This used to be a large chapel with a tended garden.  Paradoxically, an undead tender has left it with more life, longer grass, and undisturbed apples.

\toppic{Dyson_Logos/qualme_temple}{\label{qualme_temple}}

\mapentry{The Area}

Around the central area wander all the undead \gls{necromancer} has collected (depending upon which encounter the characters are on, this could be 50 -- 400.  Mostly, they stand inert.

A hundred broken arrow parts lie littered around the area, as \gls{necromancer} practices with his bow daily.

\ghoul[\npc{\D\T}{Ghouls}]

\mapentry{The Hallway}

Here, despite the undead, a shrine rests which was built be \gls{necromancer}'s predecessor, showing ten faces carved in stone, each above the other.  This magical shrine grants $1D6 + 3$ FP to anyone who prays for the guidance of an ancestor.\footnote{The items acts with Intelligence +2, Wits 0 and stores 9 MP. It spends 2 MP to cast the spell.}  Characters can work this out with an Intelligence + Academics check, TN 7.

\mapentry{The Ogre}

\Gls{necromancer}'s prized specimen -- an undead ogre.
\Gls{necromancer} killed the ogre some time ago, and pulled the body back to undeath.
Since then, \gls{necromancer} has cobbled together leather armour to glad the oversized ghoul in.
Now he stands humongous and impenetrable.

\npc{\D\N}{Undead Ogre}

\person{6}% STRENGTH
{0}% DEXTERITY
{0}% SPEED
{{0}% INTELLIGENCE
{-4}% WITS
{-5}}% CHARISMA
{2}% DR
{2}% AGGRESSION
{Deceit 1}% SKILLS
{\greatclub, \completeleather}% ABILITIES
{}%

\mapentry{The Secret Study}

After the bandits leave, \gls{necromancer} sneaks back to his private study, through a stone door, balanced on massive iron hinges.  It contains various hymns to the dead, but his favourite songs are not the prayers to Qualm\"{e}, but song magic.

One song calls all the local woodland creatures to attack the singer.%
\footnote{Page \pageref{medalofheroism}.}

\mapentry{The Prison}

This room once housed people making important decisions.  \Gls{necromancer} now uses it to house prisoners so he can feed off their souls.
Currently, it contains one terrified farmer called Laith.
He's starving, and petrified, as every day all he can hear are the shambling dead, who sometimes come to grope at the locked door.
\Gls{necromancer}, or course, holds the key.

Picking the lock requires an Intelligence + Larceny roll, TN 7.

Laith can join the characters if given a weapon, but he won't be terribly effective.

\humanfarmer[\NPC{\M}{Laith}{Pessimistic}{Mouthbreather}{Acquisition}]

\mapentry{The Watchtower}

While this place used to host a call to prayer, it now only provides a place for \gls{necromancer} to watch the world with horror.  He is convinced someone will come to try to kill him, because he feels so at odds with the world, and he practices with his bow every day.

Various crows come to see \gls{necromancer} as he takes bits of rotten meat so they can feed, then controls their minds with his prayers.

On a simple table he keeps his crow-feathered arrows and a little collection of three necklaces which allow the crows to raise the dead.

\magicitem{Silver Amulet of Return}%
  {Very Enervated Ghoul Calling}% SPELL
  {Devotion (Qualm\"{e})}% PATH
  {Instant}% SPEED
  {Talisman}% TYPE
  {4}% POWER
  {3}% MANA
  \label{ghoulNecklace}

\thenecromancer

\pic{Decky/necromancer}{\label{Decky:necromancer}}

\mapentry{The Old Mausoleum}

Ironically, no undead remain in the mausoleum.  The grounds remain fit for the living, and a group of bandits have moved into the gardener's old home.

Each of the bandits carries a Ring of Asphyxiation.
The rings are fragile things, carved from skulls of thieves who were hanged.
They function as per the first level necromancy spell: Torpor, and allow the thieves to remain invisible to the undead.

If anyone attacks the house, they will be spotted immediately, but have two rounds before the undead arrive.
After that, five ghouls arrive to tear apart anyone there, then ten, each round.

\banditking

\Gls{banditking} was the son of a nobleman in Whiteplains, but his family have been killed, so he became an outlaw.
Last year, he and his men ambushed some guards working for \gls{banditking} and took magical rings from their corpses.

The bandits are partly made from young Whiteplains nobles, orphans old enough to hold a sword, and a few dispossessed villagers who lost their villages to nura raids.

\humansoldier[\npc{\T}{15 Bandits}]

\subsection{\Glsentrytext{lostcity}}\label{lostcity}\setcounter{list}{0}

\toppic{Dyson_Logos/forgotten_city}{\label{lost_city_map}}

\Gls{forestpriest} has been staying here for some time, trying to figure out if the elves destroyed this city centuries ago, or if (as the elves say), the priests here opened a magic portal to a hellish land of nura.
Once the party arrive, \gls{forestpriest} requests they go down a tunnel he's found, to see if it contains such a portal.

\begin{boxtext}

The trees here cling to blocks of stone, as if trying to crush the last remnants of civilization.  You sometimes wander for so long without seeing a single brick that you forget you're anywhere but a normal forest, but then another errant stone, or the marble hand of a statue juts up to remind you of the ghosts resting in this place.

\end{boxtext}

The place is now heavily populated by woodspies.  Replace any random encounter that would be person (human, elf or otherwise) with a woodspy (page \pageref{woodspy}).

\mapentry{The Old Citadel}\label{lost_citadel}
While most of the ancient ruins have decayed so much that people cannot see them from the side of their path, the central citadel remains strong.
One complete room, with half a roof, has a bed and equipment for making tea.
A single scroll rests on the wall -- a children's prayer to Laiqu\"{e}.

\Gls{forestpriest} keeps two torches for venturing underground.

The nearby hidden trapdoor underground cannot be discovered from above -- in fact it's so tightly covered with rock, then earth, then well-made roofing, that it's nearly waterproof.

\mapentry{The Swarming Grove}

All of the trees in this area burst with ripe fruit, but this grove has the best.
This is why it also has all the woodspies; in total around 20 sit in these trees invisibly.
\paragraph{If anyone enters to grab the best fruits,}
they will see the trees around them mutate, as camouflaged woodspies slowly move to surround them.
Nearby, a pile of dead deer bones is loosely hidden under some leaves.%
\footnote{Remember, woodspies are rather intelligent animals, and know not to leave corpses lying around a trap.}

The party are unlikely to be in any state to defeat these creatures.  Three arrive per round, so the best anyone can do is flee.  If the woodspies are not harmed, three will venture outside of the tree cover to attack the party, but will quickly retreat at the first sign of real danger.

\woodspy[\npc{\T\A}{30 Woodspies}]

\mapentry{The Mana Lake}

This lake blossoms with magical energies, and regenerates 4 MP per turn to anyone touching its waters.

Two woodspies rest just under the water's surface, camouflaged to look exactly like the base of the pool.
Anyone getting close must make a Wits + Vigilance roll, TN 10, to spot them.
Failure, of course, means being dragged under water by the first.
Anyone helping must roll or be dragged underwater by the second.

\mapentry{The Old Door}

\begin{boxtext}

  Across the shining lake, you see a man standing next to an old wooden door in the steep hillside.
  He stares at you, then raises his hand to say `hi', but does not move from where he is.

\end{boxtext}

\paragraph{Once the party approach,}
he explains that Fenestra requires their help.
He needs to find out why \gls{lostcity} was destroyed.

\paragraph{If the party offer one explanation,}
he gives the other.
He makes sure that the party are aware of both theories.

\begin{speechtext}

  The humans say that the elves tore the city down because the elves disliked humans growing so powerful.

  The elves say that the humans used nasty magics to open portals to the nura realm, and goblins spilled out.
  They say the city was already mostly ruined, and they had to save the area through magic.

  This old door leads to the basement of some old alchemical research building.
  If a portal to the nura realm exists, it exists here.
  I need to know if it really does contain such a portal.

  If you could go down, we could know for certain what the answer is.

\end{speechtext}

\paragraph{If the party ask for payment,}
he says he's already paid all the money he can to local bandits to find out what's in the well, and they never returned.%
\footnote{Specifically, these were the woodspy bandits.}
Therefore, there should be plenty of silver on their bodies, if the party can only find the corpses.

\paragraph{If the party ask him to come with them,}
he'll only go down with the party if he has to, and if he trusts them, but will not allow himself to be surrounded by people he doesn't trust and could stab him in an instant.

\paragraph{If the party ask him to clarify the plan,}
he explains that if he can get witnesses to spread the word about what really happened in \gls{lostcity}, he will have the political backing to put a stop to expansions into this area.

If men return here to uncover the ruins containing portals, they may repeat their past mistakes.
If the elves have destroyed the area in the past, they may do so again.

\paragraph{If asked about turning people into animals,}
he refuses to speak on the subject until after the contents of the old wooden door have been thoroughly inspected.

\forestpriest

\subsection{The Old Alchemy Basement}\label{old_alchemy_basement}\setcounter{list}{0}

The entire basement of the old magical college is sodden with water, resting knee-height to a human.

\begin{boxtext}
  Descending the stairs, you find a low ceiling, and a moment later correct yourself.  It's not a low ceiling -- black, stagnant, water has flooded the entire hall.

  The torch picks up a great stone pillar in the distance, and another a little farther along.
  Great double doors along the hall, to the right.

\end{boxtext}

The characters will find movement difficult while wading through the water.  All movement is limited by 2 squares minus the character's Strength Bonus, so some will receive no penalty, while those with Strength -2 receive a 4 square penalty to movement each round.  For many, this will mean they cannot move at all without spending a full round pushing forward, or simply swimming.

Remember to note who has torches when underground, and that carrying a torch in one hand means that any medium-sized weapon in the other hand receives a weight penalty.

\paragraph{Doors} are a particular problem in the catacombs.  The water has swollen the wood, making all doors difficult to pry open.  Each one requires a Strength + Crafts check, TN 7.  This is a party roll, so the roll's result is the same for each party member.

The doors have rotted rather a lot over time, so prying them open with time is still entirely possible, it just takes a lot of time.

\paragraph{Narrow hallways}
make wielding long weapons challenging.
The \textit{Enclosure Rating} for this place is 5, so any weapon which requires 6 Initiative to wield takes a -1 penalty to Strike.

\paragraph{The Dead Chant} when not in combat.
If they stand at the other end of a hallway, they chant.
If the characters lock them in a room, the dead stand outside and chant while clawing at the door.

This strange behaviour could vex any necromancer.
The simple spirits which inhabit and animate ghouls do not usually speak.
Their strange behaviour is the result of a powerful necromancer living in the catacombs.
The undead necromancer\footnote{See page \pageref{undead_ogre}, just below.} found a scroll some time ago which contains the words to open the door, and summon creatures back from the Realm of Darkness and Fire, and other realms.
There was only one problem: the creature's tongue was too rotten to speak the words properly.
Its body has only been preserved due to the high content of peat in the water, but that was not enough to allow it to speak properly, so it tried teaching the dead to chant.

Characters versed in old languages might make a Wits + Academics roll, TN 12, to understand the words the dead are trying to say.  This is reduced to TN 10 if the characters somehow have the opportunity to hear the giant necromancer himself chant.

The words themselves simply mean `open to trade', but the characters will hear only ``Opena trei, opena trei, opena trei!''.

\paragraph{Cave-ins} present a real danger here.  If the ceiling ever collapses while the characters are inside, the falling rocks from above at first deals $1D6-2$ Damage to everyone in the room, then $1D6$, and so on, increasing by 2 each round, until it's unliveable.

\mapentry{Drowned Hallway}

\begin{exampletext}

When the city was still burning from the nura attack, some centuries ago, one necromancer raised a powerful undead spirit into the body of an ogre.
That ogre demi-lich then used the corpses around him to raise a regiment of ghouls.
The door was sealed during that time, peat-filled water flooded in, and the dead rested there, perfectly embalmed and perfectly still.

\end{exampletext}

\paragraph{Once the party have walked some distance into the hallway,}
the dead rise and attack.
It's been a long time, and their joints are stiff.
The ghouls naturally cling together, so a natural ambush forms as the characters may walk past a number of them before they notice the placid bodies on the drowned floor.
The dead will not attack them at the entrance, but stay underwater, with a natural fear of the door going above, the light outside, and acting alone.
They are hungry, but know to wait and observe, for a little while.

Five ghouls rest at the start of the hallway, and another five later on.
A further five at the other end of the hallway will arrive three rounds later.
The entire situation makes for the perfect ambush, though the dead have not planned for it.

\ghoul[\npc{\T\D}{15 Ghouls}]

\paragraph{If the party attempt to fell any pillars,}
have them roll Strength + Crafts, TN 13 (or less, if they use the right equipment, such as rope).
Once the pillars fall, the entire area floods within two rounds.

\pic{Dyson_Logos/under_lost_city}

\mapentry{Equipment Room}

\begin{boxtext}

  The shelves in this wide room are full of smashed and broken equipment, but it looks generally alchemical.

\end{boxtext}

\begin{exampletext}

The standard alchemist's equipment -- gold dust, rubies, beechwood, chitin, and black soil -- have mostly been removed from the area during the panic when people fled.

Some of those panicked people returned and were dragged back into the portal, only to return as ogres.  Those ogres were resurrected as undead, along with everyone else.

\end{exampletext}

\begin{boxtext}

  A powerful force grabs your ankle, and squeezes.
  A creature, taller than any man, stands up and turns you upside down, then pulls you in towards his teeth.

\end{boxtext}

\npc{\T\N\D}{3 Undead Ogres}

\animal{6}{-3}{0}{-3}{3}{2}{}{Death Sight}{}

\paragraph{Characters who scour the room,}
can find rare gems on the ground, although wading through all the sludge will not be pleasant.
This requires a Wits + Caving roll, TN 6.
Each marginal point means 15sp worth of items has been found.

\mapentry{Library}

\begin{boxtext}

  These two doors stand locked and refusing to budge.  A simple brass lock stands rusted on the front.

\end{boxtext}

Opening \emph{this} door requires a Strength + Crafts roll, TN 11.

\begin{boxtext}
  The doors throw inwards, revealing row upon row of rotten books.
\end{boxtext}

\paragraph{Careful perusal of the books,}
allows a few items to be discovered.
Anyone searching joins the Group Roll of Dexterity + Academics, TN 6.
Each margin allows a particular book to be carefully extracted, but destroyed the book above, so rolling `8' means the valuable book on Invocation is found, but the letter is destroyed.
Destroyed books simply fall apart due to rotten spines and damp pages, but if taken back and cared for, some could be preserved.
The party can make any number of rolls each, but each person can only make two rolls per scene to hunt for valuable books amid the dark mess.

\begin{enumerate}

\setcounter{enumi}{7}
  \item{An ancient city map, detailing sites of interest such as a Temple of Qualm\"{e}, which holds beautifully decorated, and unaging corpses (see page \pageref{green_tower})}.
  \item{A valuable book on Invocation, worth 20sp.}
  \item{A letter stating that a portal to an unknown labyrinth realm has been opened, and that trade has opened with various dwarves in exchange for food.  It also states which word will activate this portal.}
  \item{Letters of complaint from the Dean of Conjuration, stating that the Dean of Illusion must tidy his room, and that the rats he's brought in have become so bad that he's ordering no food to be permitted in the area, under any circumstances.}
  \item{A book by a priest of Qualm\"{e}, detailing various evil spirits.  It mentions the dungeon itself, saying a necromantic spirit is known to inhabit the area.}
  \item{A valuable book on late-stage Conjuration, worth 100sp, covering the highest level of the Conjuration sphere.}
  \item{Threatening letters from elves saying to be wary of opening portals.}
  \item{Hidden behind some other books: a book on Nuramancy.  This is highly illegal, but allows anyone to gain up to a single level in Nuramancy with a little study, and the right (or wrong) attitude.}
  \item{A book on opening a portal to the Realm of Darkness and Fire.}
  \item{A letter granting permission to open a portal to the Realm of Darkness and Fire, with the hopes of trading magical items for food.}

\end{enumerate}

\mapentry{The Masters' Quarters}

\begin{boxtext}
  These doors swing open effortlessly, showing a new room with three more doors; right, left and centre.

\end{boxtext}

Three rooms here used to house the various masters of alchemy.  Stairs reached down to a central pillar, then back up.  At the moment this deeper area is filled with water -- the characters must swim to any other doors they want to approach.

Unfortunately, yet another undead ogres sits at the bottom of this black water.  It won't jump up to scare the characters and battle with them, but watches then with shark-like eyes, then if one steps foot into the black water, it simply grabs the foot.

This is ogre no ordinary undead creature, but a ghast.
It thinks, plans, and knows how to cast spells.
It contains some unknown spirit, summoned into this unwholesome shell.
It is responsible for devouring the souls of the first group to enter the area, and using their energy to kill and raise the second group as undead.

\npc{\M\N\D}{Undead Ogre Mage}\label{undead_ogre}

\person{6}% STRENGTH
  {-3}% DEXTERITY
  {0}% SPEED
  {{2}% INTELLIGENCE
  {0}% WITS
  {-5}}% CHARISMA
  {3}% DR
  {1}% COMBAT
  {Aggression 2, Academics 2, Ether Lore 3, Medicine 1\Path{Nura}{\necromancy~4, \saurecanta~3}}% SKILLS
  {\longsword}% EQUIPMENT
  {\addtocounter{xpbonus}{3}}

\paragraph{Room A} used to house the master of Conjuration, who build this portal.
The ogre has kept him around for his own amusement, as a ghoul.
He still has 3gp worth of jewellery.

\paragraph{Room B} houses nothing but broken furniture and sludge.  The last room, however, is different.

\begin{boxtext}
  The heavy door creaks open to an attractive room, like an expensive upstairs room in a tavern, complete with a bed, a study, and a freshly cooked breakfast on the table.
\end{boxtext}

\paragraph{Room C} used to house an illusionist, and his spells are still going ever since he died.  Instead of cleaning his room, he would simply cast an illusion of cleanliness.  The room looks immaculate, and full of light.  The undead mage didn't like the light, so he closed the door.

Within the room, under the comfortable-looking (but filthy) bed, is a hidden little tunnel, which leads up to a secret room.
The ogrish mage cannot follow the characters here, even if he wanted to put up with the irritating light, because he is far too large to fit through the narrow opening.

\mapentry{Secret Study}

Up the stairs the area remains dry, safe and eventually leads to a regular door (no roll required to open it).  Inside, the room contains tables with extremely old scrolls, dust, and a series of very out-of-date books on alchemical theory.  These are among the scrolls:

\begin{boxtext}

  The stairs reach up, and finally you step your muddy boots out of the water and along a cold, but dry corridor.

\end{boxtext}

\begin{exampletext}

  I shall see you by Laiquea.  Have the portal completed.  We have no funds.  Five lands mapped.

\end{exampletext}

\begin{exampletext}

  Some funding came through.  They want mutton, beef, bread and soup.  Everything must be prepared before sending, except the meat.

  Prepare the food.  Destroy this letter.

\end{exampletext}

\begin{exampletext}

  The portal has been established.  Negotiations are going well, but please have more guards available than last time.  Excuses aside, we can't have a repeat of the last incident.  Three women.  It doesn't sound good in song.

  Of course if you want my advice we would put every bard in the kingdom to the sword and be done with the matter.

\end{exampletext}

A Wits + Vigilance check, TN 10, reveals a loose wooden board in the ceiling.
It used to be a secret exit to the ground floor of the Citadel above,%
\footnote{See page \pageref{lost_citadel} at the centre of area 1.}
but now the upper floor is just the ground outside\ldots after a lot of digging upwards.

\mapentry{Giftschrank}

This bare room used to store various books, including the words which open the portal.  It's flooded, like every other room on its level.

\begin{boxtext}

  The bricks fall away easily, revealing a full new room.  Two skeletons rest on a table, each clutching a book.

\end{boxtext}

The two skeletons on the table have aged worse than the other corpses, as they were never preserved in the peat-water.
They died of hunger rather than facing the dead they knew to be outside.
One holds a book of poetry, and the other holds a book of conjuration which she never managed to understand before dying.

The book of conjuration is outdated, but still worth at least 20gp to \gls{college}.  The book of poetry is pleasant, and hides one spell-song -- a poem which still functions to stop the user fearing any type of problem and regenerates 1D6+1 FP (it holds 3 mana, and costs 2 to cast).

Finding the words which unlock the portal requires an Intelligence + Vigilance roll, TN 9.\footnote{This roll can be made individual -- not a party roll -- since the information is there, and anyone has a chance to find it.}
It's hidden among a dozen rather dull books on proper etiquette with alchemy, and accountancy books concerning what the guild brings in and what is can produce.

The door to the Summoning Room is only blocked by a wooden bar, so exit is easy, though entering this room is more difficult.
Players trying to bust in the way must roll Strength + Crafts, TN 10.

\mapentry{Summoning Room}

At the moment, the arch leads nowhere  -- just the back of the room.  The enclaves are bare.
However, the words across the portal show what needs to be said -- ``Open to trade''.
The language is old but an Intelligence + Academics roll, TN 9, will allow anyone to understand it.
Once the words are spoken, every gemstone in the room shines, and the portal opens.
The ghouls then begin to echo the words after the characters in unison.\footnote{As usual, speech costs 2 Initiative points, so if the ghouls are in combat once the words are spoken, the party should enjoy the unexpected advantage they get.}

\begin{boxtext}

  The massive double doors slowly swing inwards, and the torchlight reveals a hallway of six stone pillars, two enclaves, and a stairway leading up to a stage.  The stage shows a grand stone arch, like a doorway, leading to darkness.
  You can see an writing across the top.

\end{boxtext}

This magical portal is in no state to be moved -- the magic relies on the room's composition being able to work together, with various gems on the pillars being required for the magic to work.
A single piece missing means the magic is dead.%
\footnote{The gems are worth 30gp in total.}

The water hides various dead, but these ones have been locked away for some time, and have become mummified by the peat, and so mostly unable to move.  The dungeon's necromancer has planned for the party's arrival, or someone like them, a long time ago, and has tied the dead to the first two pillars, with chains.

If undiscovered, the dead stand and begin their chant, then slowly walk towards the characters.

\begin{boxtext}

  You look behind, and note two-dozen dead men standing from the water and staring at you.
  Their skin has gone brown with age, and they look barely able to move.
  Each drags a chain behind it, tied around one of the entrance pillars.
  They pull together towards you, each uttering the same strange, chanting moan, and then stop as the chains go tight around the stone pillars.

\end{boxtext}

Each stone pillar has 10 dead pulling at it, and will collapse in 3 rounds.  Killing one of the dead buys one extra round, and at least 3 ghouls are required to pull the pillar down.

\paragraph{If the characters open the portal,}
they see a dark room, with a distant light.
What might be less obvious is that the portal opens on the \emph{ceiling} of a room in the Realm of Darkness and Fire.
Anyone throwing an item in notices it flies, then `sticks' to the far `wall' (meaning, the ground).
Characters may notice the discrepancy from the odd appearance of the doors on the other side, with a Wits + Crafts roll, TN 10.

If a single sound is made here, if an item or person drops through the portal and into the far room, the alarm is sounded in the Citadel on the other side.\footnote{See page \pageref{darknessandfire} for the realm the nura inhabit.}
Ten hobgoblins immediately arrive with a ladder and start making their way up, into the dungeon.
They know the portal can open, and they know they need a password.
They fight, but try to keep the characters alive so that they can learn the magic word which opens the portal.

If the characters drop through the portal, they have a fight on their hands.
After that, immediately roll for an encounter in the citadel.
The characters are five areas away from the Citadel's edge, and each area prompts for a new encounter.
After that, you're on your own -- you'll have to think up some opportunities for the party to make it out alive, and find another portal -- perhaps one leading to Redfall, or the portal in \gls{town} sewers.

\subsection{The Green Tower}\label{green_tower}\setcounter{list}{0}

\toppic{Dyson_Logos/green_tower}{\label{green_tower_map}}

A Temple of Qualm\"{e} once stood here, but was destroyed with the rest of \gls{lostcity}.
Now \gls{townmaster} has sent masons, some of whom are members of the Woodspy Bandits, to build a base of operations for him to begin rebuilding the old human city.

However, everyone building here is unaware that the lower parts of the temple are still active.
A portal to the nura lands below sends creatures up, and the undead Golden Priests of Qualm\"{e} have been pushing them back repeatedly over the long years.

\mapentry{Outer Grounds}

The area around the tower contains piles of rock which labourers have collected from the surrounding ruins.  The only door to the tower lies in the outhouse.

\paragraph{If anyone in the party casts spells,}
they noticed immediately that there is a magical hum of energy in the area.
Everyone here regenerates 3 MP per scene.

The source is a mana lake, slightly underground.
Any mage can easily locate where the energy is strongest by simply wandering about, and eventually identifying a large rock which blocks the path underground (and has done, for decades).
The rock has a Weight Rating of 8, so a total Strength of +4 is required to lift it.%
\footnote{Of course another option is a lot of time, digging, the proper use of a lever, with an Intelligence + Crafts roll, TN 8.}

\humanfarmer[\npc{\T\M}{6 Masons}]

\mapentry{Outhouse}

Overnight, the labouring equipment rests here.  Cunning characters can grind all work to a halt by stealing the items (picks, splitters, hammers, measuring tools, et c.), if they can manage it quietly.  The lock is a simple knot tied on the inside, and anyone slipping a knife inside can get in (Intelligence + Larceny, TN 5).

\mapentry{First Floor}

Resting slightly above ground, the first floor contains one lavish room, which nobody is allowed to enter.  It's left in preparation for the arrival of \gls{townmaster}.

\mapentry{Second Floor}

The men sleep here, though it's eventually planned as a station for lower-level archers.

\mapentry{Third Floor}

The top floor provides a place for \gls{traitor}, overseer of the operation, to get a good look at the surrounding area.
This is also where he keeps a stockpile of weapons:

\begin{itemize}

  \item{20 longbows}
  \item{50 longswords}
  \item{50 suits of partial leather armour}

\end{itemize}

\mapentry{The Stairs}
\label{underGreenTower}

Years of growth and soil-spillage have left a thin layer of mucus on the stairs.
Anyone descending must make a Dexterity + Athletics roll, TN 8, or fall down one staircase, taking $1D6-2$ Damage.
Each of the three staircases require a different roll.
Characters can get a bonus for proper equipment, such as rope.

The air down in the tomb has become so dry and foetid, that anyone spending time there gains three Fatigue Points per scene.
Of course, this can be offset with sufficient rest, but prolonged fighting can be difficult.

The stairs and all hallways here have an \textit{Enclosure Rating} of 5, meaning that any weapon which requires more than 5 Initiative to use gains a -1 penalty to hit with.
Daggers and shortswords are fine, but longswords and war axes will suffer.

\mapentry{The Watcher}

The tomb is guarded by a knight who stands in a side-cupboard, sworn to guard the undead priests.
He has spent a long time down here, and his body so seized up that he cannot move.
Sufficient time, however, will allow him to regain the use of his limbs, and open the door.

\begin{boxtext}

  You swing the door open to find a highly decorated corpse with a pendant to the god of dead, made of gold, with bone strung along the copper thread.
  The clothes look like they were once silk, and the helmet's leather covering seems to be a human face, stretched out and tanned.

\end{boxtext}

If the PCs stab the corpse, they can kill it.
More likely, they will not realize it is sentient, and leave it alone.
A moment after they leave it, Jabril animates and comes to kill them.
He is intelligent enough to know to sneak.
His use of a shortsword also means he will not gain any penalty from the narrow hallway.

If the characters enter with the Torpor spell cast, or rings of asphyxiation,\footnote{See page \pageref{ring_asphyxiation}.} they will be able to pass Jabril and the Golden Priests invisibly, so long as they do not linger too close to any of them.

\npc{\M\D}{Jabril, the Undead Watchman}

\person{3}% STRENGTH
  {2}% DEXTERITY
  {0}% SPEED
  {{0}% INTELLIGENCE
  {0}% WITS
  {-5}}% CHARISMA
  {2}% DR
  {2}% COMBAT
  {Aggression~2, Stealth~2, Tactics~2, Vigilance~1}% SKILLS
  {\shortsword, \completeplate, 200sp worth of jewellery}% EQUIPMENT
  {}

\mapentry{Hall of the Golden Priests}

\begin{boxtext}

  Five dead men, mummified and covered in golden jewels, stand in each of five enclaves at the side of the room.
  You notice head wounds and missing limbs upon some of the bodies.

  Little spears litter the area, as if a battle has taken place here.

\end{boxtext}

\begin{exampletext}

A long time ago, onlookers came here to gawk at the splendour of a glorious afterlife.  The priests of Qualm\"{e} who gained the highest honours of the temple would remain here to guard it forever.  Each is decked in golden jewellery.

\end{exampletext}

\demilich[\npc{\T\D}{3 Demiliches}]

The characters cannot tell the difference between the animate demiliches and the corpses unless they note the wounds well.

The demiliches will be initially hostile to the characters, casting spells first and asking questions later.
However, they are approachable in theory.
They cannot speak, but can understand Elvish.
They do not know the modern Trade Tongue.

\mapentry{The Exposed Doors}

\begin{exampletext}

When the priests were placed in this room, they ordered secret tunnels be made in the side to store their wealth in secret.
The tunnel then collapsed, killing the workers (as the priests knew would happen), and all the workers were given an honourable burial.

\end{exampletext}

The two tunnels stand exposed, their doors ripped down from years of underground war.  Primitive spears litter the area.

\mapentry{Side Room}

The natural tunnel was further excavated in order to horde the priests' treasures.
At present, this room contains 1,488cp, \thepage sp and \arabic{list} gp, all held in a small chest at the side of the room.\footnote{The coins date the room to the year of Rex Hunter.
See page \pageref{r_hunter}.}

The room also contains 14 undead hobgoblins, standing ready to kill any nura who come up from the tunnel, or just anyone who enters.

\undeadhobgoblin[\npc{\T\N\D}{14 Undead Hobgoblins}]

\mapentry{The Abyss}

Characters entering this area feel a warm breeze coming from the abyss, and intense magical energies.
The abyss itself is a third-level mana lake, constantly radiating magical energy.%
\footnote{See page \pageref{mana_lake} for more on mana lakes.}

\begin{exampletext}

When the tunnel collapsed, part of the ground collapsed too.  The hole does not stop -- it just keeps going down and down, and eventually reaches the Realm of Darkness and Fire.\footnote{See page \pageref{darknessandfire}.}

It wasn't long before the nura came up from below.  At first, they had gold to trade.  Soon after, they brought magical scrolls from an unnamed patron, which detailed how to gain control over the nura, or destroy one's enemies.

When the nura turned against the city, the two portals spewed hobgoblins, ogres, and the like, at the same time.
The priests argued, and considered caving the entire temple in.
They prayed to Qualm\"{e}, and saw a vision of the temple's destruction, with a hole which sucked in more and more earth, getting wider and wider, until nura filled the land.
They decided instead to remain, and guard the tunnel.
They had their servants wrap them in cloth while warriors fought against the nura.
They each took vows to never eat, open their eyes, or listen to music again.
All five became demiliches, sworn to serve their temple forever.

As the nura came up, the golden priests killed them with magic, then raised the bodies from the dead to fight against more nura.
This has left the situation in a stalemate.
Nura come up from the tunnel, they die, the golden priests raise them as ghouls to fight until all nura lie dead.
This repeats every few years.

So far, two of the golden priests have met their final end.  Their remains have been placed back in their chambers by the others.

\end{exampletext}

\mapentry{The Downward Spiral}

The party, if they venture down here, find four more undead hobgoblins in the tunnel, just standing, and waiting to open the door, descend upon the nura, grab a victim, and die with them at the bottom of the abyss.

\begin{exampletext}

  Throughout the long years, the golden priests directed the stupid undead to dig and dig into the hillside.
  They looped round, and dug into the tunnel, then fashioned a door from the broken wood from previous battles.

Any nura digging upwards can now be attacked by the undead, who hurl themselves down to attach anything climbing up, before one of the golden priests shut the door again.
While several nura would normally burst through any door, opening a stuck door while climbing can be almost impossible.

\end{exampletext}

\end{multicols}

