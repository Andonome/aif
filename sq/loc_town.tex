\section{\Glsfmttext{town} Locations}

\begin{multicols}{2}

\toppic{Dyson_Logos/town}{\label{town_map}}

\begin{table*}[t]

\begin{rollchart}

  Number & Location \\\hline
  1 & \Glsentrytext{pig}. \\
  2 & The White Horse Tavern. \\
  3 & The Citadel of \glsentrytext{townmaster}. \\
  4 & The Guard Station. \\
  5 & Butcher's entrance to the Lost Library in the sewers. \\
  6 & Temple of Ohta. \\

\end{rollchart}

\end{table*}

Every morning, wagons pour from the villages into \gls{town} with food, and enter the market.

Servants of the nobles arrive to buy large baskets.
\Gls{townmaster} and the guild masters always eat well.

The food sellers then take their coin, and move to stalls where the guilds sell their wares -- beer, weapons, contract-witnessing, blessings, clothes, carts, oil, and anything else a village might not produce.
Despite its remote location, one can buy almost anything in \gls{town}.

Next, the \gls{guard} take their share.
They may be few, but they buy a large portion with their heavy taxes on the guilds.

``No price is too high for safety'', they tell everyone inside \gls{town}'s walls.

While there isn't a legal difference between the guards in the city and the \gls{guard} outside, the difference is palatable.
Those inside are used to being heavy handed or violent with people, but they're terrified of the monsters outside, or anything unnatural.
Many have some distant family connection to \gls{townmaster}, ensuring their intense loyalty.

Servants of nobles or the guild then pay for the last of the good food at the market, and return home with what they can get.
Their employment alone keeps them out of the \gls{guard}.
When servants and lower-ranking guild members misbehave, they find themselves with a quick job outside the walls.
And the worse the deed, the closer they may come to the \gls{edge}.

The last group live in a state of constant questioning.
Will they find a little work, or steal today?
If they steal, will it be the gallows or the guard?
If someone makes them work in the \gls{guard}, what kind of creature will eat them?

Criminals taken into the \gls{guard} never return to \gls{town} alive.

\subsection{The Citadel}\label{citadel}

The citadel is massive, and contains various floors.

\begin{enumerate}

  \item{Ground Floor: Outsiders}
    \begin{itemize}
      \item{Left Wing: Ballroom.}
      \item{Left Wing: Guardroom.}
      \item{Right Wing: Dining Room.}
      \item{Right Wing: Servants' Quarters.}
      \item{Right Wing: Kitchen.}
    \end{itemize}
  \item{First Floor: Insiders}
    \begin{itemize}
      \item{Left Wing: Guest Beds.}
      \item{Left Wing: Study.}
      \item{Right Wing: \Glsentrytext{townmaster}'s Sons' 9 Quarters (a nearby tree stands tall enough to access one room).}
      \item{Right Wing: Secret Stairway up to the floor above.}
      \item{Right Wing: Winery.}
    \end{itemize}

  \item{Second Floor: Others}
    \begin{itemize}
      \item{Left Wing: \gls{alchemist}, the Alchmist's Study.}
      \item{Left Wing: \Glsentrytext{townmaster}'s close servants' quarters.}
      \item{Right Wing: \Glsentrytext{townmaster}'s room.}
      \item{Right Wing: Treasury.}
    \end{itemize}

\end{enumerate}

The lower floor holds fifteen guards in each wing.

\humansoldier[\npc{\T}{The Citadel Guards}]

\paragraph{If trouble emerges in the Citadel,}
all the guards rush to the source of the noise, ready to prove themselves.

\humandiplomat[\npc{\T\M}{\Glsentrytext{townmaster}'s Nine Sons}]

\paragraph{If \gls{townmaster}'s sons find intruders,}
they talk big, then surrender before the fight has begun, reminding the intruders that their father will pay handsomely.

\citadelAlchemist

\label{citadel_alchemist}

\paragraph{When \gls{alchemist} sees lawbreakers,}
he threatens the most awesome magic imaginable, even if his real powers leave much to be desired.

\townmaster

\subsection{The Guard Station}\label{guardstation}
The grounds are patrolled by a minimum of five guards at any point.
\Gls{captain} has an obsession with guards constantly rotating around the premise.
As a result, they've hidden a stash of whiskey in the bushes at the back, and sometimes have `rounds', while they do the rounds.

The wooden buildings tacked into the outer wall have thin rooves which constantly bend and creak -- walking silently across them is impossible for anything with a total weight of 4 or more.

Inside, \gls{captain} keeps a few magical items stashed away in his own room.

\magicitem{Scroll of Fire}{Fireball}{Alchemy}{Instant}{Pocket Spell}{4}{4}

\pic{Dyson_Logos/guard_station}{\label{dyson:guard}}

\humansoldier[\npc{\T}{30 \gls{guard}}]

Once the words on the scroll are spoken, the scroll is destroyed, and a fireball spanning 5 squares leaps out to deal $2D6$ Damage.

The guard house also contains 10 Spider Arrows and three sets of Eternal Warrior's Armour (see page \pageref{eternalwarriorarmour}).

\begin{enumerate}

  \item{Stables}
  \item{Storeroom room with handheld weapons, siege weapons, and basically every item listed in the core rules}
  \item{Toilet}
  \item{Captain's Toilet}
  \item{\gls{captain}'s Room}
  \item{Sleeping Quarters}
  \item{Dressing room, with armour}
  \item{Lecture Hall (though mostly used as a drinking hall)}
  \item{Records Room, containing lists of fugitives, laws, tax records (a copy is kept in \gls{townmaster}'s treasury), and and valuable paintings of local nobles}
  \item{Interrogation room}
  \item{Shrines to Alass\"{e}, Laiqu\"{e}, Ohta, Qualm\"{e}, and V\'{e}r\"{e}.}
  \item{Stairway down to the dungeons}
\end{enumerate}

\subsubsection{The Dungeon}

\paragraph{Background:}
Some time ago, the guards captured an ogre, and \gls{townmaster} ordered them to keep it alive so he could better understand the nura, and perhaps to bring it out as a pet one day.  Since then he's forgotten about it, but the guards have to keep feeding it.  Twice a day, they take a cart down the stairs, deal out a small portion to each of the inmates, then place the rest by the great door as a massive, grabbing hand reaches out the shutter and piles the food into its mouth.

\pic{Dyson_Logos/under_station}{\label{dyson:understation}}

\begin{boxtext}

  With a hood shoved over your head, you're taken down a set of stairs, then spun around, then down another, spun around again, then taken down another set of stairs, spun around again, then pushed hard down a long hallway.
   The stench of shit fills the room.
   You hear a horrible, inhuman, roar, feel something slippery under your feet, then pull right, down the hall, right again and two steps later the bag's pulled from your head, and you see a small pair of eyes in the darkness in front of you as the door slams shut behind you.
   The lock clinks shut in the darkness, and the little voice asks ``Hello?''.

\end{boxtext}

They didn't know what to do about a toilet, and they've never wanted to move the ogre, so the ogres shits in a bucket, and once a day it throws the contents out of the hatch and onto a guard.
This is the only fun the ogre has, and the biggest irritation the guards have to put up with.

As the characters enter the dungeon, they're separated and thrown in with thieves, townsfolk who talked badly about \gls{king}, and one indebted trader who can't stop pitying himself.

\begin{enumerate}

  \item{Food Storage}
  \item{Drunken guards}
  \item{Starving prisoners}
  \item{Empty cell}
  \item{Fake doors with locks on them}
  \item{The Ogre}

\end{enumerate}

\npc{\M\N}{The Ogre}

  \person{7}% STRENGTH
  {0}% DEXTERITY
  {4}% SPEED
  {{-3}% INTELLIGENCE
  {-2}% WITS
  {-4}}% CHARISMA
  {0}% DR
  {2}% COMBAT
  {Crafts~1, Tactics~1}% SKILLS
  {Nothing}% EQUIPMENT
  {}


\subsection{\Glsentrytext{whitehorse}}
\setcounter{list}{0}

\begin{boxtext}

  A massive man with a butter-tipped moustache throws a lanky man from the door.
  The lanky man recoils with a grimace, and shouts `you owe me my wages!'.

\end{boxtext}

\paragraph{Background:}
\Gls{sewerthief} took a job as a cook at \gls{whitehorse} to learn more about \gls{townmaster}, but \gls{townmaster} instantly took umbrage at his Whiteplains accent.

If the PCs agree with Elric about \gls{sewerthief} (and about Whiteplains in general), then Vernon lets them in.

\mapentry{Drinking Hall}

The hall contains various villagemasters playing games, and half a dozen local guards (sometimes including \gls{captain}).

\humandiplomat[\NPC{\M}{Vernon}{Proud}{Curling moustache}{Acquisition}]

\mapentry{Kitchen}

The staff sleep here during long shifts.  The lack of proper ventilation makes the air difficult to breathe.

\mapentry{The Courtyard}

\begin{boxtext}

  \Gls{townmaster} is running away from a coterie of chuckling men with his hands tied behind his back.
   A chicken runs out in front of him with a little paper hat.
   He lunges forwards and grabs the chicken in his teeth, then shakes it like a mad dog until it stops squawking.
   He gives a triumphant grin as the crowd clap and another man steps forward to have his hands tied.

\end{boxtext}

The courtyard usually contains a couple of carriages, and nobles playing ridiculous games.

\mapentry{Upstairs}

Upstairs contains two rooms, a load of equipment for the tavern, sleeping mats for favoured servants, and bookshelves.

The bookshelves contain rather a lot of history books, most focussing upon anti-elven propaganda, such as the time they destroyed the now-lost city.

\Gls{beardedalemaster} and Jubilee of the Wolf Heads sleep in the larger room.%
\footnote{See page \pageref{wolfHeads} for the Wolf Heads.}
\Gls{whitehorse} has no locks as nobody expects commoners to enter in the first place.
 
\pic{Dyson_Logos/white_horse_1}

\subsection{\Glsentrytext{pig}}

\setcounter{list}{0}

\begin{boxtext}

  Two men are pummelling each other in front of the pub's door.
   One limps and the other's nose is burst open and streaming down his shirt, but they continue circling like boxers.
   Two guards cry out and run forward to stop the public disturbance, and the two men immediately run together into \gls{pig}, a disreputable tavern near the city's entrance.
   The guards stop at the door, look at each other for a moment, and then walk away.

\Gls{pig} never treats \glspl{guard} well.

\end{boxtext}
 
The roughest and oldest pub in \gls{town} sits just across from one of the major entrances, enticing traders in with the promise of the latest news and cheap ale.

The owner, \gls{pigowner}, keeps the place in order with a mixture of social contacts with the roughest characters in \gls{town}, and rare but sudden violence.

The place gets lenient treatment from the guards as it's an official Temple of Alass\"{e}, complete with an official priestess.
The fact that she spends most of her time drunk doesn't detract from her status, or stop the occasional noble asking her to sneak contraband into the city.

Currently, \gls{pig} allows a few of the Immortal Bandits to stay under the bar, as they help her move things in and out of town without going through the \gls{guard} at the gates.%
\footnote{See page \pageref{farmExit} for the exterior route.}

\toppic{Dyson_Logos/mincing_pig}{\label{mincing_pig_map}}

\mapentry{Beerhall}

\begin{boxtext}

  Alassean song and cooked pig wafts hits you in the face as soon as the door opens.
  With only three tables in the room, people have carved out little seating circles on the ground.
  A fat cat with a brown collar sits in the rafters and eyes you suspiciously as you struggle through the disorganized crowd to get to the bar.

\end{boxtext}

The cat's name is Bob, and his collar is made from dried, woven daffodil.  The collar is activated by shouting the elvish word for `dragon', at which point Bob will turn into a nura cat until someone says the elvish word for `cat'.%
\footnote{The collar was a gift from \gls{sewerking}, who found it among the old treasures below.}

Only \gls{pigowner} knows about the collar's power, and as a result, elves and people who speak elvish (such as many academics) are not welcome in \gls{pig}.
She plans to use it for self-protection (or for her final vengeance) in case anything goes seriously wrong one day.

When in nura-form, these are his stats:

\nuracat[\NPC{\N\A}{Bob}{Slick}{Licks Paw}{Experience}]

\mapentry{Kitchen}

People cook here, mostly.

\paragraph{If the PCs observe the area carefully,}
they notice \gls{sewerthief} porting sacks to room 3 (the `private room'), followed by others.
These contain dirt, used to make fake corpses so \gls{sewerking} can swap them for real corpses.

In total, eight men enter the private room, and don't return.

\pigowner

\mapentry{Private Room}
\label{pigPrivate}

\Gls{alemaster} often sleeps in the \gls{pig}'s quieter side-room after a long night of drink and song.
She gets on well with \gls{pigowner}, helps with the books, and provides cheap ale to many of the taverns around.%
\footnote{Although, not all taverns, as the Ale Guild of the Shale is cheaper.
See page \pageref{troubleAle}.}

She has a lot of respect and time for the dispossessed nobles of Whiteplains as she detests \gls{king}, but has no idea about their plans, murders, or that the undead are wandering below her.

\begin{boxtext}

  Perfumes, spices, and stale sex fill your noses.
  The wide room practically begs for shoes to be removed as it's filled with pillows, throws, and blankets.
  The only raised platform is a table strewn with fortune-telling cards, where \gls{alemaster} sits with a headband made of gold.

\end{boxtext}

Under \gls{alemaster}'s fortune-telling card table the floorboards are loose, and lead down to a new room, muffled by a thick curtain.
The number of people entering here has lead to wild rumours about the number of men \gls{alemaster} satisfies each day, and the number of hours they spend in there.
However, the reality is that almost everyone who enters the room simply wants to go through the secret hatch to the room below.

\alemaster

\mapentry{Thieves' Den}

\paragraph{Background:}
Priests of Qualm\"{e} once lived, studied and prayed in these rooms before an underground flood made the place unlivable.
Currently it hosts card games.

A number of local thieves know of this secret and secluded room.

\begin{boxtext}
  Three men sit cross-legged on the floor, quietly playing cards.
  The second they see you, an additional layer of silence enters the room.
\end{boxtext}

\humanthief[\npc{\T}{Three Cutthroats}]

\mapentry{\Glsentrytext{pigowner}'s Room}

\Gls{pigowner} lives in a mess of old notes about what she owes to whom, chests of illegal weapons%
\footnote{Weapons are not illegal but stockpiling more weapons than an individual can use \emph{is}.}
hidden under various clothes (swords mostly), and various expensive alcohols, along with poisons, all lying about without labels.

\mapentry{Pantry}
\label{pig_pantry}

\paragraph{Background:}
\Gls{pigowner} stores additional casks of ale, the good wine, and the best salted pork in here.

\paragraph{If this encounter occurs after the ghast escapes,}
then the Immortal Bandits have trapped it in this room, along with an unfortunate maid whom the monster turned into a ghoul (see page \pageref{ghastEscape} for more on the ghast).

The thieves who live in the sewers have apologized to \gls{pigowner} for letting the creature escape, but none have volunteered to open the door.
On its way out they could see it was a different sort of undead -- not a regular ghoul, but something with a mind, and the ability to plan.

The creature looks like a regular, skinny woman, with long brown hair, dead around six months.
However, the simple shell hosts a powerful necromancer.

\npc{\D}{Monster in the Cellar}

\person{0}% STRENGTH
  {1}% DEXTERITY
  {0}% SPEED
  {{2}% INTELLIGENCE
  {0}% WITS
  {-5}}% CHARISMA
  {2}% DR
  {2}% COMBAT
  {Academics~2, Caving~1, Medicine~1, Stealth~3, Vigilance~3
  \Path{Devotion}{\aldaron~3, \fate~2, \necromancy~3}}% SKILLS
  {None}% EQUIPMENT
  {\lockedmana{2}}

The creature has already risen the maid from the dead, and has an escape plan.
It will create a magical mist as it hears anyone attempting to enter the door.
Once the door opens, the mist pours out, obscuring all vision, and the ghoul-maid will come shambling out.
Anyone attempting to kill the creature inside will probably attack the maid, and in the confusion, the monster from the cellar will attempt to stealth its way outside.
In the private room above, it will cover itself in blankets, posing as an embarrassed man who must leave unseen.
Once on the street, it can find an alley to hide in, and begin its murderous rampage before diving into the river at the first sign of a mob.

\ghoul[\npc{\D\F}{Undead Maid}]

\mapentry{Runoff}
\label{runoff}

Various little pipes, nooks, and gutters in the city lead underground.
This little drain pops out here, and heads steadily downhill, eventually landing in the sewers below (see page \pageref{slidein}).

The first member of the party to go down the tunnel rolls to spot the tripwire (Wits + Vigilance, TN 8), and failure means they immediately go tumbling downwards to the sewers.

\begin{boxtext}

  A little river can be heard ahead.
  The torchlight shines on a thin but taught rope, stretching across a passage to the right.
  The little river flows down steeply, fed by a gushing crack in the wall.
  It's not clear if the rope was to server as a poor guard-rail or as a tripwire to send people tumbling down.

\end{boxtext}

\mapentry{Ventilation Shaft}

This little tunnel reaches upwards to allow a modicum of fresh air to circulate down in the nasty little dungeon.  Characters with a Strength of 0 or greater are too large to fit through the narrow hole.  It emerges on a street, just below a rich man's house.

\mapentry{The Temple of Qualm\"e}

\paragraph{Background:}
When the Immortal Bandits cleared this hallway, they found an entrance to the Temple of Qualm\"e.
But instead of entering, they pulled out the door, and quietly replaced it with stone.

Finally, they carved through the rock underneath to create a crawl-space, and replaced the stone tile on the floor with another which would give way when pushed from below.

This ensured that their access would remain secret.
Once the waters receded, the Temple priests started to use the room for its original purpose -- cleaning the dead, in preparation for their burial.

The bandits can now sneak in here every could of nights with a fake corpse-bag, made of sack-cloth and filled with earth, and replace corpses with fakes.
The bandits then take the real corpses down to the underbelly, where \gls{sewerking} can resurrect them from the dead.

\mapentry{Sleeping Dogs}

When not getting up to no good, the Immortal Bandits sleep in these little alcove-bedrooms.
Each one has little more than blankets and clothing hooks.

\paragraph{Background:}
\Gls{sewerthief} has the room to the far right, and while everyone else was digging where \gls{sewerking} ordered, he quietly dug himself a little hiding-hole to stash money and other valuable he's taken for himself.
In total, he's collected \lootMedium, and one \lootMagic.

When \gls{sewerking} found out about this, he became annoyed, and placed a ghast in the room, instructing it to frighten (but not kill) \gls{sewerthief}.

\paragraph{If the PCs enter \gls{sewerthief}'s room,}
the ghast jumps out at them.

\npc{\D}{Ghast-in-the-Box}
\ghast

\mapentry{The Descent}
This path downwards leads to the thieves' den underground (see page \pageref{pigexit}).

\subsection{The Temple of Qualm\"e}

The temple houses priests who white biographies (mostly for nobles), handle pensions (repaid on a per-family basis), practice starvation (so others can have more), and sing prayers for their ancestors to stay somewhere nice in the afterlife.

If the temple's groundskeeper -- Boris -- ever finds out what happened to the bodies they lovingly prepare before placing in caskets, he enters a rage and demands to join the PCs to confront the perpetrators immediately.

\humanpriest[\npc{\M}{Boris}]

\toppic{Dyson_Logos/white_horse_2}{}

\subsection{The Lost Library}\label{sewers}\setcounter{list}{0}

The old temple of Qualm\"{e} stretched deep underground, and soon after it was built, a library was commissioned by local alchemists.
The two shared much of the space for some time.
The place held students, priests, alchemists, and a grand library.
However, once the nearby city was destroyed (now \gls{lostcity}), there was no longer enough money, pilgrims, or students hoping to one day see the great university of the city to sustain the underground library, or the temple.

A century later, the underground flooded, and everything below sat alone in darkness.

When the Whiteplains nobles, bereft of a home and desperate, took refuge in the mincing pig, they began exploring the tunnels below, and found they could divert the waters which had drowned the library.%
\footnote{See the `\nameref{runoff}' location, page \pageref{runoff}'.}
Once the waters began to recede into the porous earth, many of the artificial tunnels had collapsed, while some new tunnels had been dug out.

They dug their way down as quickly as they could, and cemented a stream down into an unending underground hole (area 12).
Water flows down from the \gls{pig}, and they can easily dispose of any dirt dug up.

\paragraph{The doors} are all locked by a single key type of key.
Lock-picking them requires an Intelligence + Larceny Group Roll, TN 10 if the bar's up, and TN 14 otherwise.

\paragraph{The narrow hallways}
make longer weapons difficult to use.
The \textit{Enclosure Rating} here is 4, so anyone using a weapon which requires 6 Initiative will suffer a -2 penalty to attack.%
\iftoggle{core}{
  \footnote{See the core rulebook for more on Enclosure Ratings, page \pageref{enclosedcombat}.}
}{%
  \footnote{See the core rulebook for more on Enclosure Ratings.}
}

\toppic{Dyson_Logos/sewer}{\label{sewer_map}}

\mapentry{Stairway to the Butchers}\label{butcher_exit}

This stairway has been dug upwards to a drain just outside of a butchers.
The bandits enter and exit through here.

\mapentry{Dead Cells}

\paragraph{Background:}
\Gls{sewerking} doesn't have enough space to store his undead pets, so he has his ghouls make these little alcoves.
Once he has four ghouls, he packs them all together and hammers wooden bars to keep them wandering out.
He tried making a larger alcove, but the pressure created by random movements broke the bars open; the ghouls can leave any time they have sufficient motivation, the bars just stop them wandering.

The Immortal Bandits wander in and out safely by using their rings of asphyxiation.%
\footnote{See page \pageref{ring_asphyxiation}.}
If many need to come in or out, two go through, and one carries two rings back for the next, so one by one a dozen can enter slowly.

\paragraph{When the PCs enter,}
give the first in the line a Wits + Vigilance roll, TN 16.
Every 2 margins of failure means the ghouls from a cell burst open (up to the maximum of 7), so rolling a `10' implies that 3 cells have burst open, containing a total of 12 ghouls.

\ghoul[\npc{\T\D}{4 Ghouls}]

\paragraph{If the PCs attempt to simply rush past,}
have them roll Speed + Athletics (TN 12).
Again, every 2 margins of failure mean the they have missed the end of the tunnel-section by one cell, so the dead exit and grab them.

\mapentry{Dead Rooms}

\paragraph{Background:}
\Gls{sewerking} uses these larger alcoves to store individual ghouls before he has enough to stuff into one of the cells (see above). 
He also keeps the his more dangerous undead experimentations here -- ghasts.
He can't control them all, so he simply stuffs them in, waiting to use them later.

\begin{boxtext}
  The hallway ahead has doors on the left and right, each shut with a simple bar across the front.
  The doors have a tight seal, and no window to give a clue about the contents.
\end{boxtext}

\paragraph{Anyone knocking off the bar}
will probably find a nasty surprise.
Consult the chart (no matter which door the PCs open first, that's the first door).

\begin{rollchart}
  Door & Contents \\\hline
  1st & 1 ghast hides behind the door. \\
  2nd & 3 ghouls jump out. \\
  3rd & \gls{sewerthief} is taking a nap while he should be working. \\
  4th & 2 ghasts jump out, and one flees towards \gls{town}, above. \\
  5th+ & 1 ghast jumps out. \\
\end{rollchart}

\paragraph{If the PCs listen at the door,}
they find the dead are silent when not active, and don't respond to noise.

\ghast

\ghoul

\mapentry{The Old Library }\label{oldlibrary}

\begin{boxtext}

  Empty stone shelves show where an expansive library once provided the entire city knowledge, but not a scrap of paper remains.
  Three stone pillars divide the room, each with a brazier hanging in front of them by a chain.
  The central pillar's brazier is made from a human skull.

\end{boxtext}

\paragraph{Investigating the brazier,}
shows it contains incense, ready to be lit.

\paragraph{If anyone dies in the room,}
the skull resurrects them as a ghoul.

\magicitem{Brazier of the Risen Soldiers}{Ghoul Calling}{Devotion (Qualm\"e)}{Instant}{Talisman}{2}{6}

\mapentry{Magical Item Storage}

\begin{boxtext}

  The boarded up wall pulls open -- the entire thing was a door made to look like a blocked entrance.  The rings of shelves show a strange assortment of items -- jars filled with human teeth, an old brazier, dried snowdrops, and a vial of blood.

\end{boxtext}

\Glsentrytext{sewerking} stashes most of his prizes in this room on a simple series of shelves.
Each is cast with Intelligence +1 and Wits +1.

\begin{enumerate}

  \item
  A vial of lamb's blood which makes the user invisible to the dead and immune to fatigue, marked ``Dead Wine'' (as per the Torpor spell).
  \item
  An old scroll, proclaiming elves the friends of humans, and seven reasons not to worry about nobles being assassinated.
  \item
  The Assassination Dagger, which inflicts an additional $1D3+1$ HP Damage during the round's first attack (ignoring all FP).
  This ability can be used once per scene.
  \item
  Magic Mushrooms, enchanted with Saurecanta level 2 to decrease the user's Intelligence and Charisma by 3 and increase Speed by the same amount.
  \item
  Foul alcohol in a bottle, which makes the imbiber regenerate fatigue if they eat, and otherwise inflicts hunger paints, as per Saurecanta level 1 (see page \pageref{saurecantaone}).
\end{enumerate}

\mapentry{\Glsentrytext{sewerking}'s Room}

\begin{boxtext}
  The door opens to a noble's room, bearing a striking contrast to the dungeon around.  The bed's well made, the sheets are silk, and various books sit on shelves.  On the table sits various maps.
\end{boxtext}

\begin{itemize}

  \item
  The city map shows every entry point the bandits can enter the city above, including the theoretical passage the bandits think could be found again under \glsentrytext{townmaster}'s Citadel.

  \item
  A map of the area, outlining \glsentrytext{lostcity}, the portal by \glsentrytext{redfall}, and \gls{necromancer}'s lair.

  \item
  A complete map of the current location

\end{itemize}

The books are variously written on history (real and imagined), \textit{The Art of Lies} (by an elvish author -- `Erende'), and instructions on hosting a dinner party.

\mapentry{Ogre Dust Trap}

A thin wire was stretched across the floor, leading up to a small stretch of leather, holding snowdrops.  Anyone failing a Wits + Vigilance roll, TN 12 in the twilight, feels the petals fall down.  A moment later, the character's afflicted as per Saurecanta, level 2, and gains +4 Strength at the cost of -4 Charisma.

\begin{boxtext}

  A little thread pushes against your face, like a steel spiderweb.
  A second later, something flutters around your head.
  The falling debris feels annoying beyond words, and it's difficult to say why -- you simply feel incredibly irritated, and hungry.
  \emph{Extremely} hungry.

\end{boxtext}

\mapentry{Food Storage}

\begin{boxtext}

  Barrel after barrel fill the room, along with the smell of wine, apples, and vinegar.  A little basket of choice snacks sits on top.

\end{boxtext}

The room is normal, except for the basket of choice snacks, which is poisoned with an intense laxative.
\Gls{sewerking} suspects one of his men steals food when returning from business in \gls{town}, so he's left a basket of poisoned food.
Someone can tell it's poisoned with a Wits + Medicine roll, TN 8.
Failure means the character will have a bad night, and gain 3 Fatigue Points each scene for the rest of the day.%
\footnote{Feel free to roll for the characters so they're not aware there's a problem.}

\mapentry{The Drowned Hallway}

This area recently suffered a little flood.  Most of the water has dissipated, but this lower portion of the tunnels remains flooded.  The undead hobgoblins remain locked in their cells underground.


\begin{boxtext}
  Going down the stairs you feel your feet hitting cold water.  It's not clear how far the water goes down, but it's cold.
\end{boxtext}

The water goes up to the ceiling by the last step, and for four squares after.  Each ghoul-stuffed room the characters pass the dead will lash out, with TN 12 to escape the grabbing hands, assuming the characters aren't Keeping Edgy, and have been blinded by the dark waters.

\mapentry{Grand Hall}
\label{underHall}

The thieves feast, plan, and sometimes wrestle here.

\begin{boxtext}
  Opening the door, soft lantern-light trickles into your eyes.
  The grand hall has a massive feasting table in the centre, currently occupied by a dozen rushing towards you with weapons drawn.

  ``The dead are lose!'', they cry, as one man prepares some kind of spell.
\end{boxtext}

\paragraph{As the PCs enter,}
\gls{sewerking} begins casting a \textit{Potent Sickness} spell.
After 3 rounds, he deals 1D6 HP damage to the member of the party with the highest Strength Bonus.

The other bandits rush at them immediately.

\sewerking

\humanthief[\npc{\T}{12 Sewer Bandits}]

\paragraph{Four rounds later,}
the cutthroats from room \ref{underGuard} arrive (assuming they weren't killed already).

\mapentry{Entrance to the Citadel}
\label{citadelTunnel}

\begin{boxtext}
  The picks and torches on the ground show that someone's been working their way into the ground.  At the moment, the tunnel ends in a dead end.
\end{boxtext}
 
The single square of rock ahead is made of fallen debris, so the PCs can move it far easier than most rock walls with a Strength + Crafts Group Roll, TN 9.%
\iftoggle{core}%
{\footnote{See the core rules, page \pageref{grouproll}, for Group Rolls.}}%
{}
They'll need to accumulate a total margin of 10, whether in one roll, or many, in order to clear the way.
Each roll takes 1 scene.

\mapentry{Sewer Entrance}\label{slidein}

\label{pigexit}

This artificial stream loops round from \gls{pig}, above.
The stream continues downwards to an underground area and then goes underground.
Anyone venturing down here simply dies in the unending blackness.

\mapentry{Entrance to \glsentrytext{pig}}

\paragraph{Up these stairs,}
characters can reach the bowels of \gls{pig}.

\Gls{pigowner} understands what trouble she's in, and immediately accuses the characters of theft and calls on everyone in the room to kill them.

Calming down the patrons requires a Charisma + Empathy roll (TN 10).

\paragraph{If the PCs arrest the patrons of \gls{pig},}
everyone in \gls{pig} will deny any knowledge of the deeper tunnels, and the fact that bandits lived down there.

\mapentry{Exit to the External Farm}
\label{farmExit}

\paragraph{Background:}
Outside \gls{town}'s walls the tunnel ends in a farmhouse.
The Immortal Bandits use this place to move beyond the walls without worrying about any inspections.

Angus' house above has several rooms, as he's done rather well for himself, and hopes that once the revolution comes he'll be in an even better position.

\begin{boxtext}

  The stairs go upwards for some time, and eventually arrive at a room filled with barrels of food, and a trapdoor above.
  You can hear a deep snore coming from just beyond the trapdoor.

\end{boxtext}

\humanfarmer[\npc{\M}{Angus}]

\mapentry{Guardroom}
\label{underGuard}

Here four of the thieves sit and play simple dice games to pass the time, or occasionally sleep in the foetid straw.

\paragraph{If the PCs fight,}
the cutthroats try to ram them out the door, and run to alert the others in room \ref{underHall}.

\humanthief[\npc{\T}{Four Cutthroats}]

\mapentry{The Old Temple to Qualm\"{e}}

\begin{boxtext}
  Letters in Elvish above the doors state ``We bones await yours''.
\end{boxtext}

There are twelve pillars in total in the room, and each one was formed by members of a family, over the course of generations, donating money to the Temple of Qualm\"{e}.  Those who died in its service had their skulls added to the tower.  It could take two hundred generations to create some of these towers.  Once the tower is completed, the top skull has the family's name carved into the forehead.  Family members

\begin{boxtext}

  The massive room has a strange lack of smell.  Towers of skulls stand in neat piles, each resting in a small pillar.  Some are as tall as a man, others reach nearly to the ceiling.  Each one has writing upon the top skull.

  At the far side of the room is another exit.

\end{boxtext}

There are twelve pillars in total in the room, and each one was formed by members of a family, over the course of generations, donating money to the Temple of Qualm\"{e}.
Those who died in its service had their skulls added to the tower.
It could take a dozen generations to create some of these towers.
The top skull has the family's name carved into the forehead, and speaking the family name along with religious incantations to Qualm\"{e} activates some magical effect.

Each pillar can be used once a day.
They require a short prayer (which takes some time to complete), and understanding which prayer aligns with which skull requires an Intelligence + Academics Group Roll (the TN depends upon the type of pillar).
The character requires a Margin of 2 to know what will happen before casting the spell.

\paragraph{Smaller Pillars: (TN 8)}

\begin{enumerate}

  \item{(2) Regenerate $1D6$ FP.}
  \item{The target loses $1D6$ FP (this family stopped paying temple dues).}
\end{enumerate}

\paragraph{Larger Pillars: (TN 10)}

\begin{enumerate}

  \item{The room is filled with sweet-smelling mist, as per Aldaron level 2.}
  \item{The target sees a vision of the future, as per the Fate spell Augury.}
  \item{Target regenerates $1D6+2$ FP.}
\end{enumerate}

\paragraph{Towering Pillars: (TN 12)}

\begin{enumerate}

  \item{This one pillar is topped by a trapped spirit, rather than being a simple magical item.  The spirit, once presented with any appropriate token of the dead, will enchant it to be a useful magical item.  The enchantment will come from the Necromancy sphere, and is cast with Intelligence +2 and Wits +2.  The enchantment lasts for a day.}
  \item{The target regains $1D6+3$ FP.}
  \item{Any recently deceased target is returned to life, as per Fate level 5.}

\end{enumerate}

\mapentry{The Guard Dog}

\paragraph{Background:}
The Immortal Bandits keep one of farmer Angus' dogs chained here at all times (rotating, so they don't get bored).

\begin{boxtext}

  Just ahead, you can see a dog, lying on the ground.
  It instantly stands up and begins to bark.

\end{boxtext}

\paragraph{If the characters approach,}
Rover hears them and begins barking.
If the group remained silent, they can make a Dexterity + Stealth Group Roll (TN 9) to move silently.

\paragraph{If Rover hears them,}
he begins barking and four of the bandits in room \ref{underHall} (\nameref{underHall}) come to investigate.

They won't poke about long, so the PCs simply need to pick a reasonable hiding place or roll Intelligence + Stealth (TN 8).

\paragraph{If a bandit or a ghast walks by,}
Rover does nothing -- he got used to the bandits, and even random undead wandering about, and the undead have no interest in animals.

\huntingdog[\npc{\A}{Rover}]

\vfill\null

\end{multicols}

