\makeindex[name=history,title={Chronological Events},columns=2]

\newcommand\histEvent[3]{%
    \setcounter{track}{\value{fenestraYear}}%
    \setcounter{diceNo}{#1}%
    \addtocounter{track}{-\value{diceNo}}%
  \index[history]{{\titleFont C. \Large\arabic{track}}!{\scshape\showCycle[#2]}\nobreak\\~{#3}|textbf~page~}%
}

\newcommand\caveIn{
  \sidebox[18]{
    \setcounter{track}{3}
    \begin{boxtable}
    \textbf{Roll} & \textbf{Damage} \\
    \hline
    1 in 6 & \stepcounter{track} \dmg{track} \\
    2 in 6 & \stepcounter{track} \dmg{track} \\
    3 in 6 & \stepcounter{track} \dmg{track} \\
    4 in 6 & \stepcounter{track} \dmg{track} \\
    5 in 6 & \stepcounter{track} \dmg{track} \\
    \end{boxtable}
  }
}

\newcommand\fireFuel{%
  \ifnum\value{temperature}=0%
    dry leaves%
  \else%
    sticks%
  \fi%
}

\newcommand\stateArea[1]{%
	\vpagerefnearby{#1}{`\nameref{#1}', \vpageref[below][above]{#1}}{area \vref{#1}, `\nameref{#1}'}%
}

\newcommand\wayOut[2]{
  \hspace{1em}
  \par
  \noindent
  \adforn{2}\quad#1
	\stateArea{#2}.%
}


\newcommand{\giantSnail}[1][Giant Snail]{
  \Animal{\npc{\A\R}{#1}}%
  {{5}{rn1t0}{-3}}% BODY
  {-4}% WITS
  {%
    \set{Projectiles}{r3c}%
    \ifodd\value{r4}%
      \setcounter{Vigilance}{\value{r2c}}%
    \fi%
    \addAbility{Rubbery (\gls{dr}~5 remains, even on \pgls{vitalShot})}
  }% SKILLS
  {}% KNACKS
  {\acidSpray \viscid \hide{10}}% ABILITIES
}


\newcommand{\enchantedGoblin}[1][\npc{\T[\arabic{r4t6}]\N}{\arabic{noAppearing} Goblins}]{
  \Person{#1}%
    {{r3}{r2b}{rn1t2}}% BODY
    {{rn1t0}{rn3t3}{rn4tn1}}% MIND
    {%
      \addtocounter{Strength}{-3}
      \ifodd\value{r4}
        \set{Melee}{1}
      \else
        \set{Projectiles}{1}
      \fi
      \set{Brawl}{rn1t2}
      \set{Empathy}{r2}
      \addtocounter{Empathy}{-2}
      \set{Stealth}{r2c}
      \ifnum\value{Strength}>-2
        \javelin
      \else
        \Dagger
      \fi
    }% SKILLS
    {}% KNACKS
    {licence for \weaponName, %
    \ifnum\value{noAppearing}<3%
      \ifnum\value{r3c}=3%
        dried snail meat%
      \else%
        enormous carrot%
      \fi%
    \else%
      pet \rock%
    \fi}% EQUIPMENT
    {\ifnum\value{noAppearing}<3%
      \ifodd\value{r2}%
        \mutation{r4b}%
      \fi%
    \fi}% ABILITIES
}

\newcommand{\enchantedHobgoblin}[1][\npc{\T[\arabic{r4t5}]\N}{\arabic{noAppearing} Hobgoblins}]{
  \Person{#1}%
    {{r3}{rn1t2}{r2}}% BODY
    {{rn1t0}{rn1t0}{rn4tn1}}% MIND
    {%
      \set{Melee}{r2c}
      \set{Brawl}{rn1t2}
      \set{Empathy}{r3}
      \addtocounter{Empathy}{-2}
      \set{Stealth}{r2b}
      \set{Vigilance}{rn1t2}
      \ifcase\value{Strength}
        \Dagger
      \or
        \javelin
      \or
        \glaive
      \else
        \greatsword
      \fi
      \partiallisk
    }% SKILLS
    {}% KNACKS
    {licence for \weaponName, %
      \ifodd\value{r4b}%
        whistling cane flute%
      \else%
        \goblinLight%
      \fi%
    }% EQUIPMENT
    {\ifnum\value{noAppearing}=1%
      \mutation{r4}%
    \fi}% ABILITIES
}

\newcommand{\enchantedOgre}[1][\npc{\N}{Ogre}]{
  \Person{#1}%
    {{r4t6}{rn1t2}{r3}}% BODY
    {{rn4tn1}{rn3t3}{r3c}}% MIND
    {%
      \addtocounter{Charisma}{-5}
      \ifodd\value{r3}
        \set{Melee}{rn1t2}
      \else
        \set{Projectiles}{rn1t2}
      \fi
      \set{Brawl}{rn3t3}
    }% SKILLS
    {}% KNACKS
    {}% EQUIPMENT
    {\ifnum\value{noAppearing}=1
      \mutation{r4}%
    \fi}% ABILITIES
}

\newcommand{\stupifiedElf}[1][\npc{\T[\value{r2t4}]\El}{\arabic{noAppearing} Elves}]{
  \setcounter{age}{1}
  \addtocounter{age}{-\value{r3}}
  \set{track}{r3}
  \addtocounter{track}{-\value{r2c}}
  \Person{#1}%
    {{age}{r3b}{r3c}}% BODY
    {{r3}{rn3t3}{rn1t2}}% MIND
    {%
      \set{Brawl}{rn1t2}
      \ifodd\value{r2c}
        \set{Melee}{rn3t3}
        \set{Athletics}{r3c}
        \set{Crafts}{rn3t3}
        \set{Cultivation}{rn1t2}
        \set{Vigilance}{r2}
        \set{Survival}{rn1t2}
      \else
        \set{Melee}{r3c}
        \set{Academics}{r2}
        \set{Athletics}{r3c}
        \set{Performance}{r3}
        \set{Empathy}{r2c}
        \set{Seafaring}{r3c}
        \set{Survival}{r3b}
      \fi
      \ifnum\value{noAppearing}>1
        \set{Fate}{rn1t2}
        \set{Water}{track}
      \else
        \set{Air}{rn1t2}
        \set{Fate}{rn3t3}
        \set{Earth}{r3b}
        \set{Water}{rn1t0}
        \ifodd\value{r4}
          \addtocounter{Water}{2}
        \fi
      \fi
      \ifnum\value{Strength}=0
        \rapier
      \else
        \Dagger
      \fi
    }% SKILLS
    {%
      \ifnum\value{Charisma}<1%
        \snapcaster%
      \else%
        \ifnum\value{Academics}>1%
          \specialist{star lore}%
        \else%
          \lucky%
        \fi%
      \fi, %
      \ifcase\value{r4b}\relax%
      \or%
        \manaWell%
      \or%
        \dodger%
      \or%
        \weaponmaster%
      \else%
        \specialist{\Glsfmtplural{storm}}%
      \fi, %
      \ifnum\value{Academics}>1%
        \specialist{tragic-romance tales}%
      \else%
        \ifnum\value{Cultivation}>1%
          \specialist{thorny plants}%
        \else%
          \ifnum\value{Performance}>1%
            \specialist{wood-flutes}%
          \else%
            \ifnum\value{Crafts}>0%
              \specialist{silverworking}%
            \else%
              \specialist{bee-keeping}%
            \fi%
          \fi%
        \fi%
      \fi%
    }% KNACKS
    {\ifnum\value{Performance}>1%
      flute, %
    \fi\ifnum\value{Strength}=0%
      \lootJewellery%
    \else%
      \ifodd\value{r4b}%
        \lootJewellery, \lootJewellery%
      \else%
        tinderbox%
      \fi%
    \fi}% EQUIPMENT
    {}% ABILITIES
}


\newcommand\enchantedMap[1][b]{
  \renewcommand\csComments{
    \draw[very thick,gray, dash dot] (9,0.6)
    node[anchor=north]{\outline{Miles}} -- (10,0.6)
    node[anchor=north]{\outline{1}} -- (11,0.6)
    node[anchor=north]{\outline{2}} -- (12,0.6)
    node[anchor=north]{\outline{3}} -- (13,0.6)
    node[anchor=north]{\outline{4}} -- (14,0.6)
    node[anchor=north]{\outline{5}} ;
  }

  \mapNotes{
    \normalsize Civilization/08/01,
    \Hu~\Glsfmttext{broch}/22/08,
    \Hu~\Glsfmttext{coppernut} \Glsfmttext{village}/20/30,
    \huge\Glsfmttext{sunderedForest}/35/99,
    \huge\Glsfmttext{enchantedLands}/80/99,
    \rotatebox{40}{\Large\nameref{plateauGardens}}/40/70,
    \rotatebox{40}{\El\R}/45/65,
    \rotatebox{40}{\Large\nameref{ravencops}}/55/45,
    \rotatebox{40}{\N}/58/40,
    \nameref{oathtower}/86/75,
    \El/85/65,
    \rotatebox{40}{\Large\nameref{sunway}}/50/60,
  }

  \widePic[#1]{feylands}

}

\newcommand\enchantedNotesMap[1][b]{
  \widePic[#1]{feylands}
}

\newcommand\sunderedMap[1][t]{
  \widePic[#1]{extracted/sundered}
}

\newcommand\enchantedMapSmall[1][t]{
  \widePic[#1]{extracted/enchanted}
}


\newcommand\feyCommentaryBailey{
  \playCommentary[t]{
    \begin{description}
      \item[Player 1:]
      (Sinkmaul)
      Let's jump into the gates.
      \item[\gls{gm}:]
      The people usher you in, pushing the thick wooden gate closed behind you.
      The \gls{village} looks dusky-dim, but active.
      People are rushing out with bundles of arrows in hand, and the \gls{village}'s last two cows are mooing loudly in distress; but the children all know to keep quiet until the danger has gone.
      \item[Player 2:]
      (Mildrain)
      Can we get up to the wall?
      \item[\gls{gm}:]
      You go up the stairs, take $2D6+1$ Damage.
      \item[Player 1:]
      (Sinkmaul)
      Wait, are we both up the stairs?
      \item[Player 2:]
      (Mildrain)
      Wait, do I roll the Damage?
      \item[\gls{gm}:]
      No, I can\ldots that's `12 Damage'.
      (Mildrain)
      Okay, so I'm dead\ldots
    \end{description}
  }{
    The initial description helps the players picture \pgls{village} under threat, then it all falls apart.
    \begin{itemize}
      \item
      The player wanted to know about how people access the wall, because that part of the arrangement was entirely unclear.
      They didn't mention their \gls{pc} actually going up.
      \item
      Responding with `\textit{yes, can get up the wall; do you?}' isn't much better.
      More description always works better.
      \item
      \Glsentrylongpl{pc} should take \pgls{action} before they die, not because `it's fair', but to make certain they're committed.
      Once they cast the dice, responsibility shifts from the \gls{gm} to the player.
    \end{itemize}
  }

}


\newcommand\feyCommentaryLocations{
  \playCommentary{
    \begin{description}
    \item[\gls{gm}:]
    Some miles on, and the Sun's high.
    You keep walking with that `squelch, squelch, squelch' along the snail-path, and spot a potato on the path ahead, held in someone's arms.
    \item[Player 3:]
    (Cleftbarb)
    Hide!
    \item[\gls{gm}:]
    That's \roll{Wits}{Stealth}, \glsentrylong{tn}~6.
    \item[Player 2:]
    (Mildrain)
    Scared of a potato-wielding marauder?
    Okay then\ldots\dicef{6}.
    \item[\gls{gm}:]
    Hiding in the thick darkness which surrounds the road, you wait until the small figure passes, grumbling to himself in the \gls{tradeTongue} about `the sleeping ones'.
    Once silence returns, the march continues, through hours of nearly-identical woodland.
    By the time you approach the tower, it's nearly night, but the dusk's light highlights a stone structure in the forest.
    A long cube, a rumbling stone box.
    \item[Player 2:]
    (Mildrain)
    Ignore!
    \item[Player 1:]
    (Sinkmaul)
    Right, `tower'.
    \item[Player 3:]
    What if it's treasure?
    \item[\gls{gm}:]
    The road opens, revealing a shining lake, with the tower in the centre\ldots
    \end{description}
  }{
    The pacing sped up suddenly as the players hopped through three \glspl{segment} within a couple of minutes.
    The players still received information (giant snails reduce the \gls{monster} population), gained questions (though nobody has actively asked about the source of the giant potato), and have a new location on the map which they might return to.

    That last \gls{segment} with the `stone box' (\gls{sepulchre}) has the `\gls{vlg}' symbol listed next to it.
    This indicates it should go onto the map, with the assumption it's always been there.
  }

}

\newcommand\feyCommentaryRoads{
  \playCommentary[t]{
    \begin{description}
      \item[\gls{gm}:]
      The path comes to a cross-roads.
      Do you continue forwards, or go left, or right?
      \item[Player 1:]
      (Sinkmaul)
      Left?
      \item[Player 2:]
      (Mildrain)
      Left\ldots
      \item[\gls{gm}:]
      Now the path has a turn-off to the right, or you can continue straight\ldots
    \end{description}
  }{\noindent
    This is awful, and the \gls{gm} should be fed to goblins, feet-first.
    If the players say they want to approach `the tower', then the \gls{gm} should describe approaching the tower, or describe what's stopping the \glspl{pc} from getting to the tower (and repeat, until they approach the tower).

    Instead of asking players about `left or right' turns, the \gls{gm} should abstract the entire journey into a roll.
  }
}

