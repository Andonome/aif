\chapter{Astronomy}\label{astronomy}\label{seasons}\index{Astronomy}
\index{Seasons}
\setcounter{list}{0}\setcounter{enc}{0}

\begin{multicols}{2}

\begin{boxtext}
\begin{list}{\addtocounter{enc}{1} \bf Cycle \Roman{enc}}{}

\item :

\begin{list}{\addtocounter{list}{1}\roman{list}}{}

\item \textbf{Qualmea:}  Stormy eclipse

\item \textbf{Atya:}  Mild

\item \textbf{Alassea:}  Cold, eclipse

\item \textbf{Cantea:}  Mild
\end{list}

\item :

\begin{list}{\addtocounter{list}{1}\roman{list}}{}
\item \textbf{C\'{a}lea:}  Warm

\item \textbf{V\'{e}rea:}  Mild

\item \textbf{Otsea:}  Storm

\item \textbf{Toldea:}  Mild

\end{list}

\item :

\begin{list}{\addtocounter{list}{1}\roman{list}}{}

\item \textbf{Laiquea:}  Warm

\item \textbf{Quainea:}  Mild

\item \textbf{Minquesta:}  Cold, eclipse

\item \textbf{Ohta:}  Mild

\end{list}

\end{list}

\end{boxtext}

\index{Ainumar}
\noindent
The planet \gls{ainumar} shines bright in the night sky, and appears a little larger than our moon does to us.
A raging storm moves across its face, which can be seen about half of the cycle.
Many people believe that the gods live there, inside the moving eye of the storm.

Three moons circle the planet.
One shines yellow, another grey, and the third we live on.

Our worldmoon swings wildly around \gls{ainumar}, coming so close in that it could almost kiss the gods, and then hurtles back to sit in empty space, far away from the Sun, before landing for a moment in the coldest patch of all -- the shadow of the gods.
Here, twice every three cycles, the world grows deathly cold.
For a few hours, everything goes black.
Usually, \gls{ainumar} provides dazzling light at night, and sometimes, after, a grand eclipse lasting several hours.

The planet has its own years, but referring to two types of year could get confusing -- the planet really has cycles -- it cycles around the Sun.
In each cycle are four seasons -- the first is called `Qualmea', in deference to the god of death.
It is full of brutal storms and at its height holds an eclipse so long it can often bleed into a two-day night.
After this comes `Atya' -- a milder season before a deep cold season known as Alassea, when people cheer themselves up by playing pranks, telling jokes and drinking or smoking to excess; at night \gls{ainumar} becomes a smaller orb in the sky, retreating like the Sun.
Things settle down again for Cantea, the fourth season, though \gls{ainumar} is as far away as ever.

That finishes the first cycle -- the second starts with a warm, Summery season called C\'{a}lea; during the night Fenestra spins round to view a dazzling \gls{ainumar}.
The world's shadow races across the face of the home of the gods, creating a dark eye which looks up at the night-time planet.
All the ice in the planet recedes to nothing before slowly reforming again.
The world trundles on from there until at the end of the eighth season the second cycle finishes.
The last cycle ends in Ohta -- the twelfth season, named after the goddess of war.

A further two moons can be seen on rare years, if one knows where to look.

\end{multicols}
