\chapter{\Glsfmttext{guard}}

\label{guard}

\begin{multicols}{2}

\noindent
The \Gls{guard} hosts humanity's front-line protectors, but also beckons the poor towards the jaws of all the beasts of Fenestra.

The \Gls{guard} typically dress all in black or dark greens.
The darkness helps them lay ambushes for monsters.
They tend to use weapons fitting dense foliage, such as short swords, and often carry swords on their backs rather than by their sides.

The \Gls{guard} recognise a strict hierarchy, in a very particular order.
Whoever the highest ranking person around is has the ability to override others.
The following stations are presented in order, so Scouts outrank Associates, and Associates outrank Soldiers.

\subsection{Trainees}

New recruits are typically sent to Eastlake to train for at least six months in the frigid landscape.
The journey itself comprises half the training, as they must politely request food from farms along the way.
\Gls{king} has forbidden the old tradition of adventurers proudly presenting the head of some beast, and requesting food, so many of the guard hang onto griffins' claws or a mandible from a chitincrawler instead, and attempt to pass it along to villagers who will then provide a `free' meal.

The journey is long and tiring, and the rations are so poor that many come back with less muscle than when they first arrived.
Nobody goes through a full journey without some kind of encounter with one of the beasts of Fenestra.

Trainees experience radically different journeys.
Some have to travel some distance to reach any part of the \gls{shatteredcastle}, along with a group of perhaps a dozen, depressed, clueless, landless farmers; they learn how to survive along the way, or die.
Others travel a short distance along with fifty others, making so much noise as they walk that even the woodspies flee at the sound of their feet.

\subsubsection{Duties}

\begin{itemize}
  \item{Survive}
  \item{Get to Eastlake}
\end{itemize}

\subsection{Novices}

Once training ends, and the novices return, the crown assigns them a location to live.
They sleep and eat in those barracks, and receive 20cp per week -- enough for a few strong drinks a week.

Novices are given basic leather armour and a short sword, and are expected to keep them in good condition.

\subsubsection{Duties}

\begin{itemize}
  \item{Guarding livestock overnight}
  \item{Keeping watch around the town wall}
  \item{Patrolling a town}
  \item{Forging a road through dangerous territory}
\end{itemize}

\subsection{Soldier}

Once a novice has felled their first monster with a close-quarter weapon, they are officially soldiers.
A lot of leeway is given to these soldiers, and often Lieutenants will allow someone to raise rank by killing a bandit, or by simply getting a good hit against some monster which was killed by the team.
Despite the leeway, arguments about who did what are common among new recruits.

Full members of the Guard can expect a full 100cp per week in payment, as well as the ability to sleep and eat in the soldiers' barracks.
Townsfolk and villagers alike typically respect them because of the difficult duties they perform.

Soldiers can request replacement arms at any point, but will typically have to accept whatever weapons are available -- generally a short sword and leather armour.

\subsubsection{Duties}

\begin{itemize}

  \item{Rounding up a village militia to fend off a basilisk}
  \item{Fighting monsters}
  \item{Tracking down thieves in town}
  \item{Accompanying trainees to Eastlake}
\end{itemize}

\subsection{Associate}

Associate are those who join the Guard for a particular mission, or just to advise.
Their pay varies greatly, but they are given the title and called `Associate Oscar', or `Associate Maria', in order to give them a fixed position.
Associates can tell Scouts what to scout for, but are expected to do as they are told by Lieutenants.

Having `Associate' as a formal and prestigious title also helps keep undesirables out of the \gls{guard}.
It might be tempting to hire some local thugs for a short mission, instead of paying people a long-term salary, but the rules on the matter are quite clear -- those thugs would outrank normal soldiers.
This potentially horrifying situation deters almost all Lieutenants from hiring anyone who is not minimally competent.

\subsubsection{Duties}

\begin{itemize}
  \item{Killing rogue magic users}
  \item{Hunting massive monsters}
  \item{Reporting on village masters who threaten the Rex}
\end{itemize}

\subsection{Scouts}

Those Night Guard with a special talent for sneaking and surviving in the wilderness become Scouts -- the elite troop who stay away from the base for long periods of time.

They earn 5sp per week in recognition of the additional dangers they face, and in order to pay for the expenses of travelling, such as paying for their own food and equipment.

Scouts who are caught staying with villagers when they were not called for can expect a demotion before long.

\subsubsection{Duties}

\begin{itemize}

  \item{Tracking down monster nests}
  \item{Tracking down deserters of the \gls{guard}}
  \item{Uncovering illegal weapon stockpiles}
  \item{Recovering precious items in fallen towns}
  \item{Entering nura-infested areas and reporting on findings}
  \item{Spying on other members of the \gls{guard} and reporting the local Captain.
  Scouts will often pretend to be novices or soldiers during these times}

\end{itemize}

\subsection{Masters}

Village masters, town masters, or anyone else with a noble title may command the \gls{guard} to protect, serve, or do anything else they wish, so long as no Lieutenants have given contrary orders.
Such masters are not \textit{in} the Guard in any way, but the soldiers are theoretically there as a replacement to the guards that such nobles would organize on their own, before the time of Nolan Shale.\footnote{See page \pageref{nolan} for more on the monarchy.}

\subsection{Lieutenant}

To achieve the rank of Lieutenant requires three things.

\begin{enumerate}

  \item{Literacy (meaning `Academics 1')}
  \item{A recommendation from a Captain to a Prefect or Commander}
  \item{Some outstanding achievement on the \gls{edge}}

\end{enumerate}

\noindent
If all this checks out well, Lieutenants earn 6gp each week.

The job of a Lieutenant is to lead the \gls{guard} to complete any jobs that others have failed at.
Specifically, if two or more groups fail at some task, the Lieutenant is bound to personally deal with the situation (along with as many men as they care to bring).

\subsubsection{Duties}

\begin{itemize}

  \item{Leading men into battle}
  \item{Recruiting new members}
  \item{Keeping track of soldiers and novices}

\end{itemize}

\subsection{Magus}

Typically, a Magus is a human alchemist from the \gls{college}, however priests of Ohta or Qualm\"e have been known to lend their martial skills to the battlefield.
Force mages use magical shields to protect wounded soldiers and scout ahead with magical senses.
Invocationists destroy enemies with fire.
Conjurers are rarely seen on the battlefield as \gls{king} has ordered anyone showing promise in this field to focus on the study of magical gates.

In theory, a Magus should not go onto the field without extensive training, but in reality \gls{college} often selects the least useful mage they have to go to war, then throw a few snowballs at them to prepare them for real fighting.
A magus is typically allowed free access to buy anything by request, but they are not paid.

\Gls{king} worries that with too much political power, they might become dangerous.
He has little to worry about, as very few join the Night Guard officially.
While they may wish to, the Guard are reticent to take on any because any local village masters, Lieutenants, and scouts will object to anyone joining who could outrank them.

\subsubsection{Duties}

\begin{itemize}

  \item{Strategy and planning}
  \item{Preparing temporary magical items for particularly dangerous missions}
  \item{Occasionally accompanying soldiers on dangerous missions}

\end{itemize}

\subsection{Observer}

The petty bureaucrats from \gls{shatteredcastle} who occasionally come to observe, complete reports, and pester Lieutenants, are a perpetual menace to the \gls{guard}.
They have no field experience but still outrank even a battle hardened magus.

They receive only 1gp per week.

\subsubsection{Duties}

\begin{itemize}

  \item{Routing out any challenges to the Rex}
  \item{Making sure everyone gets paid on time}
  \item{Advising Lieutenants on tactics}
  \item{Keeping an accurate count of the number of Night Guard}
  \item{Collecting taxes from masters}

\end{itemize}

\subsection{Captain}

Captains organize all of soldiers around a town.
This might include five to a hundred villages in the area.
Each village makes requests directly to the captain for aid, and the captain then decides how many men to send, and where.

Captains are selected from Lieutenants when an old Captain dies, or when a new town is founded.
Payment is 10gp per week.

\subsubsection{Duties}

\begin{itemize}

  \item{Ensuring the local town and surrounding villages remain free from nura}
  \item{Collecting reports from the scouts}
  \item{Making reports for Prefects}

\end{itemize}

\subsection{Commander}

Commanders reside in cities, and spend most of their days book-keeping, and making reports to Prefects.
They are chosen by Prefects, just whenever those prefects desire.
Commanders are never expected to fight, but should have some expertise in strategy in case of war.
Commanders nominally receive 20gp per week -- enough for a modest estate with servants -- but most bolster their pay with bribes.

\subsubsection{Duties}

\begin{itemize}

  \item{Liaising with village and town masters}

\end{itemize}

\subsection{Prefect}

Prefects are the bureaucrats from the \gls{shatteredcastle} who control the \gls{guard}.
They have no military background, but have the final say on all matters concerning war.
They travel from town to town, or area to area, often with fifty or more soldiers.

The prefects are given leave to bargain with town masters on behalf of \gls{king}, and for the most part have free reign to do as they please.

Prefects live in \gls{shatteredcastle}, typically in the Shale or the Pebbles wings.
Each of them show a lot of loyalty to the crown at every opportunity, including disciplining any soldiers suspected of an irreverent attitude towards \gls{king}.

\subsubsection{Duties}

\begin{itemize}

  \item{The peace of the realm}
  \item{Creating the budget for the observers}

\end{itemize}

\end{multicols}
