% Hyper Tempo
% Blights
%%%%% Banishment

\section{The Goblin Ways}

\begin{multicols}{2}

\subsection{The Road Less Travelled}

When goblins find a new pathway, they make a mark before exploring -- usually just a simple slash.
If they return, they make another slash over the first; this cross shows that someone returned safely.
But if goblins come to a path with three slashes and no crosses, it signals death.

When two hordes go to war, the space between them becomes a maze of lies, as each side wants to mislead the other about which routes will kill, and which lie empty.
Goblins at war may put four slashes down across a safe road, to make the enemy think it has some awful danger; or they may leave their slashes out in the open, as a double bluff; and both sides attempt new codes, perceptible only to their own side.

The multitude of empty, barren, spaces in the \gls{deep}, which have so many one-way paths, such as cliffs, cavernous-slides, and tunnels which fork in only one direction, all conspire to make space work in different ways to the lands above.
The way there is not the way back.

\subsection{Good Goblins and Bad Goblins}

If the \glspl{pc} attempt to talk with the goblins, they will talk peacefully, as long as the characters seem polite, and don't work with any of the bad goblins.

\subsubsection{Minding Your Manners}
is important in small societies -- when someone drags the rest down, they face the dangers of ostracism, or banishment.
But goblin manners don't come easily to outsiders.

These are some of the most important manners for most goblins:

\begin{itemize}
  \item
  Standing too far from people suggests you don't like them, or don't trust them.
  Real friends stand close enough to smell each other, and discuss their findings openly.
  \item
  Mud is useful for covering one's scent before a raid.
  Therefore, being covered in mud shows murderous intent.
  \item
  A lucky find should benefit everyone, especially when it comes to food.
  If the lucky one's don't share, the unlucky will become hungry, and the tribe will grow weak.
\end{itemize}

\subsubsection{Those Who Call Us `Them'}
\begin{exampletext}
are the worst.
They don't share their tools (like good goblins do), and constantly interrupt each other.
They can't negotiate or trade properly - it's all shouting and constant noise.

They are so bad that they can't even tell stories -- they just shout over each other, constantly, interrupting the narrative, and talking a bunch of nonsense.
Do you know what you call a people without stories?
They are quite literally `uncultured'.

They're degenerated creatures, full of mutants with claws so long they can't properly hold a hammer or blade.

We need to protect ourselves, and our gardens.
They come in the night, tearing and burning everything they can, because they cannot grow anything of their own.

\end{exampletext}

The two groups of goblins have no names for each other beyond `them' and `us'.

\subsubsection{We Must Stop Them}
\begin{exampletext}
so we can live in peace, sharing and helping each other.
Their \glspl{witch} want to rule everyone in this land, and they use their garden to make weapons to hurt and kill us.
Sometimes their \gls{witchcraft} kills a dozen of us at a time.
One moment you're chatting, the next you're on the ground, suffocating on poisonous fumes, or burn in a fiery explosion, then those \glspl{witch} send in their minions.

Go and try to talk with them, go see for yourself, if they don't kill you, they take everything you have.
Normal people share food and songs.
But those ones think they can just take someone's spear or drinking-horn, as if they owned it.

And they can't even make stories or songs.
They just repeat what they hear from us, because we have all the best stories.

They hate people like me -- special people.
I was born with venom in my teeth, because my mother ate up all the powerful plants in their garden.
But when one of them grows up to be special, the \glspl{witch} can't wait to send them over here to kill us.

\end{exampletext}

\end{multicols}
