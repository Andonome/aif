% Hyper Tempo
% Blights

\begin{multicols}{2}

\subsection{Minding Your Manners}
is important in small societies -- when people drag the rest down, they can face the dangers of ostracism, or banishment.
But goblin manners don't come easily to outsiders.

These are some of the most important manners for most goblins:

\begin{itemize}
  \item
  Standing too far from people suggests you don't like them, or don't trust them.
  Real friends stand close enough to smell each other, and discuss their findings openly.
  \item
  Mud is useful for covering one's scent before a raid.
  Therefore, being covered in mud shows murderous intent.
  \item
  A lucky find should benefit everyone, especially when it comes to food.
  If the lucky one's don't share, the unlucky will become hungry, and the tribe will grow weak.
\end{itemize}

\subsection{The Road Less Travelled}

When goblins find a new pathway, they make a mark before exploring -- usually just a simple slash.
If they return, they make another slash over the first; this cross shows that someone returned safely.
But if goblins come to a path with three slashes and no crosses, it signals death.

When two hordes go to war, the space between them becomes a maze of lies, as each side wants to mislead the other about which routes will kill, and which lie empty.
Goblins at war may put four slashes down across a safe road, to make the enemy think it has some awful danger; or they may leave their slashes out in the open, as a double bluff; and both sides attempt new codes, perceptible only to their own side.

The multitude of empty, barren, spaces in the \gls{deep}, which have so many one-way paths, such as cliffs, cavernous-slides, and tunnels which fork in only one direction, all conspire to make space work in different ways to the lands above.
The way there is not the way back.

\end{multicols}
