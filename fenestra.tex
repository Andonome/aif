\chapter{Fenestra}

\epigraph{Still around the corner there may wait,

A new road or a secret gate.}{Tolkien}

\iftoggle{core}{}{
  \section{Monsters \& Malthus}
  \section{Monsters \& Malthus}

\begin{multicols}{2}

\subsection*{The Primordial Forest}

\begin{exampletext}

  The forest wants to eat you, so pay attention.

  You've had an easy life, in your secluded, little patch, but here we live on the ground, where giant creatures crawl everywhere, and I can assure you that I'm the gentlest one out here.
  They're probably looking at you and licking their lips.
  Well, not `lips'.
  I don't think a single one has lips.
  Some have beaks, others have `mandibles', but you get my point.

  The chitincrawlers lay webs, but don't think they need to wait for you.
  If they get hungry enough, they'll run straight at you, grab you with their claws, and just start eating!

  And remember to be on the lookout for moving trees.
  If you see something shift that's meant to stay still, it could be a woodspy -- like an octopus, but\ldots do they have octopuses where you're from?
  No I didn't think so.
  But I'm glad to have you with us.
  I hear your people can see better in the dark than we can.

  Most of the world sits in darkness, just like this.
  Most of the world lacks roads, beer, beds, and everything that makes life worth living.
  For this reason, we exist, to push back the darkness, and make way for more civilization.


  \subsubsection*{Bandits}

  Notice the trees.
  \ifnum\value{temperature}<1
    They're just sitting there in the \seasonDesc n wind right now, but once the warmer seasons hit, they'll be full of fruits, and you'll see things growing all around.
  \else
    There's good eating up there if you can climb.
    The forest is laying a trap for us, but it's a tasty one!
  \fi
  Humans could live out here like some kind of paradise, never working, just taking food from the trees -- at least over C\'alea and Laiqea, and the other warmer seasons.
  Even Toldea has plenty to eat if you know where to look!

  And all this means the forest has laid another trap for us.
  Thieves, cut-throats, and black-alchemists who want to escape the law come into the forest, and she treats them well.
  They live here, tax-free, robbing villages over the cold seasons.

  And obviously it's \emph{our} job to come out here and route them out, by fire and sword.


  \subsubsection*{And worse\ldots}

  Magic is a horrid thing.
  Once someone knows enough of it, they can destroy a city.
  And you can never spot someone who knows it.
  Well sometimes I think you can -- a shifty look in the eye, especially if they've been away from the \gls{alchemists} too long.
  They may as well stay out there, as far as I'm concerned.

  Sometimes we lose whole cities, to curses which bring the beasts out of the forests, or maybe they'll just turn a town's walls to ice on a warm C\'alean day and watch them melt while the beasts comes in to eat the village.

  You don't know any magic do you?

  Shame.
  We could probably use some out here.
  At least a little blessing or something would be nice.
\end{exampletext}

\subsection*{The \Glsentrytext{edge}}

\begin{exampletext}
  We're coming up to civilization at last!
  See that patch of lawful land?
  No trees, vines, bushes, or anything?
  That means the town must be close.

  I was in proper \textit{big burn} once.
  We covered the area in oils last Laiquea \ifnum\value{temperature}=3 on a day as warm as this one \else on a scorcher of a day \fi and it burnt so high I swear it reached \gls{ainumar} and made the gods stink for a week of woodsmoke!

  \subsubsection*{Walled Villages}

  The great clear areas around the \gls{edge} provides a buffer. 
  Anything that wants to skitter over here in the daylight gets filled with arrows.
  Of course that won't always stop a basilisk, but it can drive them off.

  We'll see some sheep inside, maybe even cows, but there's never much meat here.
  You can take them out grazing only a short way, where it's safe, and the sight of so many animals always tempts something out of the forest sooner or later.
  Mostly, meat comes from the inside, while the outer circles send back wood, or forest fruits, or anything from the fields outside their walls.

  Most humans on the \gls{edge} learn the bow, or at least how to use a crossbow.
  Anyone who doesn't puts everyone else in danger.
  I grew up in a village like this one -- and we had a lot of sleepless nights, telling each other stories of famous adventurers from back when that sort of thing was still legal.

  Simpler times.
  But also dangerous -- we're much safer now with \gls{king} in charge of everything, and don't let anyone hear you say otherwise.
  Any time these places need help, \gls{king} sends us out to help out.

  We can't stay long, so get some rations, and we'll be on the road soon, headed inwards.

  \subsubsection*{Lonely Roads}

  That's another one of your duties, recruit -- maintaining the roads.
  Sometimes bandits slip past the outer villages and camp at the side.
  Sometimes the critters in the forest do the same.
  A lot of them are smart enough to know where we go, so they'll sit at the side of the road, picking off traders who carry meat, or just any trader.
  Traders \emph{are} meat as far as the forest is concerned.

  I wanted to be a trader once, but honestly couldn't summon up the courage.
  I survived a couple of trips, but I knew I wouldn't survive for long.
  So if you're ever travelling out to the \gls{edge}, remember to let the word around town, just like I did there.
  The more people who travel together, the safer.
  And if you end up getting eaten by something, maybe the beast will leave the trader alone, and let him get to market.
  Then you'll die a hero!

  Everyone dies a hero in the \gls{guard}.

  See that crossroads ahead?
  That's a good sign.
  We passed the \gls{edge}, now we have two roads, meaning at least two villages around us.
  They'll have walls of their own too, but the farther inwards we get, the safer.

  Sometimes these outer roads break.
  Whenever people wander down one, they might not return.
  And if nobody who went out for the last week returns, that tells the village that it's time to stay put for a while, maybe hope the \gls{guard} will come and save them.
  When that goes on too long, sometimes villages get isolated entirely, and have to try to live out their lives alone, without iron, or coal from outside.

  If that goes on too long, it's another win for the forest, and a loss for civilization.
  We'll keep on pushing out, but when we reach too far out, the forest eats our fingers.

  \subsubsection*{Quiet Hamlets}

  We're getting closer.
  See that little hamlet?
  No walls, or nothing -- just stone houses for emergencies.
  Very little makes it in this far.

  Whether it's beasts or bandits, they get tempted by the smells along the road, and end up in an altercation with one of the settlements further out.

  These inner lands provide most of the meat of Fenestra.
  I bet you've even had some back home.
  No?
  Well lets go up and say `hello'!
  Villagers always give hospitality to the \gls{guard} when they see us.

  \subsubsection*{Little Masters}

  Each area has its own master.
  It's not true what they say about humans -- we don't need leaders telling us what to do, but we have them anyway.
  They don't really do much, but I guess they look nice and fancy.
  Village masters own a few villages, and town masters own a town.
  Then they send their taxes back and have a nice dinner.

  Must be nice.
  Pointless, but nice.
\end{exampletext}

\subsection*{Hungry Towns}

\begin{exampletext}

  I suppose you've never seen a big city like Arthur's Wing before.
  No monsters live in here, so everyone can rest easy, aside from the cutthroats and thieves, who of course have to worry about the likes of me dragging them into our merry little crew and our glorious missions.
  Look at that pathetic beggar over there, asking for food.
  He can clearly walk, but refuses to sign up with us and fight for the crown.
  Remember that even if you get mauled by some creature in the forest, that still leaves an opening for your companions to get a hit against the beast that killed you.
  Everyone dies a hero in the \gls{guard}.


\subsubsection*{Guilded Temples}
  The various guild-temples obtained their monopolies long before the current Rex.
  The priests of V\'er\"e have always taken care of the court houses, and the priests of Ohta have always dealt with weapons.
  And of course, nobody else is allowed to.

  They all support \gls{king}, and he supports them.
  Of course in the olden days -- back when we still had adventurers -- people would rise through divine gifts to be a top priest, and give blessings to all the people.

  Nowadays is different.
  The guild higher-ups just want to make money.
  Some of them still hear the call, and get gifts from the gods, but then their superiors send them out on a grand mission.
  We can't have useful people doing paperwork all day, so I suppose it's for the best.
  And it's good to have them out with us in the \gls{guard}.

  Let's get some rest.
  You've got a mission already.
  Nobody's heard from Greenwell in a week, and someone needs to find out why.
  I've found a few other new recruits, so you won't be lonely.

  Time to be a hero.

\end{exampletext}

\end{multicols}

\section{Circles of Civilization}

\begin{multicols}{2}

While maps of Fenestra, made by men, focus on cities and villages, fields and coppiced trees; the truth is that the world is mostly made of dangerous terrain, running wild with dangerous creatures.

\subsubsection{The Outer Darkness}

Throughout most of Europe's history, we were the nastiest, scariest things around.
We could wander freely, and use the land as we saw fit.
We tamed forests by cutting down the under brush so we could hunt game more easily, and farm any land we could.
Fenestra, by comparison, sustains a much smaller population per square mile.
The continent remains wild, and only little blotches of civilization exist, with rare roads running between them.

These regions of outer darkness, placed on maps using guesswork and rumours, form the larger part of the world.

\subsubsection{Lonely Roads}

Roads which lead from town to village, or between villages, provide easy walking for people and horses.
But the roads which connect two great cities by cutting through a large, wild, forest, present far more danger.

Sometimes these lonely roads break.
Whenever people wander down one, they might not return.
However, if too many parties go down one but do not return, the people know that the lonely road has been closed.
Typically a  road's closure is resolved by a band of armed warriors who go to clear and cleanse whatever lies on the road to eat people.
If they don't return, then the road lies closed for good, and people have to take another road to civilization, or forge a new one through the wild forest.

\subsubsection{Chaos at the Edge of Civilization}

Exactly what lies in wait for people outside the small civilized lands depends upon the area.
Mount Arthur has bears, giant arachnids, griffins and more.
The frozen Eastlake area in the North tends to have a lot of undead.
Quennome has every creature one can name, in addition to strange monsters which defy classification.

Long roads, connecting different civilizations, wander through the forest for many miles.
These long roads are only taken by the suicidal, or by groups of armed men.
The exact number of soldiers depends upon the area, but typically six to twelve can keep themselves safe if they take turns at watch during the night.

The forests hold so much edible material -- fruits, vegetables, roots, and game -- that people could live easily within them were it not for the creatures which hunt them.
For this reason, outlaws commonly make little liveable spots, either in a self-made shelter, an abandoned stone building in the forest, or anywhere else they can put up enough of a wall to stay safe.
In this way, any group can keep themselves fed until the food in the local area runs out, or until a cold season hits.
In general, such groups do not have the organizational skills to survive, so they either die one by one, as the forest eats them, or they turn to banditry, and someone comes for their heads.

From the point of view of civilization, the greatest dangers come from any element which can organize the creatures of the forest.
Sometimes this is a necromancer, able to summon the dead, and intent on taking out villages.
At other times, an old elf has become irritated with humanity's encroachment on a nice forest, and decides to organize the creatures of the forest to attack, and trees to grow tall and reclaim the land.
These `forest masters', or `beast masters' pose such a danger that local lords must send specialized hunters after them.
Sometimes a full army will go, but smaller teams are often preferred.
When necromancers kill large armies, the lot can be turned undead, and when priests of the forest sing enchantment spells over a wide area, the extra numbers offered by an army do nothing to help the battle.

\subsubsection{Villages \& Walls}

Villages have numerous ways to stay safe, from staying on small islands to building massive wooden walls.
Many build massive moats around their lands to keep their animals safe, while others keep their animals in barns, and post watchmen with bows to guard them through the night.
Predatory creatures do not always come out during the night, but during the day people have less to fear because they can see danger a long way off, and fire arrows before it arrives.

Villages almost universally cut down all vegetation in the area to give themselves better visibility.
Farther afield, villagers allow trees to grow so they can grow them into the correct shapes for quarterstaffs, or use them to make arrows.
Massive orchards can be left safely outside, as the animals of the forest already have plenty of fruits to eat.

Villages typically surround a town in every direction, meaning that those close to a town or city can rest easy;
any creatures wandering from the forest will typically encounter trouble with those in the outer layers before getting anywhere near the inner circle.
Meanwhile, those poor villages in the outer circle can see a dark, primordial forest every day.

If a village defends itself well, it can grow, and one day may create another village farther out, pushing further into the deep forest.
However, this push-and-pull game does not always go so well for people.
When a village has too many young archers die, or too many livestock stolen to feed itself properly, it can no longer defend itself, and the remaining inhabitants must flee to neighbouring villages, or into a town, where most will have to join the \gls{guard}.

\paragraph{Dwarves} tend to live underground, with tight fortifications, and almost always maintain a direct, safe, tunnel to some nearby dwarvish city.

Despite their relative safety, dwarvish parties must still venture out in order to hunt for more seams, or establish fertile mushroom gardens.

\paragraph{Elves} build small villages almost exclusively.
Each one needs only one or two powerful spellcasters and the rest can remain safe.
The exact magics employed vary from village to village, but they might include a spellcaster who can sense any nearby dangers and incinerate them, or someone who can bless all other villagers with luck when they leave.

\paragraph{Gnomes} tend towards hidden villages, but a few cities remain within Fenestra.
They rely extensively on traps both underground and above ground.

\paragraph{Gnolls} keep plenty of fierce guard dogs around their area to alert them to wandering monsters.
Every gnoll in a village knows they must run and hunt at the first sign of danger.
Gnolls welcome such incursions more than any other race, as they enjoy meats of any creature.

\subsubsection{Temple Guilds}
\label{guilds}

Within every town in Fenestra, divine monopolies are officially enforced.
People must seek legal rulings in a temple of V\'{e}r\"{e}, swords are only sold from the temples of Ohta, and every tavern, at least in some official sense, is a temple to Alass\"{e}.

\paragraph{Alass\"{e}} governs `the ale guild', and all manner of taverns.  Officially, all taverns are temples to Alass\"{e}.

\paragraph{C\'{a}l\"{e}'s} temples doubles as paper-producing guilds, and provided all townmasters and areamasters with seneschals to count up the lord's holdings and due taxes.
Many are now being replaced with the \gls{king}'s own accountants.

\paragraph{Laiqu\"{e}} was at one point in charge of grain supply and various tasks related to farming.
The temple have since abandoned all such activities, and mostly abandoned any buildings they once held in towns.
The priesthood have stated their intention to work purely on theological matters; as a result they hold the highest portion of efficacious miracle workers.

\paragraph{Ohta} rules over warfare and many call the temple `the Sword Guild', as it has exclusive jurisdiction over the sale of all weapons.

\paragraph{Qualm\"{e}} does not deal with much beside funerals, and once dealt with death-payments, made when a murderer must make pay the expected value of the victim to the victim's family.
These services were not popular, and death-payments were soon taken over by the Verean temples.
After that, the church had borrowed too much so the temples were sold or abandoned.
A few remote priests decided to pass into undeath and remain in their abandoned monasteries in a sad, robotic, and bitter state.

\paragraph{V\'{e}r\"{e}} has become central figure of `the Justice Guild'.
People approach the temples of V\'{e}r\"{e} for marriages, court rulings, and to make public business deals.

\subsubsection{\Gls{shatteredcastle}}

\Gls{king} rules the land completely from every area at once, with the single exception of Liberty.
Not long ago, these lands had separate rulers, and in those times each lord of a land -- large or small -- created a personal militia to deal with problems.
Nowadays, all personal armies have been made illegal.
People may hold weapons, but nobody may have a standing army.

In place of the local militias, \gls{king} has created the \gls{guard} to look after the realm.
Anyone unable to find proper work, goes to work in the \gls{guard}.

\end{multicols}

}

\section{Circles of Civilization}

\begin{multicols}{2}

\iftoggle{core}{}%
{
  \widePic{Roch_Hercka/cave_fight}
}

\subsubsection{\Gls{shatteredcastle}}

\Gls{king}'s castle exists in every corner of Fenestra, all at once.
It has a magical portal to Eastlake, not far from \gls{college}; it has another to Quennome, where the elves refuse to acknowledge his rule.
It has portals all over Fenestra, so he can take a walk through the snow in the morning, go back into his castle, then wander in the Mt Arthur sunshine in the afternoon.

\Gls{shatteredcastle} seems to exist \emph{around} Fenestra, rather than anywhere in it, as its presence dictates local politics for leagues around every wing.

\subsubsection{The Law}

Not long ago, each of these lands had a king.
Every noble in those lands would raise their own militias to protect their people.
But \gls{shatteredcastle} allowed one of \gls{king}'s ancestors to move armies quickly across the land, and become a `rex' -- ruler of all of Fenestra.
He outlawed personal militias, limiting nobles to a few personal guards for their own holdings, and outlawed amassing large amounts of weapons.

In place of the individual armies, \gls{king} sends out his \gls{guard} -- the people entrusted with the safety of the realm.
Anyone unable to find proper work, goes to work in the \gls{guard}.

The \gls{guard} and those traders who can afford the privilege use \gls{shatteredcastle} to travel across Fenestra faster than anyone else possibly could.

\subsection{Temple Guilds}
\label{guilds}

Within every town in Fenestra, divine monopolies hold domain over every market place and official dealings in the cities.
Their adventuring priests have become `\glspl{nomad}` -- the guild members who wander across Fenestra, tying guild factions together, and completing missions.

\subsubsection{The Ale Guild}
Alass\"e governs everything to do with wheat, barley, and rye.
Almost every tavern also functions as her temple.

Her \glspl{nomad} once travelled and sang, but since bards received so many new laws under \gls{king}, they mostly just research better methods for brewing beer.

\subsubsection{The Justice Guild}

V\'{e}r\"{e} has become central figure for all manner of rulings.
People approach the temples of V\'{e}r\"{e} for marriages, court rulings, and to make public business deals.

\subsubsection{The Paper Guild}

C\'{a}l\"{e}'s temples doubles as paper-producing guilds, and provided all town masters and area masters with seneschals to count up their holdings and taxes due.
Many are now being replaced with \gls{king}'s own accountants.

Many priests claim that they should govern \gls{alchemists} by divine right, but so far \gls{king} has not entertained these notions.

The `Temple of Light', occasionally sends recruits along with the \gls{guard} to find information from beyond the \gls{edge}, especially where a road has broken recently.

\subsubsection{The Sword Guild}

Ohta rules over warfare and works often with the \gls{guard}, as it has exclusive jurisdiction over the sale of all weapons.

Her \glspl{nomad} travel with the \gls{guard} routinely, and regularly pretend to be one of them to check on whether small nobles have created any weapon stocks beyond what the law allows.

\iftoggle{core}{}%
  {\widePic[t]{Leonard/next_day}}

\subsubsection{The Temple of the Forest}

Laiqu\"{e} was at one point in charge of various tasks related to farming.
The temple have since abandoned all such activities, as they found handling money distasteful, and mostly abandoned any buildings they once held in towns.
The priesthood have stated their intention to work purely on theological matters; as a result they hold the highest portion of efficacious miracle workers.
Almost all are \glspl{nomad}.

Most people feel at least a little suspicion from them.
Laique herself symbolizes the forest, and it seems the `temple' never entirely takes humanity's side.

\subsubsection{The Final Guild}

Qualm\"{e}'s temple did not survive the switch into being a guild as well as the other religious orders.
It didn't deal with much beside funerals, and death-payments (made when a murderer must make pay the expected value of the victim to the victim's family).
These services were not popular, and death-payments were soon taken over by the Justice Guild.
After that, the church had borrowed too much so the temples were sold or abandoned.

Once the temple lost the people's favour, many priests refused to abandon their fading temples, and instead decided to stay, beyond their own death, to tend to the grounds until living people could return to pray again for their immortal souls.
They remain in a sad, robotic, and bitter state.

A few unfortunate incidents with these dead priests caused further reputation damage, so very few pay any respect when Qualm\"e holy day arrives.

\subsection{Villages \& Walls}

Villages have numerous ways to stay safe, from staying on small islands to building massive wooden walls.
Many build massive moats around their lands to keep their animals safe, while others keep their animals in barns, and post watchmen with bows to guard them through the night.
Predatory creatures do not always come out during the night, but during the day people have less to fear because they can see danger a long way off, and fire arrows before it arrives.

Villages almost universally cut down all vegetation in the area to give themselves better visibility.
Farther afield, villagers allow trees to grow so they can grow them into the correct shapes for quarterstaffs, or use them to make arrows.
Massive orchards can be left safely outside, as the animals of the forest already have plenty of fruits to eat.

Villages typically surround a town in every direction, meaning that those close to a town or city can rest easy;
any creatures wandering from the forest will typically encounter trouble with those in the outer layers before getting anywhere near the inner circle.
Meanwhile, those poor villages in the outer circle can see a dark, primordial forest every day.

If a village defends itself well, it can grow, and one day may create another village farther out, pushing further into the deep forest.
However, this push-and-pull game does not always go so well for people.
When a village has too many young archers die, or too many livestock stolen to feed itself properly, it can no longer defend itself, and the remaining inhabitants must flee to neighbouring villages, or into a town, where most will have to join the \gls{guard}.

\paragraph{Dwarves} tend to live underground, with tight fortifications, and almost always maintain a direct, safe, tunnel to some nearby dwarvish city.

Despite their relative safety, dwarvish parties must still venture out in order to hunt for more seams, or establish fertile mushroom gardens.

\paragraph{Elves} build small villages almost exclusively.
Each one needs only one or two powerful spellcasters and the rest can remain safe.
The exact magics employed vary from village to village, but they might include a spellcaster who can sense any nearby dangers and incinerate them, or someone who can bless all other villagers with luck when they leave.

\paragraph{Gnomes} tend towards hidden villages, but a few cities remain within Fenestra.
They rely extensively on traps both underground and above ground.

\paragraph{Gnolls} keep plenty of fierce guard dogs around their area to alert them to wandering monsters.
Every gnoll in a village knows they must run and hunt at the first sign of danger.
Gnolls welcome such incursions more than any other race, as they enjoy meats of any creature.

\subsubsection{Chaos at the Edge of Civilization}

Exactly what lies in wait for people outside the small civilized lands depends upon the area.
Mount Arthur has bears, giant arachnids, griffins and more.
The frozen Eastlake area in the North tends to have a lot of undead.
Quennome has every creature one can name, in addition to strange monsters which defy classification.

Long roads, connecting different civilizations, wander through the forest for many miles.
These long roads are only taken by the suicidal, or the \gls{guard}.
The exact number of soldiers depends upon the area, but typically six to twelve can keep themselves safe if they take turns at watch during the night.

The forests hold so much edible material -- fruits, vegetables, roots, and game -- that people could live easily within them were it not for the creatures which hunt them.
For this reason, outlaws commonly make little liveable spots, either in a self-made shelter, an abandoned stone building in the forest, or anywhere else they can put up enough of a wall to stay safe.
In this way, any group can keep themselves fed until the food in the local area runs out, or until a cold season hits.
In general, such groups do not have the organizational skills to survive, so they either die one by one, as the forest eats them, or they turn to banditry, and someone comes for their heads.

From the point of view of civilization, the greatest dangers come from any element which can organize the creatures of the forest.
Sometimes this is a necromancer, able to summon the dead, and intent on taking out villages.
At other times, an old elf has become irritated with humanity's encroachment on a nice forest, and decides to organize the creatures of the forest to attack, and trees to grow tall and reclaim the land.
These `forest masters', or `beast masters' pose such a danger that local lords must send specialized hunters after them.
Sometimes a full army will go, but smaller teams are often preferred.
When necromancers kill large armies, the lot can be turned undead, and when priests of the forest sing enchantment spells over a wide area, the extra numbers offered by an army do nothing to help the battle.

\subsection{The Outer Darkness}

While maps of Fenestra, made by men, focus on cities and villages, fields and coppiced trees; the truth is that the world is mostly made of dangerous terrain, running wild with dangerous creatures.

Throughout most of Europe's history, we were the nastiest, scariest things around.
We could wander freely, and use the land as we saw fit.
We tamed forests by cutting down the under brush so we could hunt game more easily, and farm any land we could.
Fenestra, by comparison, sustains a much smaller population per square mile.
The continent remains wild, and only little blotches of civilization exist, with rare roads running between them.

These regions of outer darkness, placed on maps using guesswork and rumours, form the larger part of the world.

\end{multicols}

\section{Regions}
\label{regionEncounters}

\begin{multicols}{2}

\widePic[t]{Tom_Prante/winter}

\iftoggle{players}{}{
  \encEastlake
}

\noindent
Fenestra contains seven regions, each of which have a slightly different ecology and different populations.

The human-populated areas tend to defend themselves and drive out dangers and monsters, but wander too far between settlements, or make a \gls{nomad} into a forest, and you may find a nasty creature, random traders, or even a mana lake.%
\iftoggle{players}{}{%
  \footnote{See page \pageref{mana_lake}.}

Each region has a set of standard encounters for times when players wander into less populated and more dangerous areas.  These encounters aren't necessarily there for combat.  If players spot wolves, the pack may simply stalk them with a mind to steal food.  Alternatively, the players may wander into a pack of wolves mid-hunt, bringing down a deer.  An encounter with griffins need not be violent -- they could simply see a nest in the distance, and make the decision to steer clear.

Encounters can flavour a journey in a multitude of ways, and allow those characters with Caving, Wyldcrafting or Caving to interact with the world around them.
}

\subsection{Eastlake}\index{Eastlake}

Soggy, miserable children -- mostly with rich parents -- are forced in their hundreds to \gls{college} to receive training in history, invocation, literature and most popular of all -- conjuration.
Most are from warmer lands, and have a hard time dealing with the cold.
Mid-way up Eastlake, overlooking the great lake which the area is named for, the University watches over many leagues.
At night, it is possible to see little hearth fires from the windows of village cottages.
Evergreen trees dot the landscape and then turn into thick forests further North.

\Gls{king}, worried about the possibility of creating a ruling class of powerful alchemists, has banned all alchemists from owning land.
For this reason, first sons are almost never sent to study in the college.
After their studies, many return to their families with a few magical tricks but often lead solitary lives as they can neither own their own land nor till another's -- such activity would be beneath the nobility.
Some few become village mages, neither owning land nor working in the normal way but rather gaining a perpetual stream of revenue from a nearby village for entertaining them, fixing problems and occasionally dealing with intruding monsters.

The official god of the region was Qualm\"{e}, but after the University was erected, C\'{a}l\"{e} became the favourite as so many followed the examples of the mages in the area.
Massive halls full of the writings of C\'{a}l\"{e} fill the college like miniature chapels full of reading rather than pews.

Farther North, snow-elves live in icy caverns or build castles made partly of ice and partly of enchanted evergreen conifers.
They hunt with a combination of spears and enchantments and occasionally battle with gnoll incursions from Whiteplains.
Humans typically count these areas as beyond the \gls{edge}, despite the elven population.

\iftoggle{players}{
  \subsubsection{Walking in the Snow}
  Eastlake's snow is a menace to anyone trying to travel, but it's never worse than when it melts.
  The warmer seasons bring floods along all rivers, and at the base of every mountain, which just about covers all of Eastlake.

  Throughout the more normal weather, you might be lucky enough to encounter a vegetable trader from some other kingd\ldots er, `region'.
  Less lucky travellers might be able to trade with one of the local snow elves for whatever they happen to have caught.
  \emph{Really} unlucky travellers might just be caught by those snow elves.
  They say the older ones sometimes eat people or dwarves.
  \footnote{They never eat gnomes. Maybe gnomes taste really bad?}

}{
  \subsection{Encounters}
  Despite being a dangerous area, Eastlake's slow climate means danger does not swing around very often.

  \begin{nametable}{General Tempo Chart}
    \textbf{Roll} & \textbf{Result} \\\hline
    6 & Encounter, and reroll! \\
    5 & Encounter. \\
    4 & 1 day peace. \\
    3 & 2 days peace. \\
    2 & 4 days peace. \\
    1 & 6 days peace. \\
  \end{nametable}

  \paragraph{Journeys}
  to the \gls{edge} add +1 to the left-hand die, while excursions into the forest add +2.

}

\subsubsection{Forgotten Temples}

A little farther North still, a handful of those forgotten temples to Qualm\"{e} remain with animate, but non-living priests.
They have decided never to die, but to study and pray forever, through death, and into undeath.
While many rest meditatively, some have become resentful of being forgotten, and raised armies of the dead by stealing corpses.
The region now has a serious problem with roaming undead.

\subsubsection{Elven Wastelands}

An empty wasteland sits between the elvish forests and the dwarvish tunnels in the nearby Shale region.
Historically, skirmishes and minor wars have opened up when dwarves came North to chop down the massive trees for wood.
The elves guard their forests fiercely, partly because fewer trees means fewer animals to hunt and partly because they feel a deep connection with the area.

\subsubsection{\Glsfmttext{college}}

The four houses -- Alisa, Kisha, Stein and Ventress -- run the college as a union, and have successfully managed to strangle a lot of the realm's alchemical potential.
Deals with the Crown have been legislated, and mages may no longer swap magic of any kind -- all magical learning must pass through the guild and pay for the privilege.
Any alchemy practised or learned outside of the Guild's Monopoly is classified as `black magic'.

Every alchemist must retake their oath each year:

\needspace{3em}
\begin{enumerate}

  \item
  To never dishonour the Crown.
  \item
  To never create false coinage.
  \item
  To never target a landowner or \gls{guard} member with magic.
  \item
  Never create long-range gates.
  \item
  Obey all \gls{guard} members of Lieutenant status, or higher.
  \item
  Obey all commands from \gls{alchemists}.

\end{enumerate}

\subsubsection{Gnoll Territory}

Local tribes have become a little tame after picking up so many human goods.
Foreigners consider them dangerous, but the local trackers understand that one just needs to know which tribe one is dealing with to predict their behaviour.
They may all look the same to outsiders, but each tribe is proud of its reputation, whether as honourable traders, or brutal killers.

\subsubsection{Snow Elves}
\index{Elves!Snow}

The elves of the region -- known for their bleach-pale skin -- hunt with spears and magic.
Typically they go after aurochs, but they have no problem eating wolves, and some of the older elves have been known to gain a taste for human, gnoll, and dwarven flesh.

Like any other elves, they live below ground, but more often replace the standard glass ceilings with sheets of ice.
There, in little caves or ice-domes, the elves live, often with a few specialized spell-casters, who provide food for the community.

Their territories encompass wide areas, which they defend fiercely, though not with numbers -- a group of thirty elves may wander and hunt in areas which could house some thousands of humans.
Humans invaded the area when elven tribes died, leaving a spot open for farmers and fishermen.
Once a walled village establishes itself, the snow elves leave it alone out of respect for another's territory.

If they trade, they usually make themselves look human with the Polymorph sphere, and cover up their strange accent by pretending to be foreign (which is not entirely untrue).

\subsubsection{Splinter Island}

Splinter Island began life as the home of the king of Eastlake, back when individual realms had kings.
A grandchild later trained as an alchemist, and opened a portal towards Whiteplains' central hub.%
\footnote{See page \pageref{whiteland_heart}.}
When \gls{king} took over, this place became just another wing of \gls{shatteredcastle}, so now those wishing to trade with other lands must take boats out and pay a tax to use the road to warmer climates, such as Quennome or Liberty.

\iftoggle{players}{
  \subsubsection{Character Concepts}

  \begin{itemize}

    \item
    A dilettante gnoll, who has gained too much of a taste for human luxuries to spend his time hunting to trade for a couple of cakes.
    She wants more.
    She wants jam, carrots, a fancy hat and a greatsword all of her own.
    \item
    A proud alchemist, too eager to talk about being from the place alchemists are trained.
    He wears no armour and thin clothing, relying on his magic to protect him from harm, and his birthright of cold-endurance to endure the elements.
    \item
    A wandering gnome, bent on showing that magic, trickery, and planning make him more of a warrior than large men carrying big sticks.

  \end{itemize}
}{}


\widePic[t]{Tom_Prante/swamp}

\iftoggle{players}{}{
  \encLiberty
}

\subsection{Liberty}
\index{Liberty}
\index{Dogland}

A generation ago, gnolls ruled Dogland, with only sparse human villages dotted around the outer rim.
Rex Nolan decided the area would be `civilized', and drafted an army of the \gls{guard}.
Humanity did badly, but the Rex continued pushing everyone he could until settlements had been established.

Since then, the Temple of V\'er\"e have rebranded themselves as `the Justice Guild', and `put the gnolls to work'.
This alternative guild-branch call themselves `the Gnoll Guild', and use forced Gnoll labour to build new settlements.
Anyone joining either the Gnoll Guild or serving in the local \gls{guard} received a home in the area.
To emphasise their newfound freedom, they named the region `Liberty'.

The region has provided excellent logging, and the nearby forests have been tamed fairly quickly.
However, the danger lurks in the foliage of the deep forest.
As a result, all human settlements within the region are surrounded by towering wooden walls and manned with archers.

\subsubsection{The Coastal University}

A second branch of \gls{alchemists} has opened at the coast, devoted entirely to martial magic.
\Gls{king} has allowed the wizards to operate fairly independently, and with a lot of funding, in order to gather information on any potential attacks upon the Pebbles islands from the Southern Kingdom.
The Coastal University provides little training to people, so \gls{king} typically staffs it with the most loyal alchemists from \gls{college}, employing them to research until he calls them for special missions.

So far, they have been unable to create another portal.
The skills to do so are rare, coming along perhaps once in a generation of alchemists.
But once the Coastal University succeed in creating a portal, this alternative University may turn into the seventh wing of \gls{shatteredcastle}.

\subsection{Encounters}
Liberty's warm, humid, climate breeds constant danger, but also brings a lot of cheer on the road as plenty of traders travel across the roads constantly.

The \gls{guard} and Justice Guild patrol the areas between villages constantly, as Liberty receives a lot of financial aid from \gls{king} to ensure everyone remains safe, and expands.
Of course, the \glspl{guard} prefer patrolling the inner areas more than the deep forest, so once travellers take a few steps beyond a known path, and approach the \gls{edge}, they can expect to see a lot more danger, including gnolls who have a serious problem with any humans on their land.

Humanity's constant logging has upset the local ecosystem, and anyone in or around human settlements now experiences sudden floods during the warm and stormy seasons.

Heatwaves also present a major problem for most humans.
Of course, Liberty

\iftoggle{players}{
  \subsubsection{Character Concepts}
  \begin{itemize}

    \item
    A betrayed warrior who did his time in the \gls{guard} already, fighting against the gnolls, but lost all his rewards due to a paperwork error in \gls{shatteredcastle}.
    He intends to fight the bare minimum, and will take any resources he can from the \gls{guard} stash until he gets back what he owes.
    \item
    A deceptive gnoll who, having escaped the wars with the humans, has been given the task of infiltrating and reporting back to his tribe after some years served in the \gls{guard}.
    \item
    A wandering orphan, with nowhere to go after the forest ate the road to his village.
    He assumes they died, but who knows?
    Either way, without any cash, it's time to join the \gls{guard}.
    \item
    A loyal gnoll who bought into every piece of human propaganda, and managed to impress a captain of the \gls{guard} enough to buy his freedom.

  \end{itemize}
}{

  \begin{nametable}{General Tempo Chart}
    \textbf{Roll} & \textbf{Result} \\\hline
    6 & Encounter, and reroll! \\
    5 & Encounter. \\
    4 & Encounter. \\
    3 & Encounter. \\
    2 & 1 days peace. \\
    1 & 3 days peace. \\
  \end{nametable}

  \paragraph{Journeys}
  to the \gls{edge} add +2 to the left-hand die, while excursions into the forest add +3.
}

\subsubsection{Swamps}

These water-forests are a particular hazard within the region.
Any time travellers encounter one, they must stop, and go the long way around, or procure a raft or boat, and glide slowly through the tangled mess of tree-roots.
Any journey across one requires an Intelligence + Wyldcrafting roll, TN 10.
Failure indicates the travellers have become lost.

Any violent encounters in swamps can prove particularly dangerous, as the assaulted travellers will most likely have to defend themself from a raft.

\subsection{Mt Arthur}\index{Mt Arthur}

\iftoggle{players}{}{
  \encArthur
}

\widePic{Johan_Jaeger/mountain_river}

Mt Arthur is a thriving region where towns keep a wide ring of little villages between them and the \gls{edge}.
Dwarven settlements are dotted about the Southern mountains, acting as a peaceful area for both sides to negotiate and trade, and often imposing little taxes and tribute demands for the freedom to do so.

The temples to the gods transformed into guilds first in Mt Arthur, so the local population feel entirely accustomed to the divine monopolies.
V\'{e}r\"{e} is by far the most popular god within the region, and temples dot the land to him, producing contracts for marriage, business deals and often acting as court-houses when problems between the nobility flare.

The large cities often host blood-sports, sometimes involving people, and always involving captured beasts of some kind.
Crowds cheers and jeer as griffins, chitincrawlers, and umberhulks tear each other apart.
They cheer even more when they can see a member of the \gls{guard} doing so in front of them.
The organizers always cripple the animals in some way, which helps (but never ensures) that civilization wins the fight, and prompts the young people watching to sign up to the \gls{guard}, thinking that they might also kill a beast with the same ease, while earning some money.

\subsubsection{The Citadel}

Mt Arthur's wing of \gls{shatteredcastle} is the only wing placed inside a city - the capital Arthurseat.
It is also the most popular destination for traders, as Mt Arthur has the highest population of any of the realms.

Locals who wander outside often report smelling the sea or feeling a frosty chill, as a sudden breeze from another wing of \gls{shatteredcastle} weaves out the door.

The inner castle holds a complete barracks, along with multiple martial alchemists, all ready to defend the castle and their Rex.
Various enchanted items detect any magical items or disguises, and the castle reacts fiercely to non-mundane intruders.

\subsection{Encounters}

Mt Arthur has sprawling, quiet, villages, with wide circles of land, even around the \gls{edge}, but two steps beyond that and the forests turn fierce quickly.
However, the abundance of game, drives villagers to hunt, and the blood sports in towns drives townsfolk to `chance an arm' in the forest.

Despite the safety of the villages here, when the forest moves in, it moves in quickly.
Travellers regularly find entire villages abandoned after one too many chitincrawlers stand outside, visibly waiting (just outside of longbow range) for someone to come out.

\iftoggle{players}{

  \subsubsection{Character Concepts}
  \begin{itemize}

    \item
    An ambitious nobody.
    Optimism springs eternal, and the reason lies in complete mystery.
    Her strength is mediocre.
    Her intelligence is mundane.
    But her hope is uncrushable.
    \item
    An elven academic who ignored his innate magical abilities, preferring to study at \glsentrytext{college}.
    His excellent performance at the Guild of Paper concerned so many of the guild leaders that they made him a \glsentrytext{nomad} in the hopes of getting rid of him.
    \item
    A tax-dodging dwarf, forced into the \gls{guard} as part of his Queen's debt to \gls{king}.
    `Such is life'.

  \end{itemize}
}{

  \begin{nametable}{Mr Arthur Tempo Chart}
    \textbf{Roll} & \textbf{Result} \\\hline
    6 & Encounter, and reroll! \\
    5 & Encounter. \\
    4 & Encounter. \\
    3 & 1 day peace. \\
    2 & 2 days peace. \\
    1 & 4 days peace. \\
  \end{nametable}

  \paragraph{Journeys}
  to the \gls{edge} add +1 to the left-hand die, while excursions into the forest add +3.

}

\subsubsection{Storms}

Mt Arthur can suffer storms almost as bad as the Shale region.
During the stormy seasons lightning and massive amounts of hail are common.
The Kingsway mountains, which divide the two regions, are rife with volcanic activity.
During the stormy seasons, the sulphurous clouds spill over massive portions of the region, choking everyone in the area who does not hide indoors.

\widePic[t]{Tom_Prante/ancient_valley}

\iftoggle{players}{}{
  \encPebbles
}

\subsection{The Pebbles}
\index{Pebbles Islands}

These little islands had their own group of languages -- very different from any of the various human languages around the mainland of Fenestra.
They joined Fenestra only a few generations ago, and many still want their old independence.
However, the old independence meant independence \emph{of each island}, so removing the rule of \gls{king} would involve (at the very least) creating a single rule for all of the islands, which is precisely what the pro-independence movement do not want.

Despite the new governance, the Pebbles continue to trade with the \gls{outerKingdom}, despite the enmity from the recent wars.

They commonly worship Alass\"{e} and have some of the grandest temples build in her honour, full to the brim with beer and joyful songs.
Many a village here hosts a single `village gnome' -- typically an alchemist who teaches the young people history and sometimes even a little magic.
Local gnomish communities consider it good practice to create ties to humans, though they do not take anyone back to their own homes, claiming gnomish warrens would be `too short by half' for humans to visit.

\subsubsection{Reef Wing}

\Gls{king}'s personal wing of \gls{shatteredcastle} sits in the main island of the Pebbles.
Any time \gls{king} feels in danger, the powerful doors which lead to this wing's portal close, and guards surround his living quarters.
From the outside, this wing appears relatively small.
Two stories high, it sits on a little island, and does not permit any boats to approach.
Watchmen patrol the ramparts all day and night, including various dwarves, employed for their superior eyesight at night.

\iftoggle{players}{
  \subsubsection{Character Concepts}
  \begin{itemize}

  \item
  A seafaring dwarf who loved nothing more than feeling the wind in his beard, but everyone has to grow up sometime.
  Fearing boredom, he decided to sign up to the \gls{guard} and make his fortune.
  Perhaps one day he'll save enough to buy an island.
  \item
  A scarred pirate who gained his wounds through an innocent misunderstanding.
  All rumours about the men he murdered are false.
  `Really they are'.
  \item
  An excellent cook who thought they'd let her stay in the barracks after training in the guard, and just let her earn a wage, quietly.
  Perhaps she can spice up the field rations.

  \end{itemize}
}{
  \subsection{Encounters}
  The people in the Pebbles always say, `when it rains, it pours', and indeed problems tend to come together in the Pebbles.
  Much of this stems from the poor weather upsetting and pushing other factors about.
  Storms can bring rain, or an opportunity for pirates to strike on the seas while a boat seems distracted mending its sails.

  \begin{nametable}{General Tempo Chart}
    \textbf{Roll} & \textbf{Result} \\\hline
    6 & Encounter, and reroll! \\
    6 & Encounter, and reroll! \\
    4 & Encounter. \\
    3 & 1 day peace. \\
    2 & 3 days peace. \\
    1 & 6 days peace. \\
  \end{nametable}


  \paragraph{Journeys}
  out to sea add +4 to the left-hand die.
  The potential encounters here change significantly.

}

\iftoggle{players}{}{
  \encQuennome
}

\subsection{Quennome}\index{Quennome}

\widePic{Tom_Prante/the_old_path}

Throughout the Quennome forests, connected trees fashioned into houses form subtle living spaces.
Under the ground, little elvish communities spring up here and there.
Mostly, the forest's base is reserved for creatures rather than people, and it is easy to understand why after one sees the kinds of creatures which roam there.
Basilisks, griffins, woodspies and sometimes nura wander the landscape, looking for food.

Nestled within Quennome's forests are little human towns.
Perhaps in imitation of the elves they build their houses slightly below the ground so that a thatched roof on top of two feet of brick wall is all that can be seen.
The people are adept at rallying round and defending their villages from attacks by larger beasts.
Archery with the long bow is popular for just this purpose and Quennome boasts the best archers in Fenestra.

The monstrous beasts in the area often like to eat humans when they are walking in smaller numbers, so they sit in waiting around human paths.
As a result, people change how they get from one place to the other very often.
Faint paths appear, are forgotten and quickly vanish.

Various temples to Laiqu\"{e} are erected with stone or wood, and statues created in honour of all the most terrifying beasts of the forest.  It is hoped that if they are treated with respect then they will leave people alone.
The elves of the region find this habit ridiculous.
The human priests maintain that this practice instils a certain level of respect for the forest in people, which indirectly helps them to survive.

\subsection{Encounters}

Quennome brings unending stimulation to its inhabitants.
The humans survive through eternal vigilance, and high walls.
Rivers -- the standard means of travel through Quennome -- bring serious dangerous, but allow for fast travel (at least down river).

When the inhabitants find a lone boat, floating down river, they assume the occupant was grabbed from below by a woodspy.
People protect themselves through various means, from standard defences like a dagger to stab at any tentacles which reach up from the water, to herbal remedies thrown into the water, can drive the woodspies away.

For most, a journey into the forest approaches suicide, though the older elves and some priests understand enough Aldaron to ensure they can enchant any wild beasts which might otherwise attack them.

\iftoggle{players}{

  \subsubsection{Character Concepts}
  \begin{itemize}

    \item
    A forlorn elf, who wishes she had more of a tight-knit family structure, like humans.
    Elves just wander away, without even saying `goodbye', and why didn't her parents take better care of her?
    It's time to make a proper family, or at least some friends, at least before they die, like all humans do before long.
    \item
    An impatient sorcerer, who does not want to slowly watch his powers grow over the decades.
    With only a few years spent in battle, he will return a force to be reckoned with!
    \ldots but should probably keep his powers hidden, as the \gls{guard} don't really approve of that sort of thing.
    \item
    A curious alchemist who wants to know more.
    He may just be a little gnome, but he plans to hide behind the `big folk' while they fight things, and he should return home with all manner of stories, insights, and maybe a few maps.

  \end{itemize}
}{

  \begin{nametable}{General Tempo Chart}
    \textbf{Roll} & \textbf{Result} \\\hline
    6 & Encounter, and reroll! \\
    5 & Encounter. \\
    4 & Encounter. \\
    3 & 1 day peace. \\
    2 & 2 days peace. \\
    1 & 4 days peace. \\
  \end{nametable}

  \paragraph{Journeys}
  on the river add +1 to the left-hand die, while excursions into the forest add +2.

}

\subsubsection{Rivers and Lakes}

Boating is undoubtedly the safest route through Quennome, so leaving the area by going downstream to either Shale or Mt Arthur is fast and reliable, so long as one knows which rivers to avoid.

\subsubsection{The Wooden Wing}

The only wooden segment of \gls{shatteredcastle} sits in a clearing.
Various political tiffs have emerged after \gls{king} demanded some form of tribute from local elves.
The result is a perpetual stalemate, with the elves not venturing close to \gls{shatteredcastle}, and the king's army not daring to invade elvish territory.
The closer a human lives to the Wooden Wing of \gls{shatteredcastle}, the more taxes they will pay to the king.
Those farther will pay fewer taxes, as fewer tax collectors have the nerve to travel that far out.

\subsection{Shale}\index{Shale}

\widePic{Tom_Prante/autumn}

\iftoggle{players}{}{
  \encShale
}

\index{Bearded Mountains}
The Shale is riddled with dwarvish tunnels and occasional little gnomish communities closer to the surface, earning the area the nickname `the Bearded Mountains'.
It is said that you can travel from anywhere to anywhere in this region via underground passages, if you have a gnomish guide.
Travelling without such a guide is, of course, likely to ensure a nasty encounter with an umber hulk or acidic ooze.
Or more likely, someone entering the wrong passage can find any number of the standard cave hazards.
Slippery mosses can shoot people down a dead-end, with little hope of return; mushroom spores can induce hallucinations; cave-ins can block passages.

Farther from the mountains and closer to the sea, seafaring humans live and trade with the Pebbles islands.

The bearded mountains are famed for their terrible storms.
During the stormy seasons -- Qualmea and Otsea -- fishermen do not go out to sea.
Worst of all, the mountains often explode, sending lava flowing downhill and great clouds of smoke and ash into the air.

\subsubsection{Underground Living}

The tunnels winding their way around the deeps can provide a relatively safe passage from place to place, though one must know which road to take.
This maze of underground forks connects gnomish warrens and dwarvish fortresses.

The dwarves extensively study which tunnels fill with magma, and eagerly await dormant tunnels where they can build new homes where the magma has subsided.
Humans in the area often herd their animals into barns or their own houses, and then put themselves into small underground bunkers to await the passing of the storm.

Dwarvish citadels here are massive, and do not suffer from the storms above.
Even when flooding occurs during storms, the drainage is adequate, and water never dampens dwarven shoes.

Gnomish warrens fare less well, as gnomes typically try to plan far in advance, but only act when something is happening.

Oozes -- they keystone species of all underground areas -- inhabit areas by water, and use their membranous bodies to extract nutrients from above.
Every so often a corpse will fall into some underground river, providing a feast, and allowing the ooze to experience a growth spurt.

Most mushrooms can grow around oozes, feeding from them, without suffering from their acidic touch.
Once an area has produced too many mushrooms for an ooze to survive comfortably, it goes searching for a different water source.

\iftoggle{players}{
  \subsubsection{Character Concepts}
  \begin{itemize}

    \item
    A lovesick dwarf who could not win the heart of his princess.
    He intends to bring her a collection of monstrous heads to build a throne so high that her head will touch the ceiling in the feasting halls of the Shale.

    Of course her chief loves are logistics and gardening, but it's the thought that counts.
    \item
    A suspicious child (in fact a clean-shaven gnome), who never speaks about the true extent of his abilities, and joined the guard as an emissary of the Ale Guild with forged paperwork.
    \item
    Guard number 46.
    Just a random guy.
    `What? Nothing to see here'.

  \end{itemize}
}{
  \subsection{Encounters}
  Shale's wide forests, flat planes, and long, deep, tunnels all bring equal chances of someone, or something, colliding with anyone wandering the land.

  \begin{nametable}{General Tempo Chart}
    \textbf{Roll} & \textbf{Result} \\\hline
    6 & Encounter, and reroll! \\
    5 & Encounter. \\
    4 & Encounter. \\
    3 & 1 day peace. \\
    2 & 2 days peace. \\
    1 & 3 days peace. \\
  \end{nametable}

  \paragraph{Journeys}
  To to the forest add -2 to the left-hand die, the underground \emph{add} +2, and venturing into the \gls{deep} adds +4.
  The \textit{Forest} adds -2 to the left-hand die, \textit{Underground} adds +4, and the \textit{Deep} adds +8.
}

\subsubsection{The Shattered Dungeon Wing}

\Gls{king}'s castle sits high on a mountain's side, viewing the sea.
In the far distance, on a clear day, it is possible to see the segment of \gls{shatteredcastle} in the Pebbles.
The castle's underground sections provides the main point of trade to various dwarvish realms, bringing food and wood.
This is the main stranglehold \gls{king} has over the dwarves, which then forces them to pay taxes on all goods sold.

\widePic[t]{Tom_Prante/inaok}

\iftoggle{players}{}{
  \encWhiteplains
}

\subsection{Whiteplains}\index{Whiteplains}

Small villages and rare towns dot this snowy expanse.  The land has grown unstable in recent years as the crown has killed all nobles in the area and left the people with a few mousey official bureaucrats who are in charge of collecting taxes and meting out punishments.

In a desolate and forgotten region of Whiteplains is a massive structure without any footpaths leading to it or from it.
It is built in a dome shape so that snow covers as much as possible.
It has no doors or windows on the outside except one -- a single door at the top goes to a walkway so guards can see attacks coming from the distance.

Inside this monolithic white dome wind stone corridors.
Some corridors have doors, and some doors have locks, but it seems like there is no plan or pattern to any of it.
Some are wide enough for a wagon, others are so narrow that the average human has trouble fitting through.
Various hallways exit to magical portals, created through alchemy.
One exits to a wing in Mt Arthur, another goes out to Quennome.
All in all each region has one portal which leads to the Heart of \gls{shatteredcastle}, and from the heart one can travel to any other.
And behind one, lonesome and secret, locked and guarded door, sits the portal to the Pebbles Wing where \gls{king} currently resides.

There are other rumours concerning the heart, of secret rooms with locked doors, with gates which lead to other worlds.  Perhaps these are places so far from Fenestra that they appear strange.  Some think that these doorways lead to Ainumar -- the great celestial orb in the sky where it is said that the gods live.

Far in the North, where the snow never melts, caves of gnolls fight and occasionally journey South to raid human villages.
They have been quieter in the years since the great war in Whiteplains, when the gnolls of the North coordinated their attacks with those of Liberty.

\subsubsection{The Castle's Heart}\label{whiteland_heart}

Every portal within \gls{shatteredcastle} leads to the heart.
Inside, a long mess of cold, sunless corridors lead to various locations.
Some rooms lie empty, some are traps, and a few host on-duty guards.
Travellers wishing to trade must all move through this area, accompanied by a guard, to make sure they reach the portal they paid to use.

The exact location of the Heart is unknown, even to those who work there, as no doors lead outside.
Those wishing to travel to Whiteplains must go by foot.

Various rumours exist of secret portals within the Heart which lead to other locations, such \gls{outerKingdom}, or strange lands beyond.

\iftoggle{players}{
  \begin{itemize}

    \item
    A lost noble, who once stood to inherit an estate until \gls{king} put all the nobility to death.
    So far none of the guard have noticed her accent.
    So far\ldots
    \item
    A loyal subject of \gls{king}, intent on proving his worth in the glorious \gls{guard}.
    With enough hard work, anyone can live a worthy life.
    \item
    A thieving orphan, who was caught stealing far too much.
    He escaped the noose by convincing his local village master that he would do honest work in the \gls{guard}.

  \end{itemize}
}{
  \subsection{Encounters}

  Whiteplains has little civilization, or interruption to travellers.
  For the most part, it is a barren wasteland, populated by sparse herds of aurochs, and the undead.

  \begin{nametable}{General Tempo Chart}
    \textbf{Roll} & \textbf{Result} \\\hline
    6 & Encounter, and reroll! \\
    5 & Encounter. \\
    4 & 1 day peace. \\
    3 & 2 days peace. \\
    2 & 4 days peace. \\
    1 & 8 days peace. \\
  \end{nametable}

  \paragraph{Journeys}
  to the \gls{edge} add +1 to the left-hand die, while excursions into the forest add +2.

}

\subsubsection{The \Glsentrytext{guard} Weathering Halls}

New guards always train in Whiteplains.
They consider this their basic right of passage.
Most never venture too far from populated areas, but rumours abound among the barracks that if one steps too far off the snowy path, portals, and strange magical landscapes await.
For more information, see \autoref{guard}.

\end{multicols}

\vfill\null
