\section{Descent into the Darkness}

\begin{multicols}{2}

\mapentry[start]{Fresh Hall}

Ask each of the \glspl{pc} to describe what their characters are doing.
If none of them mention holding a light, then none of their characters have a light source.

\begin{boxtext}
  None of the characters were holding a light.
  Where did you leave the torch?
  It must have dropped before the last tunnel.
\end{boxtext}

\noindent
The troupe will almost certainly want to stop and rest here for \pgls{interval}.
Once a light has been organized, the troupe can look around and take a rest.

\begin{boxtext}
  This fresh hall has an ambient drip of water, and moisture everywhere.
  The ceiling slopes down gently, surrounding the entire passage with a ring of low-hanging darkness.
  Large, damp, rocks litter the ground in every direction.
\end{boxtext}

\paragraph{The way out}
goes down, sharply.

The troupe should roll \roll{Dexterity}{Caving} (\tn[10]).
A tie means the first person can drop an item instead of taking the fall.
Falling down the passage inflicts $1D6-1$ Damage (a helmet provides a maximum of \gls{dr} 1, no other armour counts).

\begin{boxtext}
  Ahead, the tunnel stops sloping, and continues, flat.
  A wide shadow indicates a hole in the ground.
\end{boxtext}

\mapentry[firstBridge]{First Bridge}

\begin{exampletext}
  The human prospectors (who looked for a good place to mine) built two bridges.
  Over time, dust, rocks, and other debris covered the bridge entirely.
  Then the wood degraded, leaving a natural trap.

  The first group of \glspl{guard} to descend ran across the bridge, so half of it crumbled, sending them falling to the pit of sharp rocks below.
  One died, and his body, and equipment, were left down here to rot.
\end{exampletext}

\paragraph{If the troupe inspect the hole,}
they will see the ground here is degraded wood.

\paragraph{If the troupe go across the wood,}
anyone with a \gls{weight} of 9 or more (including equipment) will make the remains of the bridge collapse.

\paragraph{Jumping over the bridge,}
requires a \roll{Speed}{Athletics} roll (\tn[9]).
A tie means the bridge collapses, but the character has crossed safely.

\paragraph{Falling into the hole}
inflicts $1D6+2$ Damage.

\paragraph{Entering the hole}
requires an \roll{Intelligence}{Caving} roll (\tn[10]), to understand a safe route down.
Ascending requires \tn[12].

The players can make this a group roll, by having their characters discuss the safest route down, if they are prepared to spend \pgls{interval} on the discussion.

\begin{boxtext}
  The ground stops suddenly, as you reach a cliff.
  At the side, a thin rope wraps around a boulder, and droops over the cliff.
\end{boxtext}

\mapentry[ropeDown]{Rope \& Cliff}

This looks like a trap, but it isn't.
The goblins need a rope to climb down, but they can't take it down with them.

\begin{boxtext}
  The torchlight flies out, and disappears.
  The cave goes on, but you can't see how far.
  Then the echoes of your feet start to return, but seem to take an impossibly long time to return, and much longer to stop.
\end{boxtext}

\paragraph{If the rope takes a combined \gls{weight} of 11 or more,}
the rope will pull the boulder down on their head.

\paragraph{If the rope takes a combined \gls{weight} of 14 or more,}
(presumably due to multiple characters climbing down) it will snap.

\mapentry[hungryHall]{Hungry Hall}

This chamber could fit an entire \glsentrytext{village} inside it.
The troupe can go anywhere, in any direction, but they will find only damp rocks in the darkness unless they search hard.

\sidebox{
  \begin{boxtable}
  \textbf{\glsentrytext{tn}} & \textbf{Result} \\
  \hline
  13  & exit to \gls{area} \ref{secondBridge} (`\nameref{secondBridge}') \\
  10  & exit to \gls{area} \ref{murderHoles} (`\nameref{murderHoles}') \\
  \end{boxtable}
}

The players should feel aware of their lighting choices.
Keeping a lit torch means that they can see danger coming, at least a little way off, but it also means that anything in the darkness can see them, while they cannot see it.

The darkness watches them.

\paragraph{Finding the exits}
requires a \roll{Speed}{Caving}.
The troupe can use teamwork%
\exRef{core}{Core Rules}{teamwork}
only if they split up.

They can also make this \pgls{restingaction}%
\exRef{core}{Core Rules}{restingactions}
if they are prepared to spend \pgls{interval} doing it carefully.

\mapentry[gentlePassage]{Gentle Passage}

This path, like the entrance has had a little cave-in, and will have more if anyone large walks through it.
The troupe can make a \roll{Dexterity}{Stealth} roll, with a \gls{tn} equal to their total weight.%
\footnote{The total \gls{weight} equals the character with the most \glspl{hp}, plus half of the second-highest \glspl{hp}, a quarter of the third, and so on.
\iftoggle{core}{See the \textit{Core Rules}, \autopageref{stacking} for a rundown.}{}}

\begin{boxtext}
  The path widens, and you can comfortably walk, without ducking.
  It the wet cavern walls stand apart, allowing everyone to walk together, with two or three side-by-side.
  Little dry stones litter everywhere.
\end{boxtext}

\caveIn

Just like the entrance, if the \glspl{pc} don't spot the danger, the ceiling will collapse on a 1 in 6 chance, which increases every time something falls.
However, this time the cavern stretches across 60 steps.

Some characters may reach the other side before this passage collapses.
If the troupe becomes split, focus on the characters with the longest journey ahead, and resolve their actions as quickly as you can, then check the remaining characters.
This will require some careful timekeeping.

\mapentry[forkLeaf]{Fork \& Leaf}

\mapentry[gas]{Gas Chamber}

The passage here leads upwards, to a chamber of flammable gas.

The party can make a \roll{Wits}{Caving} roll (\tn[13]) to notice the smell before their torches ignite it, dealing $2D6$ Damage to the first half of the characters, and $1D6$ to the second half.

\mapentry[mudSlide]{Mud Slide}

The goblins carried some mud here, so that they could leave large, obvious, footprints, upwards to the cavern of the umber hulk (room \ref{umberHulk}).

\mapentry[murderHoles]{Murder Holes}

\mapentry[secondBridge]{Second Bridge}

The prospective miners built another bridge here.
The troupe can make a \roll{Wits}{Caving} roll to notice the rickety bridge (\tn[10]).
Any attempt to note further broken bridges should give the troupe a good Bonus.

If the troupe move over the bridge, the first character with a \gls{weight} over 6 collapses the bridge.
That character, and anyone ahead or behind them in the marching order, falls down the hole which the bridge stretched across, and receives $2D6$ Damage.

\paragraph{Entering the hole}
requires an \roll{Intelligence}{Caving} roll (\tn[10]), to understand a safe route down.
Ascending requires \tn[12].

\paragraph{Jumping over the bridge,}
requires a \roll{Speed}{Athletics} roll (\tn[9]).
A tie means the bridge collapses, but the character has crossed safely.

\mapentry[waterMaze]{Puddle of Doom}

This little passage flooded recently.
The goblins remember which way to go, but the \glspl{pc} have a serious problem on their hands.
Ahead, they will see only a puddle, and while they can easily find that it covers a passage, mapping that passage will prove difficult.
In fact, some players may take a while to notice that they will have to map the passage, in one way or another.

If multiple \glspl{pc} enter, this spells trouble.
The passage only has enough space for one person, so if they run into trouble, they cannot back up without bumping into everyone behind them.

Whoever enters the frigid waters instantly gains \pgls{fatigue}, and the exertion of swimming (as usual) inflicts another.

The scene might go like this:

\begin{itemize}
  \bf
  \item
  `Okay, I'll jump in the puddle.'
  \begin{itemize}
    \it
    \item
    `Do you take your backpack off?'
  \end{itemize}
  \item
  `Yes.
  What do I see?'
  \begin{itemize}
    \it
    \item
    `The water has a metallic cold, which pulls heat from your skin.'
    \item
    `You feel the hard, slippery rocks around, and have to descend a little before finding a passage.
    This seems to be the only passage.
    It's about the height of a crouching child.'
  \end{itemize}
  `Okay, so I'll go through, carefully.'
  \begin{itemize}
    \item
    \it
    `Going carefully, you have 3~\glspl{ap}, so you cover 3 steps this round.'
    \item
    `Put down a temporary \gls{fatigue} (you can remove it once you can breathe again).'
    \item
    `Swimming onward, you cover another 3~\glspl{step} and find a side tunnel on the left.'
    \item
    `Above, the tunnel opens upwards.'
    \item
    `Put down another \gls{fatigue}'
    \item
    `Which way do you go?'
  \end{itemize}
  `I'll move up.'
  \begin{itemize}
    \it
    \item
    `You find a sharp, spiky ceiling as you swim up and round.
    You think the tunnel moves horizontally from here, but you can't be sure of which way is up or down in the freezing, dark waters.'
    \item
    `Take another \gls{fatigue}.'
    \item
    `Do you continue?'
  \end{itemize}
  \item
  `So if I have 3 slots without \pgls{fatigue}, that means I can go another 3 rounds\ldots so then how far have I come?'
  \begin{itemize}
    \item
    \it
    `You're unsure.'
    \item
    `Take \pgls{fatigue}.'
  \end{itemize}
  \item
  `Again?
  Okay, I'm going back!'
\end{itemize}

One out of the water and rested, the character can remove all \glspl{fatigue} from the water except 2 -- they receive \pgls{fatigue} for exertion, and another from the freezing-cold water.

This process might take a while.
The player will have to decide who goes next, and plan their routes ahead.
Each one will enter blind, and have to remember where they should scout next.

Remember to ask about the backpacks every time a character enters.
Their rations almost certainly do not have waterproof coverings, nor do their tinder boxes.
The torches, however, will still light fine, even after becoming wet.
However, the water in the torch's wood will mean that torch creates a constant hiss from the water turning to steam.
This will prevent the troupe moving silently while they carry those torches.

\paragraph{For some extra tension,}
let a character enter the water and then focus on the other characters.

\begin{speechtext}
  Okay, Ratcull entered the water.
  Everyone else waits for some time, and the water feels quiet.
  What do you do now?
\end{speechtext}

Let the troupe decide how long they will wait, in the darkness, before someone else enters.
Of course, once you resolve the first character's actions (the one in the black waters), you may find that the troupe did not in fact wait for an hour.

\mapentry[darkTunnel]{Dark Tunnel}

This narrow tunnel stretches a long way -- around a mile.
The troupe can walk down it single-file, occasionally crouching.

However, if they have even a single torch lit, the oxygen in the tunnel will deplete, causing hypoxia.%
\exRef{core}{Core Rules}{hypoxia}
This will leave the party with a real problem -- tiredness, and hallucinations can follow.
Since the \glspl{fatigue} gained from hypoxia only last until the characters can breathe again, you should simply raise the \gls{tn} for all rolls secretly, instead of handing out the \glspl{fatigue} to players.

Soon after, the troop may begin to hallucinate figures in the darkness.

\begin{boxtext}
  You stop for a small breather, then notice little goblinoid silhouettes up ahead or behind.
\end{boxtext}

That little verbal slip-up -- `ahead or behind' -- should be given with intention.
The troupe will have trouble remembering which direction they came from and which they were going to.
If they ask about the odd phrasing, tell them what they're having trouble remembering; otherwise, simply wait until the troupe have dealt with the imaginary goblins, and decide to continue moving.

\mapentry[wormTunnel]{Worm Tunnel}

This tunnel has around a 1 metre diameter, and then becomes progressively smaller.
It stretches a full twenty paces, so torch light will not show the exit.

Players should roll \roll{Dexterity}{Caving}, at \gls{tn} 7 plus double their Strength score, plus 2 for wearing armour.

\paragraph{Rolling a tie}
means that the character takes a long time, and gains 2 \glspl{fatigue} from all the shuffling and scrapes.

\paragraph{Rolling a failure}
means the character has become stuck, has gained 2~\glspl{fatigue}, and will not get out without someone helping to pull them from the other side, with a rope.

Setting up a rope to help them requires another character to make their roll for them again, with +1 to the \glsentrylong{tn}.

\mapentry[umberHulk]{Umber Hulk}

Once the party are up, they steal some eggs, crawl down a small hole, and singing.

\begin{speechtext}
  Leviathan, beetle king!

  \noindent
  Hunting dun, eggies bring!
\end{speechtext}

\mapentry[blindFish]{Blind Fish}

This long, lazy river has a wicked undercurrent.
Swimming at the entrance demands a \roll{Strength}{Seafaring} roll (\tn[6]).
The \gls{tn} increases rapidly, until it reaches 12 at the exit, where the water goes underground.

\paragraph{Rolling a tie}
means the character can make no progress -- they simply fight against the river's undercurrent, and gain \pgls{fatigue}.

\paragraph{Rolling a failure}
means the river pulls the character downriver, the \gls{tn} increases by +2, and the character gains \pgls{fatigue} then rolls again.

\paragraph{Examining the river}
reveals little fish, about the size of a finger.

Any reasonable plan to catch them might work, although making a fire underground safely can prove very difficult.%
\exRef{core}{Core Rules}{place_fire}

\mapentry[smokePassage]{Smokey Passage}

The goblins light a fire at the bottom of this long, sloping passage.

\mapentry[stallagmites]{Stalagmites Point}

\Pgls{guard} corpse hangs here on ropes.
Goblins above jiggle the ropes, using him like a puppet.
The players will almost certainly think that this creature is undead.

He holds a map of the cave, created as the previous group came down.

\mapentry[mushrooms]{A Short Cut to Mushrooms}

Glowshrooms live here.

\mapentry[basiliskCave]{Basilisk Cave}

\ifnum\value{temperature}=0
  Three basilisks have curled up here to hibernate.
  The stench signals to everyone around that the creatures lie nearby, and every breath when wandering near this alcove will inflict 3~\glspl{fatigue}.

  Awakening the basilisks will almost certainly prove fatal.

  \basilisk[\npc{\T[3]\A}{\arabic{noAppearing} Basilisks}]
\else
\fi

\mapentry[sunRoof]{Sun Roof}

This semi-false exit floods sunlight into this area.
Children above look down, but the PCs look like monsters, so the children scream and run away.

\mapentry[crackExit]{Crack to Exit}

\mapentry[windyPassage]{Windy Passage}


\end{multicols}
