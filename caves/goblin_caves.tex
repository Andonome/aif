\section{The Guts of the Mountain}

\begin{multicols}{2}

\renewcommand\npcsymbol{\glsentrysymbol{yonder}}
\smolMapPic{cave_2}{
  \ref{falseExit}/85/10,
}

\mapentry[start]{Fresh Hall}
\begin{boxtext}
  This fresh hall has an ambient drip of water, and moisture everywhere.
  Large, damp, rocks on the ground demand walking carefully, while feeling the low, slanted, ceiling above.
\end{boxtext}

The troupe will almost certainly want to stop and rest here for \pgls{interval}.
The air smells dusty, but fresh enough to rest.

\wayOut{A path winds slowly up half a mile, to}{falseExit}
\wayOut{Another leads 1 mile to}{firstBridge}

\mapentry[falseExit]{False Exit}

The path travels upwards until opening on the mountain's side.
Unfortunately, the path also exits onto the side of a cliff, so the troupe cannot simply walk down.

\act{wrecklessly climb down}{Dexterity}{Athletics}{14}%
\footnote{If a player insists on attempting this check, you should roll for them, and tell them if they succeed or fail.  The other players will only find out if their characters can climb down later -- they should not know their chances of success by looking at the \gls{natural}.}
Failure means death.

\boxPair[t]{
  \smolMapPic{cave_3}{
    {\normalsize\roll{Wits}{Caving} (8/ 11)}/17/92,
    {\normalsize\glsfmttext{weight}~25: collapse}/57/75,
    \rotatebox{-21}{\normalsize\roll{Intelligence}{Caving} (7)}/71/21,
    \ref{firstBridge}/49/57,
    \ref{gentlePassage}/92/05,
  }
}{
  \smolMapPic{cave_4}{
    {\large\glsfmttext{weight}~30: fall}/71/80,
    \rotatebox{-90}{\large\glsfmttext{weight}~40: snap!}/20/50,
    \ref{ropeDown}/42/92,
  }
}

\mapentry[firstBridge]{First Bridge}

\playCommentaryAftermath[b]

If the troupe walk with a light, they instantly see the hole in the ground (but their torch is probably about to go out at this point).%
\footnote{The human miners built two bridges while exploring further into the mountain.
Over time, dust, rocks, and other debris covered this first bridge entirely.
Then the wood degraded, leaving a natural trap.}
Otherwise, they should roll \roll{Wits}{Caving} at \tn[8].

\paragraph{If the troupe go across the wood,}
anyone with \pgls{weight} of 25 or more (including their \gls{equipment}) will make the remains of the bridge collapse.

\act{jump over the bridge}{Speed}{Athletics}{9}
A tie means the character has crossed safely, but the bridge collapses.

Anyone falling into the hole receives $1D6$ Damage plus their own Strength Bonus (bigger creatures suffer more from falling).

\label{caveCoinsI}
A dry human corpse lies inside the pit.%
\footnote{When first group of \glspl{guard} came through here, the bridge collapsed, sending one down to a pit of sharp rocks below.
He died instantly, and the group moved on.}
His satchel still contains two bees-wax candles which can burn for an hour each, a tinder-box, \lootMedium, and slightly-expired \rations.

\setcounter{diceNo}{0}%
\renewcommand\npcsymbol{\currency}%

\wayOut{Some steps past the bridge lies}{ropeDown}
\wayOut{Inside the hole under the bridge, a crack reveals}{gentlePassage}

\act{gauge a safe route down}{Intelligence}{Caving}{7}
Once \pgls{pc} explains the hand-holds and where to place the legs, anyone can descend safely.
Failure means falling, and $1D6$~Damage plus the \gls{pc}'s Strength~Bonus.

\mapentry[ropeDown]{Rope \& Cliff}

\begin{boxtext}
  The small echoes from your footstep change tone gradually.
  You feel a large boulder in front, and as you navigate around it by touch, you find a rope on top.
  The rope goes around the boulder like the bow on a birthday present, then hangs down heavily, past the floor.
\end{boxtext}

The rope is 15~\glspl{step} long, and the cliff 10~\glspl{step} down.%
\exRef{core}{Core Rules}{falling}
The boulder can take up to \pgls{weight} of 30 before it falls, and the rope can take up to \pgls{weight} of 40 before it snaps (this includes items, as usual).%
\footnote{This may look like a trap, but it isn't; the goblins simply use the rope to climb up and down.}

If the \glspl{pc} have \pgls{torch} lit, they see the walls glisten faintly, due to the silver in this section.

\wayOut{The long, glinting, cavern, goes a mile to}{secondBridge}

\mapPic{b}{cave_5}{
  {\normalsize\glsfmttext{weight}~30: collapse}/67/24,
  \ref{secondBridge}/80/40,
  \ref{cliffDive}/60/85,
  {\normalsize\roll{Speed}{Athletics}}/65/94,
  {\normalsize Dive (6)}/28/45,
  {\normalsize Climb (10)}/43/70,
  \ref{wormTunnel}/58/02,
  \ref{standingTunnel}/00/05,
}

\mapentry[secondBridge]{Last Bridge}

\begin{boxtext}
  The sound of gently moving water comes from ahead, and its change in tone suggests the passage widens.
  As the water-hum sound seems right in front of you, the ground now feels like wood.
  Planks of wood stretch out, over the water.
\end{boxtext}

Anyone stepping on the bridge with a total \gls{weight} of 30 or more will break it.%
\footnote{The second old mining bridge has degraded almost as much as the first, so the goblins walk across carefully, and one at a time.}

\paragraph{Falling into the river's frigid water}
inflicts \pgls{ep}
\ifnum\value{temperature}<3%
  but anyone can swim across easily.
\else%
  and may push the characters downstream if they can't swim against the currents (\roll{Speed}{Athletics}, \tn[5]).
\fi%

\needspace{6\baselineskip}
\wayOut{Following the river inflicts another \gls{ep} and quickly leads to the banks of}{waterMaze}
\wayOut{Three \glspl{step} later, a small pile of stones and the stench of goblin shit marks the start of}{standingTunnel}
\wayOut{The slender crack leads immediately to}{wormTunnel}

\begin{boxtext}
  One step off the bridge, your left hand finds the stone wall continues.
  The smell of goblin-shit haunts your nose from the tunnel ahead, then your left hand touches a crack in the wall, the size of a window.
  It goes in deep.
\end{boxtext}

\mapentry[standingTunnel]{Standing Tunnel}

\begin{boxtext}
  Two steps in, you kick a pebble, and it clatters to the side with an interesting sound --- a chorus of little rocks shifting just to the right.
\end{boxtext}

If the character investigates the little rocks, any human will know the shape of a cairn (even a very small one).
The wide rock on top has the word `Happy' carved on its flat face.
Underneath, the skeletal remains of a canary.
\footnote{Miners use small, sensitive, birds to detect lethal gas.  If the bird dies, it's time to leave.
They returned to look for another route, and brought a box to bury `Happy' properly.}

\paragraph{If the troupe have \pgls{torch} lit}
they see a human footprint in the goblin shit (the last group of \glspl{guard} came down here).

\paragraph{Further into the passage}
the ground slops gently downwards.
\Pgls{pc} might notice with a \roll{Wits}{Caving} roll at \tn[8].

\begin{boxtext}
  You step on something soft.
  You feel a little hand, and a head with long ears, as cold as the stone floor beneath it.
\end{boxtext}

\paragraph{Carbon monoxide}
lays in the low parts of the tunnel, around waist-height to a human.
It won't do any harm to walking humans, but if a gnome or dwarf enter, they will start to feel sleepy, then die.
The same goes for anyone sitting down to rest.

The first sign of carbon monoxide poisoning is usually death, but in this case the first sign is a dead goblin (10~\glspl{step} down the passage).
It died weeks ago, but has no marks -- no cuts or bruises -- and the corpse has not decayed much, due to the infertile conditions in the cave.

\paragraph{Halfway down the passage}
a dead gnome lies face-up, with only his black leather \gls{guard} \gls{armour}.
His companions picked up his items before moving on, except a bag of shells and pebbles.%
\footnote{This portable \gls{shellGame} has excellent pieces, and any reasonable gnome will a dozen \glsentrylongpl{sp} for it.}

\paragraph{If the players fail to heed the warning signs,}
\glspl{pc} familiar with caves might get a strange feeling.
The troupe can roll \roll{Intelligence}{Caving} (\tn[13]) to at least get the impression that they should run, not walk.

Failure means gnomes, dwarves, and anyone who rests will accrue invisible \glspl{ep} --- you should note the \glspl{ep} without telling them, just apply any Penalty secretly by adding it to every \gls{tn}.
Soon after the \glsentrylongpl{ep} begin, the character dies, suddenly, and quietly.
Then anyone kneeling down to inspect them dies too.

\wayOut{After 2 miles, the tunnel meets}{waterMaze}

Before the troupe exit, they hear
\ifnum\thetemperature>2%
  the gentle noise of the river, and
\fi%
`\textit{splt-splt-splt-SPLOOSH}'
as the goblin guarding \nameref{wormTunnel} flees into \nameref{waterMaze}.

\playCommentaryPuddle[t]

\mapentry[wormTunnel]{Worm Tunnel}

The goblins cannot traverse the \nameref{standingTunnel}, so they must crawl through this passage.
The \glspl{pc} will probably have a difficult time here, and the goblins know it.
The narrow tunnel extends a full mile before the troupe can stretch their arms again, and it counts as two miles due to the cramped conditions.

\act[Characters entering]{squeeze through}{Dexterity}{Caving}{\roll[dr]{Strength}{5}}
Characters with Strength~-1 roll at \tn[4], while those with Strength~+2 and armour with \gls{dr}~3 roll at \tn[10].
On a failure, they become stuck half way through the tunnel, need an extra \gls{interval} to reach the other side (or to return).
On a tie, they immediately recognize they will become stuck and can decide to enter, or not.

\mapPic{t}{cave_6}{
  {\normalsize\roll{Intelligence}{Caving} (\tn[13])}/80/76,
  \ref{standingTunnel}/95/64,
  \ref{wormTunnel}/95/21,
  {\normalsize \roll{Dexterity}{Caving} (\tn[5] + Strength + \glsfmttext{dr})}/75/09,
  \ref{waterMaze}/05/18,
}

\act{rest}{Intelligence}{Caving}{8}.
Failure means the character does not manage to remove \pgls{ep} after the rest.

\begin{boxtext}
  Everyone at the back finds the sharp rocks stop them resting, and they have to move.
  Those in the middle try to clear the fist-sized rocks, but more keep falling down from little crevices all around, so you try further down the tunnel, but then everyone at the front contracts a headache from lying with their legs above their heads on a downward slope.
  The group repacks half-eaten \glspl{ration} and moves further into the tunnel, but the next stretch is somehow worse.
  Over an hour has passed, and nobody has managed to rest.
\end{boxtext}

\iftoggle{intro}{
  The \gls{natural} may let some rest, but not others.
  In this case, the troupe can push on (without wasting \pgls{interval}), or remain, and let some rest while the rest fail to find a comfy spot.
}{}

\paragraph{Once they near the exit,}
they must shuffle out slowly, with their head poking out.
At this point, a goblin lying in wait attacks with an automatic \gls{vitalShot}.
\ifnum\value{temperature}=3%
  The only clue about the danger, is the gentle sound of the nearby river.
\fi%
Hearing a motionless goblin is extremely unlikely, but you might ask for a \roll{Wits}{Vigilance} roll at \tn[14] from the character in the lead, if they have any chance of succeeding at the roll.

\wayOut{The goblin stands in the open, next to}{waterMaze}

\playCommentaryDrowning[t]

%\Needspace{7\baselineskip}
\mapentry[gentlePassage]{Soft Passage}

\begin{boxtext}
  The wet cavern walls stand apart, allowing everyone to walk together, with two or three side-by-side, without having to stoop.
  You tread carefully around the dry stones which cover the floor.
\end{boxtext}

Just like the entrance cave-in (\vpageref{caveIn}), the ceiling is ready to fall.

\paragraph{If they spot the danger}
(by noticing the dry rocks) and walk carefully, the troupe rolls \roll{Dexterity}{Stealth} at \tn[7].

\act[If the players don't notices the danger, the \glspl{pc}]{}{Wits}{Caving}{10}
Failure means the roof starts collapsing as they irritate the walls by stomping down together and (probably) talking, then each player rolls \roll{Speed}{Athletics} (\tn[3] for the first, then 4, 5,\ldots).
On a tie, the roof caves in, but the \gls{tn} begins at 1.

\wayOut{If the troupe return before the cave-in, they can go back to}{firstBridge}
\wayOut{The passage soon becomes more stable, and half a mile later, emerges at}{cliffDive}

\mapentry[cliffDive]{Cliff Dive}

\begin{boxtext}
  Ahead, the sound of gentle water indicates a wider passage coming up.
  The air has a razor-sharp cold feeling.
\end{boxtext}

This passage ends abruptly with a cliff, overlooking an underground lake.
The only way down is a rocky climb, and the water below is 5 degrees.
Luckily, the lake is deep, and nobody will hit the ground.

\paragraph{If the players don't mention taking special care,}
the troupe should roll \roll{Dexterity}{Caving} (\tn[5]) to avoid the first \gls{pc} stumbling over the cilff and falling into the water as everyone behind bunches up together.

\paragraph{Climbing or diving down}
demands a \roll{Dexterity}{Athletics} roll at \tn[8].%
\footnote{Planning cannot help with this roll -- the characters cannot see below well enough to plan.}
Failure inflicts $1D3$ Damage as they crash against jagged rocks on the cliff-face or a wall.

\paragraph{Once in the water,}
\ifnum\value{temperature}>1%
  the \glspl{pc} begin to feel a slow current.
  They can swim upstream, or downstream.
\else%
  \set{track}{5}%
  \ifnum\value{temperature}=3%
    \addtocounter{track}{2}%
  \fi%
  the river's current pushes the \glspl{pc} downstream unless they resist with \roll{Speed}{Athletics} (\gls{tn}~\arabic{track}).
\fi%

\wayOut{Swimming upstream leads to}{secondBridge}

\wayOut{The currents carry characters gently to the riverbank, by}{waterMaze}

\wayOut{Characters who continue down-river eventually reach}{lakeStop}

\mapentry[lakeStop]{Bone Scraper}

\Glspl{pc} who remain in the water float gently towards a deadly drop.
They can feel what's coming with a \roll{Wits}{Caving} at \tn[7] (or \tn[10] in the dark, which they probably will be, if floating down river).

\playCommentaryTeamwork[t]

\paragraph{On a tie,}
the character falls a short way onto a large rock.
Climbing back requires a \roll{Speed}{Athletics} roll (\tn[10]).

\bigbreak

\wayOut{Falling inflicts $\dmg{14}$~Damage, plus the character's Strength \gls{attribute}, then on to}{blindFish}
\wayOut{Turning back can only lead towards}{waterMaze}

\mapentry[waterMaze]{Drowned Pass}

This maze is the same shape as the letter `H', and it should terrify players.
Imagine getting from your front door to your cooker in complete darkness.
Now imagine doing that in a stranger's house.
Now imagine the house is under water.
The simple presence of water will make this a harrowing task for the \glspl{pc}.%
\footnote{The passage was normal until a recent earthquake diverted the water.}
They will have to note each short passage.

Whoever enters the frigid waters instantly gains \pgls{ep}, and the exertion of swimming (as usual) inflicts another.

Remember to ask about the backpacks every time a character enters.
Their \glspl{ration} almost certainly do not have waterproof coverings, nor do their tinder boxes.
The \glspl{torch}, however, still light fine, even while soaking.

\paragraph{Once out of the water and rested,}
the character can remove all \glspl{ep} from holding their breath in the water, but not the \gls{ep} gained for swimming\ifnum\value{temperature}<2\else, or the \glspl{ep} for entering freezing waters\fi.

\wayOut{Once out the other side, the troupe have entered}{hungryHall}

\mapPic{t}{cave_7}{
  \ref{hungryHall}/38/93,
  {\normalsize\roll{Speed}{Caving} (\tn[9])}/10/97,
  \ref{gas}/28/69,
  \ref{darkTunnel}/02/70,
  {\normalsize\roll{Intelligence}{Caving} (\tn[10])}/14/10,
  \ref{umberHulk}/64/11,
  \ref{smokePassage}/58/68,
  {\normalsize\roll{Speed}{Caving} (\tn[10])}/60/97,
  \ref{blindFish}/76/80,
  {\normalsize\roll{Speed}{Athletics} (\tn[12])}/82/69,
  {\normalsize\roll{Strength}{Seafaring} (\tn[10])}/83/61,
  {\normalsize\roll{Speed}{Vigilance}}/85/05,
  \ref{stalagmites}/90/19,
}

\mapentry[hungryHall]{Hungry Hall}

\begin{boxtext}[]
  Your eyeballs sting, perhaps from the nasty water, perhaps from the light.
  A fire burns brightly in the distance.
  But the light does not reach you, or the ceiling, or the walls.
  Despite the bright flames, the ceiling and walls lie beyond the wide darkness.

  \smallbreak
  This cavern must be at least as large as \pgls{broch}.
\end{boxtext}

The goblins' fire still burns, and the air will become unbreathable within \pgls{interval}, inflicting \gls{hypoxia} on anyone in the cavern.

\paragraph{Standing by the \gls{caveFire}}
will help \glspl{pc} dry off after the dunk from the \nameref{waterMaze}.

Inside the fire sits the charred bones of \pgls{guard}.
His black, leather armour lies nearby, with teeth-marks.
His coins lie discarded by the fire --- \lootSmall\ in total.%
\footnote{\label{caveCoinsII}The coins are a clue.  They demonstrate that the goblins do not care about coins, so they did not take the money that \gls{susjot} thinks they did.}

\setcounter{diceNo}{0}%
\renewcommand\npcsymbol{\currency}

\act{find the exits}{Speed}{Caving}{9}
Without the fire, the \gls{tn} increases by~3.
The troupe can only use \pgls{bandAct} if they split up.

\wayOut{One passage out smells like farts, and leads to}{gas}
\wayOut{Another, smaller, tunnel leads down to}{darkTunnel}

\null

\mapentry[gas]{Fire Cavern}

\begin{boxtext}[]
  The passage ascends a little, but never too steeply.
  The centre of the passage constantly shifts with falling pebbles, while the edges have no pebbles or debris.
\end{boxtext}

The tunnel continues for half a mile, to a chamber where flammable gas seeps in through little air-vents.

\act{notice the smell}{Wits}{Caving}{12}
Success means \pgls{pc} knows to hold their breath to avoid taking \pgls{ep}.
Failure means that anyone holding \pgls{torch} ignites the gas, and everyone in the tunnel takes \dmg{4}~Damage.

\wayOut{After the \nameref{gas}, the tunnel descends to}{smokePassage}

\mapentry[smokePassage]{Smokey Passage}

\begin{boxtext}[]
  The passage descends, and smells bad.
  The air feels thick, like burning fertilizer.
\end{boxtext}

The goblins wait quietly, ready to light a fire with all their \fireFuel, once the troupe descend.
Once the smoke billows, the goblins \gls{retreat} across \gls{area}~\ref{blindFish}.

\paragraph{If the \glspl{pc} have been creeping quietly,}
call for \pgls{resistedaction}: \roll{Dexterity}{Stealth}, against the goblins' \roll{Wits}{Vigilance}.

\paragraph{If the \glspl{pc} rush,}
they can roll \roll{Speed}{Caving} (\tn[10]) to hold their breath.
Every Failure Margin inflicts \pgls{ep}, and the same \gls{natural} counts for everyone.

\paragraph{If the troupe flee up hill,}
the fire goes out and the smoke clears within \pgls{interval}.

\wayOut{This alcove sits at the side of}{blindFish}

\mapentry[darkTunnel]{Dark Tunnel}

\begin{boxtext}[How will you find the right way?]
  Cracks line the tunnel, some wide enough to walk or crawl through.
  The tunnel forks like lightning.
  Many choices lie ahead.
\end{boxtext}

The goblins abandoned this narrow, meandering maze when \pgls{hulk} blocked the exit, but their footprints (and claw-prints) remain in places, hinting at the correct route out.

\act{find the exit}{Intelligence}{Caving}{10}
Each Failure Margin adds a pointless mile to the journey, but decreases the \gls{tn} by~1 for the next roll.%
\footnote{The players don't need to stick with the same \gls{natural}, because each time, the situation and location has changed.}
\iftoggle{intro}{%
  A tie increases the wasted time by a mile, but the troupe still find the exit.
}{}

Each time a player thinks of a reasonable way to find the exit path, reduce the \gls{tn} by 1 or~2.

\paragraph{And somewhere in the darkness}
\pgls{guard} still lives.

He sometimes hallucinates that people he knows are with him, but he knows he's hallucinating.
However, when the \glspl{pc} arrive, he still thinks he's hallucinating.

\begin{boxtext}
  A mutter from the darkness says ``I know you're not real''.
  You can't hurt me, because you're not real.
\end{boxtext}

\act{reassure \scaredFodder\ they're real}{Intelligence}{Empathy}{10}
If anyone approaches without reassuring him, he attacks.
If nobody manages to show they're real, he follows the troupe from a safe distance.

\paragraph{Lighting \pgls{torch}}
will instantly show \scaredFodder\ that everyone is real and adds a +1~Bonus to finding a way out, but also means \gls{hypoxia}, leading to \pgls{ep} per \gls{torch} and hallucinations.

\begin{boxtext}
  Stopping for a small breather to take in that stuffy air, you think a little goblinoid silhouettes up ahead or behind.
\end{boxtext}

That little verbal slip-up --- `you think a', `ahead or behind' --- should be given with intention.
The troupe will have trouble remembering which direction they came from and which they were going to.
If they ask about the odd phrasing, tell them what they're having trouble remembering; otherwise, simply wait until the troupe have dealt with the imaginary goblins, and decide to continue moving.

\paragraph{If asked about his journey,}
\scaredFodder\ tells the troupe everything (even if he thinks they're not real).

\begin{boxtext}[]
  The ragged voice speaks like it's reciting a prayer.
  \begin{speechtext}
    We came in and \composeHumanName\ fell down an old hole.
    {\sffamily\upshape(\gls{area}~\vref{firstBridge})}
    Down past a boulder with a rope,
    {\sffamily\upshape(\gls{area}~\ref{ropeDown})}
    then a river, and that long passage down and up.
    {\sffamily\upshape(\gls{area}~\ref{standingTunnel})}

    My little buddy Nilo died.
    {\sffamily\upshape(\gls{area}~\ref{standingTunnel})}

    When we came up through that water-filled passage,
    {\sffamily\upshape(\gls{area}~\ref{waterMaze})}
    they were waiting on us.
    They killed \composeHumanName.
    {\sffamily\upshape(\gls{area}~\ref{hungryHall})}
    I had an argument with \composeHumanName\ about going back.
    I think they got him.
    {\sffamily\upshape(they did, \gls{area}~\vref{stalagmites})}

    Now it's just me, and my imagination.
  \end{speechtext}
\end{boxtext}

\begin{minipage}{.1\linewidth}
  \Huge\gls{T}
\end{minipage}
Any player without \pgls{pc} can use \scaredFodder.

\renewcommand\rations{pebbles}
\set{wounds}{3}
\lostGuard

\wayOut{The only exit is blocked by}{umberHulk}

\playCommentaryRestI

\playCommentaryRestII[b]

\mapentry[umberHulk]{\Glsfmttext{hulk}}

\ifnum\value{temperature}=0%

  \begin{boxtext}
    Out on the other side, your hand touches a different kind of rock -- completely smooth, and a series of smaller lines coming out of it.
    The lines below have a shape like exposed tree roots.
  \end{boxtext}

  The \gls{hulk} wakes after 2~\glspl{round}.%
  \footnote{The great \gls{hulk} lies in hibernation, dreaming of laying eggs and eating goblins.}
\else%

  \begin{boxtext}
    You hear crunching ahead, like someone shuffling a cart around a pile of rocks.
    Something stinks like a tanner's.
  \end{boxtext}

  The great \gls{hulk} wanders the caverns, feeding off anything that moves.
    It has laid a large group of eggs, and returns to them periodically, to drop any food on them that it can.
\fi%

\paragraph{If the troupe approach from \stateArea{darkTunnel},}
\ifnum\thetemperature=0%
  they will encounter the \gls{hulk} by touch.
\else%
  the \gls{hulk} sniffs silently and shuffles loudly around the tiny entrance.
\fi%

The \gls{hulk} cannot fit through the passage to the \stateArea{darkTunnel}, so the \glspl{pc} can stay safely away from it once they know it exists, but will have a hard time getting past it until the goblins arrive.

\ifnum\value{temperature}=0\else
  \paragraph{Throwing stones past the \gls{hulk}}
  will make it leave the cave and search for the cause of the noise, because \glspl{hulk} are stupid, and always move towards any noises.
\fi

\wayOut{The only way out leads to the same chamber as}{smokePassage}
The goblins arrive soon, and set their fire, which begins to irritate the \gls{hulk}.
Soon after, the \gls{hulk} attacks, and the goblins flee cross the river in the \stateArea{blindFish}.

\iftoggle{intro}{
  The \gls{hulk} will stop at the unsteady bridge, but might be coaxed across with the right carrot and stick.
}{}

\set{r2}{2}
\umberhulk

\mapentry[blindFish]{Blind Fish Rapids}

\begin{boxtext}
  A wide crack in the ground ahead falls away to reveal a river, almost within touching distance.
  It moves silently, and looks the colour of black tea under the torchlight.
\end{boxtext}

A wicked undercurrent hides in the river.
The \glspl{pc} will need to vault it or swim across.
Once on the other side, they can use the wooden planks goblins have left on the other side to get across.

One wooden plank is stable, the other has a cut across the middle.
Using the planks without examining them results in \pgls{pc} falling through the water.

\act{swim}{Strength}{Seafaring}{10}
Rolling a tie means the character can make no progress -- they simply fight against the river's undercurrent, gain \pgls{ep}, and roll again.

Rolling a failure means the river pulls the character downriver, the \gls{tn} increases by +2, and the character gains \pgls{ep} then rolls again.

\paragraph{Examining the river}
reveals little fish, about the size of a finger.
Any reasonable plan to catch them works, and provides \pgls{ration} once cooked.
Placing \pgls{caveFire} safely can prove very difficult.

\act{jump over the river}{Speed}{Athletics}{10}

\paragraph{Going down-river}
means death in a black abyss.

\wayOut{Once over the river, the cavern narrows to meet}{stalagmites}
\wayOut{Following the wall leads to}{umberHulk}

\mapentry[stalagmites]{Stalagmites Point}

\begin{boxtext}
  Around the corner, patches of feint, green, light lie on the ground, behind silhouettes of spikes.
  The spike-outlines range from the height of a dagger, to the height of a human.
  And behind them all, in the distant shadows, a man in black leather stands with his arms spread wide, and his head looking straight down at the ground.
\end{boxtext}

On closer inspection, the feint, green, light emanates from little mushrooms --- \glspl{glowshroom} --- dropped around the room.
But closer inspection means closer to the goblins, waiting above with rocks.

\act{avoid the rocks}{Speed}{Vigilance}{\roll{Dexterity}{Projectiles}}
The goblins have a -4~Penalty to hit, because the rocks are \emph{big}, but a +2~Bonus while in the dark.
After throwing their rocks, they take 2~\glspl{round} for the goblins above to pass more down along the narrow ledge.

\iftoggle{intro}{
  \begin{exampletext}
  Ten goblins remain alive, so five are present with big rocks.
  \begin{description}
    \item[On \gls{round} 1]
    \pgls{pc} dies, as rocks fall on his head.
    \item[On \gls{round} 2]
    the troupe hear the goblins passing down more rocks.
    \Pgls{pc} decides to rush out, pick up \glspl{glowshroom} and throw them up at the goblins.
    \item[On \gls{round} 3]
    the goblin by the exit takes another rock, but the next spends her time kicking that \gls{glowshroom} away, so no more goblins have rocks to throw.
  \end{description}
  \end{exampletext}
}{}

\paragraph{Retreat is easy,}
but the troupe have limited supplies.
The goblins will wait \pgls{interval} before retreating\ldots silently.

%! Map 2 Handout
\paragraph{If the \glspl{pc} approach the \gls{guard}'s body,}
they can see the exit to \gls{area}~\vref{olmSwarm} just ahead.

\label{caveCoinsIII}
On the corpse, \pgls{coin}-purse has \arabic{r12}~\glspl{sp} and \arabic{r3}0~\glspl{cp}.%
\footnote{The goblins have taken the rope from his satchel, and used it to string him up like a puppet, to entertain themselves.}
\setcounter{diceNo}{0}%
\renewcommand\npcsymbol{\currency}

\wayOut{This lower passage continues for 1~mile to}{olmSwarm}
\wayOut{If the troupe run up the little ledge the goblins stand on, they find a cavern full of large rocks at the top, which goes along for a mile to}{mushroomsTunnel}

\mapentry[mushroomsTunnel]{Glowing Chamber}

This passage hosts glowing mushrooms, called `\glspl{glowshroom}'.
Each one gives off light when disturbed in anyway.

A pile of them can be used as lantern light, though the light dies after \pgls{interval}.

\paragraph{If the \glspl{pc} pick the mushrooms,}
They will have enough supplies for five meals.
The cook will need \pgls{caveFire} and then makes an \roll{Intelligence}{Cultivation} roll (\tn[10]) to prepare them properly.

Rolling a tie indicates that the chef has failed, but will not poison anyone.
Failure indicates that the resulting rancid mushroom `stew', will not count as a full meal for anyone, and instead inflict 3~\glspl{ep}.

\wayOut{The tunnel continues for a mile, to}{sunRoof}
\wayOut{Another tunnel leads down through miles of darkness, until it hits}{goblinHole}

\mapentry[goblinHole]{Goblin Hole}

The goblins initially emerged from this tunnel, and can retreat through it if all else goes wrong.
It contains narrow passages and wide passages, and connections to the \gls{deep}.
It also presents a serious problem for the \glspl{pc}' mission, with very little solution -- the goblins can retreat down here, and the troupe will struggle to follow them.

Far below, a complete society of goblins lives beside volcanic vents, which grow verdant gardens.
The \glspl{pc} cannot hope to survive down here, and if they spend \pgls{interval} descending, they should figure out that they are going the wrong way.

\wayOut{The troupe must return to}{mushroomsTunnel}

\caveMapOut

\mapentry[olmSwarm]{\Glsfmtplural{olm} Passage}

\begin{boxtext}
  Feint squeaking sounds echo from ahead.
  The echo comes from above and below, and left and right.
  It's less of a sound, and more like a tiny pain in the ear.
\end{boxtext}

\paragraph{If the troupe have any source of light,}
the \glspl{olm} attack whoever has the light as \pgls{swarm}.

\olmSwarm

\wayOut{The tunnel continues for a mile, to}{basiliskCave}

\metroMapDistances[t]

\mapentry[basiliskCave]{\Glsfmttext{basilisk} Cave}

\ifcase\thecycle% Yonders
  \ifnum\thefenestraDay>40%
    \begin{boxtext}
      The second-hand Sunlight, entering through the long cave-gullet shows bones.
      Thousands of long bones.
      And a motionless \gls{basilisk}.
    \end{boxtext}

    \Pgls{basilisk} has curled up to hibernate here before the snow hits during \gls{cTwo}.
    It has begun to sleep deep enough that its breath does not fill the cave with the usual stench, but any noise will waken it.
    If the troupe move carefully, they can sneak past the bones with a \roll{Dexterity}{Stealth} roll at \tn[9].
    If they have no light, they will awaken it by bumping into the bones.

    \basilisk
  \else%
    This cave contains ten thousand giant bones.
    Each \gls{cTwo}, \glspl{basilisk} hibernate here, and they will arrive in the coming weeks.
    But for now, they cave remains lifeless and safe.
  \fi%
\or% Sables
  \begin{boxtext}
    The matted knot of green flesh ahead sits like so many rocks.
    You cannot see a head, but you could never mistake that rancid stench -- some number of \glspl{basilisk} lie in a knotted pile.
  \end{boxtext}

  Three \glspl{basilisk} have curled up here to hibernate.
  The stench signals to everyone around that the creatures lie nearby, and every breath when wandering near this alcove will inflict 3~\glspl{ep}.

  The \glspl{pc} can sneak past quietly, with a \roll{Dexterity}{Stealth} at \tn[6] (or 9 if they have no light).
  Awakening the \glspl{basilisk} will almost certainly prove fatal.
  However, they take 3~\glspl{round} to awaken%
  \iftoggle{intro}{%
    , so the \glspl{pc} might kill all of them before any consequences, as each attack will come as a surprise to the \glspl{basilisk}.%
    \exRef{core}{Core Rules}{sneak_attackattack}
  }{.}

  The troupe will be aware that \gls{basilisk} hides make some of the best armour, and the entire corpse can fetch a lot of money (although the \gls{templeOfBeasts} will ask them to use the funds to repay their debt).

  \basilisk[\npc{\T[3]\A}{\arabic{noAppearing} \Glsfmttext{basilisk}}]
\or% Wrecans
  \Glspl{basilisk} hibernate here over \glsentrytext{cTwo}.
  Last snowfall, one \gls{basilisk} did not wake up, and so provided food for the entire caving system.

  \Pgls{swarm} of \glspl{olm} currently feast on the remains of the \gls{basilisk} corpse.
  As before, if the \glspl{pc} carry any source of light, they will attack; otherwise, they ignore them.

  \olmSwarm
\else
  This crunchy cave has ten thousand giant bones.
  Each year, the local \glspl{basilisk} hibernate here.
  But for now, it has nothing but a few more \glspl{olm}, poking about at night, and sleeping under the bones during the daylight.

  \olmSwarm
\fi

\wayOut{A slender ray of light comes from}{sunRoof}

\mapentry[sunRoof]{Sun Roof}

\begin{boxtext}
  Around the next corner, you see a speck of light in the distance.
  It looks incredibly bright and brings a fresh wind\ldots
\end{boxtext}

This passage has had a cave-in, letting Sunlight \ifnum\value{temperature}=0 and snow \fi flood in through the top.
Unfortunately the walls reach higher than a castle's.

\act{climb up}{Speed}{Athletics}{12}
Failure means the character falls 8~\glspl{step}.%
\exRef{core}{Core Rules}{falling}
The player should not know the \gls{tn} as the character cannot see enough details high up the wall --- only the lower holds.

\act{spot the best way to climb up}{Intelligence}{Athletics}{8}
Success means a +2~Bonus to climbing, and reveals the \gls{tn}.

\paragraph{If any of the \glspl{pc} fled before the initial cave-in,}
now is a good time to see if they survived.%
\footnote{Check \vpageref{pcRunaways} for notes on \glspl{pc} who fled.}

\wayOut{A nasty stench wafts out of}{basiliskCave}
\wayOut{The second cavern leads back to}{mushroomsTunnel}
\wayOut{The third leads to}{crackExit}

\mapentry[crackExit]{Exit Crack}

\begin{boxtext}
  A rotten smell seeps into the tunnel.
  Small holes at the side of the cavern lead upwards towrds light.
\end{boxtext}

Whether the tunnels lead to starlight, or daylight, they do not provide an exit for the entire troupe.
Only creatures with \pgls{weight} of 20 or less can crawl through.

On the other side, the horse-bones litter the area, taken from the raid on the caravan.
%! Page ref

\wayOut{One more mile along is}{windyPassage}

\mapentry[windyPassage]{Windy Passage}

The wind coming through this tunnel invites the troupe out of the caverns, and back to \gls{fenestra}'s usual dangers.

\end{multicols}
