\section{The Guts of the Mountain}

\begin{multicols}{2}

\mapentry[start]{Fresh Hall}

The troupe will almost certainly want to stop and rest here for \pgls{interval}.
The air smells dusty, but fresh enough to rest.

\begin{boxtext}
  This fresh hall has an ambient drip of water, and moisture everywhere.
  The ceiling slopes down gently, surrounding the entire passage with a ring of low-hanging darkness.
  Large, damp, rocks litter the ground in every direction.
\end{boxtext}

\paragraph{The way out}
goes down, sharply.

The troupe should roll \roll{Dexterity}{Caving} (\tn[10]).
A tie means the first person can drop an item instead of taking the fall.
Falling down the passage inflicts $1D6-1$ Damage (a helmet provides a maximum of \gls{dr} 1, no other armour counts).

\wayOut{A path winds slowly up half a mile, to}{falseExit}.

\wayOut{Another leads 1 mile to}{firstBridge}

\pic{cave_2}

\mapentry[falseExit]{False Exit}

The path travels upwards until opening on the mountain's side.
Unfortunately, the path also exits onto the side of a cliff, so the troupe cannot simply walk down.

Climbing down requires a \roll{Dexterity}{Athletics} at \tn[14].%
\footnote{If a player insists on attempting this check, you should roll for them, and tell them if they succeed or fail.  The other players will only find out if their characters can climb down later -- they should not know their chances of success by looking at the \gls{natural}.}

Failure means death.

\boxPair[t]{
  \pic{cave_3}
}{
  \pic{cave_4}
}

\mapentry[firstBridge]{First Bridge}

\begin{exampletext}
  The human prospectors (who looked for a good place to mine) built two bridges.
  Over time, dust, rocks, and other debris covered the bridge entirely.
  Then the wood degraded, leaving a natural trap.

  The first group of \glspl{guard} to descend ran across the bridge, so half of it crumbled, sending them falling to the pit of sharp rocks below.
  One died, and his body, and equipment, were left down here to rot.
\end{exampletext}

\paragraph{If the troupe were not running,}
then at least one torch has burnt out.%
\footnote{Each torch only lasts 1 hour.}

\paragraph{If the troupe inspect the hole,}
they will see the ground here is degraded wood.

\paragraph{If the troupe go across the wood,}
anyone with a \gls{weight} of 9 or more (including equipment) will make the remains of the bridge collapse.

\paragraph{Jumping over the bridge,}
requires a \roll{Speed}{Athletics} roll (\tn[9]).
A tie means the bridge collapses, but the character has crossed safely.

\paragraph{Falling into the hole}
inflicts $1D6+2$ Damage.

\paragraph{Entering the hole}
requires an \roll{Intelligence}{Caving} roll (\tn[10]), to understand a safe route down.
Ascending requires \tn[12].

The players can make this a group roll, by having their characters discuss the safest route down, if they are prepared to spend \pgls{interval} on the discussion.

\wayOut{Past the bridge, after a mile of fairly smooth cavern floor, the troupe find}{ropeDown}

\wayOut{Inside the hole under the bridge, a crack reveals}{gentlePassage}

\mapentry[ropeDown]{Rope \& Cliff}

\begin{exampletext}
  This cavern has a tall cliff-face, ten \glspl{step} straight down, so the goblins descend with a rope tied to a boulder.
\end{exampletext}

\begin{boxtext}
  The torchlight no longer hits the far wall -- you have come to a cliff-face, which overlooks a massive drop down.
  Right ahead, you see a boulder with rope tied around it.
  The rope droops down the cliff-face.

  The echoes of your footsteps return from the other side of the cavern.
\end{boxtext}

This may look like a trap, but it isn't.

\paragraph{If the rope takes a combined \gls{weight} of 11 or more,}
the rope will pull the boulder down on their head.

\paragraph{If the rope takes a combined \gls{weight} of 14 or more,}
(presumably due to multiple characters climbing down) it will snap.

\wayOut{The wide cavern at the bottom leads to the}{secondBridge}

\mapPic{b}{cave_5}{
  \ref{secondBridge}/80/40,
  \ref{cliffDive}/60/85,
  \ref{wormTunnel}/58/02,
  \ref{standingTunnel}/00/04,
}

\mapentry[secondBridge]{Second Bridge}

\begin{exampletext}
  The second old mining bridge has degraded almost as much as the first, and covers a wider area -- a full four \glspl{step} across, and one \gls{step} high.
\end{exampletext}

Anyone with \gls{weight} 8 or more who steps on it will make it break, and fall into the water below.
However, the goblins can walk across it, one at a time, without damaging it.

\wayOut{Falling into the river means flowing downstream to meet the riverbank leading to the}{waterMaze}

\wayOut{Past the bridge, a slender crack leads immediately to the}{wormTunnel}

\wayOut{Three \glspl{step} later, a small pile of stones and the stench of shit marks the start of the}{standingTunnel}

\mapentry[gentlePassage]{Gentle Passage}

This path, like the entrance has had a little cave-in, and will have more if anyone large walks through it.
The players may spot the danger, or you can give the characters a roll for \gls{gagingCave}.

If moving across, the troupe can make a \roll{Dexterity}{Stealth} roll, with \pgls{tn} equal to their total \gls{weight}.

\begin{boxtext}
  The path widens, and you can comfortably walk, without ducking.
  It the wet cavern walls stand apart, allowing everyone to walk together, with two or three side-by-side.
  You step carefully over the little dry stones which cover the floor.
\end{boxtext}

\caveIn

Just like the entrance, if the \glspl{pc} don't spot the danger, the ceiling will collapse on a 1 in 6 chance, which increases every time something falls.
However, this time the cavern stretches across 60 steps.

\wayOut{The passage soon becomes more stable soon after, and half a mile later, emerges at the}{cliffDive}

\mapentry[cliffDive]{Cliff Dive}

This passage ends abruptly with a cliff-ledge, with a river a stone's throw below.
The river has enough depth to let anyone land safely.

The \glspl{pc} will have a difficult decision, since they can't tell if the dive will kill them.

\paragraph{Climbing down}
demands a \roll{Speed}{Athletics} roll (\tn[9]).%
\footnote{Planning cannot help with this roll -- the characters cannot see below well enough to plan.}

\paragraph{Diving down}
requires a \roll{Dexterity}{Athletics} roll (\tn[10]), but failure only inflicts 2~\glspl{ep}.

\paragraph{Once in the water,}
the river pushes everyone naturally towards a riverbank
characters float gently downstream.
One side of the river ends in a wall, the other has a little ledge to grab onto, some distance down.

Going upstream leads to a dead-end as the cavern's ceiling descends to meet the water.
Characters who attempt to go upstream will have to swim some distance before they find the dead end, and will suffer 3~\glspl{ep}.

\wayOut{The river pushes characters gently to the riverbank, close to the}{waterMaze}

\wayOut{Characters who continue down-river receive a harsh ride down an underwater stream to the}{blindFish}

The river's rapids and rocks demand a \roll{Strength}{Athletics} roll to hold one's breath, or receive 4~\glspl{ep} (\tn[10]); then a \roll{Dexterity}{Caving} roll to avoid $1D6+2$ Damage (\tn[16] if the character is in the dark, which they almost certainly are).

\mapentry[wormTunnel]{Worm Tunnel}

The goblins cannot traverse the \nameref{standingTunnel}, so they must crawl through this passage.
The \glspl{pc} will probably have a difficult time here, and the goblins know it.
The narrow tunnel extends a full mile before the troupe can stretch their arms again, and it counts as two miles due to the cramped conditions.

To squeeze through, the players roll \roll{Dexterity}{Caving}, at \gls{tn} 7 plus double their Strength score, plus 2 for wearing armour.

\paragraph{Rolling a tie}
means that the character takes a long time, and gains 2 \glspl{ep} from all the shuffling and scrapes.

\paragraph{Rolling a failure}
means the character has become stuck, has gained 2~\glspl{ep}, and will not get out without someone helping to pull them from the other side, with a rope.

Setting up a rope to help them requires another character to make their roll for them again, with +1 to the \glsentrylong{tn}.

\mapPic{b}{cave_6}{
  \ref{standingTunnel}/95/64,
  \ref{wormTunnel}/95/21,
  \ref{waterMaze}/05/18,
}

\paragraph{Resting}
requires an \roll{Intelligence}{Caving} roll (\tn[10]), because getting people to a comfortable spot where they can rest, without standing, while clearing the rocks away, won't work.
Failure means the troupe waste \pgls{interval} trying to relax and eat.

\goblin[\npc{\N\M}{Goblin}]

\paragraph{Once they near the exit,}
they must shuffle out slowly, with their head poking out.
At this point, a goblin lying in wait attacks with an automatic \gls{vitalShot}.
The only clue about the danger, is the gentle sound of the nearby river.

\wayOut{The goblin stands in the open, in the same room as the}{waterMaze}
If a single \glspl{pc} approaches while upright and ready for him, he flees towards the exit there.

\mapentry[standingTunnel]{Standing Tunnel}

\begin{exampletext}
  The human miners stopped mining slightly beyond this point.
  Their budgie died in this tunnel, which told them that the tunnel held a lethal gas.

  The next time they returned, they came back with a little wooden box, with the budgie's name -- `Happy' -- carved onto the top, and placed it under a little cairn.

  When the goblins moved in, they did not understand the significance of the little cairn, so one died in the passage.
  Since then, they use the tunnel only as a toilet.
\end{exampletext}

\paragraph{Upon entering the tunnel,}

\begin{itemize}
  \item
  The smell of goblin-shit haunts their noses.
  \item
  Any humans in the troupe will instantly spot the pile of rocks, and understand that someone built it like a cairn.
  \item
  A few \glspl{step} later, they will see a pile of goblin droppings.
\end{itemize}

\paragraph{Carbon monoxide}
lays in the low parts of the tunnel, around waist-height to a human.
It won't do any harm to walking humans, but if a gnome or dwarf enter, they will start to feel sleepy, then die.
The same goes for anyone sitting down to rest.

The first sign of carbon monoxide poisoning is usually death, but in this case the first sign is a dead goblin.
It died recently, but has no marks -- no cuts or bruises.

If the players fail to heed the warning signs, any short characters will die first.

\paragraph{After a mile,}
the passage ascends.
Various twists mean it continues longer than its neighbouring tunnel, making it 2 miles long.

\mapentry[waterMaze]{Puddle of Doom}

\begin{exampletext}
  This passage was once a nice, wide area, with a simple passage downward, to a series of small tunnels, one of which leads out to the `\nameref{hungryHall}'.
  Since then, it has flooded, which complicates the situation.
\end{exampletext}

Imagine getting from your front door to your cooker in complete darkness.
Now imagine doing that in a stranger's house.
Now imagine the house is under water.
The simple presence of water will make this a harrowing task for the \glspl{pc}.

The players will have to map the passage, in one way or another.

Whoever enters the frigid waters instantly gains \pgls{ep}, and the exertion of swimming (as usual) inflicts another.

\null
\iftoggle{verbose}{
  The scene might go like this:

  \begin{itemize}
    \bf
    \item
    `Okay, I'll jump in the puddle.'
    \begin{itemize}
      \it
      \item
      `Do you take your backpack off?'
    \end{itemize}
    \item
    `Yes.
    What do I see?'
    \begin{itemize}
      \it
      \item
      `The water has a metallic cold, which pulls heat from your skin.'
      \item
      `You feel the hard, slippery rocks around, and have to descend a little before finding a passage.
      This seems to be the only passage.
      It's about the height of a crouching child.'
    \end{itemize}
    `Okay, so I'll go through, carefully.'
    \begin{itemize}
      \item
      \it
      `Going carefully, you have 3~\glspl{ap}, so you cover 3 steps this round.'
      \item
      `Put down a temporary \gls{ep} (you can remove it once you can breathe again).'
      \item
      `Swimming onward, you cover another 3~\glspl{step} and find a side tunnel on the left.'
      \item
      `Above, the tunnel opens upwards.'
      \item
      `Put down another \gls{ep}'
      \item
      `Which way do you go?'
    \end{itemize}
    `I'll move up.'
    \begin{itemize}
      \it
      \item
      `You find a sharp, spiky ceiling as you swim up and round.
      You think the tunnel moves horizontally from here, but you can't be sure of which way is up or down in the freezing, dark waters.'
      \item
      `Take another \gls{ep}.'
      \item
      `Do you continue?'
    \end{itemize}
    \item
    `So if I have 3 slots without \pgls{ep}, that means I can go another 3 rounds\ldots so then how far have I come?'
    \begin{itemize}
      \item
      \it
      `You're unsure.'
      \item
      `Take \pgls{ep}.'
    \end{itemize}
    \item
    `Again?
    Okay, I'm going back!'
  \end{itemize}

  This process might take a while.
  The player will have to decide who goes next, and plan their routes ahead.
  Each one will enter blind, and have to remember where they should scout next.

  Remember to ask about the backpacks every time a character enters.
  Their rations almost certainly do not have waterproof coverings, nor do their tinder boxes.

  The torches will still light fine, even after becoming wet.
  However, the water in the torch's wood will mean that torch creates a constant hiss from the water turning to steam.
  This will prevent the troupe moving silently while they carry those torches.

  \paragraph{For some extra tension,}
  let a character enter the water and then focus on the other characters.

  \begin{speechtext}
    Okay, Ratcull entered the water.
    Everyone else waits for some time, and the water feels quiet.
    What do you do now?
  \end{speechtext}

  Let the troupe decide how long they will wait, in the darkness, before someone else enters.
  Of course, if the others say `one hour', once you resolve the first character's actions (the one in the black waters), you may find that the troupe did not in fact wait for an hour, as the first character may return after a few minutes to confirm that everything's fine.
}{}

\paragraph{If multiple \glspl{pc} enter the tunnels,}
this spells trouble.
The passage only has enough space for one person, so if they run into trouble, they cannot back up without bumping into everyone behind them.

\paragraph{Once out of the water and rested,}
the character can remove all \glspl{ep} from holding their breath in the water, but not the \gls{ep} gained for swimming\ifnum\value{temperature}=2\else, or the \glspl{ep} for entering freezing waters\fi.

\wayOut{Once out the other side, the troupe have entered the}{hungryHall}

\mapentry[hungryHall]{Hungry Hall}

\begin{exampletext}
  This chamber could fit an entire \glsentrytext{village} inside it, and has plenty of fresh air, but not unlimited air.
  The goblins understand the cavern's limits well, so the last time that \glspl{guard} entered, they lit a fire, and cooked a body in here, but left before the air became unbreathable.
\end{exampletext}

The goblins have lit a fire, and within \pgls{interval}, the air will become quietly deadly, inflicting \gls{hypoxia} on anyone in the cavern.

\begin{boxtext}
  Emerging from the water, you see a fire burning in the distance.
  Despite the bright light, you cannot see the ceiling or walls of the cavern -- just the bright light in the centre.
\end{boxtext}

\paragraph{Standing by the \gls{caveFire}}
will help dry off after the dunk from area \ref{waterMaze}.

Inside the fire sits the charred bones of \pgls{guard}.
His armour lies nearby, with teeth-marks.

\paragraph{Finding the exits}
requires a \roll{Speed}{Caving} roll at \tn[9] (or \tn[14] if they put out the fire).
The troupe can only use \pgls{bandAct} if they split up.

They can also make this \pgls{restingaction} if they are prepared to spend \pgls{interval} doing it carefully.

\wayOut{One passage out smells like farts, and leads to the}{gas}

\wayOut{Another, smaller, tunnel leads down to the}{darkTunnel}
Disturbed rocks show that goblins only take one of these two passages.

These tunnels have enough good air in them to prevent \gls{hypoxia} getting any worse, but not enough to cure whatever effects have begun.

\mapPic{b}{cave_7}{
  \ref{hungryHall}/38/93,
  \ref{gas}/28/69,
  \ref{darkTunnel}/02/70,
  \ref{umberHulk}/64/11,
  \ref{smokePassage}/58/68,
  \ref{blindFish}/76/80,
  \ref{stalagmites}/90/19,
}

\mapentry[gas]{Fire Cavern}

The passage here leads upwards, to a chamber where flammable gas seeps in through little air-vents.

The party can make a \roll{Wits}{Caving} roll (\tn[12]) to notice the smell before their torches ignite it, dealing $2D6$ Damage to the first half of the characters, and $1D6$ to the second half.%
\footnote{Armour with \pgls{covering} of 3 reduces the Damage by half, while armour with \gls{covering} 5 reduces it to a quarter.}

\wayOut{The troupe will exit through a narrow crack, at the}{smokePassage}

\mapentry[darkTunnel]{Dark Tunnel}

This narrow tunnel stretches a long way -- around a mile.
The troupe can walk down it single-file, occasionally crouching.

However, if they have even a single torch lit, the oxygen in the tunnel will deplete, causing \gls{hypoxia}.
This will leave the party with a real problem -- tiredness, and hallucinations can follow.

If the troupe avoid lighting any torch in the narrow tunnel, they will have to try \gls{blackWalking}.

\begin{boxtext}
  You stop for a small breather, then notice little goblinoid silhouettes up ahead or behind.
\end{boxtext}

That little verbal slip-up -- `ahead or behind' -- should be given with intention.
The troupe will have trouble remembering which direction they came from and which they were going to.
If they ask about the odd phrasing, tell them what they're having trouble remembering; otherwise, simply wait until the troupe have dealt with the imaginary goblins, and decide to continue moving.

\wayOut{The \glspl{pc} find new problems, just before reaching the}{umberHulk}

\mapentry[umberHulk]{Umber Hulk}

\begin{exampletext}
  This giant, beetle-like creature%
  \exRef{judgement}{Judgement}{umber_hulk}
  \ifnum\value{temperature}=0
    lies in hibernation, dreaming of laying eggs and eating goblins.
  \else
    wanders the caverns, feeding off anything that moves.
    It has laid a large group of eggs, and returns to them periodically, to drop any food on them that it can.
  \fi
\end{exampletext}

\paragraph{If the troupe approach from \stateArea{darkTunnel},}
\ifnum\value{temperature}=0
  they will encounter the umber hulk by touch.

  \begin{boxtext}
    Out on the other side, your hand touches a different kind of rock -- completely smooth, and a series of smaller lines coming out of it.
    The lines below have a shape like exposed tree roots.
  \end{boxtext}

  The hulk wakes within a couple of \glspl{round}.
\else
  the umber hulk will begin sniffing silently and shuffling loudly around the tiny entrance.

  \begin{boxtext}
    You hear the sound of disturbed rocks ahead, like someone shuffling a cart around a pile of rocks.
  \end{boxtext}
\fi

The Umber Hulk will cannot fit through the passage to the \stateArea{darkTunnel}, so the \glspl{pc} can stay safely away from it once they know it exists, but will have a hard time getting past it.

\umberhulk

\wayOut{The only way out leads to the same chamber as the}{smokePassage}
The goblins there will irritate the umber hulk, causing it to chase the goblins, who flee across the river in the \stateArea{blindFish}.

\mapentry[smokePassage]{Smokey Passage}

The goblins wait at the bottom of this long passageway, to light a fire with all their \fireFuel, once the troupe descend.

\goblin[\npc{\T[2]\N\M}{\arabic{noAppearing} Goblins}]

\goblin[\npc{\N\F}{Goblin}]

\goblin[\npc{\N\F}{Goblin}]

\paragraph{If the \glspl{pc} have been creeping quietly,}
they should make a \roll{Dexterity}{Stealth}, against the goblins' \roll{Wits}{Vigilance}, \tn.

\paragraph{If the \glspl{pc} head back up,}
they will receive more and more \glspl{ep} until they die, or try another way.

\paragraph{If the \glspl{pc} want to rush down,}
they can roll \roll{Speed}{Caving} (\tn[10]) to hold their breath.
Every Failure Margin inflicts \pgls{ep}, and the same \gls{natural} counts for everyone.

\wayOut{This cavern expands slowly, and soon meets the}{blindFish}.

\mapentry[blindFish]{Blind Fish Rapids}

A long, lazy, river with a wicked undercurrent cuts across this cavern.
The \glspl{pc} will need to vault it or swim across.
Once on the other side, they can use the wooden planks goblins have left on the other side to get across.

One wooden plank is stable, the other has a cut across the middle.
Using the planks without examining them results in \pgls{pc} falling through the water.

\begin{boxtext}
  A wide crack in the ground ahead falls away to reveal a river, almost within touching distance.
  It moves silently, and looks the colour of black tea under the torchlight.
\end{boxtext}

\paragraph{Swimming}
demands a \roll{Strength}{Seafaring} roll (\tn[10]).
Rolling a tie means the character can make no progress -- they simply fight against the river's undercurrent, gain \pgls{ep}, and roll again.

Rolling a failure means the river pulls the character downriver, the \gls{tn} increases by +2, and the character gains \pgls{ep} then rolls again.

\paragraph{Examining the river}
reveals little fish, about the size of a finger.

Any reasonable plan to catch them might work, although placing \pgls{caveFire} safely can prove very difficult.

\paragraph{Jumping over}
demands a \roll{Speed}{Athletics} roll (\tn[12]).

\mapPic{b}{cave_8}{
  \ref{skeinSwarm}/14/31,
  \ref{mushroomsTunnel}/25/60,
  \ref{basiliskCave}/40/30,
  \ref{goblinHole}/25/90,
  \ref{sunRoof}/62/55,
  \ref{crackExit}/80/60,
  \ref{windyPassage}/90/60,
}

\mapentry[stalagmites]{Stalagmites Point}

\begin{exampletext}
  The last of the \glspl{guard} from the first troupe died here.
  The goblins have taken the rope from his satchel, and used it to string him up like a puppet, to entertain themselves.
\end{exampletext}

\goblin[\npc{\T[3]\N\M}{\arabic{noAppearing} Goblins}]

\begin{itemize}
  \item
  Once the troupe enter, they will see the corpse of the \gls{guard}, and may think he's an undead creature.
  \item
  The goblins will then begin to drop massive rocks on the \glspl{pc}' heads (\dmg{7} Damage).
  \item
  The characters can evade with a \roll{Dexterity}{Athletics} roll, vs the goblins' \roll{Dexterity}{Projectiles} (\tn).
  \item
  The size of the rocks means the goblins can only hold onto them for two \glspl{round}.
  \item
  The goblins also need a full two rounds to `reload', as goblins pass rocks from above.
\end{itemize}

%! Map 2 Handout
\paragraph{If the \glspl{pc} examine the \gls{guard} body,}
they find a map of the cave, created as the previous group came down.

\paragraph{Retreat is easy,}
but the troupe have limited supplies.
The goblins will wait \pgls{interval} before retreating\ldots silently.

\paragraph{Finding the ground exit}
is difficult, because the stalagmites block the view in every direction.
The troupe can move through here, each \glspl{pc} must roll \roll{Dexterity}{Athletics} (\tn[10]) to `jump' through.
Failure means they block the passage -- just for \pgls{round} -- and nobody else can move through until they adjust their position.

\wayOut{This lower passage continues for 1~mile to the}{skeinSwarm}

\wayOut{If the troupe run up the little ledge the goblins stand on, they find a cavern full of large rocks at the top, which goes along for a mile to the}{mushroomsTunnel}

\mapentry[skeinSwarm]{Skein Hole}

\begin{exampletext}
  Skein are skinny little lizards which live in caves.
  They have no eyes.
  Their little claws grasp grasp smooth rocks like a spider.
  Their skin lets light through, and even the smallest glow of light causes them pain.
\end{exampletext}

\paragraph{If the troupe have any source of light,}
the skein attack whoever has the light as \pgls{swarm}.

\skeinSwarm

\paragraph{If the troupe look around,}
they will notice large patches of mushrooms growing in this cavern.
All of them are edible if one can prepare them correctly.

\paragraph{If the \glspl{pc} pick the mushrooms,}
They will have enough supplies for five meals.
The cook will need \pgls{caveFire} and then makes an \roll{Intelligence}{Cultivation} roll (\tn[10]) to prepare them properly.

Rolling a tie indicates that the chef has failed, but will not poison anyone.
Failure indicates that the resulting rancid mushroom `stew', will not count as a full meal for anyone, and instead inflict 2~\glspl{ep}.

\wayOut{The tunnel continues for a mile, to the}{basiliskCave}

\mapentry[mushroomsTunnel]{Glowing Chamber}

This passage hosts glowing mushrooms, called `glow shrooms'.%
\exRef{judgement}{Book of Judgement}{glowshroom}
Each one gives off light when disturbed in anyway.

A pile of them can be used as lantern light, though the light will die after \pgls{interval} someone picks them, so they cannot provide a long-term torch.

\wayOut{The tunnel continues for a mile, to the}{sunRoof}

\wayOut{Another tunnel leads down through miles of darkness, until it hits the}{goblinHole}.

\mapentry[goblinHole]{Goblin Hole}

The goblins initially emerged from this tunnel, and can retreat through it if all else goes wrong.
It contains narrow passages and wide passages.
It also presents a serious problem for the \glspl{pc}' mission, with very little solution -- the goblins can retreat down here, and the troupe will struggle to follow them.

Far below, a complete society of goblins lives beside volcanic vents, which grow verdant gardens.
The \glspl{pc} cannot hope to survive down here, and if they spend \pgls{interval} descending, they should figure out that they are going the wrong way.

\mapentry[basiliskCave]{\Glsfmttext{basilisk} Cave}

\ifnum\value{temperature}=0
  Three \glspl{basilisk} have curled up here to hibernate.
  The stench signals to everyone around that the creatures lie nearby, and every breath when wandering near this alcove will inflict 3~\glspl{ep}.

  Awakening the \glspl{basilisk} will almost certainly prove fatal.
  However, they take 3 rounds to awaken, so the \glspl{pc} may kill all of them before any consequences.

  \begin{boxtext}
    The matted knot of green flesh ahead sits like so many rocks.
    You cannot see a head, but you could never mistake that rancid stench -- some number of \glspl{basilisk} lie in a knotted pile.
  \end{boxtext}

  \basilisk[\npc{\T[3]\A}{\arabic{noAppearing} \Glsfmttext{basilisk}}]

  The troupe will be aware that \gls{basilisk} hides make some of the best armour, and the entire corpse can fetch a lot of money.
  Of course, officially speaking, any money the \gls{guard} earn belongs to the \gls{templeOfBeasts}.%
\else
  \Glspl{basilisk} hibernate here over the cold season.
  Last snowfall, one \gls{basilisk} did not wake up, and so provided food for the entire caving system.

  \Pgls{swarm} of skein currently feast on the remains of the \gls{basilisk} corpse.
  As before, if the \glspl{pc} carry any source of light, they will attack; otherwise, they ignore them.

  \skeinSwarm

\fi

\begin{boxtext}
  Around the next corner, you see a speck of light in the distance.
  It looks incredibly bright\ldots
\end{boxtext}

\wayOut{The not-so-distant light is the}{sunRoof}.

\mapentry[sunRoof]{Sun Roof}

This passage has had a cave-in, and Sunlight floods in through the top.
Unfortunately it still sits farther underground than a castle.

Climbing the ragged walls begins with an \roll{Intelligence}{Athletics} roll, to understand the best route up.
A successful roll gives a further +2 Bonus to the main event: a \roll{Speed}{Athletics} check, \tn[14].
It also lets the player know the \gls{tn}.

A tie means the character figures out this climb will only hurt them before they climb.
A failed roll means the character suffers $2D6$ Damage.

\wayOut{One more mile, and the troupe find the}{crackExit}

\mapentry[crackExit]{Crack to Exit}

This sweet-smelling narrow tunnel goes forward and up, and eventually exits to the world above.
Unfortunately, it will only admit creatures with a Strength Bonus of 0 or less.

\wayOut{One more mile along is the}{windyPassage}

\mapentry[windyPassage]{Windy Passage}

The wind coming through this tunnel indicates its invitation to the outside world.
The troupe can finally move out, and be free of the goblin warren.

\subsection{Returning Home}

The troupe is now free, and only have the small task of navigating back home, through the dense forest, and a final encounter.

\iftoggle{verbose}{
  \begin{itemize}
    \item
    Is it light outside?
    \item
    How many rations do they have left?
  \end{itemize}
}{}

\ifcase\value{temperature}
  A refreshing breeze blows, but in the distance, they can hear howling wolves.

  The wolves arrive after \pgls{interval}, and stalk the troupe for an interval, hoping to steal some food.

  \wolf
  \or
  Mist covers the forest, making it nearly impossible to spot the \gls{woodspy} waiting for them to come out of the \nameref{windyPassage}.

  \woodspy
  \else
  A storm is brewing above them.
  Within \pgls{interval}, it will break, inflicting an additional \gls{ep} while travelling.

  Meanwhile, a \gls{crawler} watches them from above (they spot it immediately).
  If they retreat until the storm breaks, it leaves.
  But if they wander in the open, it attacks.
\fi

If the troupe follow the sound of the river, they will eventually find a nearby \gls{village}.

\subsection{Back at the \Glsfmttext{bothy}}

Once the troupe return, \gls{redcoin} immediately asks how many goblin heads the troupe have obtained.
If the troupe have none, one of them will need to roll \roll{Charisma}{Caving} to explain the difficulties (\tn[8]).
If the roll fails, \gls{redcoin} sends them all to the local \gls{court}, while arranging for more \gls{fodder} to go up the mountain\ldots

\end{multicols}
