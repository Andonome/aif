\section{Descent into the Darkness}

\begin{multicols}{2}

\mapentry[start]{Fresh Hall}

Ask each of the \glspl{pc} to describe what their characters are doing.
If none of them mention holding a light, then none of their characters have a light source.

\begin{boxtext}
  None of the characters were holding a light.
  Where did you leave the torch?
  It must have dropped before the last tunnel.
\end{boxtext}

\noindent
The troupe will almost certainly want to stop and rest here for \pgls{interval}.
Once a light has been organized, the troupe can look around and take a rest.

\begin{boxtext}
  This fresh hall has an ambient drip of water, and moisture everywhere.
  The ceiling slopes down gently, surrounding the entire passage with a ring of low-hanging darkness.
  Large, damp, rocks litter the ground in every direction.
\end{boxtext}

\paragraph{The way out}
goes down, sharply.

The troupe should roll \roll{Dexterity}{Caving} (\tn[10]).
A tie means the first person can drop an item instead of taking the fall.
Falling down the passage inflicts $1D6-1$ Damage (a helmet provides a maximum of \gls{dr} 1, no other armour counts).

\begin{boxtext}
  Ahead, the tunnel stops sloping, and continues, flat.
  A wide shadow indicates a hole in the ground.
\end{boxtext}

\mapentry[firstBridge]{First Bridge}

\begin{exampletext}
  The human prospectors (who looked for a good place to mine) built two bridges.
  Over time, dust, rocks, and other debris covered the bridge entirely.
  Then the wood degraded, leaving a natural trap.

  The first group of \glspl{guard} to descend ran across the bridge, so half of it crumbled, sending them falling to the pit of sharp rocks below.
  One died, and his body, and equipment, were left down here to rot.
\end{exampletext}

\paragraph{If the troupe inspect the hole,}
they will see the ground here is degraded wood.

\paragraph{If the troupe go across the wood,}
anyone with a \gls{weight} of 9 or more (including equipment) will make the remains of the bridge collapse.

\paragraph{Jumping over the bridge,}
requires a \roll{Speed}{Athletics} roll (\tn[9]).
A tie means the bridge collapses, but the character has crossed safely.

\paragraph{Falling into the hole}
inflicts $1D6+2$ Damage.

\paragraph{Entering the hole}
requires an \roll{Intelligence}{Caving} roll (\tn[10]), to understand a safe route down.

The players can make this a group roll, by having their characters discuss the safest route down, if they are prepared to spend \pgls{interval} on the discussion.

\begin{boxtext}
  The ground stops suddenly, as you reach a cliff.
  At the side, a thin rope wraps around a boulder, and droops over the cliff.
\end{boxtext}

\mapentry[ropeDown]{Rope \& Cliff}

This looks like a trap, but it isn't.
The goblins need a rope to climb down, but they can't take it down with them.

\begin{boxtext}
  The torchlight flies out, and disappears.
  The cave goes on, but you can't see how far.
  Then the echoes of your feet start to return, but seem to take an impossibly long time to return, and much longer to stop.
\end{boxtext}

\paragraph{If the rope takes a combined \gls{weight} of 11 or more,}
the rope will pull the boulder down on their head.

\paragraph{If the rope takes a combined \gls{weight} of 14 or more,}
(presumably due to multiple characters climbing down) it will snap.

\mapentry[hungryHall]{Hungry Hall}

This chamber could fit an entire \glsentrytext{village} inside it.
The troupe can go anywhere, in any direction, but they will find only damp rocks in the darkness unless they search hard.

\sidebox{
  \begin{boxtable}
  \textbf{\glsentrytext{tn}} & \textbf{Result} \\
  \hline
  13  & exit to \gls{area} \ref{secondBridge} (`\nameref{secondBridge}') \\
  10  & exit to \gls{area} \ref{murderHoles} (`\nameref{murderHoles}') \\
  \end{boxtable}
}

The players should feel aware of their lighting choices.
Keeping a lit torch means that they can see danger coming, at least a little way off, but it also means that anything in the darkness can see them, while they cannot see it.

The darkness watches them.

\paragraph{Finding the exits}
requires a \roll{Speed}{Caving}.
The troupe can use teamwork%
\exRef{core}{Core Rules}{teamwork}
only if they split up.

They can also make this \pgls{restingaction}%
\exRef{core}{Core Rules}{restingactions}
if they are prepared to spend \pgls{interval} doing it carefully.

\mapentry[gentlePassage]{Gentle Passage}

\mapentry[forkLeaf]{Fork \& Leaf}

\mapentry[gas]{Gas Chamber}

The passage here leads upwards, to a chamber of flammable gas.

The party can make a \roll{Wits}{Caving} roll (\tn[13]) to notice the smell before their torches ignite it, dealing $2D6$ Damage to the first half of the characters, and $1D6$ to the second half.

\mapentry[mudSlide]{Mud Slide}
The goblins carried some mud here, so that they could leave large, obvious, footprints, upwards to the cavern of the umber hulk (room \ref{umberHulk}).

\mapentry[murderHoles]{Murder Holes}

\mapentry[secondBridge]{Second Bridge}

\mapentry[waterMaze]{Water Maze}

\mapentry[hypoxia]{Dark Tunnel}

\mapentry[wormTunnel]{Worm Tunnel}

\mapentry[umberHulk]{Umber Hulk}

Once the party are up, they steal some eggs, crawl down a small hole, and singing.

\begin{speechtext}
  Leviathan, beetle king!

  \noindent
  Hunting dun, eggies bring!
\end{speechtext}

\mapentry[blindFish]{Blind Fish}

\mapentry[smokePassage]{Smokey Passage}

The goblins light a fire at the bottom of this long, sloping passage.

\mapentry[stallagmites]{Stalagmites Point}

\Pgls{guard} corpse hangs here on ropes.
Goblins above jiggle the ropes, using him like a puppet.
The players will almost certainly think that this creature is undead.

He holds a map of the cave, created as the previous group came down.

\mapentry[mushrooms]{A Short Cut to Mushrooms}

Glowshrooms live here.

\mapentry[basiliskCave]{Basilisk Cave}

\mapentry[sunRoof]{Sun Roof}

This semi-false exit floods sunlight into this area.
Children above look down, but the PCs look like monsters, so the children scream and run away.

\mapentry[crackExit]{Crack to Exit}

\mapentry[windyPassage]{Windy Passage}


\end{multicols}
