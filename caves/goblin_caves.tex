\section{The Guts of the Mountain}

\begin{multicols}{2}

\renewcommand\npcsymbol{\glsentrysymbol{yonder}}
\mapentry[start]{Fresh Hall}

The troupe will almost certainly want to stop and rest here for \pgls{interval}.
The air smells dusty, but fresh enough to rest.

\begin{boxtext}
  This fresh hall has an ambient drip of water, and moisture everywhere.
  The ceiling slopes down gently, surrounding the entire passage with a ring of low-hanging darkness.
  Large, damp, rocks litter the ground in every direction.
\end{boxtext}

\wayOut{A path winds slowly up half a mile, to}{falseExit}

\wayOut{Another leads 1 mile to}{firstBridge}

\smolMapPic{cave_2}{
  \ref{falseExit}/85/10,
}

\playCommentaryAftermath[t]

\mapentry[falseExit]{False Exit}

The path travels upwards until opening on the mountain's side.
Unfortunately, the path also exits onto the side of a cliff, so the troupe cannot simply walk down.

Climbing down requires a \roll{Dexterity}{Athletics} at \tn[14].%
\footnote{If a player insists on attempting this check, you should roll for them, and tell them if they succeed or fail.  The other players will only find out if their characters can climb down later -- they should not know their chances of success by looking at the \gls{natural}.}
Failure means death.

\mapentry[firstBridge]{First Bridge}

\begin{exampletext}
  The human miners built two bridges while exploring further into the mountain.
  Over time, dust, rocks, and other debris covered this first bridge entirely.
  Then the wood degraded, leaving a natural trap.

  When first group of \glspl{guard} came through here, the bridge collapsed, sending one down to a pit of sharp rocks below.
  He died instantly, and the group moved on.
\end{exampletext}

If the troupe walk with a light, they instantly see the hole in the ground (but their torch is probably about to go out at this point).
Otherwise, they should roll \roll{Wits}{Caving} at \tn[8].

\paragraph{If the troupe go across the wood,}
anyone with \pgls{weight} of 13 or more (including their \gls{equipment}) will make the remains of the bridge collapse.

\paragraph{Jumping over the bridge,}
requires a \roll{Speed}{Athletics} roll (\tn[9]).
A tie means the bridge collapses, but the character has crossed safely.

Anyone falling into the hole receives $1D6$ Damage plus their own Strength Bonus (bigger creatures suffer more from falling).

The corpse is dry, and not very rotten.
His satchel still contains two bees-wax candles which can burn for an hour each, a tinder-box, \lootMedium, and slightly-expired \rations.
\label{caveCoinsI}
\setcounter{diceNo}{0}%
\renewcommand\npcsymbol{\currency}%

\wayOut{Past the bridge, after a mile of fairly smooth cavern floor, the troupe find}{ropeDown}

\wayOut{Inside the hole under the bridge, a crack reveals}{gentlePassage}

Entering the hole requires an \roll{Intelligence}{Caving} roll (\tn[7]) to understand a safe route down.
Once \pgls{pc} explains the hand-holds and where to place the legs, anyone can descend safely.
Failure means falling, and $1D6$~Damage plus the \gls{pc} Strength \gls{attribute}.

\smolMapPic{cave_3}{
  {\normalsize\roll{Wits}{Caving} (8/ 11)}/17/92,
  {\normalsize\glsfmttext{weight}~13: collapse}/57/75,
  \rotatebox{-21}{\normalsize\roll{Intelligence}{Caving} (7)}/71/21,
  \ref{firstBridge}/49/57,
  \ref{gentlePassage}/92/05,
}

\mapPic{b}{cave_5}{
  {\normalsize\glsfmttext{weight}~10: collapse}/67/24,
  \ref{secondBridge}/80/40,
  \ref{cliffDive}/60/85,
  {\normalsize\roll{Speed}{Athletics}}/65/94,
  {\normalsize Dive (6)}/28/45,
  {\normalsize Climb (10)}/43/70,
  \ref{wormTunnel}/58/02,
  \ref{standingTunnel}/00/05,
}

\mapentry[ropeDown]{Rope \& Cliff}

\begin{exampletext}
  This cavern has a tall cliff-face, ten \glspl{step} straight down, so the goblins descend with a rope tied to a boulder.
\end{exampletext}

\begin{boxtext}
  The cavern seems to end in a wall, but as the torchlight approaches, you can see the darkness of a sheer drop ahead.
  A boulder stands next to the drop with a rope tied around it.
  The rope descends into the edge.
\end{boxtext}

This may look like a trap, but it isn't; the goblins simply use the rope to climb up and down.
The boulder can take up to \pgls{weight} of 11 before it falls, and the rope can take up to \pgls{weight} of 14 before it snaps (this includes items, as usual).

\wayOut{The wide cavern leads to the}{secondBridge}

\smolMapPic{cave_4}{
  {\large\glsfmttext{weight}~11: fall}/71/80,
  \rotatebox{-90}{\large\glsfmttext{weight}~14: snap!}/20/50,
  \ref{ropeDown}/42/92,
}

\mapentry[secondBridge]{Last Bridge}

\begin{exampletext}
  The second old mining bridge has degraded almost as much as the first, and covers a wider area -- a full four \glspl{step} across, and one \gls{step} high.
\end{exampletext}

Anyone stepping on the bridge with \pgls{weight} of 10 or more will make it break, and fall into the water below.
Goblins walk across it, one at a time, without damaging it.

On the other side, the tunnel continues, and a crack appears in the wall.


\paragraph{Falling into the river's frigid water}
inflicts \pgls{ap}
\ifnum\value{temperature}<3%
  but anyone can swim across easily.
\else%
  and may push the characters downstream if they can't swim against the currents (\roll{Speed}{Athletics}, \tn[5]).
\fi%

\wayOut{Following the river inflicts another \gls{ep} and quickly leads to banks of the}{waterMaze}

\wayOut{Three \glspl{step} later, a small pile of stones and the stench of goblin shit marks the start of the}{standingTunnel}

\wayOut{The slender crack leads immediately to the}{wormTunnel}

\mapentry[standingTunnel]{Standing Tunnel}

\begin{exampletext}
  The human miners stopped at this point when their budgies died.%
  \footnote{Miners use small, sensitive, birds to detect lethal gas.  If the bird dies, it's time to leave.}

  The next time they returned, they came back with a little wooden box, with the budgie's name -- `Happy' -- carved onto the top, and placed it under a little cairn.

  When the goblins moved in, they did not understand the significance of the little cairn, so one died in the passage.
  Since then, they use the tunnel only as a toilet.
\end{exampletext}

\begin{boxtext}
  Once off the bridge, the smell of goblin-shit haunts your noses.
  It covers the ground, and drips off the tall pile of stones to the side, then stops suddenly; the passage beyond seems clear.
\end{boxtext}

\paragraph{Carbon monoxide}
lays in the low parts of the tunnel, around waist-height to a human.
It won't do any harm to walking humans, but if a gnome or dwarf enter, they will start to feel sleepy, then die.
The same goes for anyone sitting down to rest.

The first sign of carbon monoxide poisoning is usually death, but in this case the first sign is a dead goblin (10~\glspl{step} down the passage).
It died weeks ago, but has no marks -- no cuts or bruises -- and the corpse has not decayed much, due to the infertile conditions in the cave.

The second warning sign is a dead gnome (20~\glspl{step} down the passage).
He has no items, as his companions picked up his bag before moving on.

If the players fail to heed the warning signs, any short characters will die first.

\paragraph{After a mile,}
the passage ascends.
Various twists mean it continues longer than its neighbouring tunnel, making it 2 miles long.

\mapentry[wormTunnel]{Worm Tunnel}

The goblins cannot traverse the \nameref{standingTunnel}, so they must crawl through this passage.
The \glspl{pc} will probably have a difficult time here, and the goblins know it.
The narrow tunnel extends a full mile before the troupe can stretch their arms again, and it counts as two miles due to the cramped conditions.

\paragraph{To squeeze through,}
each player rolls \roll{Dexterity}{Caving}, at \tn[7] plus double their character's Strength Bonus, plus 2 if wearing armour.

On a tie, characters gain 2~\glspl{ep} from the shuffling and scrapes.
One a failure, they also become stuck, and others will have to help them.

Setting up a rope to help them requires another character to make their roll for them again, with +1 to the \glsentrylong{tn}.

\mapPic{t}{cave_6}{
  \ref{standingTunnel}/95/64,
  \ref{wormTunnel}/95/21,
  \ref{waterMaze}/05/18,
}

\paragraph{Resting}
requires an \roll{Intelligence}{Caving} roll (\tn[10]), because getting people to a comfortable spot where they can rest, without standing, while clearing the rocks away, won't work.
Failure means the troupe waste \pgls{interval} trying to relax and eat.

\paragraph{Once they near the exit,}
they must shuffle out slowly, with their head poking out.
At this point, a goblin lying in wait attacks with an automatic \gls{vitalShot}.
\ifnum\value{temperature}=3%
  The only clue about the danger, is the gentle sound of the nearby river.
\fi%

\wayOut{The goblin stands in the open, in the same room as the}{waterMaze}
If a single \glspl{pc} approaches while upright and ready for him, he flees towards the exit there.

\playCommentaryPuddle

\mapentry[gentlePassage]{Soft Passage}

This path, like the entrance, has had a little cave-in, and will have more if anyone large walks through it.
The players may spot the danger, or you can give the characters a roll for \gls{gagingCave}.

\begin{boxtext}
  The path widens, and you can comfortably walk, without ducking.
  The wet cavern walls stand apart, allowing everyone to walk together, with two or three side-by-side.
  You step carefully over the little dry stones which cover the floor.
\end{boxtext}

Just like the entrance cave-in (\vpageref{caveIn}), the ceiling is ready to fall.

\paragraph{If they spot the danger}
(by noticing the dry rocks) and walk carefully, the troupe rolls \roll{Dexterity}{Stealth} at \tn[7].

\paragraph{If nobody notices the danger}
characters can roll \roll{Wits}{Caving} at \tn[10].
Failure means the roof starts collapsing, and each player rolls \roll{Speed}{Athletics} (\tn[3] for the first, then 4, 5,\ldots).

\wayOut{If the troupe return before the cave-in, they can go back to the}{firstBridge}

\wayOut{The passage soon becomes more stable, and half a mile later, emerges at the}{cliffDive}

\mapentry[cliffDive]{Cliff Dive}

This passage ends abruptly with a cliff, overlooking an underground lake.
The only way down is a rocky climb, and the water below is 5 degrees.
Luckily, the lake is deep, and nobody will hit the ground.

\begin{boxtext}
  The cave widens, then halts at a cliff, the height of two men.
  \ifnum\value{temperature}>1%
    The darkness below looks perfectly smooth -- a flat, black, surface.
  \else%
    The black waters below show very little movement, but the echoes of running water suggest a strong undercurrent.
  \fi%
\end{boxtext}

The \glspl{pc} will have a difficult decision, since they can't tell if the dive will kill them.

\paragraph{Climbing or diving down}
demands a \roll{Dexterity}{Athletics} roll at \tn[8].%
\footnote{Planning cannot help with this roll -- the characters cannot see below well enough to plan.}
Failure inflicts $1D3$ Damage as they crash against jagged rocks on the cliff-face or a wall.

\paragraph{Once in the water,}
\ifnum\value{temperature}>1%
  the \glspl{pc} begin to feel a slow current.
  They can swim upstream, or downstream.
\else%
  \set{track}{5}%
  \ifnum\value{temperature}=3%
    \addtocounter{track}{2}%
  \fi%
  the river's current pushes the \glspl{pc} downstream unless they resist with \roll{Speed}{Athletics} (\gls{tn}~\arabic{track}).
\fi%

\playCommentaryDrowning[t]

\wayOut{Swimming upstream leads to the}{secondBridge}

\wayOut{The currents carry characters gently to the riverbank, by the}{waterMaze}

\wayOut{Characters who continue down-river eventually reach the}{lakeStop}

\mapentry[lakeStop]{Bone Scraper}

\Glspl{pc} who remain in the water float gently towards a deadly drop.
They can feel what's coming with a \roll{Wits}{Caving} at \tn[7] (or \tn[10] in the dark, which they probably will be, if floating down river).

\paragraph{On a tie,}
the character falls a short way onto a large rock.
Climbing back requires a \roll{Speed}{Athletics} roll (\tn[10]).

\wayOut{Falling inflicts $\dmg{14}$~Damage, plus the character's Strength \gls{attribute}, then on to the}{blindFish}

\wayOut{Turning back can only lead towards the}{waterMaze}

\mapentry[waterMaze]{Drowned Pass}

\begin{exampletext}
  This passage was once a nice, wide area, with a simple passage downward, to a series of small tunnels, one of which leads out to the `\nameref{hungryHall}'.
  Since then, a minor earthquake has changed the ground just enough to let the river drip into the tunnel, leaving it completely underwater.
\end{exampletext}

Imagine getting from your front door to your cooker in complete darkness.
Now imagine doing that in a stranger's house.
Now imagine the house is under water.
The simple presence of water will make this a harrowing task for the \glspl{pc}.

The players will have to map the passage, in one way or another.

Whoever enters the frigid waters instantly gains \pgls{ep}, and the exertion of swimming (as usual) inflicts another.

Remember to ask about the backpacks every time a character enters.
Their \glspl{ration} almost certainly do not have waterproof coverings, nor do their tinder boxes.

The torches will still light fine, even after becoming wet.
However, the water in the torch's wood will mean that torch creates a constant hiss from the water turning to steam.
This will prevent the troupe moving silently while they carry those torches.


\paragraph{Once out of the water and rested,}
the character can remove all \glspl{ep} from holding their breath in the water, but not the \gls{ep} gained for swimming\ifnum\value{temperature}=2\else, or the \glspl{ep} for entering freezing waters\fi.

\wayOut{Once out the other side, the troupe have entered the}{hungryHall}

\mapPic{b}{cave_7}{
  \ref{hungryHall}/38/93,
  \ref{gas}/28/69,
  \ref{darkTunnel}/02/70,
  \ref{umberHulk}/64/11,
  \ref{smokePassage}/58/68,
  \ref{blindFish}/76/80,
  \ref{stalagmites}/90/19,
}

\mapentry[hungryHall]{Hungry Hall}

\begin{exampletext}
  This chamber could fit an entire \glsentrytext{village} inside it, and has plenty of fresh air, but not unlimited air.
  The goblins understand the cavern's limits well, so they lit a fire, and cooked the body of \pgls{guard} who died here days ago, then left.
\end{exampletext}

The goblins' fire still burns, and the air will become unbreathable within \pgls{interval}, inflicting \gls{hypoxia} on anyone in the cavern.

\begin{boxtext}
  Emerging from the water, you see a fire burning in the distance.
  Despite the bright flames, you cannot see the ceiling or walls of the cavern.
\end{boxtext}

\paragraph{Standing by the \gls{caveFire}}
will help \glspl{pc} dry off after the dunk from the \nameref{waterMaze}.

Inside the fire sits the charred bones of \pgls{guard}.
His armour lies nearby, with teeth-marks.
His coins lie discarded by the fire -- \lootSmall\ in total.%
\footnote{The coins are a clue.  They demonstrate that the goblins do not care about coins, so they did not take the money that \gls{susjot} thinks they did.}
\label{caveCoinsII}
\setcounter{diceNo}{0}%
\renewcommand\npcsymbol{\currency}

\paragraph{Finding the exits}
requires a \roll{Speed}{Caving} roll at \tn[9] (or \tn[14] if they put out the fire).
The troupe can only use \pgls{bandAct} if they split up.

They can also make this \pgls{restingaction} if they are prepared to spend \pgls{interval} doing it carefully.

\wayOut{One passage out smells like farts, and leads to the}{gas}

\wayOut{Another, smaller, tunnel leads down to the}{darkTunnel}
Disturbed rocks show that goblins only take one of these two passages.

These tunnels have enough good air in them to prevent \gls{hypoxia} getting any worse, but not enough to cure whatever effects have begun.

\mapentry[gas]{Fire Cavern}

The passage here leads upwards for half a mile, to a chamber where flammable gas seeps in through little air-vents.

The party can make a \roll{Wits}{Caving} roll (\tn[12]) to notice the smell before their torches ignite it, dealing $2D6$ Damage to the first half of the characters, and $1D6$ to the second half.
\Glspl{pc} with armour can reduce the Damage by an amount equal to its \gls{covering} (so minus 3 in most cases).

\wayOut{After the \nameref{gas}, the tunnel descends to the}{smokePassage}

\mapentry[smokePassage]{Smokey Passage}

The goblins wait at the bottom of this long passageway, to light a fire with all their \fireFuel, once the troupe descend.

\paragraph{If the \glspl{pc} have been creeping quietly,}
they should make a \roll{Dexterity}{Stealth}, against the goblins' \roll{Wits}{Vigilance}.

\paragraph{If the \glspl{pc} head back up,}
they will receive more and more \glspl{ep} until they die, or try another way.

\paragraph{If the \glspl{pc} want to rush down,}
they can roll \roll{Speed}{Caving} (\tn[10]) to hold their breath.
Every Failure Margin inflicts \pgls{ep}, and the same \gls{natural} counts for everyone.

\wayOut{This alcove sits at the side of the}{blindFish}

\mapentry[darkTunnel]{Dark Tunnel}

The goblins abandoned this narrow, meandering maze when an umber hulk blocked the exit, but their footprints (and claw-prints) remain in places, hinting at the correct route out.

Finding that exit needs an \roll{Intelligence}{Caving} roll at \tn[10].
Each failure adds a pointless mile to the journey, but decreases the \gls{tn} by~1 for the next roll.%
\footnote{The players don't need to stick with the same \gls{natural}, because each time, the situation and location has changed.}

\begin{boxtext}
  A tunnel cuts across the current one, letting you divert left or right.

  Ahead, the torchlight barely illuminates another branching cavern ahead.
  How will you find the right way out, with so many paths in the cave?
\end{boxtext}

Each reasonable answer the \glspl{pc} give should reduce the \gls{tn} by 1 or~2.

\paragraph{If the troupe have \pgls{torch} lit,}
the oxygen in the tunnels will deplete, causing \gls{hypoxia}.
This will leave the party with a real problem -- tiredness, and hallucinations can follow.

If the troupe avoid lighting any torch in the narrow tunnel, they will have to try \gls{blackWalking}.%
\footnote{This \glsentrydesc{blackWalking}.}

\begin{boxtext}
  You stop for a small breather, and smell the stuffy air, then notice little goblinoid silhouettes up ahead or behind.
\end{boxtext}

That little verbal slip-up -- `ahead or behind' -- should be given with intention.
The troupe will have trouble remembering which direction they came from and which they were going to.
If they ask about the odd phrasing, tell them what they're having trouble remembering; otherwise, simply wait until the troupe have dealt with the imaginary goblins, and decide to continue moving.

\wayOut{The only exit is blocked by the}{umberHulk}

\mapentry[umberHulk]{Umber Hulk}

\begin{exampletext}
  The great umber hulk%
  \exRef{judgement}{Judgement}{umber_hulk}
  \ifnum\value{temperature}=0
    lies in hibernation, dreaming of laying eggs and eating goblins.
  \else
    wanders the caverns, feeding off anything that moves.
    It has laid a large group of eggs, and returns to them periodically, to drop any food on them that it can.
  \fi
\end{exampletext}

\paragraph{If the troupe approach from \stateArea{darkTunnel},}
\ifnum\value{temperature}=0
  they will encounter the umber hulk by touch.

  \begin{boxtext}
    Out on the other side, your hand touches a different kind of rock -- completely smooth, and a series of smaller lines coming out of it.
    The lines below have a shape like exposed tree roots.
  \end{boxtext}

  The hulk wakes within a couple of \glspl{round}.
\else
  the umber hulk will begin sniffing silently and shuffling loudly around the tiny entrance.

  \begin{boxtext}
    You hear crunching ahead, like someone shuffling a cart around a pile of rocks.
    Something stinks like a tanner's.
  \end{boxtext}
\fi

The Umber Hulk cannot fit through the passage to the \stateArea{darkTunnel}, so the \glspl{pc} can stay safely away from it once they know it exists, but will have a hard time getting past it.

\umberhulk

\wayOut{The only way out leads to the same chamber as the}{smokePassage}
The goblins there will irritate the umber hulk, causing it to chase the goblins, who flee across the river in the \stateArea{blindFish}.

\mapentry[blindFish]{Blind Fish Rapids}

A long, lazy, river with a wicked undercurrent cuts across this cavern.
The \glspl{pc} will need to vault it or swim across.
Once on the other side, they can use the wooden planks goblins have left on the other side to get across.

One wooden plank is stable, the other has a cut across the middle.
Using the planks without examining them results in \pgls{pc} falling through the water.

\begin{boxtext}
  A wide crack in the ground ahead falls away to reveal a river, almost within touching distance.
  It moves silently, and looks the colour of black tea under the torchlight.
\end{boxtext}

\paragraph{Swimming}
demands a \roll{Strength}{Seafaring} roll (\tn[10]).
Rolling a tie means the character can make no progress -- they simply fight against the river's undercurrent, gain \pgls{ep}, and roll again.

Rolling a failure means the river pulls the character downriver, the \gls{tn} increases by +2, and the character gains \pgls{ep} then rolls again.

\paragraph{Examining the river}
reveals little fish, about the size of a finger.

Any reasonable plan to catch them might work, although placing \pgls{caveFire} safely can prove very difficult.

\paragraph{Jumping over}
demands a \roll{Speed}{Athletics} roll (\tn[12]).

\mapPic{b}{cave_8}{
  \ref{skeinSwarm}/14/31,
  \ref{mushroomsTunnel}/25/60,
  \ref{basiliskCave}/40/30,
  \ref{goblinHole}/25/90,
  \ref{sunRoof}/62/55,
  \ref{crackExit}/80/60,
  \ref{windyPassage}/90/60,
}

\paragraph{Going down-river}
allows the character to feel ever-stronger water-pressure as they near an underground waterfall, then plunge to darkness and death.

\wayOut{This cavern becomes narrower, and quickly meets}{stalagmites}

\mapentry[stalagmites]{Stalagmites Point}

\begin{exampletext}
  The last of the \glspl{guard} from the first troupe died here.
  The goblins have taken the rope from his satchel, and used it to string him up like a puppet, to entertain themselves.
\end{exampletext}

\begin{itemize}
  \item
  Once the troupe enter, they will see the corpse of the \gls{guard} waving its arms, and may think he's an undead creature.
  \item
  The goblins will then begin to drop massive rocks on the \glspl{pc}' heads (\dmg{7} Damage).
  \item
  The characters can evade with a \roll{Dexterity}{Athletics} roll, vs the goblins' \roll{Dexterity}{Projectiles} (\tn).
  \item
  The size of the rocks means the goblins can only hold onto them for two \glspl{round}.
  \item
  The goblins also need a full two rounds to `reload', as goblins pass rocks from above.
\end{itemize}

%! Map 2 Handout
\paragraph{If the \glspl{pc} examine the \gls{guard}'s body,}
they find \pgls{coin} purse with \lootMedium.
\label{caveCoinsIII}
\setcounter{diceNo}{0}%
\renewcommand\npcsymbol{\currency}

\paragraph{Retreat is easy,}
but the troupe have limited supplies.
The goblins will wait \pgls{interval} before retreating\ldots silently.

\paragraph{Finding the ground exit}
is difficult, because the stalagmites block the view in every direction.
The troupe can move through here, each \glspl{pc} must roll \roll{Dexterity}{Athletics} (\tn[10]) to `jump' through.
Failure means they block the passage -- just for \pgls{round} -- and nobody else can move through until they adjust their position.

\wayOut{This lower passage continues for 1~mile to the}{skeinSwarm}

\wayOut{If the troupe run up the little ledge the goblins stand on, they find a cavern full of large rocks at the top, which goes along for a mile to the}{mushroomsTunnel}

\mapentry[mushroomsTunnel]{Glowing Chamber}

This passage hosts glowing mushrooms, called `\glspl{glowshroom}'.
Each one gives off light when disturbed in anyway.

A pile of them can be used as lantern light, though the light will die after \pgls{interval} someone picks them, so they cannot provide a long-term torch.

\wayOut{The tunnel continues for a mile, to}{sunRoof}

\wayOut{Another tunnel leads down through miles of darkness, until it hits the}{goblinHole}

\mapentry[goblinHole]{Goblin Hole}

The goblins initially emerged from this tunnel, and can retreat through it if all else goes wrong.
It contains narrow passages and wide passages, and connections to the \gls{deep}.
It also presents a serious problem for the \glspl{pc}' mission, with very little solution -- the goblins can retreat down here, and the troupe will struggle to follow them.

Far below, a complete society of goblins lives beside volcanic vents, which grow verdant gardens.
The \glspl{pc} cannot hope to survive down here, and if they spend \pgls{interval} descending, they should figure out that they are going the wrong way.

\playCommentaryRestI[t]

\mapentry[skeinSwarm]{Skein Hole}

\begin{exampletext}
  Skein are skinny little lizards which live in caves.
  They have no eyes.
  Their little claws grasp smooth rocks like a spider.
  Their skin lets light through, and even the smallest glow of light causes them pain.
\end{exampletext}

\paragraph{If the troupe have any source of light,}
the skein attack whoever has the light as \pgls{swarm}.

\skeinSwarm

\paragraph{If the troupe look around,}
they will notice large patches of mushrooms growing in this cavern.
All of them are edible if one can prepare them correctly.

\paragraph{If the \glspl{pc} pick the mushrooms,}
They will have enough supplies for five meals.
The cook will need \pgls{caveFire} and then makes an \roll{Intelligence}{Cultivation} roll (\tn[10]) to prepare them properly.

Rolling a tie indicates that the chef has failed, but will not poison anyone.
Failure indicates that the resulting rancid mushroom `stew', will not count as a full meal for anyone, and instead inflict 2~\glspl{ep}.

\wayOut{The tunnel continues for a mile, to the}{basiliskCave}

\playCommentaryRestII

\metroMapDistances[t]

\mapentry[basiliskCave]{\Glsfmttext{basilisk} Cave}

\ifnum\value{temperature}=0
  Three \glspl{basilisk} have curled up here to hibernate.
  The stench signals to everyone around that the creatures lie nearby, and every breath when wandering near this alcove will inflict 3~\glspl{ep}.

  Awakening the \glspl{basilisk} will almost certainly prove fatal.
  However, they take 3 rounds to awaken, so the \glspl{pc} may kill all of them before any consequences.

  \begin{boxtext}
    The matted knot of green flesh ahead sits like so many rocks.
    You cannot see a head, but you could never mistake that rancid stench -- some number of \glspl{basilisk} lie in a knotted pile.
  \end{boxtext}

  The troupe will be aware that \gls{basilisk} hides make some of the best armour, and the entire corpse can fetch a lot of money (although the \gls{templeOfBeasts} will ask them to use the funds to repay their debt).

  \basilisk[\npc{\T[3]\A}{\arabic{noAppearing} \Glsfmttext{basilisk}}]
\else
  \Glspl{basilisk} hibernate here over \glsentrytext{cTwo}.
  Last snowfall, one \gls{basilisk} did not wake up, and so provided food for the entire caving system.

  \Pgls{swarm} of skein currently feast on the remains of the \gls{basilisk} corpse.
  As before, if the \glspl{pc} carry any source of light, they will attack; otherwise, they ignore them.

  \skeinSwarm
\fi

\begin{boxtext}
  Around the next corner, you see a speck of light in the distance.
  It looks incredibly bright and brings a fresh wind\ldots
\end{boxtext}

\wayOut{The not-so-distant light comes from the}{sunRoof}

\mapentry[sunRoof]{Sun Roof}

This passage has had a cave-in, letting Sunlight \ifnum\value{temperature}=0 and snow \fi flood in through the top.
Unfortunately it still sits farther underground than a castle.

Climbing the ragged walls begins with an \roll{Intelligence}{Athletics} roll (\tn[10]), to understand the best route up.
A successful roll gives a further +2 Bonus to the main event: a \roll{Speed}{Athletics} check, \tn[14].
It also lets the player know the \gls{tn}.

A tie means the character figures out this climb will only hurt them before they climb.
A failed roll means the character falls 8~\glspl{step}.%
\exRef{core}{Core Rules}{falling}

\paragraph{If any of the \glspl{pc} fled before the initial cave-in,}
now is a good time to see if they survived.%
\footnote{Check \vpageref{pcRunaways} for notes on \glspl{pc} who fled.}

\wayOut{A nasty stench wafts out of the}{basiliskCave}

\wayOut{The second cavern leads back to the}{mushroomsTunnel}

\wayOut{The third leads to the}{crackExit}

\mapentry[crackExit]{Exit Crack}

This sweet-smelling narrow tunnel goes forward and up, and eventually exits to the world above.
Unfortunately, it will only admit creatures with a Strength \gls{attribute} of 0 or less.

\wayOut{One more mile along is the}{windyPassage}

\mapentry[windyPassage]{Windy Passage}

The wind coming through this tunnel invites the troupe out of the caverns, and back to \gls{fenestra}'s usual dangers.

\end{multicols}

