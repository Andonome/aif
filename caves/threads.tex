\thread[Town,Mountain,Forest]{Goblin Gossip}

Goblins banish their criminals to the world above, which has earned them a bad reputation.
One such group clambers skyward, and soon begin stealing from nearby \glspl{village}.
The behaviour escalates until they kill a human, and everyone demands the \gls{guard} ascend to slay the lot.
However, goblins can be extremely dangerous on their home turf.

\segment{Mountain}% AREA
{Little by Little}% NAME
{Goblins steal milk from \glsfmtplural{village}}% SUMMARY

By the mountain's side (in \pgls{village} or on the road), one farmer shouts about her stolen milk, while the other tries to stop her making accusations.

\begin{speechtext}
  What kind of monster steals milk left outside a house?
  No kind!
  Not a single \gls{monster} which walks, crawls, or flies takes milk.

  Cows?  Yes
  Babies? Absolutely!
  But never \emph{milk}!
\end{speechtext}

\paragraph{Examining the area}
with a \roll{Wits}{Survival} roll at \tn[12] show goblin footprints going up to the mountain.
A Margin of 2 (i.e. \tn[14]) means the \glspl{pc} can trace the goblins back through the forest, and eventually to their mountainous entrance (\vpageref{goblinCaveEntrance}: `\nameref{goblinCaveEntrance}').

\segment{Town}% AREA
{\squash~Cooked Innards}% NAME
{\Glsfmttext{smuggler} speaks about the deer bones found in the forest, covered with goblin tracks}% SUMMARY

The \glspl{pc} overhear \gls{smuggler}, shouting about goblin tracks around some half-digested deer bones, close to the mountain.
Everyone in the market wants to know what the \glspl{pc} intend to do about it.

\segment[\gls{afternoon}]{Mountain}% AREA
{Dead Horses}% NAME
{Goblins drop rocks on traders' horses}% SUMMARY

\begin{exampletext}
  The goblins have discovered a foolproof way to hunt for food.
  They haul large rocks up trees, then along a branch close to the road, and drop the rocks on a trader's horse.

  If one horse dies, the caravan moves on, leaving the goblins a dead horse to eat.
  If more horses die, the caravan never has enough horse-power to pull all the wagons, and has to leave one behind.
\end{exampletext}

As the troupe travel close to the mountain, goblins stand on the treetops, ready and waiting with boulders.
If the \glspl{pc} are with a caravan, the goblins drop rocks on the horses, otherwise, they remain in the shadowy treetops, silently.

Spotting the goblins requires a \roll{Wits}{Vigilance} roll (\tn[10]).
But doing something about the problem requires a plan, because hitting the goblins won't be easy.

If the \glspl{pc} have \glspl{projectile}, one goblin can drop their big rock and take cover each \gls{round}.
On the treetops, 10~\glspl{step} in the air, is \tn[9], but once they take cover the players will have to roll at \tn[11].

\goblin

\goblin

In addition to the listed rocks, three goblins have a boulder which deals $2D6$ Damage.
However, they can only drop it on someone directly below them.

\segment{Mountain}% AREA
{Over the Line}% NAME
{Goblins kill a shepherd}% SUMMARY

\begin{exampletext}
  A shepherd saw goblins trying to disturb his cows, and rushed in to stop them.
  The goblins killed him quickly, then fled.
\end{exampletext}

Every \glspl{village} round the mountain talks about \composeHumanName's death, and all agree the \gls{guard} must sort the problem immediately.

One group of \glspl{fodder} have already gone up the mountain, but haven't returned.
The next session should begin with \nameref{goblinsBegin}, \vpageref{goblinsBegin}.
