\section{Induction}
\begin{multicols}{2}

\noindent
This module presents a baptism-by-napalm for BIND.
Players should each begin with two to three characters.
If that doesn't work for you, add some extra \glspl{guard} to the troupe, and hand out those character sheets to your players.
They will need them!

Our duplicitious misadventure begins with an apparently easy job.
The troupe must enter goblin-infested caves, and kill the lot.
However, once they enter, a cave-in traps them inside, forcing them to find another route out of the caves.

Players familiar with RPGs will find themselves surprised and confused by the dangers in the these caves, as they are entirely natural, normal caves.%
\footnote{Caveat: I have been down a couple of caves, read a few articles, and asked caving friends, but do not know a lot about caves.
The caves here should seem plausible to everyone except extremely pedantic geologists.
If you are or know a pedantic geologist, please get in touch and tell me any pedantic thoughts which might make the module more interesting.}
They will find these kinds of scenes:

\begin{boxtext}
  Despite the wet cavern walls and floor, big dry rocks sit everywhere.
\end{boxtext}

\noindent
These rocks have fallen from the cracked cave ceiling.
In a moment, more will fall, potentially killing some \glspl{pc}, and wounding others.

\begin{boxtext}
  Following the goblins, the narrow tunnel becomes smaller and smaller, until it reaches about the width of your forearm.
  You think you might be able to squeeze through by shifting along like a worm, if you remove your armour, but you can't see how long the crawl will be\ldots
\end{boxtext}

\noindent
On the other side, a goblin awaits with a spear.

\begin{boxtext}
  The fleeing goblins jump into the darkness with a `sploosh!'.
  Walking carefully, you find a dead-end, and a puddle, as wide as a man.
\end{boxtext}

\noindent
The puddle was a tunnel, but the cavern has flooded.
The flooded area only goes on for a short way, but any \gls{pc} who enters will have to grope blindly to find their way, and feel about up, down, left, and right.

A single, short dead-end, presents a serious danger, and the \glspl{pc} will have to co\"{o}rdinate multiple dead ends.

\begin{boxtext}
  From a little speck of mud here, and some disturbed rocks, you can see the goblins must have gone down this tunnel.
  It's wide enough for a human to walk down, while minding their head of course.
\end{boxtext}

\noindent
This tunnel stretches on so long the group will become weary, and have to rest, then count their rations.
And once they reach the bottom, the goblins wait for them with fire.
They won't use it to burn the \glspl{pc}, but to send smoke up the tunnel.

\subsection{Implicit Scenes}

\subsubsection{Difficult Decisions}

While the \glspl{pc} endure a long journey through miles of caverns, the players will have to many many decisions, in rapid succession.
As time goes on, they will no double notice that any wounded or weak members of their troupe slow the group down considerably, as they must stop to rest often.

\iftoggle{core}{\caveTravelChart}{}
Each \gls{interval}, the \glspl{pc} will have decide how far and how fast they want to travel, and make a \roll{Dexterity}{Caving} roll.
The more miles covered, the more dangerous the journey, but the fewer rations they will lose along the way.%
\exRef{core}{Core Rules}{walk_underground}

Whenever a character dies, the others will doubtless take their rations, making every death bitter-sweet.

If the troupe manage to back any of the goblins into a dead-end, they can fight, but their fatigued bodies will make the fight far more challenging than they might expect.

\subsubsection{Finding Tracks}

If the \glspl{pc} want to make it out of the caverns alive, they may try to follow the goblins.
You should give them a reasonable chance; let them roll \roll{Intelligence}{Caving} at every fork in the route where the goblins fled.%
\footnote{We use Intelligence here, because there is no question about whether or not they spot some small clue.
They only need to search for tracks or signs of disturbance over a fork in the long tunnels, so we only need to know if they can interpret the rock-disturbances, or understand the meaning of wet vs dry mud.}
Of course, they may want to avoid the goblins, in order to avoid an ambush.
Either way, wait until the players ask before giving out this information.

The goblins may show their passing by disturbing rocks, or drop their \fireFuel\ (for the fire they set in area \ref{smokePassage}).

\subsubsection{Silence}

The module will not present these scenes explicitly.
These kinds of events and decisions naturally arise, just by narrating the rules.
Make sure that the players record their characters' rations, \glspl{fatigue}, and the total \gls{weight} each character carries at all times.
And whenever this presents new challenges, they should be made aware of what this means in terms of their characters' prospects.

\subsection{History}
\label{caves_history}

\noindent
Fifty years ago, humans tried to mine this mountain, but found nothing of value, and left.
A couple of wooden bridges remain.

Goblins have stolen milk, rustled pigs, and even killed a few people and nobody knew where in the mountains they came from, until recently.
\Pgls{guard} ranger discovered a cave entrance, some miles up a mountain.
In fact, the goblins know of a tunnel which exits much closer to local \glspl{village}, but as the goblins keep their secrets, the \glspl{pc} will have to enter via the mountainside.

\Pgls{redcoin} has already dispatched two groups of \glspl{guard} have to enter the mountainside tunnel.
The first group entered and died two weeks ago.
The second group went a little into the darkness, lost two men, then returned terrified.
The \pgls{jotter} has ordered their execution, so \glspl{sunGuard} have arrived to take them away from Stoatfen \gls{bothy}.%
\footnote{All \glspl{bothy} are named after someone who died on the road.
They found Stoatfen died in the many arms of a woodspy.}

\end{multicols}

