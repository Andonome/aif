\section{Induction}

\begin{multicols}{2}

\noindent
This module presents a baptism-by-napalm for BIND.
Players should each begin with two to three characters.
If that doesn't work for you, add some extra \glspl{guard} to the troupe, and hand out those character sheets to your players.
They will need them!%
\footnote{I think about 10 characters, divided between the players, should work, but playtesting data may prove me wrong.}

Our duplicitous misadventure begins with an apparently easy job.
The troupe must enter goblin-infested caves, and kill the lot.
However, once they enter, a cave-in traps them inside, forcing them to find another route out of the caves.
Players familiar with RPGs will find themselves surprised and confused by the dangers in the these caves, as they are entirely natural, normal caves.%
\footnote{Caveat: I have been down a couple of caves, read a few articles, and asked caving friends, but do not know a lot about caves.
The caves here should seem plausible to everyone except extremely pedantic geologists.
If you are or know a pedantic geologist, I'd love to receive any pedantic thoughts which might make the module more interesting.}
They will find these kinds of scenes:

\begin{itemize}
	\item
	In a long cavern, the air grows stale.
	A group of people, breathing in the remainder makes things worse.
	If they carry a torch through that long tunnel, hypoxia sets in, causing drowsiness, and hallucinations.

	The come out safely, they will have to understand their condition, and feel through the long, narrow, cavern, with their fingers and feet.
	\item
	Someone breaks an ankle.
	They can still move, but they will have to take breaks more often, and they want to rest for a night.
	But nobody knows how much further they will have to climb, and the rations are running low.
	Nobody asks the difficult questions, but everyone knows what those questions entail.%
	\footnote{One option would be to split up at a dangerous impasse.
	The wounded might jump into a river in the hopes of getting down quickly.
	Be prepared to freeze the action, and catch up with a secondary troupe sometime later.}
	\item
	The tunnel splits, and someone finds goblin tracks going down one of the passages.
	Players will probably discuss what to do with this information.
	\begin{speechtext}
		``We should follow the tracks, so we avoid dead-ends.''
	\end{speechtext}
	There is a lot to discuss, but limited time.%
	\footnote{If you want to guarantee that this chapter takes precisely one evening, you might want to set a timer.
	If you have a tall candle, you can drive three nails into it, to indicate four \glsfmtplural{interval} passing.
	Of course, if the troupe split into two, you may need to add an extra candle to the evening.}
	\item
	The troupe took a left turn a long time ago, and the tunnel has been getting narrower for a long time.
	If they turn back, they lose hours of progress, and have to attempt another unknown tunnel.
	If they continue, they will have to squeeze through a shaft so narrow that they will struggle to squeeze through while naked.

	\ldots and on the other side, goblins silently wait for an immobile head to poke out as people slither out the crack like worms.
\end{itemize}

Many of these scenes will not appear in the text.
I don't know when the \glspl{pc} will run out of light, or if they will discuss splitting up.
But scenes like these will emerge naturally, by reporting to the players exactly what their character can see at all times.

\end{multicols}

\widePic{unknown/bothy_river}
