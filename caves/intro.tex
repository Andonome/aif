\section{Induction}

\begin{multicols}{2}

\begin{boxtext}
  Sleepy-grey clouds give up the last of their rain.
  Spots of blue start to show.
  Tomorrow will be beautiful, but the town will spend it looking down at the \gls{court} before the hanging.
\end{boxtext}

\noindent
This module presents a baptism-by-napalm for BIND.
Players should each begin with two to three characters.
If that doesn't work for you, add some extra \glspl{guard} to the troupe, and hand out those character sheets to your players.
They will need them!%
\footnote{About 10 characters, divided between the players, should work, but I've seen fewer characters survive.}

Our duplicitous misadventure begins with an apparently easy job.
The troupe must enter goblin-infested caves, and kill the lot.
However, once they enter, a cave-in traps them inside, forcing them to find another route out of the caves.
Players familiar with RPGs will find themselves surprised and confused by the dangers in the these caves, as they are entirely natural, normal caves.%
\footnote{Caveat: I have been down a couple of caves, read a few articles, and asked caving friends, but do not know a lot about caves.
The caves here should seem plausible to everyone except extremely pedantic geologists.
If you are or know a pedantic geologist, I'd love to receive any pedantic thoughts which might make the module more interesting.}
They will find these kinds of scenes:

\begin{itemize}
	\item
  Deep underground, air grows stale.
  People walking through, talking, sighing, and panting, makes the air thinner still.
  If they carry a torch, \glsentrytext{hypoxia} sets in soon, causing drowsiness and hallucinations.
	\item
	Someone breaks an ankle, and they need to slow down and rest more often.
  \Glspl{pc} may agree, while others want to abandon the wounded so the rest can survive.
	Or perhaps the wounded character will attempt a dangerous, but potentially shorter, route.
	\item
	The tunnel splits, and someone finds goblin tracks going down one of the passages.
	Players will probably discuss what to do with this information.
	\begin{speechtext}
		``Let's go the other way, so we can avoid the goblins.''
	\end{speechtext}

	\begin{speechtext}
		``We should follow the tracks, so we avoid dead-ends.''
	\end{speechtext}
	There is a lot to discuss, but limited time.%
	\footnote{If you want to guarantee that this chapter takes precisely one evening, you might want to set a timer.
	If you have a tall candle, you can drive three nails into it, to indicate four \glsfmtplural{interval} passing.
	Of course, if the troupe split into two, you may need to add an extra candle to the evening.}
	\item
	The troupe took a left turn a long time ago, and the tunnel has been getting narrower for a long time.
	If they turn back, they lose hours of progress, and have to attempt another unknown tunnel.
	If they continue, they will have to squeeze through a shaft so narrow that they will struggle to squeeze through while naked.

	\ldots and on the other side, goblins silently wait for an immobile head to poke out as people slither out the crack like worms.
\end{itemize}

Many of these scenes will not appear in the text explicitly.
They come up naturally when \glspl{pc} receive Damage and \glsfmtfullpl{ep}, or when the players notice how little light or food they have left.
As the \gls{gm}, you must keep reinterpreting the rules so the players understand what those \glsfmtfullpl{ep} coins mean in terms of how their character feels.

\end{multicols}
