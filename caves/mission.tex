\section{The Ascent}
\label{goblinsBegin}

\begin{multicols}{2}
\renewcommand\npcsymbol{\glsfmttext{morning}}

\sidequest[Mountain]{Busy Morning}

\noindent
\iftoggle{verbose}{
  Run these three \glspl{segment} in rapid succession.
  They establish the links between the goblin-infested mine and the outside world.
}{
  Goblins banish their criminals to the world above, which has earned them a bad reputation.
  One such group clambers skyward, and soon begin stealing from nearby \glspl{village}.
  The behaviour escalates until they kill a human, and everyone demands the \gls{guard} ascend to slay the lot.
  However, goblins can be extremely dangerous on their home turf.

  \sqpart{Mountain}% AREA
  {Little by Little}% NAME
  {Goblins steal milk from \glsfmtplural{village}}% SUMMARY

  By the mountain's side (in \pgls{village} or on the road), one farmer shouts about her stolen milk, while the other tries to stop her making accusations.

  \begin{speechtext}
    What kind of monster steals milk left outside a house?
    No kind!
    Not a single \gls{monster} which walks, crawls, or flies takes milk.

    Cows?  Yes
    Babies? Absolutely!
    But never \emph{milk}!
  \end{speechtext}

  \paragraph{Examining the area}
  with a \roll{Wits}{Survival} roll at \tn[12] show goblin footprints going up to the mountain.
  A Margin of 2 (i.e. \tn[14]) means the \glspl{pc} can trace the goblins back through the forest, and eventually to their mountainous entrance (\vpageref{goblinCaveEntrance}: `\nameref{goblinCaveEntrance}').

  \sqpart{Town}% AREA
  {\squash~Cooked Innards}% NAME
  {\Glsfmttext{smuggler} speaks about the deer bones found in the forest, covered with goblin tracks}% SUMMARY

  The \glspl{pc} overhear \gls{smuggler}, shouting about goblin tracks around some half-digested deer bones, close to the mountain.
  Everyone in the market wants to know what the \glspl{pc} intend to do about it.

  \sqpart{Mountain}% AREA
  {\Glsentrytext{afternoon}~Dead Horses}% NAME
  {Goblins drop rocks on traders' horses}% SUMMARY

  \begin{exampletext}
    The goblins have discovered a foolproof way to hunt for food.
    They haul large rocks up trees, then along a branch close to the road, and drop the rocks on a trader's horse.

    If one horse dies, the caravan moves on, leaving the goblins a dead horse to eat.
    If more horses die, the caravan never has enough horse-power to pull all the wagons, and has to leave one behind.
  \end{exampletext}

  As the troupe travel close to the mountain, goblins stand on the treetops, ready and waiting with boulders.
  If the \glspl{pc} are with a caravan, the goblins drop rocks on the horses, otherwise, they remain in the shadowy treetops, silently.

  Spotting the goblins requires a \roll{Wits}{Vigilance} roll (\tn[10]).
  But doing something about the problem requires a plan, because hitting the goblins won't be easy.

  If the \glspl{pc} have \glspl{projectile}, one goblin can drop their big rock and take cover each \gls{round}.
  On the treetops, 10~\glspl{step} in the air, is \tn[9], but once they take cover the players will have to roll at \tn[11].

  \goblin

  \goblin

  In addition to the listed rocks, three goblins have a boulder which deals $2D6$ Damage.
  However, they can only drop it on someone directly below them.

  \sqpart{Mountain}% AREA
  {Over the Line}% NAME
  {Goblins kill a shepherd}% SUMMARY

  \begin{exampletext}
    A shepherd saw goblins trying to disturb his cows, and rushed in to stop them.
    The goblins killed him quickly, then fled.
  \end{exampletext}

  Every \glspl{village} round the mountain talks about \composeHumanName's death, and all agree the \gls{guard} must sort the problem immediately.

  One group of \glspl{fodder} have already gone up the mountain, but haven't returned.
  The next session should begin with \nameref{goblinsBegin}, \vpageref{goblinsBegin}.
}

\sqpart[\gls{vlg}\gls{morning}]{Mountain}% AREA
{The Trader's Tragedy}% NAME
{``I was the lone survivor when goblins attacked, and took everyone's money''}% SUMMARY

People here feel scared that goblins will kill their animals, or steal their \glspl{gp}.
So \gls{woetide} is riding North with \glspl{soldier} from the \gls{guard}, so she can witness the \glspl{soldier} taking the \glspl{coin}.

And of course, she asks the \glspl{pc} to join so she can tell them her tragic story.

\begin{speechtext}
  Awful things, goblins.
  I was in a caravan, on the way to start helping with a new \gls{village}, not far from here.
  The goblins managed to haul some rocks up the trees, and as we passed by, they dropped the rocks on the horses' heads.
  Then came those javelins\ldots

  I fell out the wagon and fled.
  Lost my life savings, but at least left alive, unlike most of the others.
\end{speechtext}

\woetide[Travel straight, never curious]

\begin{speechtext}
  Here we are at last -- Stoatfen \Gls{bothy}, where we part ways.
  \Pgls{susjot} will be along soon, I expect.
  He works around here.
\end{speechtext}

Soon the caravan reaches Stoatfen~\Gls{bothy}.
The traders get out and ask the characters to perform the traditional \gls{guard} duties:

\begin{itemize}
  \item
  Check inside the \gls{bothy} for \glspl{monster}.
  \item
  Check the perimeter for \glspl{monster}.
  \ifnum\value{temperature}<2
    \item
    Check the \gls{bothy} has firewood (it does not, but does have an axe).
  \fi
\end{itemize}

\Gls{woetide} stretches her legs while the \glspl{soldier} tend to her three horses while the \glspl{pc} attend to their duties.

\iftoggle{verbose}{
  This makes a good opportunity to introduce new players to the system, by making a few rolls.
}{}

\paragraph{Shortly after,}
the wagon moves on.

\sqpart[\gls{morning}]{Mountain}% AREA
{The Innocent Man in the Shadows}% NAME
{\Glsfmttext{fodder} Scarstain fled from the goblin caves, and wants to know if he can return}% SUMMARY

\begin{exampletext}
  Last week, Scarstain entered the \gls{guard} after smuggling swords into a town for a friend, so they didn't have to pay tax (or explain the \glspl{weapon}).
  Four days ago, \pgls{susjot} ordered him and all the new \glspl{fodder} to enter an abandoned mine, kill the goblins, and return with the treasure the goblins stole from \gls{woetide} and the rest of that caravan.

  The goblins killed most of the new \glspl{guard}.
  Only Scarstain made it out alive.
\end{exampletext}

Scarstain knows he should have returned with goblin heads, and doesn't want to report mission failure.
He waits near the \gls{bothy}, in the shadows of the forest, for the right moment to talk with someone who might be sympathetic.

\begin{boxtext}
  Snapping twigs.
  Gentle footsteps approach.
\end{boxtext}

\humanthief[\NPC{\glsentrysymbol{sylf}\M\Hu}{\Glsfmttext{fodder}~Scarstain}%
  {mid-forties, sunburnt face, and flustered}% DESCRIPTION
  {grimaced-smile}% MANNERISM
  {to tell everyone he's not a criminal}]% WANTS

\begin{boxtext}
  A sunburnt man steps into a ray of Sunlight, mostly bald and half-grey on what remains, wearing a black cassock just like yours.

  \begin{speechtext}
    Hi.
    My name is Scarstain.
    \Gls{fodder}~Scarstain.

    \ldots

    \Pgls{susjot} sent the five of us into the old mine, to kill the goblins and return with the treasure they stole.
    But we didn't get the treasure.
    We died.
    Then I ran away.
    A trader told me they're looking for me.
    She said I should run away.
    Can you check if it's safe for me to come back?
  \end{speechtext}

  The sunburnt man pulls a smiling grimace, as if forcing himself to stay hopeful.
\end{boxtext}

The `trader' who told Scarstain that he's a wanted fugitive is \gls{woetide}.
Scarstain does not know anything about her, but can describe her big eye-bags.

\paragraph{If anyone asks about the details at the mines,}
Scarstain explains a little, but won't give more details until someone reassures him that nobody thinks he's guilty.

\begin{speechtext}
  \begin{itemize}
    \item
    The entrance is easy.
    It's a hole in the ground, so bring something to get down with -- a rope, a stake, and a mallet.
    \item
    The goblins won't fight, they just run away.
    \item
    The darkness in caves isn't like the darkness at night.
    At night, the sky is dark, but in a cave, the darkness stands right next to you.
    \item
    The ground is rough -- you can't run anywhere.
    That means the goblins can stand back and throw things at you.
    \item
    And if you carry \pgls{torch}, they can see you, but you can't see them.
    And when you put it out, they can hear you, and you still get hit by rocks.
    \item
    I swear I heard \gls{yonder}'s bell down there, calling everyone downwards into the dark.
  \end{itemize}
\end{speechtext}

\paragraph{Reassuring Scarstain}
requires a \roll{Charisma}{Empathy} roll at \tn[10].
If that succeeds, he offers more details.

\begin{speechtext}
  \begin{itemize}
    \item
    \composeHumanName\ died falling through a rotten bridge that the miner's left.
    Look out for a hole in the ground, and tread carefully around it, one at a time.
    
    \hint{\autopageref{firstBridge}, area \ref{firstBridge}}
    \item
    The goblins left a rope for us to get down.
    We thought it was a trap, but it's not.
    
    \hint{\autopageref{ropeDown}, area \ref{ropeDown}}
    \item
    We went down a tunnel with a dead goblin, then Tukala died.
    He was my little gnome-buddy, and he just lay down and died.
    I didn't see anything happen to him, he just died.
    
    \hint{\autopageref{standingTunnel}, area \ref{standingTunnel}}
    \item
    The next bit, we had to crawl under a narrow pass.
    They were waiting for us at the other side, throwing rocks and javelins as we came up.
    That's when I ran back up.
    
    \hint{\autopageref{hungryHall}, area \ref{hungryHall}}
    \item
    I waited a while, but nobody else came back up.
    Five of us went in.
    All four, dead.
    \item
    I'm never going back there again!
  \end{itemize}
\end{speechtext}

\sqpart[\gls{afternoon}]{Mountain}% AREA
{Gold in the Mines}% NAME
{The \glsfmttext{jotter} demands the troupe go to the mines and kill the goblins}% SUMMARY

\begin{exampletext}
  Two days ago, \gls{woetide} informed \gls{susjot} that three \glspl{guard} were walking down the road, near the mine, carrying a heavy chest.
  \Gls{susjot} has figured they took the chest of \glsentrylongpl{gp} and decided to take it, and live beyond the \gls{edge} (effectively making them bandits).
\end{exampletext}

\begin{boxtext}
  Hooves and wheels smash through the
  \ifnum\value{temperature}=0%
    frost
  \else%
    mud
  \fi.
  Behind, six \glspl{guard}, all in black, struggle to keep up.
  The coachman arrives with a writing tablet under his arms and an angry scowl.
  \begin{speechtext}
    \begin{itemize}
      \item
      Are you the new \glspl{fodder}?
      \item
      Did you restock the logs for the \gls{bothy}?
    \end{itemize}
  \end{speechtext}
\end{boxtext}

Once \gls{susjot} feels satisfied with the state of Stoatfen~\Gls{bothy}, he gives the troupe their first real mission.

\begin{boxtext}
  \begin{speechtext}
    It's time to go up the mine, and kill some goblins!

    Listen, it's an easy job.
    Just go and kill the goblins, and take any remaining \gls{coin} they might have.
    It's just like stealing from an ugly baby, okay?

    You may keep one \glsentrylong{sp} for each goblin head you return with.

    The last lot of \glspl{fodder} who went up just left with the gold, and didn't even kill the goblins.
    So now it's no more mister nice-\gls{jotter}.
    Now anyone who comes back without a goblin head gets sent back to the \gls{court} for a good neck-stretch, okay?
  \end{speechtext}
\end{boxtext}

\paragraph{If the \glspl{pc} ask for more supplies,}
they can make a \roll{Charisma}{Survival} check to convince \gls{susjot} to give them some of the equipment from his cart.
\iftoggle{verbose}{%
  Everyone uses the same \gls{natural}.
}{}

\begin{boxtable}[l|L]
  \textbf{\glsentrytext{tn}} & \textbf{Equipment} \\
  \hline
  4 & 50' of rope (only 2 available) \\
  \ifnum\value{temperature}=0
    5 & Warm clothes \\
  \fi
  6 & Mallet \\
  7 & two more \glspl{torch} \\
  9 & One extra day's \glsfmtplural{ration} (\rations, or \rations) \\
  10 & Leather armour (\glsfmttext{dr}~3, \glsfmttext{covering}~3, fits anyone with 6-8~\glsfmtplural{hp}) \\
  11 & One more \glsfmttext{torch} \\
  12 & Shortsword (2 available) \\
  13 & Chainmail (\glsfmttext{dr}~4, \glsfmttext{covering}~3, fits anyone with 7-9~\glsfmtplural{hp}) \\
\end{boxtable}

\paragraph{If the \glspl{pc} ask about the missing \glspl{fodder},}
\gls{susjot} questions them about what they know.

\Gls{soldier}~\composeHumanName\ happily mentions the reward for bringing in the deserters: 5~\glspl{gp} dead, 20~\glspl{gp} alive.

\susjot

\stopcontents[sq]

\end{multicols}

