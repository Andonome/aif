% The 'random' human names in all the modules will be identical, unless we push
% the counters which determine the names.
\setcounter{humanNameNo}{\day}
\setcounter{humanNameSuffNo}{\month}
\randomize

\newcommand\scaredFodder{Scarstain}

%%% Maps %%%

\newcommand\caveMapOut[1][b]{
  \mapPic{#1}{cave_8}{
    \ref{olmSwarm}/14/31,
    \ref{mushroomsTunnel}/25/60,
    \ref{basiliskCave}/40/30,
    \ref{goblinHole}/25/90,
    \ref{sunRoof}/62/55,
    \ref{crackExit}/80/60,
    \ref{windyPassage}/90/60,
    \ifnum\value{temperature}=0{\normalsize\roll{Dexterity}{Stealth} (\tn[12])}\fi/50/05,
    \ifcase\value{temperature} Wolves\or\Glsentrytext{woodspy}\else\Glsentrytext{crawler}\fi/85/05,
  }
}


\newcommand\metroMapNames[1][b]{
  \mapNotes[\normalsize]{
    \nameref{start}/12/97,
    \nameref{falseExit}/22/65,
    \nameref{firstBridge}/05/65,
    \rotatebox{-59}{\small\nameref{ropeDown}}/18/48,
    \nameref{standingTunnel}/33/00,
    \nameref{wormTunnel}/32/39,
    \rotatebox{-59}{\small\nameref{gentlePassage}}/11/45,
    \nameref{cliffDive}/11/20,
    \nameref{secondBridge}/23/20,
    \nameref{waterMaze}/42/20,
    \nameref{hungryHall}/49/40,
    \nameref{gas}/57/20,
    \nameref{darkTunnel}/53/05,
    \nameref{umberHulk}/66/05,
    \nameref{blindFish}/74/18,
    \large\nameref{goblinHole}/50/80,
    \nameref{stalagmites}/75/35,
    \nameref{mushroomsTunnel}/54/66,
    \nameref{basiliskCave}/86/65,
    \nameref{olmSwarm}/74/51,
    \nameref{sunRoof}/81/75,
    \nameref{crackExit}/90/80,
    \nameref{windyPassage}/99/48,
  }

  \widePic[#1]{Irina/metro_cave}
}

\newcommand\metroMapDistances[1][t]{
  \mapNotes[\normalsize]{
    %\tiny 1/29/90,
    %\tiny 2/30/90,
    %\tiny 3/31/90,
    %\tiny 4/32/90,
    %\tiny 5/33/90,
    %\tiny 2/29/88,
    %\tiny 4/29/86,
    %\tiny 6/29/84,
    %\tiny 8/29/82,
    \Large\ref{start}/16/95,
    \rotatebox{60}{1 mile}/15/85,% 1 mile to bridge
    $\frac{1}{2}$ mile/21/84,% 1/2 mile
    \Large\ref{falseExit}/17/73,
    \Large\ref{firstBridge}/09/63,
    \rotatebox{-60}{1 mile}/18/48,% 1 to ropeDown
    \Large\ref{ropeDown}/16/39,
    \Large\ref{secondBridge}/21/20,
    \rotatebox{-60}{$\frac{1}{2}$ mile}/11/45,% 1/2 mile to cliffDive
    \Large\ref{cliffDive}/12/21,
    \Large\ref{standingTunnel}/31/00,
    \Large\ref{wormTunnel}/31/40,
    \rotatebox{-60}{2 miles}/24/10,% 2 to waterMaze
    \rotatebox{60}{1 mile}/24/40,% 1 to waterMaze
    \Large\ref{gentlePassage}/10/39,
    \Large\ref{waterMaze}/41/19,
    \Large\ref{lakeStop}/53/50,
    \Large\ref{hungryHall}/46/19,
    $\frac{1}{2}$ mile/53/33,% 1/2 mile to gas
    \Large\ref{gas}/56/21,
    \Large\ref{darkTunnel}/59/10,
    1 mile/50/12,% 1 to umberHulk
    \ref{umberHulk}/62/04,
    \Large\ref{blindFish}/61/22,
    \Large\ref{stalagmites}/63/33,
    \rotatebox{60}{1 mile}/79/36,% 1 mile to olmSwarm
    \rotatebox{-60}{1 mile}/63/51,% 1 to mushroomsTunnel
    \Large\ref{mushroomsTunnel}/59/64,
    \Large\ref{goblinHole}/51/80,
    \Large\ref{olmSwarm}/77/48,
    1 mile/84/57,% 1 mile to basiliskCave
    1 mile/68/78,% 1 mile to sunRoof
    \Large\ref{basiliskCave}/82/65,
    \Large\ref{sunRoof}/73/73,
    1 mile/82/78,% 1 mile to crackExit
    \rotatebox{-60}{1 mile}/91/70,% 1 to windyPassage
    \Large\ref{crackExit}/87/82, % 1 mile to windy Passage
    \Large\ref{windyPassage}/96/46,
  }


  \widePic[#1]{Irina/metro_cave}
}

%%% Commentaries %%%

\newcommand\playCommentaryCaveIn[1][b]{
  \playCommentary[#1]{
    \begin{description}
      \item[\Gls{gm}:]
      The ceiling falls apart, like
      \ifcase\value{temperature}%
        snow falling off a roof.
      \or%
        a chestnut tree when it's kicked.
      \else%
        a broken biscuit.
      \fi%
      \twoDice{\value{r12}}

      The falling rocks will inflict \arabic{r12}~Damage, but you can avoid by rolling \roll{Speed}{Athletics} at \glsfmtlong{tn}~3, \ldots
      \item[Player 1:]
      Why even roll for that?
      \twoDice{9}
      I pass.
      \item[\Gls{gm}:]
      \Glsfmtlong{tn}~4 for you,
      \randomize\twoDice{12}
      or take 12~Damage,

      \glsfmtlong{tn}~5 for you, \ldots
      \item[Player 2:]
      Ah, okay.
      \twoDice{3}
      That's a failure from me.
      \item[Player 3:]
      \twoDice{5}
      I hit the \glsfmtlong{tn} exactly.
      What happens?
      \item[\Gls{gm}:]
      You see a companion's head crushed by a falling rock for 12~Damage.
      The skull shatters across the floor.
      I suppose that explains why your character hesitates, going nowhere.
      You can roll again at the same \glsfmtlong{tn}.
      \item[Player 2:]
      I'm dead!?
      Already?
      \item[\Gls{gm}:]
      You all have multiple characters, remember?
      Lets roll for them \ldots \glsfmtlong{tn}~6.
    \end{description}
  }{
    Rolling Damage before \pgls{action} is not in the rules, or against the rules.
    It just helps to let the players know what's happening.

    Hopefully everyone in your group elects to have a second character, because the dice can be fickle.
  }
}

\newcommand\playCommentaryDust[1][b]{
  \playCommentary[#1]{
    \begin{description}
      \item[\Gls{gm}:]
      The sound of coughing takes over the cave as dust fills the air.
      Everyone gain one \glsentrylong{ep}!

      \item[Player 2:]
      Can't I just hold my breath?

      \item[\Gls{gm}:]
      After sprinting, and the shock of the ceiling collapsing?
      Maybe?
      \roll{Strength}{Athletics}, \glsentrylong{tn}~10.

      What do you do\ldots in the darkness?
      \item[Player 2:]
      $\twoDice{9} + 2$\ldots
      hold my breath, apparently.
      \item[Player 3:]
      Time to get out of here, while holding my breath.
      \twoDice{6}
      \item[\Gls{gm}:]
      That's a failure, so that character gains \pgls{ep}.

      The passage out is quite narrow.
      Roll \roll{Dexterity}{Caving}, \glsentrylong{tn}~6 plus your Strength.
      \item[Player 2:]
      So \glsentrylong{tn}~8?
      Okay, \twoDice{5}.
      That's a `no'.
      \item[Player 3:]
      Okay, I'll go out.
      \item[\Gls{gm}:]
      I'm afraid you can't while someone is struggling to squeeze through.
      The first character in the exit-hole can spend \pgls{ap} to try again if you like, or spend \pgls{ap} to crawl back.
      \item[Player 2:]
      Okay, I'll go back.
      \item[Player 3:]
      Good,
      \twoDice{10}
      looks like I'm out.
      And the other character\ldots
      \twoDice{2}
      \ldots not so much?
      \item[\Gls{gm}:]
      That's the end of the \gls{round}, so everyone holds their breath at \glsentrylong{tn}~10, or gain another \glsentrylong{ep}.
    \end{description}
  }{
    If players spend a while discussing the pros and cons of the optimal order to exit the cavern, remember that speech during \pgls{round} costs \pgls{ap}.
  }
}

\newcommand\playCommentaryAftermath[1][b]{
  \playCommentary[#1]{
    \begin{description}
      \item[\Gls{gm}:]
      Who had \pgls{torch} lit?
      I don't think anyone did?
      \item[Player 1:]
      No, but I'll light one now.
      \item[\Gls{gm}:]
      Pulling a nice, dry, tinder-box from your bag, you get to work.
      Soon the \gls{torch}-light illuminates a low ceiling -- darkness hangs around the edges.
      \item[Player 1:]
      Looks like we'll need to explore.
      \item[\Gls{gm}:]
      Out of interest, how many \glspl{ration} does everyone have?
      How many tinder boxes?
      \item[Player 2:]
      How long are we going to be in here?
      \item[\Gls{gm}:]
      Well it's not for me to say, really.
    \end{description}
  }{
    Noting that the tinder-box is `dry' helps remind people that it needs to be dry, and won't work as well if the character holding it goes underwater.
    That's the kind of detail which the characters must find obvious, but the players might not.
  }
}

\newcommand\playCommentaryPuddle[1][b]{
  \playCommentary[#1]{
    The troupe have emerged at the \nameref{waterMaze} (\vpageref{waterMaze}).

    \begin{description}
      \item[Player 2:]
      Okay, I'll jump in the puddle.
      \item[\Gls{gm}:]
      Do you take your backpack off?
      \item[Player 2:]
      Yes.
      What do I see?
      \item[\Gls{gm}:]
      One moment\ldots

      Okay, guys!
      You hear the splash as he goes in.
      The \ifcase\value{temperature}%
        freezing
      \or%
        cold
      \else%
      \fi
      water ripples against your shins, then the cavern returns to the feint noise of the nearby river.

      What do you do?
      \item[Player 1:]
      Aren't we going in?
      \item[\Gls{gm}:]
      Yes, if you want.
      Are you going into the water?
      \item[Player 3:]
      The goblin must be on the other side, so doesn't that mean he'll have to fight alone?
      \item[\Gls{gm}:]
      What is your character doing?
      \item[Player 1:]
      If there's another ambush like the last one, someone else could die.
      We should all go in quickly.
      \item[\Gls{gm}:]
      You've spent a moment talking already, but you can still go in now, if you want to.
      \item[Player 3:]
      Okay, I'll go.
      I'm going in now.
    \end{description}
  }{
    For some extra tension, let a character enter the water and then focus on the other players.
    Let the troupe decide how long they will wait, in the darkness, before someone else enters.
    Of course, if the others say `one hour', once you resolve the first character's actions (the one in the black waters), you may find that the troupe did not in fact wait for an hour, as the first character may return after a few minutes to tell everyone how to move through.
  }

}

\newcommand\playCommentaryDrowning[1][t]{
  \playCommentary[#1]{
    \Pgls{pc} has entered the \nameref{waterMaze} (\vpageref{waterMaze}).

    \begin{description}
      \item[Player 2:]
      Okay, so what do I see?
      \item[\Gls{gm}:]
      Nothing -- you need to feel.
      The water has a metallic cold, which pulls heat from your skin.
      You feel the hard, slippery rocks around, and have to descend a little before finding a passage.
      This seems to be the only passage.
      It's about the height of a crouching child.
      \item[Player 2:]
      Okay, so I'll go through, carefully.
      \item[\Gls{gm}:]
      Going carefully, you have 4~\glspl{ap}, so you cover 3~\glspl{step} this \gls{round}.
      Put down a temporary \gls{ep} (you can remove it once you can breathe again).
      Swimming onward, you cover another 4~\glspl{step} and find a side tunnel on the left.

      Above, the tunnel opens upwards.
      Put down another \gls{ep}.
      Which way do you go?
      \item[Player 2:]
      I'll move up.

      You find a sharp, spiky ceiling as you swim up and round.
      You think the tunnel moves horizontally from here, but you can't be sure of which way is up or down in the freezing, dark waters.
      Take another \gls{ep}.
      Do you continue?
      \item[Player 2:]
      So if I have 3 slots without \pgls{ep}, that means I can go another 3~\glspl{round}\ldots so then how far have I come?'
      \item[\Gls{gm}:]
      You're unsure.
      Take \pgls{ep}.
      \item[Player 2:]
      Again?
      Okay, I'm going back!
      \item[\Gls{gm}:]
      Turning round, your head knocks into something new, something hard and floating in the dark.
      \item[Player 2:]
      I'll push it aside, and go back to the air.
      \item[\Gls{gm}:]
      Roll Strength plus Brawl, \glsentrylong{tn}\ldots what's \underline{your} Strength and Brawl?
      \item[Player 3:]
      Oh, it's me?
      Okay, it's {\ldots\footnotesize one and one\ldots} two.
      \item[\Gls{gm}:]
      \Glsentrylong{tn} 9.
    \end{description}
  }{
    This simple maze is the same shape as the letter `H', but in the frozen darkness, it provides a nasty enemy.
    The player will have to decide who goes next, and plan their routes ahead.
    Each one will enter blind, and have to remember where they should scout next.
  }
}

\newcommand\playCommentaryTeamwork[1][b]{
  \playCommentary[#1]{
    The troupe want to find an exit in the `\nameref{hungryHall}' (\vpageref{hungryHall}), but first, they want to rest.
    \begin{description}
      \item[\Gls{gm}:]
      Sitting by the fire feels nice, and soon everything feels dry again, though you do feel light-headed.
      \dicef{\value{r6}}
      Everyone remove \pgls{ep} and regain \arabic{r6}~\glspl{fp}.

      You search together for a way out in the darkness, feeling dizzy as you stand.
      Roll \roll{Speed}{Caving} at \glsentrylong{tn}~nine.
      \item[Player 1:]
      \twoDice{3}
      I have minus one Speed, so the total is `two'.
      \item[Player 2:]
      I have plus four in total so\ldots
      \twoDice{10}
      Got it!
      \item[\Gls{gm}:]
      I'm afraid we just use the same \gls{natural} for everyone, so your total is seven.
      \item[Player 3:]
      I'm going to put out the fire before we suffocate.
      \item[\Gls{gm}:]
      Searching in the dark is \glsentrylong{tn}~fourteen.
      \item[Player 1:]
      Okay, so let's not do that.
      What if we split up and look?
      \item[\Gls{gm}:]
      Splitting up means you're working together, so that's \pgls{bandAct}.
      The next character adds half their score to the total.
      \item[Player 3:]
      I have a plus one in total.
      \item[\Gls{gm}:]
      Round it up to plus one.
      \item[Player 4:]
      I have a plus two.
      \item[\Gls{gm}:]
      The next person adds a quarter of their bonus, so that's another plus one.
      That's a tie.
      \item[Player 3:]
      I'm putting the fire out by handing everyone a flaming log.
      We've all take our armour off, so we can carry them in the armour.
      \item[\Gls{gm}:]
      It's only embers, but it's worth another plus one, so that beats the \glsentrylong{tn}.
    \end{description}
  }{
    \Glspl{bandAct} can be ordered any way the players like, which often lets them pull out another `plus one', if they work together.
  }
}

\newcommand\playCommentaryRestI[1][t]{
  \playCommentary[#1]{
    \begin{description}
      \item[\Gls{gm}:]
      The \gls{torch} sputters and grows dim.
      That's another three miles travelled, and the end of the \gls{interval}.
      There's one \glsfmtlong{mp} available\ldots wait nobody needs \glsfmtlongpl{mp} here\ldots
      And everyone regain
      \dicef{\arabic{r6}}
      \arabic{r6}~\glspl{fp}.
      \item[Player 1:]
      Fabulous.
      \item[Player 2:]
      That's the third \gls{interval} in here.
      You said \pgls{interval}'s six hours, so it'll be dark outside again.
      I don't have any food left.
      \item[Player 3:]
      I don't either.
      We should hurry it up a bit.
      \item[\Gls{gm}:]
      That was the last \gls{torch}, wasn't it?
      If you're in the dark, how will you find your way?
      \item[Player 1:]
      Use the sword-scabbard as a walking stick?
      \item[\Gls{gm}:]
      Okay, but everyone will still slow down to feel the next footstep carefully, and travel just a mile over the \gls{interval}.
      \item[Player 1:]
      One?
      No way!
      We're going at least another three.
      \item[\Gls{gm}:]
      Okay, if you're leading then that's a \roll{Dexterity}{Caving} at \glsentrylong{tn}~9, but you should add +1 for the walking stick.
      \item[Player 1:]
      Okay -- I knew the sword would come in handy.
      \twoDice{7} + 2 means a tie.
      What happens?
      \item[\Gls{gm}:]
      For now?
      You press on.
      And after a mile, you hear movement ahead, small and indistinct, but real\ldots
    \end{description}
  }{
    Every \gls{interval}, the troupe gamble time against safety.
    If you're not walking all the time, are you even in a fantasy land?
  }
}


\newcommand\playCommentaryRestII[1][t]{
  %%%%%%%%
  \playCommentary[#1]{
    \begin{description}
      \item[\Gls{gm}:]
      The passage out smells less retched.
      It has a glimmer of light.
      It's about the end of the \gls{interval}\ldots what am I forgetting?
      \item[Player 1:]
      \Glsentrylongpl{fp}?
      \item[\Gls{gm}:]
      No, before that -- the \gls{blackWalking}.
      $\dicef{6} + 1$
      Rushing towards the light, you miss a rock and bang your head for seven damage.
      \item[Player 1:]
      Seven?
      I'm down to my last two \glsentrylongpl{hp}, and I have four \glsentrylongpl{ep}, and three items.
      \item[\Gls{gm}:]
      That's a minus five penalty to everything.
      That's a pretty bad state.
      Can't move or think properly.
      Killer migraine.
      \item[Player 2:]
      You'll be fine.
      There's the Sun.
      We're out, right?
      \item[\Gls{gm}:]
      Stepping out into the light, you're at the bottom of a deep shaft\ldots
    \end{description}
  }{
    The players may have forgotten about the goblin heads at this point, but best not to tell them.
  }
}
