\section{Example of Play}

\begin{multicols}{2}

\begin{description}\sf
  \item[Player 3:]
  Guys, it's been twenty minutes.
  Can  we just pick a tunnel?
  \item[\Glsentrytext{gm}:]
  You're the tie-breaker in a vote, so vote.
  \item[Player 3:]
  Okay, em---`left'.
  \item[Player 2:]
  We're going to die down there\ldots
  lead the way then.
  \item[Player 3:]
  Sure, whatever.
  I crawl into the crack in the wall.
  \item[\Glsentrytext{gm}:]
  In you get, inching round the corner.
  The rest of the troupe follow behind as you inch round the next tight squeeze.
  \item[Player 3:]
  How long does the narrow bit go on for?
  \item[\Glsentrytext{gm}:]
  You just keep going sideways.
  As the wall starts leaning to the side, you feel yourself leaning backwards to accomodate the slant, while you walk sideways.
  \item[Player 3:]
  That sounds like a bad idea.
  Probably best to turn around and lean down, if I have to lean one way or another.
  \item[\Glsentrytext{gm}:]
  You can't turn around.
  The tunnel doesn't have enough room -- everyone can only squeeze sideways, and you're starting to lean back more as the wall slants more.
  \item[Player 3:]
  Okay, I'll get out and walk back in the correct way round.
  \item[\Glsentrytext{gm}:]
  You can't.
  The rest are behind you, and you can't go through them.
  \item[Player 3:]
  Guys, can you say your characters are going back?
  \item[Player 2:]
  Can we just see if this tunnel has another side?
  It doesn't matter what way you're facing.
  \item[\Glsentrytext{gm}:]
  The tunnel stops.
  You stop.
  You have no more room for your head.
  \item[Player 1:]
  They can't have gone the other way.
  `No room for your head'\ldots
  \item[Player 3:]
  Can I feel about with my legs?
  \item[\Glsentrytext{gm}:]
  Yes -- the tunnel goes on, but with much less ceiling-room.
  You can enter if you go right-leg first.
  \item[Player 3:]
  Okay\ldots in we all go\ldots right legs first.
  \item[Player 2:]
  Nah, you've moved ahead, so there's more room.
  I'm going head-first.
  \item[\Glsentrytext{gm}:]
  Okay, that works!
  \item[Player 3:]
  Total dick.
\end{description}

\bigLine
\vspace{2em}
\noindent
The real purpose of this module is to give players their first claustrophobic panic-attack, from the comfort of the table.
You might want to ask the players how claustrophobic they are, on a scale of 1 to 10, before the session starts.
The sweet-spot is probably 3 to 6.%
\footnote{Never let it be said that safety tools were not provided.}

\bigLine

\begin{description}\sf
  \item[\Glsentrytext{gm}:]
  The tunnel continues for some time, then widens.
  The stone feels like slate in places -- smooth, but brittle, and prone to fracturing.
  Moving too fast makes the air rocky.
  \item[Player 1:]
  You mean `dusty'?
  \item[\Glsentrytext{gm}:]
  No it's much thicker than that.
  The air gets `rocky', and the tunnel becomes narrower.
  Anyone with a Strength Bonus of +3 will have serious trouble moving.
  \item[Player 3:]
  Why did it have to be me?
  \item[\Glsentrytext{gm}:]
  Because that character is a large, strong, man, who is wearing padded leather armour.
  \item[Player 3:]
  Okay, I'll remove it.
  \item[\Glsentrytext{gm}:]
  How?
  I mean, you can try with a Dexterity and Athletics roll at \glsentrylong{tn}~10.
  \item[Player 3:]
  Okay that's \twoDice{8}\ldots not quite enough.
  I'll take \pgls{restingaction}.
  Can we rest in here for the evening?
  \item[\Glsentrytext{gm}:]
  Maybe?
  It's wet and sandy, and everyone's lying at a weird slant.
  Someone can try to cobble together some basic comfort with Intelligence and Crafts, \glsentrylong{tn}~10?
  \item[Player 1:]
  I can get that one\ldots or if I fail then maybe I could just cut your armour off with a knife?

  How long does this tunnel go on for?
  \item[\Glsentrytext{gm}:]
  Nearly a mile, but I bet it feels like more!
  \item[Player 1:]
  I bet it does!
  \twoDice{9}
  Okay, I beat the \glsentrylong{tn}, I'll get everyone ready for a nap.
  \item[\Glsentrytext{gm}:]
  Everyone settles for \pgls{interval}, then continues worming their way through the darkness.
  Soon the air ahead smells more fresh.
  The sound of running water joins, and the tunnel ahead sounds \emph{wide}.
  \item[Player 3:]
  Finally!
  Right, my first character squeezes out.
  \item[\Glsentrytext{gm}:]
  Your first character is stabbed in the neck.
  \item[Player 3:]
  What!?
  \item[\Glsentrytext{gm}:]
  That's $\dicef{4}+3$\ldots 7 Damage.
  \item[Player 3:]
  That's all my \glsentrylongpl{fp} gone, and I lost 3~\glsentrylongpl{hp}.
  What stabbed me?
  I'm crawling out!
  \item[\Glsentrytext{gm}:]
  Luckily, you blocked the blade with your shoulder-blade.
  You start crawling out, but currently the character lies prone, so you need to spend \pgls{ap} to right yourself.
  The creature attacks again\ldots
  \item[Player 3:]
  I have 4~\glsentrylongpl{ap}, so I'm going first.
  I'll spend the first \glsentrytext{ap} to move out, and another to stand up.
  \item[\Glsentrytext{gm}:]
  A swish in the dark comes.
  Roll Brawl at \glsentrylong{tn}~10.
  \item[Player 3:]
  No `Melee'?
  \item[\Glsentrytext{gm}:]
  `Melee' is \pgls{skill} for weapons.
  You don't have yours ready in-hand.
  The character can only fight with arms and legs.
  \item[Player 3:]
  If I hit a goblin, I'm going to grab the thing, and throw it into that water.
  \twoDice{6}
  \item[Player 3:]
  Unfortunately, that is a miss,
  \item[\Glsentrytext{gm}:]
  \dicef{4}
  and a blade hits stabs into your guts.
  5~Damage.
  \item[Player 1:]
  I'm crawling out now.
  Do I still have all my \glsentrylongpl{ap}?
  \item[\Glsentrytext{gm}:]
  No, you emerged after the \gls{round} had gone down to 2, so you have one \glsentrylong{ap} left.
  \item[Player 1:]
  I'll yank the next person out, so next \gls{round}, there will be two of us standing.
  \item[\Glsentrytext{gm}:]
  And on the next \gls{round} you hear a `pad-pad-pad` away from the running water, and it turns into a `splash-splash-SPLOOSH'.
  Speckles of water shower over your faces.
\end{description}

\bigLine
\vspace{2em}
\noindent
Do not, dear reader, give me that look.
The module did not promise fairness; it promised a realistic cave and offered each player multiple characters.

Besides -- look at the situation from the goblins' point of view.
Why should they fight fair?
If they were the \glsentrylongpl{pc}, you wouldn't complain about them doing something devious.

\bigLine

\begin{description}\sf
  \item[Player 3:]
  Do I get a `death save'?
  \item[\Glsentrytext{gm}:]
  Sort of?
  It's more of a first-aid attempt, but nobody can help in the dark, and getting a torch lit would take time.
  You can still try to roll at \glsentrylong{tn}~7, without any help, to see if the character lives.
  \item[Player 3:]
  \twoDice{8}
  Made it.
  Sort of.
\end{description}

\bigLine
\vspace{2em}

\noindent
Now the real drama begins.
What will the other \glspl{pc} do?
Will they let the wounded member die, or try to carry him?
If they carry him, will they be prepared to drop him if something dangerous approaches?
Will they take his food?

Players having a second character can really help set the tone here.
If two players debate abandoning or looting a third player's character, that's not fun.
But if the third player has a second \gls{pc}, who is currently voting to loot the body and move on, then that seals the drama `in-game'.

\bigLine

This module (and the rest) comes with the necessary resolution rules at the start (\vpageref{mechanicsGlossary}), but nothing more.
The \textit{Core Rules} book has more detail, but everything there remains a interpretation of these basic rules.
For example, the final roll to see if a character survives after losing all \glspl{hp} is just a Medicine roll for another character.

\subsubsection{The Goblin Blocks}
appear \vpageref{goblinBlocks}, so you can track how many remain alive, and so you can keep the \glspl{statblock} to hand at all time.
Each time the troupe encounter the goblins, they should encounter half of the total goblins.
At the end of the encounter, the goblins flee.

\end{multicols}
