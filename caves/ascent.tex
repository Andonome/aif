\section{The Ascent}

\begin{multicols}{2}

\subsection[The Hanging Procession]{\gls{justice}~The Hanging Procession~\gls{justice}}

\begin{boxtext}
  This morning, the \gls{bothy} hosts two traders, a visiting \gls{jotter}, a dozen new recruits, and ten \glspl{sunGuard}.
  The \glspl{sunGuard} have travelled from a town's \gls{court}, all the way here to hang half the new recruits, for dereliction of duty.
  You can hear them complaining bitterly about all the muck on their white tabard.

  A shutter on the \gls{bothy}'s upper floor opens, and the \gls{jotter} shouts to you to come up for orders.
  The rest of the recruits start lazily making their into the \gls{bothy}.
\end{boxtext}

\Pgls{redcoin} explains the following points:

\begin{itemize}
  \it
  \item
  Goblins have been thieving, looting, and even killing around a number of nearby \glspl{village}.
  \item
  Those last recruits went up, but returned as cowards, without a single goblin head to show for their efforts, so the \gls{sunGuard} will take them to the \gls{court} of the nearest town, for justice.
  \item
  A \gls{guard} ranger has already found an entrance to the goblins' lair.
  You must find this spot, enter their cave, kill all of them, and return with their heads.
  \item
  Everyone can pick up \ifnum\value{temperature}=0 four \else three \fi days' rations, and this map.
\end{itemize}

\noindent
As he explains all this, the \glspl{sunGuard} take the doomed prisoners down, and prepare to leave.
The \glspl{pc} hear all this from \gls{redcoin}'s office, and witness the rope-bound prisoners from the window.

\humansoldier[\npc{\T[8]\Hu}{\glsentrytext{sunGuard}}]

\paragraph{If the \glspl{pc} argue,}
\gls{redcoin} will have none of it.
He's busy, and wants the \glspl{pc} to get going, quickly.

\paragraph{If the \glspl{pc} try to plead for the lives of the ex-\gls{guard} prisoners,}
nobody will listen.

\paragraph{Asking the prisoners about what happened}
requires a \roll{Charisma}{Tactics} roll (\tn[8]) (or whatever Skill seems appropriate), to convince the \glspl{sunGuard} to allow the conversation.
They can tell the \glspl{pc} the following:

\begin{itemize}
  \it
  \item
  The entrance easy.
  It's a hole in the ground, so bring something to get down with -- a rope, a stake, and a mallet.
  \item
  The goblins won't fight, they just run away.
  \item
  The darkness in caves isn't like the darkness at night.
  At night, the sky is dark, but in a cave, the darkness stands right next to you.
  \item
  The ground is rough -- you can't run anywhere.
  That means the goblins can stand back and throw things at you.
  \item
  And if you carry a torch, they can see you, but you can't see them.
  And when you put it out, they can hear you, and you still get hit by rocks.
  \item
  Better to die in a town, than down some horrible cave.
  Better to join \gls{paik} than whatever strange god takes you when goblins eat you.
\end{itemize}

\paragraph{If the \glspl{pc} ask for more supplies,}
\gls{redcoin} will give them the following items, with a successful \roll{Charisma}{Tactics} roll.

\redcoin

\begin{itemize}
  \item
  1 torch each (\tn[3])
  \item
  50' of rope (\tn[4])
  \ifnum\value{temperature}=0
    \item
    Warm clothes (\tn[5])
  \fi
  \item
  Mallet (\tn[6])
  \item
  2 torches each (\tn[8])
  \item
  One extra day's rations (\tn[10])
  \item
  3 torches each (\tn[11])
  \item
  2 shortswords (\tn[12])
  \item
  A partial chainmail suit (\tn[13])
\end{itemize}

You can use a single roll, rather than having the \glspl{pc} making loads of dice-rolls.

\subsection{The Road Up}

\subsubsection{Marching}
The \glspl{pc} will start marching that very morning.
Have the \glspl{pc} roll \roll{Intelligence}{Wyldcrafting} to co\"{o}rdinate themselves.
The \gls{tn} is 8 if they have the map, and 9 otherwise.
Each Margin of failure adds 2 miles to the journey time.%
\exRef{core}{the Core Rules}{marching}

\subsubsection{Camping}
By afternoon they will have to make camp.
Ask them who takes watch, and assign 3 \glspl{fatigue} between whoever keeps watch.

\subsubsection{The Mountain}
The next day, the troupe will reach the mountain (unless they have become lost).
As they stop for a rest, part-way up the mountain,
\ifcase\value{temperature}
  the skies are clear and the air feels fresh.
  Snow covers everything.
  They should each take \pgls{fatigue}, but no other dangers will emerge.
  \or
  a hurricane blows fiercely.
  And in the distance, an umber hulk%
  \exRef{judgement}{the book of Judgement}{umber_hulk}
  has already passed by the goblins (and eaten a few), exited the cave, and now stalks the mountainside.
  It attacks the troupe, with a morale score of 12.
  The \glspl{pc} will see it coming a long way off.

  \umberhulk
  \else
  warm rain falls, and new streams have formed down the mountain.
  The troupe will know that this means flooding all across the land, but they won't suffer from it at this altitude.

  Unfortunately, all the floods have agitated a nearby hive of stirges.
  These vicious flying insects, the size of a man's fist, will swarm%
  \exRef{core}{Core Rules}{swarms}
  and sting the party, inflicting \glspl{fatigue}.%
  \exRef{judgement}{Book of Judgement}{stirge}

  \stirgeSwarm
\fi

If the troupe are still active the next afternoon, roll yet another encounter.%
\exRef{judgement}{the book of Judgement}{encounters}

Remember to have the players apply \glspl{fatigue}, even if they will just mark them off again a moment later due to resting.

\subsection{The Cave Mouth}

\begin{exampletext}
  Past boulder leaning on rock.
  By rotten tree.
  Up to rock-line.
  The hole has ferns.
\end{exampletext}

The directions on the map don't make for easy navigation, but the \glspl{pc} will find their destination eventually.

\subsubsection{Ferns}
Give the players two false-starts.
Tell them they see a big patch of ferns in the distance, and ask what they do.
The third set of ferns contains the cave-mouth.

\subsubsection{The Rock}
At the cave-mouth, if any of the goblins see or hear the troupe, they will wait to throw rocks at them.

The cave-mouth is a hole in the ground.
The drop to the ground is the height of two adult humans.%
\footnote{The goblins emerge by sending one to climb, and then helping the others with a rope.}

\end{multicols}
