\section{The Ascent}

\begin{multicols}{2}

\subsection[The Hanging Procession]{\gls{justice}~The Hanging Procession~\gls{justice}}

\begin{boxtext}
  This morning, the \gls{bothy} hosts two traders, a visiting \gls{jotter}, a dozen new recruits, and ten \glspl{sunGuard}.
  The \glspl{sunGuard} have travelled from a town's \gls{court}, all the way here to hang half the new recruits, for dereliction of duty.
  You can hear them complaining bitterly about all the muck on their white tabard.

  A shutter on the \gls{bothy}'s upper floor opens, and the \gls{jotter} shouts to you to come up for orders.
  The rest of the recruits start lazily making their into the \gls{bothy}.
\end{boxtext}

\Pgls{redcoin} explains the following points:

\begin{itemize}
  \it
  \item
  Goblins have been thieving, looting, and even killing around a number of nearby \glspl{village}.
  \item
  Those last recruits went up, but returned as cowards, without a single goblin head to show for their efforts, so the \gls{sunGuard} will take them to the \gls{court} of the nearest town, for justice.
  \item
  A \gls{guard} ranger has already found an entrance to the goblins' lair.
  You must find this spot, enter their cave, kill all of them, and return with their heads.
  \item
  Everyone can pick up \ifnum\value{temperature}=0 four \else three \fi days' rations, and this map.
\end{itemize}

\noindent
As he explains all this, the \glspl{sunGuard} take the doomed prisoners down, and prepare to leave.
The \glspl{pc} hear all this from \gls{redcoin}'s office, and witness the rope-bound prisoners from the window.

\humansoldier[\npc{\T[8]\Hu}{\glsentrytext{sunGuard}}]

\paragraph{If the \glspl{pc} argue,}
\gls{redcoin} will have none of it.
He's busy, and wants the \glspl{pc} to get going, quickly.

\paragraph{If the \glspl{pc} try to plead for the lives of the ex-\gls{guard} prisoners,}
nobody will listen.

\paragraph{Asking the prisoners about what happened}
requires a \roll{Charisma}{Tactics} roll (\tn[8]) (or whatever Skill seems appropriate), to convince the \glspl{sunGuard} to allow the conversation.
They can tell the \glspl{pc} the following:

\begin{itemize}
  \it
  \item
  The entrance is easy.
  It's a hole in the ground, so bring something to get down with -- a rope, a stake, and a mallet.
  \item
  The goblins won't fight, they just run away.
  \item
  The darkness in caves isn't like the darkness at night.
  At night, the sky is dark, but in a cave, the darkness stands right next to you.
  \item
  The ground is rough -- you can't run anywhere.
  That means the goblins can stand back and throw things at you.
  \item
  And if you carry a torch, they can see you, but you can't see them.
  And when you put it out, they can hear you, and you still get hit by rocks.
  \item
  Better to die in a town, than down some horrible cave.
  Better to join \gls{paik} than whatever strange god takes you when goblins eat you.
\end{itemize}

\paragraph{If the \glspl{pc} ask for more supplies,}
\gls{redcoin} will give them the following items, with a successful \roll{Charisma}{Tactics} roll.

\redcoin

\begin{boxtable}
  \textbf{\glsentrytext{tn}} & \textbf{Equipment} \\
  \hline
  \tn[3] & 1 torch each \\
  \tn[4] & 50' of rope \\
  \ifnum\value{temperature}=0
    \tn[5] & Warm clothes \\
  \fi
  \tn[6] & Mallet \\
  \tn[8] & 2 torches each \\
  \tn[10] & One extra day's rations \\
  \tn[11] & 3 torches each \\
  \tn[12] & 2 shortswords \\
  \tn[13] & A partial chainmail suit \\
\end{boxtable}

You can use a single roll, rather than having the \glspl{pc} making loads of dice-rolls.

\paragraph{If anyone asks about the history of the caves,}
let them make an \roll{Intelligence}{Academics} roll (\tn[12]).
Success means they know that humans once scouted the area, to check for valuable metals to mine.
A tie means the character thinks that humans built a copper mine down there, a century ago.%
\footnote{See `\nameref{caves_history}', \vpageref{caves_history}.}

\subsection{The Road Up}

\subsubsection{Marching}
The \glspl{pc} will start marching that very morning.
Have the \glspl{pc} roll \roll{Intelligence}{Wyldcrafting} to co\"{o}rdinate themselves.
The \gls{tn} is 8 if they have the map, and 9 otherwise.
Each Margin of failure adds 2 miles to the journey time.%
\exRef{core}{the Core Rules}{marching}

\subsubsection{Camping}
By afternoon they will have to make camp.
Ask them who takes watch, and assign 3 \glspl{fatigue} between whoever keeps watch.

\subsubsection{The Mountain}
The next day, the troupe will reach the mountain (unless they have become lost).
As they stop for a rest, part-way up the mountain,
\ifcase\value{temperature}
  the skies are clear and the air feels fresh.
  Snow covers everything.
  They should each take \pgls{fatigue}, but no other dangers will emerge.
  \or
  a hurricane blows fiercely.
  And in the distance, an umber hulk%
  \exRef{judgement}{the book of Judgement}{umber_hulk}
  has already passed by the goblins (and eaten a few), exited the cave, and now stalks the mountainside.
  It attacks the troupe, with a morale score of 12.
  The \glspl{pc} will see it coming a long way off.

  \umberhulk
  \else
  warm rain falls, and new streams have formed down the mountain.
  The troupe will know that this means flooding all across the land, but they won't suffer from it at this altitude.

  Unfortunately, all the floods have agitated a nearby hive of stirges.
  These vicious flying insects, the size of a man's fist, will swarm%
  \exRef{core}{Core Rules}{swarms}
  and sting the party, inflicting \glspl{fatigue}.%
  \exRef{judgement}{Book of Judgement}{stirge}

  \stirgeSwarm
\fi

If the troupe are still active the next afternoon, roll yet another encounter.%
\exRef{judgement}{the book of Judgement}{encounters}

Remember to have the players apply \glspl{fatigue}, even if they will just mark them off again a moment later due to resting.

\subsection{The Cave Mouth}

\begin{exampletext}
  Past boulder leaning on rock.
  By rotten tree.
  Up to rock-line.
  The hole has ferns.
\end{exampletext}

The directions on the map don't make for easy navigation, but the \glspl{pc} will find their destination eventually.

\subsubsection{Ferns}
Give the players two false-starts.
Tell them they see a big patch of ferns in the distance, and ask what they do.
The third set of ferns contains the cave-mouth.

\begin{boxtext}
  Pushing back the ferns, you see a hole, and a rock flies at your face.
\end{boxtext}

\subsubsection{The Rock}
At the cave-mouth, if any of the goblins see or hear the troupe, they will wait to throw rocks at them.

\goblin[\npc{\N\F}{Goblin}]

The first \gls{pc} will have to roll \roll{Wits}{Athletics}, against the goblin's \roll{Dexterity}{Projectiles} (\tn).

The cave-mouth is a hole in the ground.
The drop to the ground is the height of two adult humans.%
\footnote{The goblins emerge by sending one to climb, and then helping the others with a rope.}

\end{multicols}

\needspace{15em}
\section{Inside the Dark Hole}

\begin{multicols}{2}

\subsection{Last of the Sunlight}

\begin{boxtext}
  You stand in a shaft of Sunlight.
  Darkness surrounds.
  The sound of your landing echoes back to you.
  High-pitched, mocking giggling noises follow it.
  More rocks follow.
\end{boxtext}

\sidebox{
  \begin{boxtable}
  \textbf{\glsentrytext{tn}} & \textbf{Result} \\
  \hline
  10  & 1 rock ($1D6-1$) \\
  9  & 2 rock ($2D6-2$) \\
  8  & 3 rock ($3D6-3$) \\
  7 & 4 rock ($4D6-4$) \\
  \end{boxtable}
}

\noindent
The first \gls{pc} down receives 5 rocks thrown at them, and will have to make a single \roll{Dexterity}{Athletics} roll to avoid them (\tn[10]).
Each failure margin means another rock, and each rock means $1D6-1$ Damage.

The \glspl{pc} can only enter one at a time, and the rocks will keep on coming.
The players will have to decide, one at a time, who goes next into the dark hole.

\goblin[\npc{\T[5]\N}{Goblins}]

\paragraph{Running in the cave}
demands a \roll{Wits}{Caving} roll (\tn[12]).
Failure inflicts $1D6$ Damage, plus the character's Speed Bonus (running faster means more Damage).

\subsubsection{Screams Above}

Once most of the \glspl{pc} enter the dark hole, a couple of goblins crawl out of a tiny side-tunnel, and cause trouble at the top.
They can only squeeze through the tunnel, as they have a Strength Penalty of -1.
Anything larger than this cannot fit through the narrow tunnel.

\npc{\T[2]\N}{Skinny Goblins}
\person{-2}% STRENGTH
{1}% DEXTERITY
{1}% SPEED
{{-1}% INTELLIGENCE
{0}% WITS
{-2}}% CHARISMA
{0}% DR
{1}% COMBAT
{}% SKILLS
{\Dagger}% EQUIPMENT
{
  \setcounter{Combat}{1}
  \setcounter{Brawl}{2}
  \setcounter{Caving}{1}
  \setcounter{Tactics}{1}
}

\subsection{The Cave-In}

\subsubsection{Down the Tunnel}
The second after two characters get down the hole, the goblins flee.
If the \glspl{pc} don't give chase, they have failed in their mission.
The goblins will not return to this area for two days, so the \glspl{pc} will eventually have to follow.

\paragraph{If the \glspl{pc} have lit a torch,}
they will spot dry rocks.

\begin{boxtext}
  The flickering torch illuminates a soaking wet cave, littered with bone-dry rocks of every size.
\end{boxtext}

The rocks have come from the ceiling.
The various rocks, shouts, and \ifnum\value{temperature}=1 flooding \else stomps \fi in cavern have begun a cave-in.

\subsubsection{Rocks Fall}

It begins once the troupe have gone 60 steps into the cavern.

\begin{itemize}
  \item
  Each round, roll $1D6$ for each \gls{pc}.
  On the roll of a 1, a rock lands on the \gls{pc}, inflicting $1D6+1$ Damage.
  \item
  Every time a rock falls, the threshold for the roll, and Damage, increase by 1.
  \item
  The third time someone gets hit by a rock, the roll will increase to a 4 in 6 chance, and inflict $1D6+4 Damage$
\end{itemize}

\caveIn

Some \glspl{pc} may try to run back to the entrance, but they won't be safe there -- nowhere remains safe except further down the tunnel.

\paragraph{Once the cavern collapses,}
the dust thrown up inflicts 2 \glspl{fatigue}, each round, until the \glspl{pc} leave the \gls{area}.

\paragraph{Leaving this area}
means they must find the exit.
It looks like another shadow among shadows, so the troupe must roll \roll{Wits}{Caving} (\tn[10], +2 bonus for having a torch).
They can work together%
\exRef{core}{Core Rules}{teamwork}
but each new person helping with the roll will take more time.

\paragraph{The passage out}
becomes smaller and smaller, until the head of the troupe must crawl.
Whoever went first must roll \roll{Dexterity}{Caving}, against \gls{tn} 8, plus their own Strength Bonus (larger people will struggle more).

Every character must make this roll, and every time they fail, every character behind them takes \pgls{fatigue} due to the dust.

\subsection{The Realization}

Once out the other side, give the players a moment to understand what has happened, and the implications.
Describe the scene again.
They will need a clear view in order to prepare.

\begin{itemize}
  \item
  Does anyone have a lit torch?
  \item
  How many torches do they have?
  \item
  How many characters survived?
  How many are dead?
  \item
  Who is wounded?
  \item
  How many tinder boxes?
  Tell them that the tinder box is dry, so it still functions.
  This will remind them that they must keep the tinder box dry.
  \item
  Did anyone take their backpack off to fight?
  Where did they leave that backpack?
  \item
  How many rations do they have in total?
  \item
  Do they have writing equipment, to make a map?
  (players who want to make a map should still be allowed, even if their character cannot)
\end{itemize}

\paragraph{Digging their way out}
seems obviously futile -- the characters have a clear view of the debris, and can feel the weight of the rock around them, so no \glspl{fatigue} should be given out for attempting to dig, unless the players insist.
And in this case, simply keep giving out \glspl{fatigue}, and allow them to remove another small patch of dirt.

\paragraph{If any characters remain above,}
they will have to find out what happens to them later.

\subsubsection{Counting Supplies}

Ask the players about each of their supplies, so they remain aware of the decisions ahead of them.

\end{multicols}
