\section{The Ascent}

\begin{multicols}{2}


\subsection[The Hanging Procession]{\glssymbol{paik}~The Hanging Procession~\glssymbol{paik}}

\begin{boxtext}
  This morning, Stoatfen \gls{bothy} hosts two traders, \gls{redcoin} (the visiting \gls{jotter}), six new recruits, six prisoners and eight \glspl{sunGuard}.
  The \glspl{sunGuard} have travelled from a town's \gls{court}, all the way here to hang the prisoners, for dereliction of duty.
  You can hear the guards complaining bitterly about all the muck on their white tabards.

  A shutter on the \gls{bothy}'s upper floor opens, and the \gls{jotter} shouts to you to come up for orders.
  The rest of the recruits start lazily making their into the \gls{bothy}.
\end{boxtext}

\Pgls{redcoin} explains the mission:

\begin{exampletext}
  Your mission is to clear out a goblin-warren, half-way up Failpeak Mountain.
  Make sure you kill all of them.

  Everyone can pick up \ifnum\value{temperature}=0 four \else three \fi days' rations, and a torch.
\end{exampletext}

\noindent
As he explains all this, the \glspl{sunGuard} take the doomed prisoners down, and prepare to leave.
The \glspl{pc} hear all this from \gls{redcoin}'s office, and witness the rope-bound prisoners from the window.

\humansoldier[\npc{\T[8]\Hu}{\glsentrytext{sunGuard}}]

\paragraph{If the \glspl{pc} argue,}
\gls{redcoin} will have none of it.
He's busy, and wants the \glspl{pc} to get going, quickly.

\paragraph{If the \glspl{pc} try to plead for the lives of the ex-\gls{guard} prisoners,}
nobody will listen.

\sidebox{
  \begin{boxtable}[lX]
    \textbf{\glsentrytext{tn}} & \textbf{Equipment} \\
    \hline
    \tn[4] & 50' of rope (only 2 available) \\
    \ifnum\value{temperature}=0
      \tn[5] & Warm clothes \\
    \fi
    \tn[6] & Mallet \\
    \tn[7] & 2 more torches \\
    \tn[9] & One extra day's rations \\
    \tn[10] & partial leather armour \\
    \tn[11] & 2 more torches \\
    \tn[12] & shortsword \\
    \tn[13] & partial chainmail armour \\
  \end{boxtable}
}

\paragraph{If the \glspl{pc} ask for more supplies,}
\gls{redcoin} will give them the following items, with a successful \roll{Charisma}{Wyldcrafting} roll.

You can use a single roll, rather than having the \glspl{pc} making loads of dice-rolls, so the party will be able to ask for anything on their roll, plus anything beneath that \glsentrylong{tn} (including all of those torches), and everyone who makes the roll receives the item.
However, they will have to ask for an item by name in order to get it.

Inform the players that each torch burns for an hour.

\paragraph{Asking the prisoners about what happened}
requires a \roll{Charisma}{Empathy} roll (\tn[8]) (or whatever Skill seems appropriate), to convince the \glspl{sunGuard} to allow the conversation.

Scarstain (a bulky, Sunburnt man in his forties) can tell the \glspl{pc} the following:

\begin{itemize}
  \item
  The entrance is easy.
  It's a hole in the ground, so bring something to get down with -- a rope, a stake, and a mallet.
  \item
  The goblins won't fight, they just run away.
  \item
  The darkness in caves isn't like the darkness at night.
  At night, the sky is dark, but in a cave, the darkness stands right next to you.
  \item
  The ground is rough -- you can't run anywhere.
  That means the goblins can stand back and throw things at you.
  \item
  And if you carry a torch, they can see you, but you can't see them.
  And when you put it out, they can hear you, and you still get hit by rocks.
  \item
  Better to die in a town, than down some horrible cave.
  Better to join \gls{paik} than whatever strange god takes you when goblins eat you.
\end{itemize}

\redcoin

\subsection{History}
\label{caves_history}

If anyone asks about the history of the caves, let roll \roll{Intelligence}{Academics}.

\begin{boxtable}
  11 & Fifty years ago, humans tried to mine this mountain, but found nothing of value, and left.
    A couple of wooden bridges remain.
  \\
  8 & Last year, goblins climbed up from the \gls{deep}.
    They have stolen milk, rustled pigs, and even killed a few people and nobody knew where in the mountains they came from, until recently.
  \\
  7 & \Pgls{guard} ranger discovered a cave entrance, some miles up a mountain.
    In fact, the goblins know of a tunnel which exits much closer to local \glspl{village}, but as the goblins keep their secrets, the \glspl{pc} will have to enter via the mountainside.
  \\
  5 & \Pgls{redcoin} has already dispatched two groups of \glspl{guard} have to enter the mountainside tunnel.
    The first group entered and died two weeks ago.
    The second group went a little into the darkness, lost two men, then returned terrified.
  The \pgls{jotter} has ordered their execution, so \glspl{sunGuard} have arrived to take them away from Stoatfen \gls{bothy}.
  \\
  3 & Each \glsentrytext{bothy} is named after someone who died on the road.
  Stoatfen died in the many tentacles of some \glsentrytext{woodspy}.
  \\
\end{boxtable}

\glsresetall

\subsection{The Road Up}

\subsubsection{Marching}
The \glspl{pc} will start marching that very morning.
Have the \glspl{pc} roll \roll{Intelligence}{Wyldcrafting} to co\"{o}rdinate themselves.
The \gls{tn} is 8 if they have the map, and 9 otherwise.
Each Margin of failure adds 2 miles to the journey time.%
\exRef{core}{the Core Rules}{march}

\subsubsection{Camping}
By afternoon they will have to make camp.
Ask them who takes watch, and assign 2 \glspl{ep} between whoever keeps watch.

\subsubsection{The Mountain}
The next day, the troupe will reach the mountain.
As they stop for a rest, part-way up the mountain,
\ifcase\value{temperature}
  the skies are clear and the air feels fresh.
  Snow covers everything.
  They should each take \pgls{ep}, but no other dangers will emerge.
  \or
  a hurricane blows fiercely.
  And in the distance, an umber hulk has already passed by the goblins (and eaten a few), exited the cave, and now stalks the mountainside.
  It attacks the troupe, with a morale score of 12.
  The \glspl{pc} will see it coming a long way off.

  \umberhulk
  \else
  warm rain falls, and new streams have formed down the mountain.
  The troupe will know that this means flooding all across the land, but they won't suffer from it at this altitude.

  Unfortunately, all the floods have agitated a nearby hive of stirges.
  These vicious flying insects, the size of a man's fist, will swarm%
  \exRef{core}{Core Rules}{swarms}
  and sting the party, inflicting \glspl{ep}.%
  \exRef{judgement}{Book of Judgement}{stirge}

  \stirgeSwarm
\fi

If the troupe are still active the next afternoon, roll yet another encounter.%
\exRef{judgement}{the book of Judgement}{encounters}

Remember to have the players apply \glspl{ep}, even if they will just mark them off again a moment later due to resting.

\subsection{The Cave Mouth}

\begin{exampletext}
  Past boulder leaning on rock.
  By rotten tree.
  Up to rock-line.
  The hole has ferns.
\end{exampletext}

The directions on the map don't make for easy navigation, but the \glspl{pc} will find their destination eventually.

\subsubsection{Ferns}
Give the players two false-starts.
Tell them they see a big patch of ferns in the distance, and ask what they do.
The third set of ferns contains the cave-mouth.

\begin{boxtext}
  Pushing back the ferns, you see a hole, and a rock flies at your face.
\end{boxtext}

\subsubsection{The Rock}
At the cave-mouth, if any of the goblins see or hear the troupe, they will wait to throw rocks at them.

\goblin[\npc{\N\F}{Goblin}]

The first \gls{pc} will have to roll \roll{Wits}{Athletics}, against the goblin's \roll{Dexterity}{Projectiles} (\tn).

The cave-mouth is a hole in the ground.
The drop to the ground is the height of two adult humans.%
\footnote{The goblins emerge by sending one to climb, and then helping the others with a rope.}

\end{multicols}

\needspace{15em}
\section{A Small Hole in the Ground}

\begin{multicols}{2}

\subsection{Last of the Sunlight}

\begin{boxtext}
  You stand in a shaft of Sunlight.
  Darkness surrounds.
  The sound of your landing echoes back to you.
  High-pitched, mocking, giggling noises follow it.
  More rocks follow.
\end{boxtext}

\sidebox{
  \begin{boxtable}
  \textbf{\glsentrytext{tn}} & \textbf{Result} \\
  \hline
  10  & 1 rock ($1D6-1$) \\
  9  & 2 rock ($2D6-2$) \\
  8  & 3 rock ($3D6-3$) \\
  7 & 4 rock ($4D6-4$) \\
  \end{boxtable}
}
The first \gls{pc} down receives 5 rocks thrown at them, and will have to make a single \roll{Dexterity}{Athletics} roll to avoid them (\tn[10]).
Each failure margin means another rock, and each rock means $1D6-1$ Damage.

The \glspl{pc} can only enter one at a time, and the rocks will keep on coming.
The players will have to decide, one at a time, who goes next into the dark hole.

\goblin[\npc{\T[5]\N}{Goblins}]

\paragraph{Running in the cave}
demands a \roll{Wits}{Caving} roll (\tn[12]).
Failure inflicts $1D6$ Damage, plus the character's Speed Bonus (running faster means more Damage).

\subsubsection{Screams Above}

Once most of the \glspl{pc} enter the dark hole, a couple of goblins crawl out of a tiny side-tunnel, and cause trouble at the top.
They can only squeeze through the tunnel, as they have a Strength Penalty of -1.
Anything larger than this cannot fit through the narrow tunnel.

\goblin[\npc{\T[2]\N}{Skinny Goblins}]

\subsection{The Cave-In}

The second after two characters get down the hole, the goblins flee.
If the \glspl{pc} don't give chase, they have failed in their mission.
The goblins will not return to this area for two days, so the \glspl{pc} will eventually have to follow.

\paragraph{If the \glspl{pc} have lit a torch,}
they will spot dry rocks.

\begin{boxtext}
  The flickering torch illuminates a soaking wet cave, littered with bone-dry rocks of every size.
\end{boxtext}

The dry rocks on the ground have fallen from the ceiling recently.
The shouts and \ifnum\value{temperature}=1 flooding \else stomps \fi in the cavern have begun a cave-in.

\subsubsection{Rocks Fall}
once the troupe have gone 60 steps into the cavern.
They will have to keep running forward to survive.

\begin{itemize}
  \item
  Each round, roll $1D6$ for each \gls{pc} who don't run towards the goblins.
  On the roll of a 1, a rock lands on the \gls{pc}, inflicting $1D6+1$ Damage.
  \item
  Every time a rock falls, the threshold for the roll, and Damage, increase by 1.
  \item
    The third time someone gets hit by a rock, the roll will increase to a 4 in 6 chance, and inflict \dmg{8} Damage.
\end{itemize}

\caveIn

Some \glspl{pc} may try to run back to the entrance, and they may succeed.
In this case, they will find themselves alone, at night, with a couple of goblins waiting nearby.
Those goblins will wait for \pgls{interval} for a moment of weakness, but if that does not arrive, they become bored, and head down to the other cave entrance, far down the mountain.

You can leave these \glspl{pc} as wildcards, until the very end of the session, then tell their entire story, just as the troupe begin to exit the mountain, far below (assuming anyone survives).

\paragraph{Once the cavern collapses,}
the dust thrown up inflicts 2 \glspl{ep}, each round, until the \glspl{pc} leave the \gls{area}.

\paragraph{Leaving this area}
means they must find the exit.
It looks like any other shadow among shadows, so the troupe must roll \roll{Wits}{Caving} (\tn[10], +2 bonus for having a torch).
They can use \pgls{bandAct}, so they will succeed eventually, as long as enough people help.

\paragraph{The passage out}
becomes smaller and smaller, until the head of the troupe must crawl.
Whoever went first must roll \roll{Dexterity}{Caving}, against \gls{tn} 8, plus their own Strength Bonus (larger people will struggle more).

Every character must make this roll, and every time they fail, every character behind them takes \pgls{ep} due to the dust.

\subsection{The Realization}
begins once they reach the other side.
Give the players a moment to understand what has happened, and the implications.
Describe the scene again.
They will need a clear view in order to prepare.

\begin{itemize}
  \item
  Does anyone have a lit torch?
  \item
  How many torches do they have?
  \item
  How many characters survived?
  How many are dead?
  \item
  Who is wounded?
  \item
  How many tinder boxes?
  Tell them that the tinder box is dry, so it still functions.
  This will remind them that they must keep the tinder box dry.
  \item
  Did anyone take their backpack off to fight?
  Where did they leave that backpack?
  \item
  How many rations do they have in total?
  \item
  Do they have writing equipment, to make a map?
  (players who want to make a map should still be allowed, even if their character cannot)
\end{itemize}

\paragraph{Digging their way out}
seems obviously futile -- the characters have a clear view of the debris, and can feel the weight of the rock around them, so no \glspl{ep} should be given out for attempting to dig, unless the players insist.
And in this case, simply keep giving out \glspl{ep}, and allow them to remove another small patch of dirt.

\paragraph{If any characters remain above,}
they will have to find out what happens to them later.

\subsection{The Goblin Feast}

The goblins know the full cave layout, and will use this to their advantage.
They will send scouts out to get updates on where the \glspl{pc} have got to.
They will move as quietly as they can, but may grow over-confident as they think humans are all large and noisy.

\paragraph{If the troupe stay silent and still for long enough,}
they may manage to capture a goblin scout as it claws past them through the darkness.

The \glspl{pc} should roll \pgls{bandAct} \roll{Strength}{Vigilance} against \tn[5] plus the goblin's \roll{Dexterity}{Stealth} (the \gls{natural} determines everyone's result).

Success means that the goblins will become confused about where the troupe have gone for a while, but once scouts return from every passage except one, they will at least know where the troupe used to be.

The goblins will have to travel up to the troupe \emph{and back}, which makes them slower.
However, they all know these caverns well, and move easily through narrow spaces, so they receive a +3 Bonus to any Caving rolls (in case you need to work out specifics).

\paragraph{If a member of the troupe dies,}
the goblins will want the corpse for their dinner.
It goes like this:

\begin{description}
  \item[After \pgls{interval},]
  a scout finds the corpse, and screeches loud, notifying the other goblins of the find (they have a system of screeches, and will understand the meaning).
  The troupe will hear the echoes along the cavern, but may not understand the meaning (at least when the first character falls).
  \item[\Pgls{interval} later,]
  five scouts meet at the corpse, strip it (to make it lighter), then carry it back to the others.
  \item[Two more \glspl{interval} after that,]
  the scouts join the other goblins.
\end{description}

Once the troupe encounter the goblins, they may witness a feast of a former comrade.

\subsubsection{Keeping Pace with the Goblins}
requires marching fast enough to cover 2~miles in \pgls{interval}.
You can keep track of the various goblin movements, including multiple scout groups, simply by writing a number to represent \pgls{interval}, on the map.

If you've reached the 7th \gls{interval} and think the goblins would sent out a wave of scouts, write down the number `8' where the scouts will arrive in a moment, and then `9' where one might find a corpse.
Once the troupe arrives at a spot on the 8th \gls{interval}, you'll know exactly how many goblins wander nearby, and what they can hear.

\end{multicols}
