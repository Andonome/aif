\section{Returning Home}

\begin{multicols}{2}
\renewcommand\npcsymbol{\gls{evening}}

\noindent
The troupe emerge into a
\ifcase\value{temperature}%
  snowy
\or%
  dense 
\else%
  verdant, noisy 
\fi%
forest, and a final encounter.

\paragraph{If the troupe emerge during the morning or evening,}
the local \gls{broch} sounds their pipes (as usual), which lets the \glspl{pc} know where to go.
At any other time of day, the troupe can hear a nearby river, which travels 5~miles to \pgls{village}, not far from Stoatfen~\Gls{bothy}.

\begin{boxtext}
  \ifcase\value{temperature}
    A vicious breeze blows through the frozen forest, and in the distance, wolves howl.

    \or
    Mist covers the forest, making navigation difficult.
    \else
    A storm is brewing above.
    It looks set to break before long.

    In the trees, far above but not far along, \pgls{crawler} hangs upside-down and motionless.

  \fi
\end{boxtext}

\ifcase\value{temperature}
  The wolves arrive after \pgls{interval}, and stalk the troupe for \pgls{interval} more, hoping to steal some food.

  \wolf
  \or
  The mist makes it nearly impossible to spot the \gls{woodspy} waiting for them to come out of the \nameref{windyPassage}.

  \woodspy

  \set{tn}{7}
  \addtocounter{tn}{\value{Dexterity}}
  \addtocounter{tn}{\value{Stealth}}
  The troupe rolls \roll{Wits}{Vigilance} at \gls{tn}~\arabic{tn} to avoid being surprised.
  Failure means the \gls{woodspy} gains a +2 Attack Bonus, and (if successful) pulls one character up a tree.
  All subsequent attacks deal Damage and the other characters can do nothing, unless they can climb or have ranged weapons.

  \else
  \chitincrawler

  The \gls{crawler} waits for a while before attacking, but if they leave immediately, it forgets about them.

  Within \pgls{interval}, the storm breaks, inflicting \pgls{ep} while travelling.
\fi

\subsection{Back at Stoatfen \Glsfmttext{bothy}}

Once the troupe return from their harrowing journey past the \gls{edge}, \gls{susjot} immediately asks how many goblin heads the troupe have obtained.
If the troupe have none, one of them will need to roll \roll{Charisma}{Caving} to explain the difficulties (\tn[8]).
If the roll fails, \gls{susjot} sends them all to the local \gls{court}, while arranging for more \gls{fodder} to go up the mountain\ldots

\ifnum\value{temperature}=0
  \paragraph{If the troupe report the hibernating \glspl{basilisk},}
  \gls{susjot} makes a deal with \pgls{ranger} to have them gutted and sold.
  He later states the \glspl{basilisk} had `just gone, like all the gold in that mine'.
\fi

\subsubsection{The Chest of Gold}
\label{goldConspiracy}
never entered the mines with the goblins.
In truth, \gls{woetide} lied.
When goblins raided the caravan she travelled with, she ran away, then returned and took the chest of gold, then left it nearby in the forest.

The chest remains there, in a bush, with \arabic{r12}0~\glspl{gp}, and 
\randomize\arabic{r12}00~\glspl{sp}.
Finding the chest requires knowing where that attack took place, and an \roll{Intelligence}{Empathy} roll at \gls{tn}~14.

\Gls{woetide} will take the chest to a safe location by the end of the \glsentrytext{cycle}.

The players will probably realize what has happened, because:


\begin{itemize}
  \item
  The dead \glspl{guard} have their \glspl{coin} (\glspl{area} \ref{caveCoinsI}, \ref{caveCoinsII},~and~\ref{caveCoinsIII}).
  \item
  \Glsfmtname{coin} is useless to people who don't trade with other settlements.
  \item
  \Pgls{susjot} heard that three \glspl{guard} left with a chest of \glspl{coin}, but all the \glspl{guard} died in the mine.
  \item
  The person to make the claim was `\pgls{notary}', and must have had a reason to fabricate a story about why the money was not in the mines.
  \item
  If \pgls{woetide} had lost her money, and knew that \gls{susjot} would arrive at Stoatfen~\Gls{bothy} soon, she would have waited there to ask him if her money had been recovered.
\end{itemize}

\end{multicols}
