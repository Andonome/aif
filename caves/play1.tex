\section{Example: The Game Begins}

\begin{multicols}{2}

\noindent
The table has two players because one is running late.
Such is life.

\begin{description}\sf
  \item[Player 1:]
  Shall we start character creation now?
  \item[\Glsentrytext{gm}:]
  Yes!
  There you go, one character, already made.
  She sits in \pgls{bothy} -- a kind of half-way house, where people can sleep while travelling.
  \item[Player 2:]
  So, we're starting in a tavern?
  I order an ale.
  \item[\Glsentrytext{gm}:]
  Only characters can order ale, so here's your character.
  He finds out that he can't order ale, because you two are waiting in Stoatfen~\Gls{bothy} for \pgls{jotter} to arrive with orders.

  Which character did you get?
  I just picked the first one on the pile.
  \item[Player 2:]
  Says here I'm a thief.
  \item[Player 1:]
  Same here.
  Hopefully the next one's a fighter.
  What's \pgls{jotter}?
  \item[\Glsentrytext{gm}:]
  All the characters are thieves, cutthroats, dick-heads, and bandits, and most of them fighers as well.
  You were sentenced to repay your debt to society in the \gls{templeOfBeasts}, where \glspl{jotter} keep the records, and pass the orders along.
  \item[Player 1:]
  Apparently the debt is `100~\glsentrylongpl{sp}'; it's at the top of the character sheet.
  \item[\Glsentrytext{gm}:]
  Wheels turn in the distance.
  Through the window, three horses \ifnum\value{temperature}=0 push through the fresh snow with a wagon sliding behind them\else pull a wagon along the muddy road\fi.
  The tall trees cover them in shadow until near Stoatfen~\Gls{bothy}.
  Coming forward, the wagon has three on foot, with long spears.
  \item[Player 1:]
  So, these guys are the fighters? 
  \item[\Glsentrytext{gm}:]
  Traders in \gls{fenestra} travel well-armed.
  \Glspl{monster} are wandering.
  \item[Player 1:]
  So people here are aware of random encounters?
  \item[\Glsentrytext{gm}:]
  Very.
  Once or twice a week, something comes from the forest to feed.
  This trader, for example, may put his horses inside the \gls{bothy}, leave his goods outside, and bar the door and shutters.
  \item[Player 2:]
  So this place is \gls{monster}-proof?
  \item[\Glsentrytext{gm}:]
  Mostly, but they still try.
  Sometimes one feels along the walls and roof, searching for an opening during the night, while people inside debate what kind of creature is trying to come in.
  \item[Player 1:]
  So they only come out at night.
  \item[\Glsentrytext{gm}:]
  No, they often wait until morning.
  That's where you come in!
  In \pgls{bothy}, people expect the \gls{guard} to open the door in the morning, and give the all-clear.
  \item[Player 2:]
  How the hell does anyone live in this place?
  \item[\Glsentrytext{gm}:]
  How would you live?
  How would you farm?
  \item[Player 3:]
  Hey guys!
  Did I miss the start?
  Have you done character creation?
  \item[\Glsentrytext{gm}:]
  Here's your character.
  Now, how do you think you'd farm in a world of wandering monsters?
  \item[Player 3:]
  Why's my name `\composeHumanName'?
\end{description}

\bigLine
\vspace{2em}

\noindent
The `wandering monsters' question makes a nice opener, as people can have their own ideas about it.
But initial sessions should not have too much lore -- it's always better to describe more of what the \glspl{pc} \emph{experience} in \gls{fenestra} and trust them to pick it up naturally.

\bigLine

\begin{description}\sf
  \item[\Glsentrytext{gm}:]
  A woman with a spear hails you while the older man on the cart slowly climbs down.
  \item[Player 3:]
  Um\ldots `hail'?
  \item[\Glsentrytext{gm}:]
  The older man \ifcase\value{temperature} shivers, and enters the \gls{bothy}, complaining that you haven't lit the fire\or removes his bright-red hat to you, and unbuckles the horses\or smiles warmly, and introduces you to his horses\else pulls off his hat to fan himself in the shadow of the \gls{bothy}\fi.
  Two with the spears tend to the horses, while the third scouts around the perimiter.
\end{description}

\bigLine
\vspace{2em}

\noindent
We don't need to stop and `role-play' a meeting, with individual introductions, and \glsentryfirst{npc} names which nobody will remember.
Trust the players to engage with the scene at the level you set.
  The \gls{gm} should always feel free to shift focus and push the narrative to the next meaningful event.

\bigLine

\begin{description}\sf
  \item[\Glsentrytext{gm}:]
  The trader immediately settles into his tale of woe, while the walkers roll their eyes, and disappear.
  He comes from a good family, with many friends in the \gls{templeOfFrost}.
  But last month, goblins killed his brother, not far from here.
  He was on his way to make a good purchase, and had the family's saving with him, in a single bag of gold coins.
  He died, and the family has been poor, and in debt, since then.
  \item[Player 2:]
  Does he have any ale?
  \item[\Glsentrytext{gm}:]
  No, and his horses have rested and eaten, so he bids you goodbye with the usual good wishes.

  ``Travel straight, never curious.''
  \item[Player 2:]
  `Never curious'?
  How is that a standard wish?
  \item[Player 1:]
  Curiosity's dangerous.
  \item[\Glsentrytext{gm}:]
  As the wagon continues South, the sound of galloping comes from the North.
  The rider's dressed like \pgls{guard} with blackened lisk-hide, with focussed, accusing eyes.
  He storms up to you on the large grey horse then stops with some distance, like he's afraid.
  He says ``My name is \Gls{builder}~Sourgroat%
  \ifnum\value{temperature}=0 '', his breath steams up the frozen air.
  ``\else.\fi
  Are you the \glspl{fodder} who went to the mountain?''.
  \item[Player 2:]
  ``\Gls{fodder}?
  What's his problem?''
  \item[Player 1:]
  That's your rank.
  Character sheet.
  Top right.

  We haven't been to the mountain, have we?

  \item[\Glsentrytext{gm}:]
  The \gls{ranger} gives you a confirming look, then says ``no, you're not the deserters.
  \Pgls{jotter} will arrive soon with your duties; gather logs for the \gls{bothy}'', then he gallops South.
  \item[Player 2:]
  What an asshole.
  \item[\Glsentrytext{gm}:]
  The cold wind bites, and nobody has very appropriate clothes.
  \item[Player 3:]
  What are we wearing?
  \item[\Glsentrytext{gm}:]
  Just the standard black cassock of the \gls{guard}, and anything else on the character sheet.
  The black uniforms help you hide from \glspl{monster}, and help people on the road notice who you are.
  \item[Player 1:]
  Okay.
  We light a fire.
  \item[\Glsentrytext{gm}:]
  Stoatfen~\Gls{bothy} doesn't have a single log to burn, but it does have an axe.
  \ifnum\value{temperature}=0%
    Meanwhile, everyone's feeling freezing, so take these \glsentrylongpl{ep} -- one each.
    These represent feeling frozen and tired, but they work just like the weight of an item.
  \fi
  \item[Player 2:]
  I'll take the axe, and chop some logs.
  \item[\Glsentrytext{gm}:]
  Outside the \gls{bothy}, the ground lies flat for a hundred \glspl{step} in every direction.
  The cloests trees stand as tall as castles%
  \ifnum\value{temperature}=0%
    and frozen solid%
  \fi, but further inside, you can see smaller trees, and a dead tree, fallen over and half-rotten.
  \item[Player 2:]
  Jackpot!
  I'll get the fallen tree.
  Do I even need to roll for that?
  Why are you rolling?
  What's happening?
  \item[\Glsentrytext{gm}:]
  Just checking for wandering \glspl{monster}.
  What's everyone else up to?
  \item[Player 1:]
  Do I see this?
  What happened to him?
  \item[\Glsentrytext{gm}:]
  He went into the forest with the last axe, and hasn't come back.
  \item[Player 1:]
  Well I'm not going in, so I'll just shout to see if he's there.
  Why are you rolling?
  \item[\Glsentrytext{gm}:]
  \Glspl{monster} love bangs and shouts, so I have to check if any are close enough to hear.
  \item[Player 1:]
  Back in the \gls{bothy}!
  \item[Player 3:]
  Me too.
  \item[\Glsentrytext{gm}:]
  The Stoatfen~\Gls{bothy} doors are barred, and the shutters shut.
  Inside is pitch black\ifcase\value{temperature} and freezing\or\else and smells of donkey\fi.

  Meanwhile, in the dark forest, did you get any logs?
  Roll Strength and Survival, \glsentrylongpl{tn}~7.
  \item[Player 2:]
  Okay, so\ldots Strength first?
  How do I roll that?
\end{description}

\bigLine
\vspace{2em}

\noindent
All of BIND's systems are just reinterpretations of the same basic $2D6$ roll, so it's best to make sure all the players understand the base system properly.
Spending two minutes to explain how to `add these four numbers' can feel aggonizingly slow, but if it's worth it if any of the players don't feel certain about how the base system operates.

\begin{enumerate}
  \item
  The \gls{pc} wants some prize (e.g. `logs').
  \item
  The \gls{pc} wants to avoid a danger (\ifnum\value{temperature}=0 spending time in the cold\else returning late\fi).
  \item
  The \gls{gm} always decides on the most appropriate \gls{attribute} and \gls{skill} to use, then the \glsentrylong{tn}.
\end{enumerate}

\vspace{\baselineskip}
\noindent
\begin{tabularx}{\linewidth}{ccccccr}
\hline
$2D6$ & + & \gls{attribute} & + & \gls{skill} & vs & \Glsentrylong{tn} \\
\hline
$2D6$ & + & Strength & + & Survival & vs & \glsentrylong{tn}~7 \\
$2D6$ & + & 1 & + & 1 & vs & \glsentrylong{tn}~7 \\
\twoDice{5} & + & 1 & + & 1 & vs & \glsentrylong{tn}~7 \\
\hline
5 & + & 2 & = & 7 & vs & \glsentrylong{tn}~7 \\
\hline
\end{tabularx}

\vspace{\baselineskip}
\noindent
When rolling \pgls{tn} exactly, the player often chooses to accept both the prize and the danger, or neither.
However, in this case, the choice would make no sense, as the \gls{pc} would keep collecting logs until they have some. 

\bigLine

\begin{description}\sf
  \item[\Glsentrytext{gm}:]
  The rotten tree smashes open.
  Wood-flakes scatter.
  A few more strikes, and you have a usable chunk.

  Then snapping twigs, footsteps approach from the dark forest, from the North.
  \item[Player 2:]
  How much Damage does an axe do?
  \item[\Glsentrytext{gm}:]
  Loads!
  But he asks you not to.
  He asks you to put the axe down.
  He's holding a sword, and dressed in black and asking you to put the axe down.
\end{description}

\bigLine
\vspace{2em}
\noindent
The \glspl{pc} might meet with the \gls{guard} deserter, and learn what he knows about the cave.
Or they may miss him entirely.

Moments later, the \gls{jotter} finally arrives.
He explains that goblins routinely raid nearby \glspl{village} and the \gls{guard} must kill them.
The he explains that the last group to enter the tunnel fled, and will be killed along with the goblins.

Then comes the most terrifying of foes; it is long and cunning, and known only as `glossary'.

\end{multicols}
