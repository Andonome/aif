\section*{Example: The Game Begins}

\begin{multicols}{3}

\noindent
The table has two players because one is running late.
Such is life.


\begin{description}\sf
  \item[Player 1:]
  Shall we start character creation now?
  \item[\Glsentrytext{gm}:]
  Yes!
  There's your characters, already made.
  Who are they?
  I haven't looked at any of them.
  \item[Player 1:]
  Says here I'm a thief called `\composeHumanName'.
  \item[Player 2:]
  I got `\composeHumanName'; also a thief.
  Hopefully the next one's a fighter.
  \item[\Glsentrytext{gm}:]
  All the characters are thieves, cutthroats, dick-heads, and bandits, and most of them fighters as well.
  You were sentenced in the \gls{court} to repay your debt to society by joining the \gls{templeOfBeasts}.
  \item[Player 1:]
  It says on the sheet that the debt's `100~\glsentrylongpl{sp}'.
  \item[\Gls{gm}:]
  \Glspl{guard} may not set foot in towns, \glspl{village}, or other civilized places until they repay the debt.
  You live in the between-places; \glspl{bothy} and \glspl{broch}, keeping others safe by hunting \glspl{monster}.
  \item[Player 1:]
  And we pay back the debt with our wages from this \gls{temple}?
  \item[Player 2:]
  What's \pgls{bothy}?
  \item[\Gls{gm}:]
  No, you pay the \gls{temple} half of what you receive from selling monster bodies, and do missions.

  \Pgls{bothy}'s a half-way house, where people sleep just to stay safe from the \glspl{monster}.
  You're coming up to Stoatfen \Gls{bothy} now, where you'll receive your first mission.

  \item[Player 1:]
  What's a `Stoatfen'?
  \item[\Gls{gm}:]
  Each \gls{bothy} starts as a cairn --- a pile of rocks --- to remember someone who died on the road.
  If enough people bring enough rocks, then someone builds \pgls{bothy}.
  So something ate someone called `Stoatfen', just up ahead, and now there's \pgls{bothy}.
  \item[Player 2:]
  Why are the names weird?
  \item[\Gls{gm}:]
  Local custom.
  People say the forest eats the best things in life, so naming your kid `Sunbeam' is just asking for something to come out the forest and eat it.
  \item[Player 2:]
  Are we there yet?
  \item[\Glsentrytext{gm}:]
  Yes.
  The road widens, and the shadows of the tall trees disperse.
  Stoatfen~\Gls{bothy} has the standard clearing of 50~\glspl{step} in all directions.
  Trees felled, and bushes burnt, to push back the forest, leaving a circle of
  \ifnum\value{temperature}=0%
    white%
  \else%
    yellowed grass%
  \fi.%
  Where most \glspl{bothy} have only one room, Stoatfen has many.
  But it performs the same function; it lets people sleep safely.
  \Glspl{monster} are wandering.
  \item[Player 1:]
  So people here are aware of random encounters?
  \item[\Glsentrytext{gm}:]
  Very.
  Once or twice a week on average, something comes from the forest to feed.
  \item[Player 2:]
  So this place is \gls{monster}-proof?
  \item[\Glsentrytext{gm}:]
  Mostly, but they still try.
  Sometimes one feels along the walls and roof, searching for an opening during the night, while people inside debate what kind of creature is trying to come in.
  \item[Player 1:]
  So they only come out at night.
  \item[\Glsentrytext{gm}:]
  No, they often wait until morning.
  That's where you come in!
  In \pgls{bothy}, people expect the \gls{guard} to open the door in the morning, maybe wave \pgls{dawnDolly} around, and give the all-clear.
  \item[Player 2:]
  A what?
  \item[\Glsentrytext{gm}:]
  Like a scarecrow no a stick.
  You waggle it outside the door in the morning, and if something reaches down from the roof and grabs it, then you know there's something on the roof.
  \item[Player 2:]
  How the hell does anyone live in this place?
  \item[\Glsentrytext{gm}:]
  How would you live?
  How would you farm?
  \item[Player 3:]
  Hey guys!
  Did I miss the start?
  Have you done character creation?
  \item[\Glsentrytext{gm}:]
  Here's your character.
  Now, how do you think you'd farm in a world of wandering monsters?
  \item[Player 3:]
  Why's my name `\composeHumanName'?
\end{description}

\bigLine
\vspace{2em}

\noindent
The `wandering monsters' question makes a nice opener, as people can have their own ideas about it.
It also explains most of what the \glspl{pc} do, and why society needs them.

\bigLine

\begin{description}\sf
  \item[\Glsentrytext{gm}:]
  As you approach Stoatfen, a carriage arrives from ahead.
  A single horse and carriage, with four senior \glspl{guard} walking beside it.
  The greasy driver steps off the wagon before it's come to a stop, and shouts `\textit{have you checked inside?
  Are you the new \glspl{fodder}?}'.
  \item[Player 3:]
  The what?
  \item[Player 1:]
  No manners.
  What happened to `hello?'.
  \item[Player 2:]
  Yea, we have the rank of `\glspl{fodder}'.
  It's on the sheet.
  \item[Player 1:]
  Nevertheless.
  \item[\Glsentrytext{gm}:]
  The \glspl{guard} around him are rolling their eyes, clearly used to his behaviour.
  He tells you to go check inside, untying a little \gls{dawnDolly} from his belt, then chucks it at you.

  He says `\textit{we don't have all day.
  Two can go inside, another two can check around the perimeter for footprints.}
  \item[Player 1:]
  I guess we'll check inside then.
  \item[Player 2:]
  What do we roll.
  \item[\Glsentrytext{gm}:]
  Two dice, but here's the catch: just one of you roll the dice to check the situation, then add intelligence and vigilance to go figure out the best way to check the \gls{bothy}, without putting yourself at risk in case something's in there.
  \item[Player 2:]
  \twoDice{4}
  That's a four, and I have neither intelligence, nor vigilance.
  \item[\Glsentrytext{gm}:]
  That just means no Bonus, which means the attribute's average.
  But the dice mean that the situation is bad --- the Sun's in the wrong spot, and doesn't penetrate the windows.

  What is \emph{your} total, using the same dice roll?
  \item[Player 1:]
  I have a plus two to intelligence, so six in total.
  The tie number was seven, so you don't find anything interesting.
  Luckily, nothing dangerous was inside Stoatfen~\gls{bothy}.

  Meanwhile, \pgls{soldier} has joined you to look around the perimeter for tracks.
  Roll the dice, and add your wits and survival skill.
  The tie number is ten, so if you get over ten, the character finds any interesting tracks.
  Get under, and the morning's wasted wandering and misinterpreting what you see.

  \item[Player 3:]
  Looks like a wasted morning then.
  Shouldn't I just not bother?
  \twoDice{10}
  Yea, so with my penalty, that's a complete failure.
  \item[\Glsentrytext{gm}:]
  Yes.
  Gimme a charisma and deceit roll, against \gls{susjot}.
  The tie number is seven.
  \item[Player 3:]
  Really?
  What about the \gls{soldier}?
  \twoDice{8}
  Okay, the total is nine.
  \item[\Glsentrytext{gm}:]
  \Gls{soldier} \composeHumanName\ supports the plan.
  How does your character handle this?
  How does she go about this?
\end{description}

\end{multicols}

\bigLine
\vspace{2em}

\begin{multicols}{2}
\noindent
Nearly all of BIND's systems are just reinterpretations of the same basic roll: $2D6$ plus \gls{attribute} plus \gls{skill}.
Sometimes \gls{equipment} adds a Bonus, like \pgls{weapon} during combat, or using a map for \gls{navigation}.

\begin{enumerate}
  \item
  The \gls{pc} wants some prize (e.g. `check building', or `notice someone stayed the night alone, without a fire').
  \item
  The \gls{pc} wants to avoid a danger (`not jumped by \pgls{monster}', or `avoid being tricked').
  \item
  The \gls{gm} decides on the most appropriate \gls{attribute} and \gls{skill} to use, then the \glsentrylong{tn}.
\end{enumerate}

\begin{tabularx}{\linewidth}{ccccccL}
\hline
$2D6$ & + & \gls{attribute} & + & \gls{skill} & vs & \Glsfmttext{tn} \\
\hline
$2D6$ & + & Intelligence & + & Vigilance & vs & 7 \\
$2D6$ & + & Intelligence -1 & + & Vigilance 2 & vs & 7 \\
$2D6$ & + & -1 & + & 2 & vs & 7 \\
\twoDice{6} & + & -1 & + & 2 & vs & 7 \\
\hline
6 & + & 1 & = & 7 & vs & 7 \\
\hline
\end{tabularx}

\medskip

\noindent
If the roll on the \gls{tn} (i.e. `a tie'), both results occur, or neither, depending on what makes sense.
If both make sense, the player usually chooses.

Interpretation is key, as each roll should be explained.
A skilled hunter missing tracks should be explained by the \gls{gm} with bad conditions, while a successful roll to steal a coin-purse, or a failed roll to convince \pgls{jotter} to grant another \gls{torch} can be fleshed out by the player.

\end{multicols}

\bigLine

\begin{multicols}{3}

\begin{description}\sf
  \item[\Gls{gm}:]
  ``\textit{Time to go}'', \pgls{susjot} shouts.
  \item[Player 1:]
  We can take that out while checking the perimeter then.
  \item[\Gls{gm}:]
  By the time you're half-way round the perimeter, \gls{woetide} has fed her horses, and the \glspl{soldier} are ready to move on.
  They tell you to wait here for \pgls{jotter} to arrive with orders, and she waves goodbye.

  ``Travel straight, never curious.''
  \item[Player 2:]
  `Never curious'?
  What kind of farewell is that?
  \item[Player 1:]
  Curiosity's dangerous.

  \item[\Gls{gm}:]
  While you finish checking the forest's edge, snapping twigs and footsteps
  \ifnum\value{temperature}=0%
    on snow
  \fi%
  approach from the dark.
  \item[Player 2:]
  How much Damage does an axe do?
  \item[\Glsentrytext{gm}:]
  Loads!
  But he asks you not to.
  He asks you to put the axe down.
  He's holding a sword, and dressed in the same black cassock you have and asking you to put the axe down.
\end{description}
\nobreak\bigLine

\noindent
The \glspl{pc} might meet with the \gls{guard} deserter, and learn what he knows about the cave.
Or they may miss him entirely.

Moments later, \pgls{jotter} finally arrives with more new \glspl{guard}.
He explains that goblins routinely raid nearby \glspl{village} and they recently stole a lot of money from traders.
The last group of \glspl{guard} to enter the tunnel took the stolen \glspl{coin} and fled, and will be killed along with the goblins.

The new \glspl{guard} are additional characters.
Everyone can take a second \gls{pc}.

\end{multicols}
