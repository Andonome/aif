\section{Example: The Game Begins}

\begin{multicols}{3}

\noindent
The table has two players because one is running late.
Such is life.


\begin{description}\sf
  \item[Player 1:]
  Shall we start character creation now?
  \item[\Glsentrytext{gm}:]
  Yes!
  There you go, two characters, already made.

  Which did you get?
  I just grabbed the first ones on the pile.
  \item[Player 2:]
  Says here I'm a thief.
  \item[Player 1:]
  Same here.
  Hopefully the next one's a fighter.
  \item[\Glsentrytext{gm}:]
  All the characters are thieves, cutthroats, dick-heads, and bandits, and most of them fighters as well.
  You were sentenced to repay your debt to society in the \gls{templeOfBeasts}.
  \item[Player 1:]
  Apparently the debt is `100~\glsentrylongpl{sp}'; it's at the top of the character sheet.
  \item[\Gls{gm}:]
  After sentencing in the \gls{court}, the \gls{guard} may not set foot in any town, or \gls{village}, or other civilized places.
  You live in the between-places, keeping others safe by hunting \glspl{monster}.
  \item[Player 1:]
  And we pay back the debt with our wages from this \gls{temple}?
  \item[\Gls{gm}:]
  No, you pay the \gls{temple} 1~\gls{sp} each week.
  But you'll hear about that when you get your first mission from \pgls{jotter}, and you haven't arrived yet.
  The \gls{guard} \glspl{soldier} are complaining about the squeaking wagon-wheels, and how it might attract something from the forest.
  \item[Player 2:]
  Isn't that our job?
  Killing things from the forest?
  \item[\Gls{gm}:]
  They give you a contemptuous look, and suggest you head into the forest right now, if you're that keen.
  \item[Player 2:]
  Work-shy, clearly.
  \item[Player 1:]
  Clearly.
  So everyone walking with us is in the \gls{templeOfBeasts}?
  \item[\Gls{gm}:]
  Everyone except the \gls{notary}.
  \Gls{woetide} works for the \gls{templeOfFrost}, who help people prepare for the snow.
  And she has a tale for you.

  She says
  ``I was in a caravan, on the way to start helping with a new \gls{village}, not far from here.
  Some goblins managed to haul some rocks up the trees, and as we passed by, they dropped the rocks on the horses' heads.
  Then came those javelins\ldots

  I fell out the wagon and fled.
  Lost my life savings, but at least left alive, unlike most of the others.''

  \item[Player 1:]
  That's a shame.
  \item[Player 2:]
  Are we there yet?
  \item[\Glsentrytext{gm}:]
  Yes.

  The road widens, and the shadows of the tall trees disperse.
  Stoatfen~\Gls{bothy} has the standard clearing of 50~\glspl{step} in all directions.
  Trees felled, and bushes burnt, to push back the forest, leaving a circle of
  \ifnum\value{temperature}=0%
    white%
  \else%
    yellowed grass.
  \fi.%
  Where most \glspl{bothy} have only one room, Stoatfen has many.
  But despite its size, it performs the same function; it lets people sleep safely.
  \Glspl{monster} are wandering.
  \item[Player 1:]
  So people here are aware of random encounters?
  \item[\Glsentrytext{gm}:]
  Very.
  Once or twice a week, something comes from the forest to feed.
  \Gls{woetide}, for example, may put her horses inside the \gls{bothy}, leave her goods outside, and bar the door and shutters.
  \item[Player 2:]
  So this place is \gls{monster}-proof?
  \item[\Glsentrytext{gm}:]
  Mostly, but they still try.
  Sometimes one feels along the walls and roof, searching for an opening during the night, while people inside debate what kind of creature is trying to come in.
  \item[Player 1:]
  So they only come out at night.
  \item[\Glsentrytext{gm}:]
  No, they often wait until morning.
  That's where you come in!
  In \pgls{bothy}, people expect the \gls{guard} to open the door in the morning, and give the all-clear.
  \item[Player 2:]
  How the hell does anyone live in this place?
  \item[\Glsentrytext{gm}:]
  How would you live?
  How would you farm?
  \item[Player 3:]
  Hey guys!
  Did I miss the start?
  Have you done character creation?
  \item[\Glsentrytext{gm}:]
  Here's your character.
  Now, how do you think you'd farm in a world of wandering monsters?
  \item[Player 3:]
  Why's my name `\composeHumanName'?
\end{description}

\bigLine
\vspace{2em}

\noindent
The `wandering monsters' question makes a nice opener, as people can have their own ideas about it.
It also explains most of what the \glspl{pc} do, and why society needs them.

\bigLine

\begin{description}\sf
  \item[\Glsentrytext{gm}:]
  Arriving at Stoatfen, the senior \glspl{guard} ask you to do the usual checks.
  Go round the building, then check inside, then check around the forest's perimeter.

  ``Not feeling work-shy, are you?''
  \item[Player 3:]
  Okay, so we check around the edges, and then\ldots wait are there \glspl{monster} inside the building?
  \item[Player 1:]
  Maybe something could crawl in.
  We'll go from room to room I guess, \glspl{weapon} out.
  \item[\Gls{gm}:]
  First roll!
  Okay, gimme intelligence and vigilance to plan the route.
  \item[Player 1:]
  Give what?
\end{description}

\end{multicols}

\bigLine
\vspace{2em}

\begin{multicols}{2}
\noindent
Nearly all of BIND's systems are just reinterpretations of the same basic $2D6$ roll, so it's best to make sure all the players understand that base system properly.
Spending two minutes to explain `add these four numbers' can feel agonizingly slow, but it's worth every second to make the players comfortable with how things operate.

\begin{enumerate}
  \item
  The \gls{pc} wants some prize (e.g. `check building').
  \item
  The \gls{pc} wants to avoid a danger (`not jumped by \pgls{monster}').
  \item
  The \gls{gm} decides on the most appropriate \gls{attribute} and \gls{skill} to use, then the \glsentrylong{tn}.
\end{enumerate}

\begin{tabularx}{\linewidth}{ccccccL}
\hline
$2D6$ & + & \gls{attribute} & + & \gls{skill} & vs & \Glsfmttext{tn} \\
\hline
$2D6$ & + & Intelligence & + & Vigilance & vs & 7 \\
$2D6$ & + & Intelligence -1 & + & Vigilance 2 & vs & 7 \\
$2D6$ & + & -1 & + & 2 & vs & 7 \\
\twoDice{6} & + & -1 & + & 2 & vs & 7 \\
\hline
6 & + & 1 & = & 7 & vs & 7 \\
\hline
\end{tabularx}

\vspace{\baselineskip}
\noindent
When rolling \pgls{tn} exactly, the player often chooses to accept both the prize and the danger, or neither.
However, in this case, the \gls{bothy} lies empty, and the thing in the forest is not \pgls{monster}.

\end{multicols}

\bigLine

\begin{multicols}{3}

\begin{description}\sf
  \item[\Gls{gm}:]
  The \gls{bothy}'s completely empty.
  Not even any logs, but it still has an axe by the fire.
  \item[Player 1:]
  We can take that out while checking the perimeter then.
  \item[\Gls{gm}:]
  By the time you're half-way round the perimeter, \gls{woetide} has fed her horses, and the \glspl{soldier} are ready to move on.
  They tell you to wait here for \pgls{jotter} to arrive with orders, and she waves goodbye.

  ``Travel straight, never curious.''
  \item[Player 2:]
  `Never curious'?
  What kind of farewell is that?
  \item[Player 1:]
  Curiosity's dangerous.

  \item[\Gls{gm}:]
  While you finish checking the forest's edge, snapping twigs and footsteps
  \ifnum\value{temperature}=0%
    on snow
  \fi%
  approach from the dark.
  \item[Player 2:]
  How much Damage does an axe do?
  \item[\Glsentrytext{gm}:]
  Loads!
  But he asks you not to.
  He asks you to put the axe down.
  He's holding a sword, and dressed in the same black cassock you have and asking you to put the axe down.
\end{description}

\bigLine\nobreak
\vspace{2em}
\noindent
The \glspl{pc} might meet with the \gls{guard} deserter, and learn what he knows about the cave.
Or they may miss him entirely.

Moments later, the \gls{jotter} finally arrives with more new \glspl{guard}.
He explains that goblins routinely raid nearby \glspl{village} have stolen a lot of money from traders.
The last group to enter the tunnel took the stolen \glspl{coin} and fled, and will be killed along with the goblins.

And the new \glspl{guard} are additional characters.
Everyone can take a second \gls{pc}.

\end{multicols}
