\section{Ascending to the Pit}

\begin{multicols}{2}

\subsection{Getting There}

The opening to the old mine is 5 miles away, which the troupe can walk in \pgls{interval}.
However, if they don't walk extra-fast (and gain \pgls{ep}) then they will have to hunt for the cave-mouth by twilight.
But either way, a goblin scout will notice them walking in the open, and summon more.

\begin{boxtext}
  Ascending up the \ifnum\value{temperature}=0 snowy \fi mountain, the \showTemperature\ breeze clears lumps of low-lying cloud, leaving the face barren except for patches of pine trees.
\end{boxtext}


Finding the entrance requires a \roll{Wits}{Caving} roll at \tn[6] during the day, 8 in the twilight, and 10 at night.

\paragraph{If the troupe camp overnight,}
the goblins come, one every hour, to throw a javelin or rock at them.%
\footnote{All goblin statblocks appear \vpageref{goblinBlocks}.}
They stand well back from the camp, and suffer a -4~Penalty to the throw, but the point is not to kill the \glspl{pc}; the point is to stop them sleeping.

\subsection{The Mouth}

The cave-mouth is a muddy hole, sloping mostly down, like a reed in the wind.
The drop to the ground is 4~\glspl{step} down, and not easy to climb up.%
\footnote{The goblins emerge by sending one with claws to climb, and then helping the others with a rope.}

\end{multicols}

\needspace{15em}
\section{A Small Hole in the Ground}
\label{goblinCaveEntrance}

\begin{multicols}{2}
\renewcommand\npcsymbol{\gls{night}}

\subsection{Last of the Sunlight}

\begin{boxtext}
  You stand in a shaft of Sunlight.
  Darkness surrounds.
  The sound of your landing echoes back to you.
  High-pitched, mocking, giggling noises follow it.
  More rocks follow.
\end{boxtext}

The first \gls{pc} down receives 5 rocks thrown at them, and will have to make a \roll{Dexterity}{Athletics} roll (\tn[10]) or receive \dmg{6} Damage.

The \glspl{pc} can only enter one at a time, and the rocks will keep on coming.%
\footnote{The goblins spend 1~\gls{ap} to pick up a rock, and 1~\gls{ap} to throw.
\Glspl{pc} spend 1~\gls{ap} to move, as usual.}
The players will have to decide, one at a time, who goes next into the dark hole.

\subsubsection{Screams Above}

Once most of the \glspl{pc} enter the dark hole, two skinny goblins crawl out of a tiny side-tunnel, and cause trouble at the top.
They can only squeeze through the tunnel, as they have a Strength Penalty of -1.
Anything larger than this cannot fit through the narrow tunnel.

\playCommentaryCaveIn

\subsection{The Cave-In}

The second after two characters get down the hole, the goblins flee.
If the \glspl{pc} don't give chase, they have failed in their mission.
The goblins will not return to this area for two days, so the \glspl{pc} will eventually have to follow.

\paragraph{If the \glspl{pc} have lit a torch,}
they will spot dry rocks.

\begin{boxtext}
  The flickering torch illuminates a soaking wet cave, littered with bone-dry rocks of every size.
  The air is thick with rocky dust.
\end{boxtext}

The dry rocks on the ground have fallen from the ceiling recently.
The shouts and \ifnum\value{temperature}=1 flooding \else stomps \fi in the cavern have begun a cave-in.

\playCommentaryAftermath

\subsubsection{Rocks Fall}
once the troupe have gone 60~\glspl{step} into the cavern.
They will have to keep running forward to survive, and they are 20~\glspl{step} from the next, safer cavern \gls{area}, where the rocks won't fall on them.
\label{caveIn}

\begin{itemize}
  \item
  Each \gls{round}, have each player roll $1D6$ for each of their characters.
  On the roll of a 1, a rock lands on the character, inflicting $1D6$ Damage.
  \item
  Any \gls{pc} hit by a rock stops, stunned, and loses all movement for the \gls{round}.
  \item
  Each \gls{round}, the Damage and threshold increase~by~1.
\end{itemize}

\caveIn

\paragraph{If any \glspl{pc} flees the 60~\glspl{step} back to the entrance,}
they will find themselves alone, at night, with a couple of goblins waiting nearby.
Those goblins will wait for \pgls{interval} for a moment of weakness, but if that does not arrive, they become bored, and head down to the other cave entrance, far down the mountain.

\label{pcRunaways}
You can leave these \glspl{pc} as wildcards, until the end of the session, then try to fill in their day with a couple of rolls, just as the troupe begin to exit the mountain, far below (assuming anyone survives).

Fleeing \glspl{pc} who go follow the goblin%
\ifnum\value{temperature}=0%
 -footprints in the snow
\else%
  s
\fi%
will be able to chase them to `\nameref{sunRoof}' (\vpageref{sunRoof}).

\pic{cave_1}

\paragraph{Once the cavern collapses,}
the dust thrown up inflicts \pgls{ep} each \gls{round}, until the \glspl{pc} leave the \gls{area}.

\paragraph{Leaving this area}
means they must find the exit.
It looks like any other shadow among shadows, so the troupe must roll \roll{Wits}{Caving} (\tn[10], +2~Bonus for having \pgls{torch}).
They can use \pgls{bandAct}, so they will succeed eventually, as long as enough people help.

\paragraph{The passage out}
becomes smaller and smaller, until the head of the troupe must crawl.
Whoever went first must roll \roll{Dexterity}{Caving}, against \gls{tn} 8, plus their own Strength Bonus (larger people will struggle more).

Every character can roll repeatedly, and every time they fail, every character behind them takes \pgls{ep} due to the dust.

\paragraph{Digging their way out}
only leads to \glsentrylongpl{ep}.

\paragraph{If any characters remain above,}
just leave that \gls{pc}, and focus on those down the goblin hole.
The troupe can find out what happens to them later.

\end{multicols}

