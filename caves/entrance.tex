\section{Ascending to the Pit}

\begin{multicols}{2}

\subsection{Getting There}

The opening to the old mine is 5 miles away, which the troupe can walk in \pgls{interval}.
However, if they don't walk extra-fast (and gain \pgls{ep}) then they will have to hunt for the cave-mouth by twilight.
But either way, a goblin scout will notice them walking in the open, and call for more goblins to wait in ambush.

\begin{boxtext}
  Ascending up the \ifnum\value{temperature}=0 snowy \fi slope, the \showTemperature\ breeze clears lumps of low-lying cloud, leaving the face barren except for patches of pine trees.
\end{boxtext}

Finding the entrance requires a \roll{Wits}{Caving} roll at \tn[6] during the day, 8 in the twilight, and 10 at night.

\paragraph{If the troupe camp overnight,}
the goblins come, one every hour, to throw a javelin or rock at them.%
\footnote{All goblin statblocks appear \vpageref{goblinBlocks}.}
They stand well back from the camp, and suffer a -4~Penalty to the throw, but the point is not to kill the \glspl{pc}; the point is to stop them sleeping.

\ifnum\value{temperature}<2%
  Of course, they will need a fire to stay warm, so staying out of sight is not an option.
\fi

\playCommentaryCaveIn

\subsection{The Mouth}

The cave-mouth is a muddy hole, sloping mostly down, like a reed in the wind.
The drop to the ground is 4~\glspl{step} directly down, and not easy to climb up.%
\footnote{The goblins emerge by sending one with claws to climb, and then helping the others with a rope.}

Entering the cave and staying on one's feet requires a \roll{Dexterity}{Caving} roll at \tn[6].
Failure means the \gls{pc} falls prone%
\exRef{core}{Core Rules}{prone}
and receives 2~Damage plus their Strength~\gls{attribute},

\end{multicols}

\needspace{15em}
\section{A Small Hole in the Ground}
\label{goblinCaveEntrance}

\begin{multicols}{2}
\renewcommand\npcsymbol{\gls{night}}

\subsection{Last of the Sunlight}

\begin{boxtext}
  You stand in a shaft of Sunlight.
  Darkness surrounds.
  The sound of your landing echoes back to you.
  High-pitched, mocking, giggling noises follow it.
  Then come the rocks.
\end{boxtext}

The first \gls{pc} down receives 5 rocks thrown at them, and will have to make a \roll{Dexterity}{Athletics} roll (\tn[10]) or receive \dmg{6} Damage.

The \glspl{pc} can only enter one at a time, and the rocks will keep on coming.%
\footnote{The goblins spend 1~\gls{ap} to pick up a rock, and 1~\gls{ap} to throw.
\Glspl{pc} spend 1~\gls{ap} to move, as usual.}
The players will have to decide, one at a time, who goes next into the dark hole.

\playCommentaryDust

\subsubsection{Screams Above}

Once most of the \glspl{pc} enter the dark hole, two skinny goblins crawl out of a tiny side-tunnel, and cause trouble at the top.
The tunnel is only big enough for those with Strength~-1 or lower.
Anything larger than this cannot fit through the narrow tunnel.

\subsection{The Cave-In}

The second after two characters get down the hole, the goblins flee, and the \glspl{pc} will almost certainly follow.

\paragraph{If the \glspl{pc} decide to wait,}
the goblins simply don't return.

\paragraph{If the \glspl{pc} have lit a torch,}
they will spot dry rocks.

\begin{boxtext}
  The flickering torch illuminates a soaking wet cave, littered with bone-dry rocks of every size.
  The air is thick with rocky dust.
\end{boxtext}

The dry rocks on the ground have fallen from the ceiling recently.
The shouting and \ifnum\value{temperature}=1 recent flooding \else stomping \fi in the cavern have begun a cave-in.

\playCommentaryAftermath

\subsubsection{The Cavern Collapses}
once the troupe follow the goblins down.
Fleeing to safety requires a \roll{Speed}{Athletics} roll at \tn[3] for the first \gls{pc} who descended.
Every \gls{pc} after that has +1 to the \gls{tn} (so the fifth \gls{pc} rolls at \tn[7]).
Failure means the rocks fall on them, inflicting $2D6$~Damage, and they roll again with +1 to the \gls{tn}.

\paragraph{If any \glspl{pc} flees back to the entrance,}
the \gls{tn} is 17 for the first \gls{pc} who descended, with -1 to the \gls{tn} for subsequent \glspl{pc} (so the fifth rolls at \tn[13]).

Once out, they find themselves alone on the \showTemperature\ mountain slope.
Alone except for the two goblins who sneaked out earlier.
The goblins hear the cave-in, and flee homeward.

\label{pcRunaways}
You can leave these \glspl{pc} as wildcards, until the end of the session, then try to fill in their day with a couple of rolls, just as the troupe begin to exit the mountain, far below (assuming anyone survives).

Fleeing \glspl{pc} who go follow the goblin%
\ifnum\value{temperature}=0%
 -footprints in the snow
\else%
  s
\fi%
will be able to chase them to `the \nameref{sunRoof}' (\vpageref{sunRoof}).

\paragraph{Once the cavern collapses,}
the dust thrown up inflicts \pgls{ep} each \gls{round}, until the \glspl{pc} leave the \gls{area}.

\paragraph{The passage out}
becomes smaller and smaller, until the head of the troupe must crawl.
\Glspl{pc} must spend \pgls{ap} (just like in combat) to crawl through.
They roll \roll{Dexterity}{Caving} at \gls{tn} 8, plus their own Strength Bonus (larger people will struggle more).

Each \gls{pc} can re-roll by spending another \gls{ap}, but each \gls{round} will inflict another \gls{ep} on everyone still stuck inside.

\pic{cave_1}

\end{multicols}
