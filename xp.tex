\chapter{Experience Rewards}

\begin{multicols}{2}

\noindent When making antagonists with pen and paper, I'd recommend eyeballing the XP, as enforcing a procedure can turn into a pain.

However, if you're interested in how I've arrived at the XP values here, they are as follows:

Additional points are added for anything which aids these basic Attributes, such as weapons' damage increase, or knacks.

The broad idea is that XP should be proportional to how much of a challenge an antagonist presents.
Additionally, some stats work better together than others.
A creature which can reliably hit people but cannot do much damage is less of a threat than a creature which can strike well \emph{and} deal a lot of Damage.
The three attacking stats -- the creature's final \textit{Strike} score, Damage, and Initiative, are multiplied together, as are the defensive stats -- HP, DR, and Evasion.
The result is that if a creature has a lot of HP, adding some more DR will increase the challenge considerably, but adding more Damage will not make the creature much more dangerous if it cannot hit properly.

\end{multicols}

\begin{equation}
\frac{(( Strike + 8 ) \times ( Initiative + 5) \times ( Damage + 5 ) ) + ( ( TN -2 ) \times ( DR + 2 ) \times ( HP )) - 350}{120}
\end{equation}

