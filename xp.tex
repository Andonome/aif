\chapter{Experience Rewards}

\begin{multicols}{2}

\noindent When making antagonists with pen and paper, I'd recommend eyeballing the XP, as enforcing a procedure can turn into a pain.

However, if you're interested in how I've arrived at the XP values here, they are as follows:

The broad idea is that XP should be proportional to how much of a challenge an antagonist presents.
Additionally, some stats work better together than others.
A creature which can reliably hit people but cannot do much damage is less of a threat than a creature which can strike well \emph{and} deal a lot of Damage.
We calculate attack proficiency by the Damage (from both weapon and Strength), then multiply that by the AP twice.
Then `heft' (a weapon's AP cost to swing) negates some of the utility of the AP.

Defence is calculated by multiplying HP by the DR + 3.
The DR receives this bonus so that it's never 0, and so that it doesn't swing the result too much.

Finally, with Speed and Strength both taken into account twice already (HP derive from Strength), we add those two together, and multiply the result by the attack bonus +3.

The result is that if a creature has a lot of HP, adding some more DR will increase the challenge considerably, but adding more Damage will not make the creature much more dangerous if it cannot hit properly.

\end{multicols}

\begin{equation}
\frac{((Damage + 3 \times AP^2 - Heft) + ((DR + 3) * HP) \times (Dex + Combat + 4)) - 100}{160}
\end{equation}

