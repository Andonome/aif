\chapter{Experience Rewards}

\begin{multicols}{2}

\noindent When making antagonists with pen and paper, I'd recommend eyeballing the XP, as enforcing a procedure can turn into a pain.

However, if you're interested in how I've arrived at the XP values here, they are as follows:

The broad idea is that XP should be proportional to how much of a challenge an antagonist presents.
Additionally, some stats work better together than others.
A creature which can reliably hit people but cannot do much damage is less of a threat than a creature which can strike well \emph{and} deal a lot of Damage, so Attack and Damage are multiplied together.

Spellcasting creatures have their ammunition measured (\glspl{mp}), and if that's higher than their Speed, it gets added instead, and everything summed.
Creatures which rely on their speed have their weapon's \gls{ap} cost negated before this point.

Finally, the ability to kill a creature relies on hitting it (attack returns!), the \gls{dr}, and \glspl{hp}.

It's not an impeccable measure, but it's been working well enough so far.

\end{multicols}

\begin{equation}
\frac{
    \begin{array}[b]{lc}
        & Att \times Dam \\
      + & \max( (Spd - AP cost), MP )^2 \\
      + & (Att - 7 + DR) \times HP\\
      - & 30
    \end{array}
  }{
    13
  }
\end{equation}

