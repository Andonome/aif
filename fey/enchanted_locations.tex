\sidequest[ravencops]{Locations in \glsfmttext{enchantedLands}}

Each \gls{segment} has a location to add to the map \vpageref{extracted/enchanted}.

\sqpart[\gls{vlg}]{ravencops}% AREA
{The House of Grand Stories}% NAME
{Underground elves tell painless stories in perfect rhythm}% SUMMARY

Beehives buzz around a flowery garden.
Three large boulders (which look quite out of place) hide stairs down to a long hall, where perfect elves tell perfect stories of perfect people.
The stories rhyme in a precise pattern, and follow the hero's journey exactly.
The characters in the stories do no wrong, and have no fights, because fighting hurts people, and the storytellers never think about hurting people.

\begin{boxtext}
  Ahead, a tree glows in the dark, lit from below, but you see no fire.

  \ldots

  Closer now, the rock is \pgls{step} wide, made of glass, and covers something below.
  The rock is a roof, and under the earth there is a room with a fire, and people.
  The people have stopped to look up, and stare at you.
\end{boxtext}

This elven home has three chambers, for three elves, and a central area for cooking.
Goblins occasionally visit, bringing supplies of snail-meat and stolen vegetables.
The various cupboards also have $1D6-3$ of the following items:

\begin{itemize}
  \item
  Smoked meats (usable as a day's \glspl{ration}).
  \item
  Stormy moonlight from a storm, captured in a large, glass, phial (usable as a Water \gls{ingredient}).
  \item
  Auroch hooves (usable as an Earth \gls{ingredient}).
\end{itemize}

\elf[\npc{\F\El}{Estel}]

\elf[\npc{\M\El}{Luston \& Silmon}]

\enchantedGoblin[\npc{\F\N}{Sillaberry}]%

\sqpart[\gls{vlg}]{ravencops}% AREA
{Subtle \Glsfmtplural{sepulchre}}% NAME
{Most goblins have forgotten about these \glsfmtplural{ogre}}% SUMMARY

\begin{exampletext}
  \Gls{MindElder} requested the first \gls{sepulchre} before he understood how fragile the dreams of \glspl{ogre} are.
\end{exampletext}
\Gls{MindElder} ordered this \gls{sepulchre} made before he understood how easily the dreams of \glspl{ogre} break.
It houses three, who snore quietly.

\sqpart[\gls{vlg}]{ravencops}% AREA
{The Icebox House}% NAME
{Underground elves live to guard food packed in ice}% SUMMARY
\label{iceboxHouse}

\histEvent{130}{3}{%
  With the old lich killed, \glsfmttext{MindElder} decided to settle down, build the perfect tower, and raise perfect children in a perfect land.
  Unfortunately, the children stole, fought, and disobeyed his orders to stay at home and keep safe.
  He made them all swear oaths to uphold the law, never harming any elf, nor taking property, nor singing out of key.
  Without the ability to sing out of key, none have learnt to sing, but this turned out to be an improvement, as \glsfmttext{MindElder} always enjoyed silence more than song%
}

Thick, glass tiles, a full step wide, pepper the land; these tiles are the roof-windows of elven houses.%
\exRef{stories}{Stories}{elvenGlades}
Smoke rises from a chimney, which juts out through a tree.
Three tall trees surround and hide a stone stairway, leading down to a little door.
Three short taps permits entry.

\begin{boxtext}
  One elf takes water, and whispers gently until the water sleeps, and turns to ice.
  Another prepares a little food, using a rapier's broken-off tip as a knife.
  The rest of the rapier remains mounted on the wall, above the fire; but these elves have no use for weapons.
  Harming people causes pain, and they have promised not to harm anyone.
\end{boxtext}

A little goblin sleeps in a hammock, muttering in his sleep.
`\textit{Carapace pies, tentacle-fry\ldots}'
The elves will have something cooked for him by the time he wakes, and then he must fetch more water, using the ornate bucket, carved from carapace, with an abstract map of the land chiselled around its side.

\elf

\paragraph{If the \glspl{pc} ignore the smoking chimney,}
that's fine.
This \gls{segment} does not advance any plot, and there is nothing the \glspl{pc} need to do.
This \gls{segment} exists simply because elves live in \gls{enchantedLands}, and they will greet guests who knock on their door in a friendly manner, and start telling long, boring stories.

\paragraph{If the \glspl{pc} inspect the bucket-map}
they receive a +2~Bonus to all \gls{navigation} checks within the surrounding \glspl{area}.

\paragraph{If they ask for help,}
they receive it, as long as they make small, reasonable requests.

\paragraph{If they ask questions,}
the elves answer happily.
They know most of the history of the area (find the summary \vpageref{chronologicalEvents}).

\sqpart[\gls{vlg}]{ravencops}% AREA
{Grey Borders}% NAME
{The subtle marsh is hard to spot}% SUMMARY

The snail road widens here, and the trees grow thin where the giant snails swarmed around in little knots, eating and bathing any time they passed this location.

\begin{boxtext}
  The trees are thin, Sunlight floods in.
  It looks like the snails made a hundred little detours around the trees here, clearing bushes and thinning trees.
  You can see over twenty \glspl{step} in most directions.
  Despite the long shadows of the remaining, towering trees, an ambush seems almost impossible.
\end{boxtext}

Soon their knots became a marsh, but slowly -- the ground looks normal for some time.

\begin{boxtext}
  The ground squelches, but does not feel slippery -- it's just muddy, not slimy.
  The centre of the road lets you see for twenty or forty \glspl{step} all around, making a sudden ambush almost impossible, despite the long shadows from the towering trees.

  Unfortunately, the middle of the road makes loud squelching noises, while the roadside -- closer to the shadows -- looks dry and quiet.
\end{boxtext}

The marsh has \pgls{tn} of 10 for all rolls, and the \glspl{trait} depend on the approach.

\paragraph{Walking along the centre}
uses \roll{Strength}{Survival} to wade through.
Failure means the character has been forced off the centre and found a sudden, watery, cavity, and sunk right in.

In this case, others in the troupe will have to rescue them, with \roll{Speed}{Survival} (again, at \tn[10]).
Each Failure Margin means \pgls{ration} lost to the mud, or \pgls{ep} as the character struggles to breathe (player's choice).

\paragraph{Walking near the shadowy tree-roots and along raised, dry, paths}
uses \roll{Wits}{Survival} to spot continuous paths, without becoming tricked.

\paragraph{Finding the way back}
proves difficult, as everywhere looks a little damp and barren.
Every patch of brown earth may be part of the road back, or may be a deep hole, filled with thin mud.

\paragraph{If the troupe remain out here at night,}
they cannot rest properly.

\paragraph{If the troupe find a way through,}
the \gls{tn} reduces by~2 each time (some of the goblins know how to move through it easily).

\stopcontents[sq]


