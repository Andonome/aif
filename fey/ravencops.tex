\section{\Glsfmttext{ravencops}}
\label{ravencops}

\Gls{ravencops} received its name from the guttural bird-calls in the area.
Despite being a verdant forest, it feels bleak, and has little edible food.
All attempts at \gls{foraging} are at \gls{tn}~14.

\Gls{navigation} through the snail paths is risky -- various paths go nowhere, while others loop pointlessly.
Characters should roll \roll{Intelligence}{Survival} (\tn[10]) to head in the right direction.
Each Failure Margin sends the troupe in the wrong direction, adding a pointless mile.
However, every time the troupe ask for directions from friendly \glspl{npc} within \gls{ravencops}, they should receive a permanent +2~Bonus to \gls{navigation} rolls within \gls{ravencops}.

\Gls{oathtower} sucks up most of the \glspl{mp} in the surrounding \gls{region}, which makes the air feel thin.
Unless \pgls{witch} is missing 6~\glspl{mp} or more, they can only receive 2~\glspl{mp} at the end of \pgls{interval}.%
\exRef{core}{Core Rules}{manaVacuum}

\printThreadsInRegion{ravencops}

\histEvent{105}{4}{%
  \Gls{LifeElder} loved the little paths snails make, and wanted to walk across them, but she was too big.
  To create her roads, she used Life \glspl{spell} to grow the snails to monstrous proportions.
  But once they were as big as a house, they just got stuck in the tall trees%
}

\widePic{shadows/ravencops}

\begin{multicols}{2}

\thread[ravencops,oathtower]{Snails \& Goblins}
\label{ravencopsIntro}

\segment{ravencops}% AREA
{The Square of Life}% NAME
{Giant snail devours \gls{crawler} in acid attack}% SUMMARY

\begin{exampletext}
  \Gls{MindElder} has taken to twisting the `minds' of snails to make them crave flesh.
  Most snails do not hunt well, but the acidic spray, and disarming appearance, means they regularly consume \glspl{crawler}.
  This helps keep the \gls{crawler} population low, as the goblins would otherwise feed on the \glspl{crawler}.
\end{exampletext}

\begin{boxtext}
  The echo of a distant crow's cry reaches you, just before fading to nothing.
  The trees look a kind of uniform-brown, without any mottling or variation.
  And the road feels as smooth as pond-scum.
\end{boxtext}

\Gls{running} now requires a \roll{Dexterity}{Athletics} roll at \tn[7] to avoid slipping face-first into the slime, and going prone.%
\exRef{core}{Core}{prone}
\Glspl{pc} can avoid the actual roll by walking slowly (i.e. `\pgls{restingaction}'), or locking arms (`\pgls{bandAct}'), but the moment they need to move fast, they'll have to start with \pgls{natural}, and make individual rolls afterwards.

Failing a \roll{Dexterity}{Athletics} roll (\tn[7]) inflicts 1~\gls{ep}.

\begin{boxtext}
  You find a cross-roads, as the path splits left and right.
  The road to the left looks older, and less slimy, but has \pgls{crawler} running towards you.
  The road to the right also looks old, until the giant snail, approaching silently.
\end{boxtext}

If the \glspl{pc} are near the giant snail when it sprays acid, they can roll \roll{Wits}{Survival} (\tn[7]) to leap into the woods as the snail prepares to spray.

\begin{boxtext}
  As the snail-spittle hits the trees and bushes in a messy gush, it lets off a tiny hiss and the plants wilt fast.
  Leaves wither and bark turns black.
  Behind, the \gls{crawler} turns to flee into the woods with two left-legs melded together by the acid.
\end{boxtext}

\giantSnail

\chitincrawler

If the troupe let the situation unfold, the giant snail ignores them, and chases the wounded \gls{crawler} up a tree.

\segment[\gls{afternoon}]{ravencops}% AREA
{Got a Permit, Mate?}% NAME
{An oathkeeper goblin wants to check troupe's weapon licences}% SUMMARY

\histEvent{55}{3}{%
  Goblins approached \glsfmttext{enchantedLands}, looking for food.
  However, \gls{MindElder} forced them to swear oaths to uphold the myriad laws of the land.
  They remained, and multiplied, and soon the land held an army of goblins%
}

\begin{exampletext}
  The lack of \glspl{crawler} and plentiful giant snails soon brought a lot of goblins to the area.
  \Gls{MindElder} also turned this problem to his advantage by making the goblins swear to capture or kill anyone disturbing the peace.

  The goblins responded with sarcasm, but the spell worked anyway, soon all the goblins lay dead or agreed to take oaths of good behaviour, which work fine as long as the goblins eat regularly.
\end{exampletext}

\begin{boxtext}
  The sky rumbles with thunder, but the air feels thin.
  In the distance, a small person in a long, green cloak walks towards you, carrying a potato so large that it cannot see you.
  A long nose points up, just above the top of the potato, and pasty-white ears flop at shoulder-length.
\end{boxtext}

If the goblin (Abjad) sees the troop, it switches the potato to a one-arm hold and points accusingly at the troupe, then speaks in an unusually deep voice.

%!
\null
\begin{speechtext}
  I hope you got a licence for those weapons!
  Show me!
  Show me the licence, lanky scum!
\end{speechtext}

Abjad (the goblin) observes and insists on the following local laws:%
\footnote{Find the litany of \glsentrytext{ravencops} \vpageref{ravencopsLaws}.}
\label{ravencopsLaws}

\begin{itemize}
  \item
  No high-pitched noises.
  \item
  No jokes, nor words which move to laughter.
  \item
  No singing out-of-lock.
  \item
  No unlicensed weapons.
  \item
  No wandering without clothes on.
\end{itemize}

\enchantedGoblin[\npc{\M\N}{Abjad}]

\paragraph{Dealing with Abjad}
requires a \roll{Charisma}{Empathy} roll at \tn[9].
Failure means he will insist on them going to \gls{oathtower} to admit their criminal behaviour, while giving him all of their \glspl{weapon}.

If the \glspl{pc} calm Abjad successfully, then he won't insist on accompanying them to the tower, but they \emph{will} have to promise to go there while he finds some elves to cook his potato.

\paragraph{At the slightest hint of aggression,}
Abjad flees and tries to find reinforcements.
The goblins will try to track them down, using their \roll{Intelligence}{Athletics}, so the \glspl{pc} will roll at \tn, however they handle it (perhaps \roll{Speed}{Athletics} to simply run from \gls{ravencops}, or \roll{Intelligence}{Stealth} to hide where goblins won't find them).

\paragraph{At the end of the \gls{interval},}
the troupe receive an additional \gls{mp}, as the thunder above releases more mana.

\segment[\gls{vlg}]{oathtower}% AREA
{Groaning \Glsfmtplural{sepulchre}}% NAME
{The troupe must walk quietly and avoid the groaning sepulchres}% SUMMARY

\histEvent{20}{2}{%
  \Glsentrytext{MindElder} finds the goblins overpopulating the area, and fears the day they run out of food; even the most powerful enchantments cannot withstand goblin hunger.
  He rewards loyal goblins with obscene amounts of food, which lets them grow and grow, into hobgoblins, and eventually into \glsfmtplural{ogre}.
  Trials include service in \glsfmttext{oathtower}, hunting dangerous creatures, and plenty of duels (which really helps reduce the population).
  Once the goblins ascend, \glsentrytext{MindElder} places them in an enchanted sleep inside a stone sepulchre (where the goblins can check on them)}

\begin{exampletext}
  Goblins have no natural height limits, so when they eat too much, they just grow and grow, until one day, without any clear cut-off point, people call them a `hobgoblin', and soon after, `\gls{ogre}'.

  When the goblins become \glspl{ogre}, \gls{MindElder} puts them into an enchanted sleep, and tells the goblins their big brothers will awaken when the time of grand feasting comes.
  The goblins must see the sleeping \glspl{ogre} from time to time, or they will suspect murder and betrayal, and even their oaths will not keep them passive.

  So the forest around \gls{oathtower} has slowly filled up with snoring sepulchres, and everyone must tread quietly, lest they wake and ask for breakfast\ldots
\end{exampletext}

\begin{boxtext}
  Past the trees, an arrow's flight away, a mossy tower stands as tall as a feasting hall turns on its end.
  A low groaning noise, like a distant earthquake, \glspl{flood} through the trees, surrounding you.
\end{boxtext}

\Gls{oathtower} has a lot of \glspl{sepulchre} dotted around it, often hidden by trees, and always with little paths leading towards them.
Each one has three \glspl{ogre}, cramped in together.
The sound of snoring emanates for a few hours each day.

\iftoggle{verbose}{
  \paragraph{Once the \glspl{pc} approach \gls{oathtower},}
  place \pgls{sepulchre} on the map \vpageref{extracted/enchanted}.
  It should be inside the forest, near the \glspl{pc}' current location.
}{}

\paragraph{High-pitched noises near \gls{oathtower}}
awaken sleeping \glspl{ogre}.
Every time the \glspl{pc} make a high-pitched noise, mark \pgls{segment} ready \vpageref{wakingOgres} (`\nameref{wakingOgres}').

In order to avoid waking the \glspl{ogre}, \gls{MindElder} has told the goblins to slay anyone making high-pitched noises, such as whistling or laughing.
Farting is also banned, as it makes the goblins giggle, which then wakes the \glspl{ogre}.



\end{multicols}

\stopcontents[segments]
