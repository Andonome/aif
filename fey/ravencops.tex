\section{\Glsfmttext{ravencops}}
\label{ravencops}

\Gls{ravencops} received its name from the guttural bird-calls in the area.
Despite being a verdant forest, it feels bleak, and has little edible food.
All attempts at \gls{foraging} are at \gls{tn}~14.

\Gls{navigation} through the snail paths is risky -- various paths go nowhere, while others loop pointlessly.
Characters should roll \roll{Intelligence}{Survival} (\tn[10]) to head in the right direction.
Each Failure Margin sends the troupe in the wrong direction, adding a pointless mile.
However, every time the troupe ask for directions from friendly \glspl{npc} within \gls{ravencops}, they should receive a permanent +2~Bonus to \gls{navigation} rolls within \gls{ravencops}.

\Gls{oathtower} sucks up most of the \glspl{mp} in the surrounding \gls{region}, which makes the air feel thin.
Unless \pgls{witch} is missing 6~\glspl{mp} or more, they can only receive 2~\glspl{mp} at the end of \pgls{interval}.%
\exRef{core}{Core Rules}{manaVacuum}

\printSideQuestsInRegion{ravencops}

\histEvent{105}{4}{%
  \Gls{LifeElder} loved the little paths snails make, and wanted to walk across them, but she was too big.
  To create her roads, she used Life \glspl{spell} to grow the snails to monstrous proportions.
  But once they were as big as a house, they just got stuck in the tall trees%
}

\begin{multicols}{2}

\playCommentary[t]{
  \begin{description}
    \item[\gls{gm}:]
    Ahead on the road\ldots one moment, \vpageref{ravencopsIntro}\ldots
    ``The echo of a distant crow's cry reaches you, just before fading.
    The trees look a kind of uniform-brown, without any mottling or variation.
    And the road feels as smooth as pond-scum.''
    \item[Player 1:]
    (Sinkmaul)
    Can we stand on that?
    \item[\gls{gm}:]
    Take it easy with \pgls{restingaction}, and set a die to six, now roll the other one.
    Since the \glsentrylong{tn} is 7, you'll pass this automatically, as long as you're walking with care.
    \item[Player 1:]
    (Sinkmaul)
    So the road enforces chill\ldots
    \item[\gls{gm}:]
    The trees obscure the tower, so you can't see which way gets you closest to the tower right now, but you can see a giant arachnid crawling forward as deep-brown as the trees, it looks nearly as tough.
    It spots you and sprints, front-legs raised.
    \item[Player 2:]
    (Mildrain)
    Are you sure I don't have fireball?
    \item[Player 1:]
    (Sinkmaul)
    I'm running in the opposite direction.
    \item[\gls{gm}:]
    Spend \pgls{ap}!
    You turn and move towards another giant snail as it opens its face-flaps towards you.
    \item[Player 3:]
    (Cleftbarb)
    Let's run back the way we came.
    \dicef{9}
    \item[\gls{gm}:]
    As you run back, the crawler gallops towards you like a horse, but as it rounds the corner, brownish-green liquid sprays onto it.
    It collapses into a ball, and two of its hind legs melt together.
    \item[Player 1:]
    (Sinkmaul)
    Is it dead?
    \item[\gls{gm}:]
    It uncurls, and moves into the forest, as the snail pursues it, belching the acid to clear its path.
    \item[Player 2:]
    (Mildrain)
    So this road\ldots
    \item[Player 3:]
    (Cleftbarb)
    Let's just press on.
  \end{description}
}{
  Explaining rules up-front never works.
  But making a small pause to show how to use something makes it \emph{relevant}.
}

\noindent
The first \gls{segment} only starts once the \glspl{pc} approach the \gls{oathtower}, after which new places arrive on the map fast.
Until then, the \gls{region} introduces itself through the eclectic scenes in \nameref{ravencopsIntro}, \vpageref{ravencopsIntro}.

\sidequest[ravencops]{Snails \& Goblins}
\label{ravencopsIntro}

\sqpart{ravencops}% AREA
{The Square of Life}% NAME
{Giant snail devours \gls{crawler} in acid attack}% SUMMARY

\begin{exampletext}
  \Gls{MindElder} has taken to twisting the `minds' of snails to make them crave flesh.
  Most snails do not hunt well, but the acidic spray, and disarming appearance, means they regularly consume \glspl{crawler}.
  This helps keep the \gls{crawler} population low, as the goblins would otherwise feed on the \glspl{crawler}.
\end{exampletext}

\begin{boxtext}
  The echo of a distant crow's cry reaches you, just before fading to nothing.
  The trees look a kind of uniform-brown, without any mottling or variation.
  And the road feels as smooth as pond-scum.
\end{boxtext}

\Gls{running} now requires a \roll{Dexterity}{Athletics} roll at \tn[7] to avoid slipping face-first into the slime, and going prone.%
\exRef{core}{Core}{prone}
\Glspl{pc} can avoid the actual roll by walking slowly (i.e. `\pgls{restingaction}'), or locking arms (`\pgls{bandAct}'), but the moment they need to move fast, they'll have to start with \pgls{natural}, and make individual rolls afterwards.

Failing a \roll{Dexterity}{Athletics} roll (\tn[7]) inflicts 1~\gls{ep}.

\begin{boxtext}
  You find a cross-roads, as the path splits left and right.
  The road to the left looks older, and less slimy, but has \pgls{crawler} running towards you.
  The road to the right also looks old, until the giant snail, approaching silently.
\end{boxtext}

If the \glspl{pc} are near the giant snail when it sprays acid, they can roll \roll{Wits}{Survival} (\tn[7]) to leap into the woods as the snail prepares to spray.

\begin{boxtext}
  As the snail-spittle hits the trees and bushes in a messy gush, it lets off a tiny hiss and the plants wilt fast.
  Leaves wither and bark turns black.
  Behind, the \gls{crawler} turns to flee into the woods with two left-legs melded together by the acid.
\end{boxtext}

\giantSnail

\chitincrawler

If the troupe let the situation unfold, the giant snail ignores them, and chases the wounded \gls{crawler} up a tree.

\sqpart[\gls{afternoon}]{ravencops}% AREA
{Got a Permit, Mate?}% NAME
{An oathkeeper goblin wants to check troupe's weapon licences}% SUMMARY

\histEvent{55}{3}{%
  Goblins approached \glsfmttext{enchantedLands}, looking for food.
  However, \gls{MindElder} forced them to swear oaths to uphold the myriad laws of the land.
  They remains, and multiplied, and soon the land held an army of goblins%
}

\begin{exampletext}
  The lack of \glspl{crawler} and plentiful giant snails soon brought a lot of goblins to the area.
  \Gls{MindElder} also turned this problem to his advantage by making the goblins swear to capture or kill anyone disturbing the peace.

  The goblins responded with sarcasm, but the spell worked anyway, soon all the goblins lay dead or agreed to take oaths of good behaviour, which work fine as long as the goblins eat regularly.
\end{exampletext}

\begin{boxtext}
  The sky rumbles with thunder, but the air feels thin.
  In the distance, a small person in a long, green cloak walks towards you, carrying a potato so large that it cannot see you.
  A long nose points up, just above the top of the potato, and pasty-white ears flop at shoulder-length.
\end{boxtext}

If the goblin (Abjad) sees the troop, it switches the potato to a one-arm hold and points accusingly at the troupe, then speaks in an unusually deep voice.

%!
\null
\begin{speechtext}
  I hope you got a licence for those weapons!
  Show me!
  Show me the licence, lanky scum!
\end{speechtext}

Abjad (the goblin) observes and insists on the following local laws:

\begin{itemize}
  \item
  No high-pitched noises.
  \item
  No jokes, nor words which move to laughter.
  \item
  No unlicensed weapons.
  \item
  No wandering without clothes on.
  \item
  No singing out-of-lock.
\end{itemize}

\enchantedGoblin[\npc{\M\N}{Abjad}]

\paragraph{Dealing with Abjad}
requires a \roll{Charisma}{Empathy} roll at \tn[9].
Failure means he will insist on them going to \gls{oathtower} to admit their criminal behaviour, while giving him all of their \glspl{weapon}.

If the \glspl{pc} calm Abjad successfully, then he won't insist on accompanying them to the tower, but they \emph{will} have to promise to go there while he finds some elves to cook his potato.

\paragraph{At the slightest hint of aggression,}
Abjad flees and tries to find reinforcements.
The goblins will try to track them down, using their \roll{Intelligence}{Athletics}, so the \glspl{pc} will roll at \tn, however they handle it (perhaps \roll{Speed}{Athletics} to simply run from \gls{ravencops}, or \roll{Intelligence}{Stealth} to hide where goblins won't find them).

\paragraph{At the end of the \gls{interval},}
the troupe receive an additional \gls{mp}, as the thunder above releases more mana.


\end{multicols}

\stopcontents[sq]
