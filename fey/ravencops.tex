\section{\Glsfmttext{ravencops}}
\label{ravencops}

\Gls{ravencops} received its name from the guttural bird-calls in the area.
Despite being a verdant forest, it feels bleak, and has little edible food.
All attempts at \gls{foraging} are at \gls{tn}~14.

\Gls{oathtower} sucks up most of the \glspl{mp} in the surrounding \gls{region}, which makes the air feel thin.
Unless \pgls{witch} is missing 6~\glspl{mp} or more, they can only receive 2~\glspl{mp} at the end of \pgls{interval}.

\printSideQuestsInRegion{ravencops}

\histEvent{105}{4}{%
  \Gls{LifeElder} loved the little paths snails make, and wanted to walk across them, but she was too big.
  To create her roads, she used Life \glspl{spell} to grow the snails to monstrous proportions.
  But once they were as big as a house, they just got stuck in the tall trees%
}

\begin{multicols}{2}

\iftoggle{verbose}{
  When the eye of the story moves North, it finds a stone ziggurat.
  Then it moves West and finds a quarry, surrounded by giant snails.
  The land seems full of life, because \glspl{sq} below tricked it.
  The map had only five locations, but every time the \glspl{pc} move, another \gls{segment} in the story places a location on the map.

  Every \gls{interval}, down the list of \glspl{segment} in the \gls{region} (marked `\gls{sqr}'), and make it happen.
  Once the \gls{segment} ends, mark the \emph{next \gls{segment} in the \gls{sq} as ready} (not the next \gls{segment} in the list).

  The order of the \glspl{segment} depends a lot on the players.
  As an example:

  \begin{itemize}
    \item
    The players enter \gls{ravencops}, and the \gls{gm} runs the first two available \glspl{segment}.
    \item
    The players decide to investigate \gls{oathtower}, so the \gls{gm} finds that location \vpageref{oathtower}\ldots
    \item
    As the \gls{gm} runs \gls{oathtower}'s only available \gls{segment}, the \glspl{pc} get into a lot of trouble and have to flee.
    The \gls{gm} finds no more \glspl{segment} marked as ready in that \gls{region}, so nothing more happens.
    \item
    The \glspl{pc} want to investigate \gls{plateauGardens}, but they have to journey through \gls{ravencops} again to reach it.
    A new \gls{segment} has been made available after journeying to the \gls{oathtower}.
  \end{itemize}

  Try jumping through the \glspl{region} quickly, and checking off two \glspl{segment} in each one.
  What happens if the players decide to ignore \gls{oathtower} and go straight from \gls{ravencops} to \gls{sunway} (\vpageref{sunway}) then \gls{plateauGardens} (\vpageref{plateauGardens})?
}{}

\sidequest[ravencops,oathtower,sunway]{\Glsfmttext{enchantedLands}}

\sqpart{oathtower}% AREA
{Groaning Ziggurats}% NAME
{The troupe must walk quietly and avoid the groaning ziggurats}% SUMMARY

\histEvent{20}{2}{%
  \Glsentrytext{MindElder} finds the goblins overpopulating the area, and fears the day they run out of food; even the most powerful enchantments cannot withstand goblin hunger.
  He rewards loyal goblins with obscene amounts of food, which lets them grow and grow, into hobgoblins, and eventually into \glsfmtplural{ogre}.
  Trials include service in \glsfmttext{oathtower}, hunting dangerous creatures, and plenty of duels (which really helps reduce the population).
  Once the goblins ascend, \glsentrytext{MindElder} places them in an enchanted sleep inside a stone ziggurat (where the goblins can check on them)}

\begin{exampletext}
  Goblins have no natural height limits, so when they eat too much, they just grow and grow, until one day, without any clear cut-off point, people call them a `hobgoblin', and soon after, `\gls{ogre}'.

  When the goblins become \glspl{ogre}, \gls{MindElder} puts them into an enchanted sleep, and tells the goblins their big brothers will awaken when the time of grand feasting comes.

  The goblins must see the sleeping \glspl{ogre} from time to time, or they will suspect murder and betrayal, and even their oaths will not keep them passive.
  Hunger and murder can break even the \gls{MindElder}'s enchanted oaths, and high-pitched noises will wake the sleeping \glspl{ogre}.
\end{exampletext}

\begin{boxtext}
  Past the trees, an arrow's flight away, a mossy tower stands as tall as a feasting hall turns on its end.
  A low groaning noise, like a distant earthquake, floods through the trees, surrounding you.
\end{boxtext}

\Gls{oathtower} has a lot of ziggurats dotted around it, often hidden by trees, and always with little paths leading towards them.
Each one has three \glspl{ogre}, cramped in together.
They stand as wide as a cottage, but not as tall, and the sound of snoring emanates for a few hours each day.

\paragraph{High-pitched noises near \gls{oathtower}}
have a 1 in 6 chance of waking \pgls{ogre}.
The chances increase by 1 for loud noises, or noises closer to the ziggurats.

In order to avoid waking the \glspl{ogre}, \gls{MindElder} has told the goblins to slay anyone making high-pitched noises, such as whistling or laughing.
Farting is also banned, as it makes the goblins giggle, which then wakes the \glspl{ogre}.

\enchantedOgre[\npc{\M\N}{`The Grave'}]

\enchantedOgre[\npc{\F\N}{Kerning}]

\sqpart{ravencops}% AREA
{Goblins in the Quarry}% NAME
{\Glsfmttext{romeo} should be working, but needs to complete the perfect poem}% SUMMARY


\Gls{MindElder} has sent his son \gls{romeo} to oversee the goblins, excavating rock at the quarry, and cutting long slabs to construct more ziggurats.
But \gls{romeo} can't think of anything but the poem he needs to write, to tell \gls{juliet} how he feels.
Unfortunately, his father raised him to be a perfectionist, which means he can't write perfect poetry, or good poetry, or bad poetry, or any poetry at all.

\begin{speechtext}
  What rhymes with snail?
  Mail, sail, bail\ldots hay-bail?
  Are hey-bails a thing?
  But `hey' is too informal.
  Better to say `hello'.
  `Hello-bail'\ldots no it sounds non-committal.
\end{speechtext}

So he stands looking at a blank sheet of paper, while twenty goblins ignore him, and bicker about pick-axes and the proper way to use a cart.

\begin{boxtext}
  In the near-distance, around this corner (or possibly two), someone, or something, is hitting metal on rocks.
  The metal sounds strange, butt probably iron.
  The rocks give that satisfying crack that rocks give with a long, clean cut.
\end{boxtext}

\paragraph{If the \glspl{pc} help him with the poem,}
then he perks up and starts doing his job.
First, he stops the goblins bickering.
Second, he helps them lift the rocks with Force magic (while muttering `light as a feather, stiff as a board').
%! Maybe footnote?
This \gls{segment} tells the \glspl{pc} that \gls{romeo} understands the Force Sphere, which indicates the potential they both have as ritual casters to work together.

\paragraph{If the \glspl{pc} ask about the poem's recipient,}
\gls{romeo} becomes evasive.
His father doesn't approve of the `oathless' types, and he feels ashamed of loving such an `air-headed' person, despite all she's taught him.

\paragraph{If the \glspl{pc} do nothing,}
\gls{romeo} remains at the quarry, thinking of the perfect words.

\paragraph{If the troupe commit crimes,}
the goblins will ignore them as long as they can.
They have taken an oath to dig rocks, and they will continue to dig until something shakes them from their oath.

\romeo

%\showStdSpells

\sqpart{sunway}% AREA
{The Elven Way}% NAME
{Hidden stairs lead to \glsfmttext{plateauGardens} above}% SUMMARY

In the causeway between the \gls{ravencops} forest and the plateau gardens, a single plateau has a hidden stairway, going up.
\Glspl{crawler} cannot make much use of the narrow stairs, with occasional hand-holds for little fingers.
People who don't know about the stairs cannot usually see them, as every step blends into the tall rock-face from below.
But once someone notices the first step, they see the next, and then the next, and so on.

Spotting the rocks requires a \roll{Wits}{Vigilance} roll at \tn[12].
The \gls{tn} increases by~+1 in the rain, and by~+3 at night.


\sqpart{ravencops}% AREA
{The Icebox House}% NAME
{Underground elves live to guard food packed in ice}% SUMMARY

\histEvent{130}{3}{%
  With the old lich killed, \glsfmttext{MindElder} decided to settle down, build the perfect tower, and raise perfect children in a perfect land.
  Unfortunately, the children stole, fought, and disobeyed his orders to stay at home and keep safe.
  He made them all swear oaths to uphold the law, never harming any elf, nor taking property, nor singing out of key.
  Without the ability to sing out of key, none have learnt to sing, but this turned out to be an improvement, as \glsfmttext{MindElder} always enjoyed silence more than song%
}

Smoke rises from a chimney, which juts out through a tree.
Thick, glass tiles, a full step wide, pepper the land.
Three tall trees surround and hide a stone stairway, leading down to a little door.
Three short taps permits entry.

One elf takes water, and whispers gently until it sleeps, and turns to ice.
Another prepares a little food, using a rapier's tip as a knife.
The rest of the rapier remains mounted on the wall, above the fire; but these elves have no use for weapons.
Harming people causes pain, and they have promised not to harm anyone.

A little goblin sleeps in a hammock, muttering in his sleep.
`Carapace pies, tentacle-fry\ldots'
The elves will have something cooked for him by the time he wakes, and then he must fetch more water, using the ornate bucket, carved from carapace, with an abstract map of the land chiselled around its side.

\elf

\sqpart{ravencops}% AREA
{The House of Grand Stories}% NAME
{Underground elves tell painless stories in perfect rhythm}% SUMMARY

Perfect elves tell perfect stories of perfect people.
The stories rhyme in a precise pattern, and follow the hero's journey exactly.
The characters in the stories do no wrong, and have no fights, because fighting hurts people, and the storytellers never think about hurting people.

The elven home has three chambers, for three elves, and a central area for cooking.
Goblins occasionally visit, bringing supplies of snail-meat and stolen vegetables.
The various cupboards also have $1D6-3$ of the following items:

\begin{itemize}
  \item
  Smoked meats (usable as a day's rations).
  \item
  Stormy moonlight from a storm, captured in a large, glass, phial (usable as a Water \gls{ingredient}).
  \item
  Auroch hooves (usable as an Earth \gls{ingredient}).
\end{itemize}

\elf

\end{multicols}

