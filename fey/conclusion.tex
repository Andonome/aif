\section{Considerations}

\begin{multicols}{2}

\subsection{The Controlled Collapse of an Ecosystem}

The grotesque stack of \glspl{spell} and plans infecting the Elven lands have all the balance of a two-legged spider.
And like any unbalanced thing, if it falls wrong, then it can fall on people nearby.
The players will need to understand the ecosystem and all its parts in order to make it fall down in the right way.

\subsubsection{Messing with the Goblins}
proves easy.
Goblins tend not to follow rules and laws smoothly,%
\exRef{judgement}{Judgement}{goblin}
and don't fully grasp the rules of \gls{enchantedLands}.

If the \glspl{pc} steal a goblin's weapon's licence, the goblin drops the weapon.
If they tell a goblin that she just made a joke, she will believe them, and hand herself in to \gls{oathtower}.
If the \glspl{pc} feed the goblins beans from the vine-bridges,
they begin to fart, making others explode in laughter, which prompts arrests, arguments over who dealt it, and wakes \pgls{ogre} or two if the goblins are anywhere near \gls{oathtower}.

With enough in-fighting, the goblin population will decrease, but never by much, and not for long.

\subsubsection{Waking the \Glsfmtplural{ogre}}
means death to everyone in \gls{enchantedLands}, and many beyond.
If \pgls{ogre} wakes, hungry and confused after its long sleep, the goblins will do damage-control by leading it to food (perhaps in \gls{oathtower}, or the Icebox House \vpageref{iceboxHouse}).
But if too many wake, a hungry horde will form.

The \glspl{pc} might try to kill the sleeping \glspl{ogre} quietly, but the sepulchres have only tiny grates at the side, enough for a goblin to hear their breathing but not enough to crawl through, or even extend an arm into comfortably.
The \glspl{pc} also don't know where \gls{MindElder} placed all of them.

In total, eight sepulchres remain in \gls{ravencops}, mostly near \gls{oathtower}.

\subsubsection{\Glsfmtname{MindElder}'s Death}
would be disastrous.
Goblin oaths would start to break, little by little, while the elves in \gls{enchantedLands} would lose their enchantments much more slowly.
The goblins would awaken the \glspl{ogre} with screeches and wails held in for half of their lives.

\begin{itemize}
  \item
  Within a week, they would eat through the giant snails in \gls{ravencops} (while many continue to uphold bits and pieces of their various oaths).
  \item
  Within a week and one day, the goblin horde would eat through every elf in \gls{enchantedLands}.
  \item
  The ninth day would begin with a siege upon \gls{coppernut}.
\end{itemize}

\paragraph{Without the \Glsfmttext{disgnome}}
the elves of \gls{plateauGardens} wake up and act a little (but just a little) more proactively.
If the \glspl{pc} convince them to do something, they may convince \gls{LifeElder}.
However, \gls{LifeElder} will not take suggestions from outsiders.

If the elves remain affected by all the \gls{disgnome}, they will be unable to defend themselves, and any goblin attacks will soon find them, as their desperate fingers claw along the cliffs, and find the stairway (\vpageref{hiddenStairs}).

\subsubsection{Making the Elders Talk}
will take perseverance, but it can work.
If \gls{LifeElder} and \gls{MindElder} work together, they can  reduce the giant snails in the area slowly, while \gls{LifeElder} introduces a few, subtle, `goblin-traps'.

\subsubsection{Elven Death by Human Hands}
will prompt a vicious reaction from \gls{romeo} and \gls{juliet}, along with a couple of other elves from the \gls{plateauGardens} if the \gls{disgnome} has been quelled.
The elves will begin planning to wipe out \gls{coppernut} in order to remove all humans from the area.
The plot will begin by casting strange spells on the road out, so they can ensure complete destruction.%
\footnote{If ten refugees return, a battalion of humans may return.
But if nobody returns from the road for a month, then people will simply decide to not travel along that road rather than investigate.}

\subsubsection{The Grand Spell}
\label{grandSpell}
planned by \gls{romeo} and \gls{juliet} will require 16~\glspl{ingredient} in total.
Once they have all they need, they make each one into \pgls{boon}, and begin a grand ritual of song-magic, harmonizing together.

\spell{Shattered Identity in G Minor}% Name
  {Detailed, Distant, Divergent, Duplicated}% Enhancements
  {Warp}% Action
  {Water, Fate, Earth, Fire}% Spheres
  {\roll{Wits}{Survival}}% Resist with
  {The chorus alone is twenty minutes long, but once the spell really gets going, it binds everyone and everything at \spellRange\ into a self-jealous, psychic lump of non-space.}% Description
  {Everyone involved drops pieces of their memories, desires, and their perception of time into the fractured space, and each space involved loops together.

  Everyone involved gains a +\arabic{spellPlusOne}~Bonus to Empathy tasks, and a -\arabic{spellPlusOne}~Penalty to all other tasks.
  Both the Bonus and the Penalty reduce by one each time someone travels between one of the connected area, and finds another part of themselves.}

\begin{boxtext}
  Elves from \gls{plateauGardens} look down at you, scared and confused.
  Their usual blank-eyed cool has gone and you remember kissing \pgls{ogre} good night, and hoping to become like him one day.

  \Gls{oathtower}'s stairs lead up to the \gls{plateauGardens}, or you can go into the elven home through the trees.
  Two goblins stand on the cooker holding your memory of learning the Air \gls{sphere}.

    ``\textit{Sorry, wrong person!}'', the little goblin says in Elvish.

    Up the stairs, and out of the elvish house, \gls{oathtower}'s library is full of goblins, all of them are \gls{MindElder}, shouting for everyone to leave his home.
    Then they turn and point at you in unison, and begin to shriek, ``\textit{Identify! Identify}''.

    An eyeball reaches in, to return your name, if you want it\ldots ?
    The other eyeball searches the room upstairs.

    The name feels right, but the snail has another, and that one feels right too.
    The snail wonders how you know which name is your name, and wonders if you would share.
    The snail is ashamed of its nakedness, so the eyeball slithers back out of the window.

    The goblin hunger was deep, now infectious.
    You could enjoy the taste of your own arm, and it wouldn't actually be \emph{yours}, so it would be fine, and you would get a whole arm to yourself.

    \Gls{SnailTamer} arrives, telling you he loves all of you, and you're all doing really well.
    He's not really there, as he was taking a nap, and just decided to have a little dream with you guys to help everyone out.

\end{boxtext}

The spell ends before long, or it should if the players aren't into Dada-Taoism.
Or if they're getting the vibe, continue handing out memories until they've pieced their characters together.

After the experience ends, \gls{LifeElder} wanders past, stops to observe a spider-web, then continues.

\iftoggle{verbose}{
  \subsection{Broken Resolutions}

  If your table runs out of time for the night, and you need to wrap everything up quickly, switch to \gls{downtime} plans, and give each \gls{pc} \pgls{action}.
  If they want to negotiate with \gls{MindElder}, they might roll \roll{Intelligence}{Empathy}, or if they think it's time to kill those \glspl{ogre}, sleeping in their sepulchres.

  For some added drama, ask each player to roll under a cup, napkin, or sleeve.
  Leave the results hidden until everyone has decided what they want to do.
  One all results are in, determine the outcomes based entirely on the rolls.
  If \pgls{pc} spent the entire \gls{downtime} trying to redirect the river through \gls{shadepaths}, it could flood the \gls{sunway} and block snails from accessing \gls{ravencops}.
  Or if they spent their time trying to kill \glspl{ogre}, a failed roll could mean that \pgls{ogre} eats the \gls{pc} alive before marauding around \gls{ravencops}.
}{

  \subsection{No Way Home}

  If you want to run this arc over multiple sessions, or if the troupe find themselves stuck here over \gls{downtime}, the elven lands have plenty of reasons for characters to leave, disappear or get distracted.

  \subsubsection{In \glsfmttext{plateauGardens}}

  Characters leave because,

  \begin{enumerate}
    \item
    they ate the wrong thing, and their limbs shrivelled.
    The elves hope to fix them `soon' (and will, once the player returns).
    \item
    they wandered off with an elf (leading to gossip and laughter), and have not returned.
    The next session resumes once all hope is lost for finding them.
    \item
    a clerical error resulted in an immediate summoning from an overseer.
    A new \gls{pc} arrived to deliver the letter.
  \end{enumerate}

  Characters arrive because,

  \begin{enumerate}
    \item
    the \gls{jotter} wants to know what the hold-up is.
    \item
    the \gls{pc} was lost, some time ago.
    Hunger and monsters killed the rest of their troupe.
  \end{enumerate}

  \subsubsection{In the pale forest}

  Characters leave because,

  \begin{enumerate}
    \item
    they whistled out of key, and the goblins told them it was time to leave.
    \item
    \pgls{ogre} woke up, her hand reached out, and she grabbed the character.
    \gls{MindElder} reacted quickly, putting everyone in enchanted sleep, but she holds the character like a child with a doll; best not to wake them.
    \item
    goblins prepared food for Winter, grabbing all that was edible, and stuffing it in ice.
    The `it' included the character, who is `edible'; but the \gls{MindElder}'s enchanted sleep spell should deep them safe.
  \end{enumerate}

  Characters arrive because,

  \begin{enumerate}
    \item
    twelve oathkeeper goblins caught them in the forest, along with a bear and \pgls{griffin}.
    The goblins take their prizes -- all tied to poles -- back to share with the rest.
    \item
    the local overseer sent the character with a letter for \gls{MindElder}.
    It contains a generic proposal for mutual aid in killing beasts.
  \end{enumerate}

}

\end{multicols}

\printglossary[
  title={Factions},
  type=people,
  style=topicmcols,
  ]

