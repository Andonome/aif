\stopcontents[sq]

\chapter{Loose Ends}
\epigraph{
  Can you name the nameless one?

  %Can you draw a guilt?

  Can you shoe a snail?
}

\label{looseThreads}


\section{In Closing}
\label{feyClosing}

\begin{multicols}{2}

\subsection{The Controlled Collapse of an Ecosystem}

The grotesque stack of \glspl{spell} and plans infecting the Elven lands have all the balance of a two-legged spider.
And like any unbalanced thing, if it falls wrong, then it can fall on people nearby.
The players will need to understand the ecosystem and all its parts in order to make it fall down in the right way.

The following \glspl{sq} do not begin ready-to-go.
You can mark the \glspl{segment} each ready (`\sqr') as the eco-system begins to collapse.
Most have the `\squash' symbol, meaning they should be run at the same time as the part below.

\subsubsection{Messing with the Goblins}
proves easy.
Goblins tend not to follow rules and laws smoothly,%
\exRef{judgement}{Judgement}{goblin}
and don't fully grasp the rules of \gls{enchantedLands}.

If the \glspl{pc} steal a goblin's weapon's licence, the goblin drops the weapon.
If they tell a goblin that she just made a joke, she will believe them, and hand herself in to \gls{oathtower}.

\paragraph{With enough in-fighting,}
the goblin population will decrease, and you can mark \pgls{segment} from \nameref{goblinsRise} as not ready (`\sqn').

\sidequest[ravencops,oathtower]{Waking the \Glsfmtplural{ogre}}

\noindent
Every high-pitched noise around \gls{enchantedLands} activates another \gls{segment}, below.
This may occur, for example, if goblins raid \gls{plateauGardens} and take some of the beans from the bean-vine bridges, which makes them fart (which then makes other goblins giggle like a rusty gate).

The \glspl{pc} might try to kill the sleeping \glspl{ogre} quietly, but the \glspl{sepulchre} have only tiny grates at the side, enough for a goblin to hear their breathing but not enough to crawl through, or even extend an arm into comfortably.
The \glspl{pc} also don't know where \gls{MindElder} placed all of them.
In total, eight \glspl{sepulchre} remain in \gls{ravencops}, mostly near \gls{oathtower}.

\paragraph{Every time the \glspl{pc} destroy \pgls{sepulchre},}
the goblins notice within a day, and become enraged.
Mark one \gls{segment} below as ready (as they begin to wake \glspl{ogre} to protect them), and make another \gls{segment} ready in \nameref{goblinsRise} (\vpageref{goblinsRise}).

Once the \glspl{pc} destroy \ref{lastOgreSegment} \glspl{sepulchre}, you can start marking the \glspl{segment} below as unavailable, scoring them out (starting with the last).
However, \glspl{segment} from \nameref{goblinsRise} remain active.

\setcounter{sqNo}{-1}
\sqpart[\squash]{ravencops}% AREA
{Breakfast to Go}% NAME
{Three \glsfmtplural{ogre} awaken and need food}% SUMMARY
\stepcounter{sqNo}

When the \glspl{ogre} awaken initially, they feel hungry, but still bound by their oaths.
They ask the troupe for food, and will follow them anywhere, or go anywhere they point to.

\enchantedOgre[\npc{\T[3]\M\F\N}{\Glsfmtplural{ogre}}]

\paragraph{Any goblin present}
leads them to food (perhaps in \gls{oathtower}, or the Icebox House \vpageref{iceboxHouse}).
\Gls{MindElder} soon puts them back to sleep, and inquires about who might have been making high-pitches noises.

\sqpart[\squash]{ravencops}% AREA
{Hunting for the Ice Box}% NAME
{Three \glsfmtplural{ogre} hunt for the ice box house}% SUMMARY
\label{ogresEatIcebox}

Three \glspl{ogre} have awakened, and hunt for the Icebox House, where \gls{MindElder} has food packed in ice, below the ground.
If you have that house on the map already, the \glspl{ogre} head there, quickly.
If not, the \glspl{ogre} are trying to find it, but can't remember where it is.

Find the Icebox house \vpageref{iceboxHouse}.

\enchantedOgre[\npc{\T[3]\M\F\N}{\Glsfmtplural{ogre}}]

\paragraph{If the \glspl{pc} speak with the \glspl{ogre},}
they should roll \roll{Wits}{Empathy} (\tn[10]) to avoid agitating the \glspl{ogre}.
Failure means the \glspl{pc} are lunch, and a tie means the \glspl{ogre} discuss eating them for a few \glspl{round} before attacking.

\paragraph{If the \glspl{ogre} arrive at the Icebox house,}
they find it empty -- other \glspl{ogre} have already eaten everything inside.
At that point, nothing can stop them eating everyone in sight.

The \glspl{pc} may simply flee, and let the \glspl{ogre} eat the elves who live there.

\sqpart[\squash]{ravencops}% REGION
{Elf Smash}% NAME
{\Glsfmtplural{ogre} eat through an elven house}% SUMMARY

\begin{exampletext}
  Three \glspl{ogre} entered an elven home, famished and angry.
  The elves were not prepared for violence, due to all the oaths of peace.
  The \glspl{ogre} killed the lot, and searched for food and \glspl{weapon}, before taking the elves up to eat them.
\end{exampletext}

The \glspl{pc} find three \glspl{ogre} gnawing on five elf corpses.

\paragraph{If the \glspl{pc} leave quickly,}
the \glspl{ogre} leave them alone.

\ogre[\npc{\T[3]\M\F\N}{\Glsfmtplural{ogre}}]

\sqpart[\squash]{ravencops}% REGION
{The Questing Oath}% NAME
{\Glsfmtname{MindElder} requests the troupe kill rogue \glsfmtplural{ogre}}% SUMMARY

At \gls{oathtower}, \gls{MindElder} spots the \glspl{pc} from his balcony, and asks them to come up to speak.
Once Ha\^{c}ek, the boat goblin, has ferried them into the tower, an elf leads them up for a personal audience with \gls{MindElder} where he requests the \glspl{pc} hunt down and kill the \glspl{ogre}.

\Gls{MindElder} has no money, but he will bargain with anything else he might have in the tower -- \glspl{spell}, \glspl{weapon}, \glspl{ration}, or just `a very big favour'.

\sqpart[\squash]{ravencops}% REGION
{Everyone Up!}% NAME
{All the \glsfmtplural{ogre} have awakened, and formed a war band}% SUMMARY

The band of eight \glspl{ogre} have eaten through most of the giant snails in \gls{enchantedLands}, and hunt for what remains.

\ogre[\npc{\T[4]\M\F\N}{\Glsfmtplural{ogre}}]

\enchantedOgre[\npc{\T[4]\M\F\N}{\Glsfmtplural{ogre}}]

The \glspl{pc} will probably spot them in the distance.
Hiding requires a \roll{Wits}{Stealth} roll at
\setTN{Wits}{Vigilance}
\tn.

\label{lastOgreSegment}

\stopcontents[sq]

\subsubsection{\Glsfmtname{MindElder}'s Death}
would be disastrous.
Goblin oaths would start to break, little by little, while the elves in \gls{enchantedLands} would lose their enchantments much more slowly.
The goblins would awaken the \glspl{ogre} with screeches and wails held in for half of their lives.

\index{\expandafter\Glsfmtname{MindElder}'s Death}

\begin{itemize}
  \item
  Within a week, they would eat through the giant snails in \gls{ravencops} (while many continue to uphold bits and pieces of their various oaths).
  \item
  Within a week and one day, the goblin horde would eat through every elf in \gls{enchantedLands}.
  \item
  The ninth day would begin with a siege upon \gls{coppernut}.
\end{itemize}

\sidequest[ravencops,plateauGardens]{Goblins Rise}
\label{goblinsRise}

\renewcommand\enchantedRations{empty bowl}

If the giant snail population decreases, mark one of the \glspl{segment} as ready (`\sqr') after a week.
If the goblins become agitated by finding \pgls{sepulchre} destroyed, mark \pgls{segment} as ready immediately.

\setcounter{sqNo}{-1}
\sqpart[\squash]{ravencops}% AREA
{Goblins Don't Share}% NAME
{Goblins eat animal, root, and berry in \glsfmttext{ravencops}}% SUMMARY
\stepcounter{sqNo}

\Gls{foraging} in \gls{ravencops} rises to \tn[16] as the goblins pick the forest bare and hunt everything that moves.

\sqpart[\squash]{ravencops}% AREA
{Husk}% NAME
{A starving goblin demands food}% SUMMARY

Pica hasn't eaten in two days, and hasn't eaten well in four.

\begin{boxtext}
  A goblin approaches ahead.
  Its robes are slipping from the shoulders
\end{boxtext}

\begin{speechtext}
  I know you got something nice.
  I smell it.
  Give me a bite.

  Make it legal -- make it a gift!
\end{speechtext}

If the \glspl{pc} don't `gift' her some food, her strained oath snaps, and she becomes stupefied.
She attacks with the \weaponName, but the first attack automatically fails, as she receives the crushing weight of a -5~Penalty to thinking straight.

\set{r3}{3}
\enchantedGoblin[\npc{\F\N}{Pica}]%

\sqpart{ravencops}% AREA
{Rebalancing Grass}% NAME
{Growths along the snail paths shows the declining snail population}% SUMMARY

Without the snails, grass and bushes begin to grow across the snail paths.
In the first week, the new growth is subtle, but still possible to notice with a \roll{Wits}{Survival} roll (\tn[12]).

\sqpart[\squash]{shadepaths}% AREA
{Hunting by Shadow}% NAME
{Hungry goblins hunt for a snail, then eat until they become hobgoblins}% SUMMARY

This band feel starving, so they hunt through \gls{shadepaths} for a snail to feast on.
The snails here may have acidic vomit, but they don't attack people.

Once a snail is found (which won't be long) they kill it, and take two \glspl{interval} to eat through the corpse, belching and groaning loudly the entire time.

If the elves remain affected by the \glspl{disgnome}, they watch and slowly talk about potential solutions.
But any who awakened will think about safe methods of attack, from the high vantage point of \gls{plateauGardens}.

\sqpart{plateauGardens}% AREA
{Night Raids}% NAME
{Goblins come to remove the foods of \glsfmttext{plateauGardens}}% SUMMARY

The goblins arrive and try to take food.
If the elves have shaken off their sleep from the \gls{disgnome}, they object to the goblins taking food and may turn violent.
The situation is tense, so however the \glspl{pc} approach it, set the \gls{tn} to 10 or more.

\enchantedGoblin[\npc{\T[12]\M\F\N}{Grotesk, Gadzook, \& c.}]

\begin{boxtext}
  Put some clothes on, filthy elf!
  Take a hike!
\end{boxtext}

\sqpart{ravencops}% AREA
{Chewing Ice}% NAME
{Goblins raid the Icebox house}% SUMMARY

A dozen famished goblins try to find the Icebox house%
\footnote{Find the location \vpageref{iceboxHouse}.}
to take the food.
However, this breaks their oaths to uphold the law, which gives them a -4~Penalty to all Mental~\glspl{attribute}.

\begin{boxtext}
  A quiet \textit{thck-thck-thck} sound signals feed moving across the snail path.
  A dozen goblins approach, with an animalistic look to them -- vacant, unfocussed, but full of purpose.
\end{boxtext}

The goblins do not speak, except to say `Icebox elves', and `food'.

\paragraph{If the \glspl{pc} lead them to the Icebox house,}
they follow.
However, they remember enough to know if they're going the wrong way.

Of course, once they arrive they must eat.
If \glspl{ogre} ate through the Icebox food stores (\gls{segment}~\vref{ogresEatIcebox}) then the goblins become feral instantly.

\enchantedGoblin[\npc{\T[12]\M\F\N}{Ligature, Spur, \& c.}]

\paragraph{If they don't arrive within \pgls{interval},}
they attack.

\paragraph{If the goblins receive \pgls{ration} each}
they calm down, get back to \gls{oathtower}, and renew their oaths.


\sqpart{ravencops}% AREA
{The Horde Rises}% NAME
{Everyone bands together to feed}% SUMMARY

The goblins have broken their oaths, and begun to wake the \glspl{ogre} who sleep an enchanted sleep in the \glspl{sepulchre} near \gls{oathtower}.

\goblin

\sidequest[plateauGardens]{Removing the \Glsfmtplural{disgnome}}

\noindent
If the elves remain affected by all the \gls{disgnome}, they will be far less able to spot attacks, and use magic to defend themselves.
But if the \glspl{pc} remove the \glspl{disgnome},
many of the elves of \gls{plateauGardens} wake up and decide to get proactive.

Removing the \gls{disgnome} around \glspl{plateauGardens} takes time -- at least \pgls{interval} -- and a \roll{Strength}{Cultivation} roll at \tn[10].

\setcounter{sqNo}{-1}

\sqpart[\squash]{plateauGardens}% AREA
{Eyes Open}% NAME
{Orv\"e is awake and irate}% SUMMARY

Orv\"e has recently shaken off the effects of repeated \gls{disgnome} contact.
She mutters about how difficult finding water has become, since the giant snails make nearby rivers and lakes disgusting, and unusable.

Orv\"e mentions feeling better since avoiding the \gls{disgnome}, and asks the \glspl{pc} if they could remove any they see.

\elf[\NPC{\F\El}{Orv\"e}% Name
  {skeletal face, purple eyes}% Description
  {wrinkles nose}% Mannerism
  {to take a bath}]% Wants

\paragraph{If the \glspl{pc} encourage her irritation with the snails,}
she asks \gls{LifeElder} to stop making the giant snails.

\stepcounter{sqNo}

\sqpart[\squash\gls{vlg}]{plateauGardens}% AREA
{Defences Up}% NAME
{The elves fortify \glsfmtplural{plateauGardens}}% SUMMARY

Ontamon does not like the goblins stealing food from the gardens.
A few of the elves have decided to fortify \gls{plateauGardens}, so now he's double-checking the fortifications, to give them some feedback.

Along the nearest \glspl{plateauGardens}, thorny bushes grow along the edges, encouraged by \glspl{spell}.
Most just provide an irritating barrier (and +2 to any \gls{tn} to climb past them), but 1 in 6 have a venom, which inflicts $1D6-1$~\glspl{ep}.

\paragraph{If goblins raid \gls{plateauGardens},}
they die, and you can remove a goblin encounter (\vpageref{goblinsRise}).

\elf[\NPC{\M\El}{Ontamon}% Name
  {Elaborate, brown, braids}% Description
  {pleats hair}% Mannerism
  {to defend against those goblins}]% Wants
\label{ontamon}

{\small
  \showStdSpells
}

\sqpart[\squash]{plateauGardens}% AREA
{Death to Sleep}% NAME
{An elf removes the last of the \glsfmttext{disgnome}}% SUMMARY

Ontamon (\gls{statblock} \vpageref{ontamon}) has decided to remove the last of the \gls{disgnome} himself.
The \glspl{pc} find him pouring a plant-poison onto a patch.

The last of the \gls{disgnome} will soon disappear, and the elves will lose all Wits~Penalties.

\sqpart[\squash]{plateauGardens}% AREA
{Ready for War}% NAME
{The elves discuss the ramifications of snails dying}% SUMMARY

If the giant snails all die, the goblins will go hungry, and the enchanted oaths which keep them peaceful will break.
The elves don't want that, and they also don't want the elves of \gls{enchantedLands} to die.

\stopcontents[sq]

\subsubsection{Making the Elders Talk}
will take perseverance, but it can work.
If \gls{LifeElder} and \gls{MindElder} work together, they can  reduce the giant snails in the area slowly, while \gls{LifeElder} introduces a few, subtle, `goblin-traps'.

\subsubsection{Elven Death by Human Hands}
will prompt a vicious reaction from \gls{romeo} and \gls{juliet}, along with a couple of other elves from \gls{plateauGardens} if the \gls{disgnome} has been quelled.
The elves will begin planning to wipe out \gls{coppernut} in order to remove all humans from the area.
The plot will begin by casting strange spells on the road out, so they can ensure complete destruction.%
\footnote{If ten refugees return, a battalion of humans may return.
But if nobody returns from the road for a month, then people will simply decide to not travel along that road rather than investigate.}

\iftoggle{verbose}{
  \subsection{Broken Resolutions}

  If your table runs out of time for the night, and you need to wrap everything up quickly, switch to \gls{downtime} plans, and give each \gls{pc} \pgls{action}.
  If they want to negotiate with \gls{MindElder}, they might roll \roll{Intelligence}{Empathy}, or if they think it's time to kill those \glspl{ogre}, sleeping in their \glspl{sepulchre}.

  For some added drama, ask each player to roll under a cup, napkin, or sleeve.
  Leave the \gls{natural} hidden until everyone has decided what they want to do.
  One all results are in, determine the outcomes based entirely on the rolls.
  If \pgls{pc} spent the entire \gls{downtime} trying to redirect the river through \gls{shadepaths}, it could \gls{flood} the \gls{sunway} and block snails from accessing \gls{ravencops}.
  Or if they spent their time trying to kill \glspl{ogre}, a failed roll could mean that \pgls{ogre} eats the \gls{pc} alive before marauding around \gls{ravencops}.
}{

  \subsection{No Way Home}

  If you want to run this arc over multiple sessions, or if the troupe find themselves stuck here over \gls{downtime}, the elven lands have plenty of reasons for characters to leave, disappear or get distracted.

  \subsubsection{In \glsfmttext{plateauGardens}}

  Characters leave because,

  \begin{enumerate}
    \item
    they ate the wrong thing, and their limbs shrivelled.
    The elves hope to fix them `soon' (and will, once the player returns).
    \item
    they wandered off with an elf (leading to gossip and laughter), and have not returned.
    The next session resumes once all hope is lost for finding them.
    \item
    a clerical error resulted in an immediate summoning from an overseer.
    A new \gls{pc} arrived to deliver the letter.
  \end{enumerate}

  Characters arrive because,

  \begin{enumerate}
    \item
    the \gls{jotter} wants to know what the hold-up is.
    \item
    the \gls{pc} was lost, some time ago.
    Hunger and monsters killed the rest of their troupe.
  \end{enumerate}

  \subsubsection{In the pale forest}

  Characters leave because,

  \begin{enumerate}
    \item
    they whistled out of key, and the goblins told them it was time to leave.
    \item
    \pgls{ogre} woke up, her hand reached out, and she grabbed the character.
    \gls{MindElder} reacted quickly, putting everyone in enchanted sleep, but she holds the character like a child with a doll; best not to wake them.
    \item
    goblins prepared food for Winter, grabbing all that was edible, and stuffing it in ice.
    The `it' included the character, who is `edible'; but the \gls{MindElder}'s enchanted sleep spell should deep them safe.
  \end{enumerate}

  Characters arrive because,

  \begin{enumerate}
    \item
    twelve oathkeeper goblins caught them in the forest, along with a bear and \pgls{griffin}.
    The goblins take their prizes -- all tied to poles -- back to share with the rest.
    \item
    the local overseer sent the character with a letter for \gls{MindElder}.
    It contains a generic proposal for mutual aid in killing beasts.
  \end{enumerate}

}

\end{multicols}

