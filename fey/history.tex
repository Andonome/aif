\subsection{In the Kingdom of Oaths}

\begin{exampletext}
  Welcome to my Kingdom.
  There is no crime here, because everyone who wishes to stay must promise to obey the law, and I bind them to their word by enchanted oaths.
  This helps instil the lessons I wish I had when I was younger, and sets them in the right frame of mind for when they grow up and leave.
  I use similar oaths on my goblin-police, but they cannot understand the subtleties, so I ask much less of their minds, and more of their bodies.

\end{exampletext}

\subsection{In the Land of Plenty}

\begin{exampletext}
  There is no crime here, because `crime' is just a thing in your mind, man.
  Just let it go.
  You only believe in theft because of your attachment to your things, but those things are just things.
  When you see the world for what it is, all problems vanish, and you will have everything and nothing.
  You don't need stuff, like swords, or cheese, or legs.
  Legs are also just stuff, but they don't take you anywhere, you only get somewhere by deciding you are where you want to be.
\end{exampletext}

\subsection{Conclusions}

The \glspl{pc} have many ways to cause chaos, but must put the chaos aside, and focus on what will stop giant, carnivorous snails from leaving the area (or from existing).
They must also avoid ecological changes which bring \emph{more} monsters to human lands.

\section{History}

\subsection{In the Land of Plenty}

\subsection{In the Kingdom of Oaths}


The Kingdom of Oaths remained peaceful, until giant snails entered, ate through all of the gardens.

%\gls{MindElder} stopped the snails eating through his gardens by twisting their tiny minds towards eating flesh rather than plants.  This helped clear out \glspl{crawler} from the area, as the slugs would destroy their webs with acidic vomit, and sometimes destroy the \gls{crawler} at the same time.  Unfortunately, the safe forests, full of giant snails, attracted goblins.
%\gls{MindElder} stopped the goblin incursions by forcing them to swear oaths to capture or kill lawbreakers, including other goblins.  Soon, he had an army of law-abiding, healthy goblins, hunting the giant snails, and growing into hobgoblins.  He feared the day that they ran out of giant snails to eat, because goblin hunger can break any enchantment.
%\gls{MindElder} began to kill goblins, through enchanted sleep, and by rallying his children to kill them.  This did not work well, and many elves died.
%\gls{MindElder} created a new deal with the goblins, where those who ate enough to grow into hobgoblins would enter a pit, and fight each other for the prize: a giant snail.  This created a much larger problem: ogres, with ogre-sized appetites.
%Killing the ogres would remove all hope from the goblins, and cause another rebellion.  \gls{MindElder} had to give them hope, so he sent the ogres into an enchanted sleep, and allowed the goblins to place them into stone monuments with tiny crawlspace-tunnels, so the ogres can breathe, and the goblins can check on them.
%The sound of goblin chatter, elven children playing, and high-pitched birdsong would often wake the ogres from their sleep.  With the growing number of ogre-stuff monuments, \gls{MindElder} could see no option but to ban all high-pitched noises.
%    * Starlings, robins and other high-pitched birds, he enchanted to stop singing.  As a result, the region has no birds except crows, ravens, and a few high-flying predators.
%    * Goblins must speak in a deep voice whenever they approach the monuments.
%    * Jokes are banned (they make the goblins giggle).
%    * Farting is also banned (it makes the goblins giggle more than the knock-knock jokes).


