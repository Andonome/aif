\section{Introduction}

\begin{multicols}{2}

Elves move slowly through the dense forest.
They don't have the numbers or strength to build a proper road -- but they their ways.

\begin{exampletext}
  Snails leave little roads\ldots
  If I grew giant snails, they could forge a road as they move.
\end{exampletext}

Unfortunately, the giant snails could not move through the dense trees.

\begin{exampletext}
  I'll grow them differently next time, with acidic vomit to burn away the trees.
  They only eat plants, and they move slowly, so they won't hurt anyone.
\end{exampletext}

However, the snails proceeded to eat all the food in every garden, leaving the elves with nothing.

\begin{exampletext}
  I will ask the earth to rise up around the gardens, creating raised platforms.
  The giant snails can wander through the little canyons between the gardens, keeping the paths clear, while we enjoy the gardens.
  Finally, the plan is perfect.
\end{exampletext}

So the elves cultivated their raised gardens, though they had trouble getting down, and soon made a new plan\ldots


But what happens when the cat dies?
What does the dog chase, and where does the bird go?

Every \gls{spell} adds a new problem, and every problem can be fixed with more magic.
And eventually a giant snail escapes to the human lands, where it becomes the players' problem.
They journey back to the source, going back down the chain of magical fixes until they reach the source.

And after that, I have no idea what happens.
Hopefully they won't just fix one random problem, creating a domino-effect throughout the land.

The chain of \glspl{spell} follows the pattern of the children's song, \textit{There Was an Old Lady Who Swallowed a Fly}.
Or perhaps it's more like a bad programmer, who patches bugs with dirty fixes, and patches those with more fixes, until nothing can be changed without the entire program collapsing.


\spell{Hyalmahta}% Name
  {Detailed, Devious, Duplicated}% Enhancements
  {Wax}% Action
  {Earth, Water}% Spheres
  {available plant quantities}% Resist with
  {The caster thinks about past dreams of salad and \arabic{spellTargets} snails in the vicinity begin to eat and grow.
  Within a few days, they reach the size of a human, and after one \showOnset\ reach the size of a house.
  Each snail has Strength~+5, Speed~-3, \gls{dr}~5 (or 10 on the shell), along with the abilities Viscid and Acid Spray}% Description
  {}

\spell{Quamahta}% Name
  {Detailed, Duplicated}% Enhancements
  {Warp}% Action
  {Fate, Water}% Spheres
  {fullness of snail belly}% Resist with
  {The caster describes meat-based salads, and \arabic{spellTargets} snails begin to hunt for crawling things in the bushes as much as they do the bushes themselves}% Description
  {They don't eat people who are not surrounded by bushes (because they expect meat only in regular food).}


\subsection{\Glsfmtname{coppernut}}

\histEvent{30}{5}{%
  The river stretching through \glsfmttext{shadepaths} becomes so polluted with snail-muck that the residents of \glsfmttext{coppernut} contract diseases%
}

Named after the crown of copper spikes along its outer wall, and the ring of walnut trees around the farmland perimeter, this little \gls{village} holds 100 humans, out here at the \gls{edge}.
They struggle to survive, as \glspl{monster} attack while they farm, and crawl over the houses at night, trying to find an opening.
And over the last few decades, many have contracted Mindflash Syndrome, or Guardbane.%
\exRef{judgement}{Judgement}{diseases}

However, they do benefit from being close to \gls{ravencops}, as no \glspl{monster} come from that forest.
Nothing ever came from that forest, until today.
