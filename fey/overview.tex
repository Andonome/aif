\section{Introduction}

\begin{exampletext}
  There was an old lady who swallowed a fly,
  I don't know why she swallowed a fly – perhaps she'll die!
\end{exampletext}

Elves move slowly through the dense forest.
They don't have the numbers or strength to build a proper road - but they their ways.

\begin{exampletext}
  Snails leave little roads...
  If I grew giant snails, they could forge a road as they move.
\end{exampletext}

Unfortunately, the giant snails could not move through the dense trees.

\begin{exampletext}
  I'll grow them differently next time, with acidic vomit to burn away the trees.
  They only eat plants, and they move slowly, so they won't hurt anyone.
\end{exampletext}

However, the snails proceeded to eat all the food in every garden, leaving the elves with nothing.

\begin{exampletext}
  I will ask the earth to rise up around the gardens, creating raised platforms.
  The giant snails can wander through the little canyons between the gardens, keeping the paths clear, while we enjoy the gardens.
  Finally, the plan is perfect.
\end{exampletext}

So the elves cultivated their raised gardens, though they had trouble getting down, and soon made a new plan...

\begin{exampletext}
  There was an old lady who swallowed a goat; \\
  Just opened her throat and swallowed a goat! \\
  She swallowed the goat to catch the dog, \\
  She swallowed the dog to catch the cat, \\
  She swallowed the cat to catch the bird, \\
  She swallowed the bird to catch the spider \\
  That wriggled and jiggled and tickled inside her, \\
  She swallowed the spider to catch the fly; \\
  I don't know why she swallowed a fly – perhaps she'll die! \\
\end{exampletext}

But what happens when the cat dies?
What does the dog chase, and where does the bird go?

Every spell adds a new problem, and every problem can be fixed with more magic.
And eventually a giant snail escapes to the human lands, where it becomes the players' problem.
They journey back to the source, going back down the chain of magical fixes until they reach the source.

And after that, I have no idea what happens.
Hopefully they won't just fix one random problem, creating a domino-effect throughout the land.


The chain of spells follows the pattern of the children's song, There Was an Old Lady Who Swallowed a Fly.
Or perhaps it's more like a bad programmer, who patches bugs with dirty fixes, and patches those with more fixes, until nothing can be changed without the entire program collapsing.

\subsection{In the Kingdom of Oaths}

\begin{exampletext}
  Welcome to my Kingdom.
  There is no crime here, because everyone who wishes to stay must promise to obey the law, and I bind them to their word by enchanted oaths.
  This helps instil the lessons I wish I had when I was younger, and sets them in the right frame of mind for when they grow up and leave.
  I use similar oaths on my goblin-police, but they cannot understand the subtleties, so I ask much less of their minds, and more of their bodies.

\end{exampletext}

\subsection{In the Land of Plenty}

\begin{exampletext}
  There is no crime here, because 'crime' is just a thing in your mind, man.
  Just let it go.
  You only believe in theft because of your attachment to your things, but those things are just things.
  When you see the world for what it is, all problems vanish, and you will have everything and nothing.
  You don't need stuff, like swords, or cheese, or legs.
  Legs are also just stuff, but they don't take you anywhere, you only get somewhere by deciding you are where you want to be.
\end{exampletext}

\subsection{Conclusions}

The \glspl{pc} have many ways to cause chaos, but must put the chaos aside, and focus on what will stop giant, carnivorous snails from leaving the area (or from existing).
They must also avoid ecological changes which bring *more* monsters to human lands.

\section{History}

\subsection{In the Land of Plenty}


\begin{itemize}
  \item
  LifeElder loved the little paths snails make, and wanted to walk across them, but she was too big.
  To create her roads, she used Life spells to grow the snails to monstrous proportions.
  But once they were as big as a house, they just got stuck in the tall trees.
  \item
  To fix the massive snails getting stuck, she gave them acidic vomit, so they could dissolve bushes and trees, burn through \gls{crawler} webs, and in general move freely.
  Unfortunately, they ate all her vegetable patches.
  \item
  To stop the giant snails eating all the vegetables, LifeElder cracked the land, sundering the soil and creating raised plateaus, where she and the other elves could live, cultivating plants.
  Meanwhile, the snails remained in the lower regions.
  Unfortunately, the elves could not get from one plateau to the other due the tall, sheer walls.
  \item
  LifeElder did not like seeing the elves trapped on different plateaus, like some kind of jail.
  She solved the problem by enchanting bean-vines to bridge nearby spaces between the plateaus, creating actual bridges.
  \item
  The bridges have held fine, the elves grow plants and sing songs (and sometimes, *vice versa*) on their plateaus.
  And LifeElder feels content that she has solved all the problems with her spells.
\end{itemize}

\subsection{In the Kingdom of Oaths}

- With the old lich killed, MindElder decided to settle down, build the perfect tower, and raise perfect children in a perfect land.
- Unfortunately, the children stole, fought, and disobeyed his orders to stay at home and keep safe.  He made them all swear oaths to uphold the law, never harming any elf, nor taking property, nor singing out of key.
- Without the ability to sing out of key, none have learnt to sing, but this turned out to be an improvement, as MindElder always enjoyed silence more than song.

The Kingdom of Oaths remained peaceful, until giant snails entered, ate through all of the gardens.


- MindElder stopped the snails eating through his gardens by twisting their tiny minds towards eating flesh rather than plants.  This helped clear out \glspl{crawler} from the area, as the slugs would destroy their webs with acidic vomit, and sometimes destroy the \gls{crawler} at the same time.  Unfortunately, the safe forests, full of giant snails, attracted goblins.
- MindElder stopped the goblin incursions by forcing them to swear oaths to capture or kill lawbreakers, including other goblins.  Soon, he had an army of law-abiding, healthy goblins, hunting the giant snails, and growing into hobgoblins.  He feared the day that they ran out of giant snails to eat, because goblin hunger can break any enchantment.
- MindElder began to kill goblins, through enchanted sleep, and by rallying his children to kill them.  This did not work well, and many elves died.
- MindElder created a new deal with the goblins, where those who ate enough to grow into hobgoblins would enter a pit, and fight each other for the prize: a giant snail.  This created a much larger problem: ogres, with ogre-sized appetites.
- Killing the ogres would remove all hope from the goblins, and cause another rebellion.  MindElder had to give them hope, so he sent the ogres into an enchanted sleep, and allowed the goblins to place them into stone monuments with tiny crawlspace-tunnels, so the ogres can breathe, and the goblins can check on them.
- The sound of goblin chatter, elven children playing, and high-pitched birdsong would often wake the ogres from their sleep.  With the growing number of ogre-stuff monuments, MindElder could see no option but to ban all high-pitched noises.
    - Starlings, robins and other high-pitched birds, he enchanted to stop singing.  As a result, the region has no birds except crows, ravens, and a few high-flying predators.
    - Goblins must speak in a deep voice whenever they approach the monuments.
    - Jokes are banned (they make the goblins giggle).
    - Farting is also banned (it makes the goblins giggle more than the knock-knock jokes).



\spell{Hyalmahta}% Name
  {Detailed, Devious, Duplicated}% Enhancements
  {Wax}% Action
  {Earth, Water}% Spheres
  {available plant quantities}% Resist with
  {The caster thinks about past dreams of salad and \arabic{spellTargets} snails in the vicinity begin to eat and grow.
  Within a few days, they reach the size of a human, and after one \showOnset\ reach the size of a house.
  Each snail has Strength +5, Speed -3, \gls{dr}~5 (or 10 on the shell), along with the abilities \viscid\ and \acidSpray}% Description
  {}

\spell{Quamahta}% Name
  {Detailed, Duplicated}% Enhancements
  {Warp}% Action
  {Fate, Water}% Spheres
  {fullness of snail belly}% Resist with
  {The caster describes meat-based salads, and \arabic{spellTargets} snails begin to hunt for crawling things in the bushes as much as they do the bushes themselves}% Description
  {They don't eat people who are not surrounded by bushes (because they expect meat only in regular food).}



