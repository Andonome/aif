\enchantedMap[t]

\section{Introduction}

\begin{multicols}{2}

\noindent
Let me start from the beginning.
The \glspl{pc} see giant, acid-vomiting snail, headed towards \pgls{village}; they manage to fend it off, but why does it exist?
Who or what made it?
A long journey into elven lands gives them a series of clues; the snails are carnivorous because a local enchanter needs to keep his goblin population under control.
But why are they here?
\ldots because they eat giant snails.
But then why\ldots

Let me start at the end, and we'll meet in the middle.
A powerful elven caster wanted roads, so she cast spells on snails to make them big and make them vomit acid.
It was great until they ate all the vegetables, so she cracked the land in two, raising the gardens into giant pillars of earth, while the snails crept through the cracks between \gls{plateauGardens}.
Soon the snails wandered through another elven land -- \gls{enchantedLands} -- where the lord of that land loved seeing the snails come in, and the roads they made.
Unfortunately the snails attracted a lot of goblins, so he cast enchantments on them (hence the name, `\gls{enchantedLands}') to make them follow the local laws, and enforce those laws.
But even powerful enchantments won't stop goblins once they get hungry, so the enchanter warped the minds of the snails, making them carnivorous, which makes them much harder to kill, and also removed a lot of the goblins' food supply when the now-carnivorous giant snails began eating all the \glspl{monster} in the forest.

Does that make sense?
If it helps, imagine an overly-proud programmer, who never fixes anything, but just adds more layers of program on top of programs.
Now imagine two of them, both with poor communication skills.

The \glspl{pc} enter this complete mess and climb their way back up the causal-chain with each new \gls{segment} in the story.
They can mess with the ecosystem easily, but can they make sure the giant snails stop approaching the local \gls{village} (`\gls{coppernut}') without creating a horde of ravenous goblins?

\subsection{Summary of the Tapestry}

This module is made from a bunch of little \glspl{sq}, which meander through the land, and then begin to zig-zag, back-and-forth across the two elvish lands.
It ends with broad suggestions on how to handle the collapsing ecosystem, \vpageref{feyClosing}.

\begin{itemize}
  \item
  In \autoref{callToAdventure}: `\nameref{callToAdventure}', the players get \pgls{pc}, and a giant snail attacks \pgls{village} called `\gls{coppernut}'.
  \item
  In \autoref{elvenForests}: `\nameref{elvenForests}', the \glspl{pc} see both elven lands, and you (the \gls{gm}) finish the map.
  Every time they move somewhere new, you can check the next available \gls{segment} in the \gls{region}, and read out what they've discovered in \gls{ravencops}, or \gls{oathtower}, and so on.
  Each elven land has two \glspl{region}, and \pgls{region} sits between them.
  \item
  In \autoref{looseThreads}: `\nameref{looseThreads}', a boat-load of \glspl{sq} mean that every \gls{region} will be full of developments.
  And at the end of \autoref{looseThreads}, \autoref{feyClosing} suggests vectors for the \glspl{pc} to destabilize the environment, or bring order (and perhaps those are the same).
  The conclusion is very open-ended, but by the time players understand the unbalanced ecosystem, they will probably have opinions about adjusting it.
\end{itemize}

\end{multicols}

\section{Action!}

\begin{multicols}{2}

\subsection{\Glsfmtname{coppernut}}

\histEvent{30}{5}{%
  The river stretching through \glsfmttext{shadepaths} becomes so polluted with snail-muck that the residents of \glsfmttext{coppernut} contract diseases%
}

Named after the crown of copper spikes along its outer wall, and the ring of walnut trees around the farmland perimeter, this little \gls{village} holds 100 humans, out here at the \gls{edge}.
They struggle to survive, as \glspl{monster} attack while they farm, and crawl over the houses at night, trying to find an opening.
And over the last few decades, many have contracted Mindflash Syndrome, or Guardbane.%
\exRef{judgement}{Judgement}{diseases}

However, the \gls{village} benefits from being close to \gls{ravencops}.
Not a single \gls{monster} has come from that forest, until today.

