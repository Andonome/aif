\enchantedMap

\section*{The Picture on the Tapestry}

\begin{multicols}{2}

\begin{description}
  \item[The tail begins]
  when a giant snail emerges from \gls{ravencops} and tries to eat \pgls{village}'s crops.

  \textit{If nobody stops the snails, more will come, destroying the \gls{village}'s crops.}

  \item[In \autoref{elvenForests}]
  the troupe enter \gls{ravencops} to find out where that giant snail came from.
  They find a labyrinth of wide roads, all created by giant snail-trails.

  Within a day, a giant snail approaches, and attacks \pgls{crawler}, which explains why \gls{ravencops} has so few monsters.

  \textit{And if the snails die, \glspl{monster} will populate \gls{ravencops} again, attacking the nearby \gls{village}.}

  \item[Closer to \gls{oathtower}]
  a goblin approaches, demanding to see the troupe's weapons licences.
  If they speak politely, he leads them to \gls{oathtower} to obtain licences.

  \textit{The lord of \gls{oathtower} keeps the goblins under enchanted oaths, so and they enforce the local laws.}

  \item[Near \gls{oathtower}'s shining lake]
  the \glspl{pc} find \pgls{sepulchre} that snores like an avalanche.
  Their goblin-guide explains that three \glspl{ogre} rest inside, in an enchanted sleep, and they will awaken during the time of feasting.

  \textit{High-pitched noises in \gls{ravencops} will awaken a horde of hungry \glspl{ogre}.}

  \item[The boatman by \gls{oathtower}]
  can give the \glspl{pc} weapons licences, and tell about the snails.

  \textit{The enchanter at the top of \gls{oathtower} makes the snails attack \glspl{monster}, but they come from the lawless elves in the West.}

  \item[To the West]
  they find the earth juts upwards, in tall pillars.
  Climbing the roots stings with a venom, that dulls the senses.
  At the top, elves languor in misty gardens, their thoughts slowed by the \gls{disgnome}.

  \textit{Without the poisonous roots, the local elves will become less useless and dull, and may help the \glspl{pc}.}
  \item[In shadows below,]
  giant snails roam in \gls{shadepaths}.

  \textit{A nameless elf wanders below.
  She makes the snails grow.}

  \item[The crux]
  arrives when the players consider what levers they have.
  \begin{itemize}
    \item
    Removing the snails means ravenous goblins.
    \item
    Removing the goblins means more snails, and more monsters in \gls{ravencops}, which will travel to human lands.
    \item
    Removing the \glspl{ogre} requires delicacy.
  \end{itemize}
  \item[In \autoref{looseThreads}]
  \autoref{feyClosing}'s little \glspl{thread} cover the dangers of ecological collapse.

  Finally, \autoref{feyClosingThreads} has fall-back \glspl{segment} which place locations and show the daily life of each \gls{region}.

  \item[The end]
  depends on your group.
  Perhaps they kill the giant snails.
  Perhaps they balance another fix on top of the problems, and the world wobbles on.
\end{description}


\end{multicols}
