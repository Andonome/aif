\section{Shadepaths}

\printSideQuestsInRegion{Shadepaths}

\histEvent{100}{5}{%
  To stop the giant snails eating all the vegetables, \gls{LifeElder} cracked the land, sundering the soil and creating raised plateaus, where she and the other elves could live, cultivating plants.
  Meanwhile, the snails remained in the lower regions.
  Unfortunately, the elves could not get from one plateau to the other due the tall, sheer walls%
}

\sidequest[plateauGardens,Shadepaths]{\Glsfmttext{plateauGardens}}

\printSideQuestsInRegion{plateauGardens}


% Note roads.
\histEvent{100}{5}{%
  To fix the massive snails getting stuck, she gave them acidic vomit, so they could dissolve bushes and trees, burn through \glsentrytext{crawler} webs, and in general move freely.
  Unfortunately, they ate all her vegetable patches%
}

% Bean Vine Bridges
\histEvent{95}{5}{%
  \Glsfmttext{LifeElder} did not like seeing the elves trapped on different plateaus, like some kind of jail.
  She solved the problem by enchanting bean-vines to bridge nearby spaces between the plateaus, creating actual bridges%
}

\begin{multicols}{2}

Each \gls{segment} in this \gls{sq} shows an area within \gls{plateauGardens}, and the \gls{sq} `continues' as long as the troupe remain in the \gls{area}.
You should note each piece on the map, so that when the troupe return, they find the same place again.

\sqpart{plateauGardens}% AREA
{\Glsfmttext{disgnome} Thickets}% NAME
{The mind-rending yellow thickets hides within the tomatoes}% SUMMARY

A tiny prick from the needle-tips can slow someone's mental capacity, making them feel like everyone around them is babbling fast-paced nonsense, and makes the day seem to last an hour.
A number of the elves have been affected, but they blame their headaches and confusion on enchantments or don't even consider the cause.

The yellow, spiky plants hide among a bed of tomato plants, ready to prick anyone who wanders by.

\paragraph{If the troupe destroy the \glspl{disgnome},}
\gls{SnailTamer} soon returns to his natural human-paced thinking (about the same as the average human).

\sqpart{Shadepaths}% AREA
{The Watering Hole}% NAME
{Clear pool now an undrinkable snail-bath}% SUMMARY

Little rivers gather into a little pool.
The shallow basin stretches only twenty steps wide, just enough for a few snails at a time.
They slither around and drink, replenishing their slime.

This leaves the pool filthy, and the elves don't like the grime.
Carrots, on the other hand, love the water's brine.

The troupe see a thin trickle of water running through their mossy path.
Following it leads to the watering hole.

\paragraph{Each time the troupe arrive at the watering hole,}
they find 1D6-2 giant snails bathing.


\sqpart{plateauGardens}% AREA
{Enlightenment}% NAME
{The flowers of the garden of light make you float}% SUMMARY

\histEvent{50}{1}{%
  \Glsfmttext{juliet} became bored of trying to manipulate bodies, and focussed herself on the Force \glsfmttext{sphere}%
}

\Gls{juliet} has cultivated these flowers using the Force Sphere.
The flowers of enlightenment make you light, and let you float; but first they must wilt and go brown.
The decaying process requires two \glspl{interval} of Sunlight, after which someone can eat the plants.
Doing so reduces their \gls{weight} by~5.

\begin{boxtext}
  A bed of bright-red flowers, with long petals like the floppy ears on a dog.
  It looks like someone made space for them, and spent a lot of time on their soil bed.
\end{boxtext}

\sqpart{plateauGardens}% AREA
{Purple, Yellow Beds}% NAME
{More mind-rending plants hide by a carrot-patch}% SUMMARY

Long, green plants spring up, indicating massive carrots below.
The weak elves find pulling them up to be very difficult, and only do so in groups, or by speaking sweetly to the earth, and asking it to let the carrot go.

\Glspl{disgnome} hide around the side of the carrot beds.
Anyone passing through must roll to notice them, or suffer the usual consequences.

\sqpart{Shadepaths}% AREA
{Guardian Stones}% NAME
{Last hope of the elves: hidden lake uncovered behind seeping-wet wall}% SUMMARY

\histEvent{40}{5}{%
  \Glsfmttext{LifeElder} walled off the last clean lake in \glsfmttext{plateauGardens} to stop the giant snails infecting it%
}

With little clean water left, \gls{LifeElder} guarded the last pool of water by summoning stony walls around it.
Water escapes through little holes at the base, which will give the characters a clue about this hidden lake.

\begin{boxtext}
  A shining, tiny, rivulet meanders through the barren, dry canal.
  It smells and tastes fresh!
\end{boxtext}

Nobody can see the lake from the outside.
Trees in the plateau gardens merge seamlessly with trees around the lake.
From a distance, it all looks like a continuous canopy.

If the elves cannot use this lake -- due to snail-access, or poisoning, or some other catastrophe -- they will find themselves without a good source of water, and \gls{LifeElder} will have to stop supporting the snails.

\sqpart{plateauGardens}% AREA
{The Great Snail Lake}% NAME
{Lake spotted from a garden plateau}% SUMMARY

The canyon widens here, and a barren, slimy land (stripped bare by giant snails) holds a great lake in the centre.
It stretches as far as an arrow's flight, and glistens with a thick film of slime across most of the surface.

Garden plateaus surround the lake, and each one holds a narrow staircase down.
The crack in the plateaus where the stairs descend is very narrow.
Characters with Strength~+1 can only enter the staircase by removing all armour and squeezing through.
Anyone with a higher Strength Bonus cannot enter.

Each time the troupe arrive at the lake,
they see 2D6-2 giant snails bathing, and 1D6-3 elves collecting water.

The elves purify the water with spells when they can, but this requires \glspl{mp}, which are in scarce supply in the area.

\end{multicols}

