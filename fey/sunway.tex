\section{\Glsfmttext{sunway}}
\label{sunway}

At the border between the two lands the trees grow thin.
Bushes dominate, grasses grow, and small circles of Sunlight form where the canopy breaks.
\Gls{sunway} cuts a line across many miles, providing easy passage for aurochs, who help widen the passage every time they pass, by eating young trees and stamping down everything they can.

Here at last, \pgls{witch} can breathe easily.
\Glspl{mp} regenerate at the normal rate.

\printSideQuestsInRegion{sunway}

\begin{multicols}{2}

\subsection{\Glsfmttext{plateauGardens} Walls}

\begin{boxtext}
  \Gls{sunway} comes to a harsh stop at a wall of perfectly vertical, and strangely solid, earth.
  It stands half as tall as \pgls{broch}, and tree-branches sway even higher up, their roots poking a crown around the edges.

  The great wall continues down this Sunlit part of the sparse forest farther than you can see, in both directions.
  But it also has crevices, or cracks, or some darkness dotted along it.
  Through one crack, the silhouette of a half-dead tree droops down the wall half way.
\end{boxtext}

The raised plateaus of the \gls{plateauGardens} form this wall, and the cracks which separate the gardens are \gls{shadepaths}.
Dangling foliage lets bear-light elves clamour up, but heavier people risk breaking the delicate foliage.

\paragraph{When someone climbs,}
they roll $1D6 + \gls{weight}$; a roll of 12 or more breaks the branch (or root, or vine) and they tumble down, and hit the ground four \glspl{step} below, while earth or plants dislodge and falls on them after the fall.

The total Damage is 2 plus the character's Strength%
\exRef{core}{Core}{falling}
and everyone standing close receives the die just rolled as Damage too.

\sidequest[sunway]{Places to See in \glsfmttext{sunway}}

\sqpart[\gls{vlg}]{sunway}% AREA
{Elven Steps}% NAME
{A hidden path leads to \glsfmttext{plateauGardens} above}% SUMMARY
\label{hiddenStairs}

In the causeway between the \gls{ravencops} forest and the \gls{plateauGardens}, a single plateau has a hidden stairway, going up.
\Glspl{crawler} cannot make much use of the narrow stairs, with occasional hand-holds for little fingers.
People who don't know about the stairs cannot usually see them, as every step blends into the tall rock-face from below.
But once someone notices the first step, they see the next, and then the next, and so on.

\paragraph{Spotting the rocks}
requires a \roll{Wits}{Vigilance} roll at \tn[12].
The \gls{tn} increases by~+1 in the rain, and by~+3 at night.

\end{multicols}
