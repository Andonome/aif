\section{\Glsfmttext{sunway}}
\label{sunway}

At the border between the two lands the trees grow thin.
Bushes dominate, grasses grow, and small circles of Sunlight form where the canopy breaks.
\Gls{sunway} cuts a line across many miles, providing easy passage for aurochs, who help widen the passage every time they pass, by eating young trees and stamping down everything they can.

Here at last, \pgls{witch} can breathe easily.
\Glspl{mp} regenerate at the normal rate.

\printThreadsInRegion{sunway}

\begin{multicols}{2}
\widePic{shadows/sunway}
\subsection{\Glsfmttext{plateauGardens} Walls}

The raised plateaus of \gls{plateauGardens} form this wall, and the cracks which separate the gardens are \gls{shadepaths}.
Dangling foliage lets bear-light elves clamour up, but heavier people risk breaking the delicate foliage.

\begin{boxtext}
  \Gls{sunway} comes to a harsh stop at a wall of perfectly vertical, and strangely solid, earth.
  It stands half as tall as \pgls{broch}, and tree-branches sway even higher up, their roots poking a crown around the edges.

  The great wall continues down this Sunlit part of the sparse forest farther than you can see, in both directions.
  But it also has crevices, or cracks, or some darkness dotted along it.
  Through one crack, the silhouette of a half-dead tree droops down the wall half way.
\end{boxtext}

\paragraph{When someone climbs,}
they roll $1D6 + \glsentrytext{weight}$; a roll of 12 or more breaks the branch (or root, or vine) and they tumble down, and hit the ground four \glspl{step} below, while earth or plants dislodge and falls on them after the fall.

The total Damage is 2 plus the character's Strength%
\exRef{core}{Core}{falling}
and everyone standing close receives the die just rolled as Damage too.

\thread[oathtower,sunway,shadepaths,plateauGardens,ravencops]{The \Glsfmttext{ranger}}

\noindent
Soon after the troupe left their \gls{broch} to find the source of the giant snail, \gls{susjot} sent \gls{dickhead} out on the same mission, alone, because nobody at the \gls{broch} likes \gls{dickhead}.

\segment[\gls{vlg}]{sunway}% AREA
{An Old Acquaintance}% NAME
{\Glsfmttext{dickhead} arrives to scope out the situation}% SUMMARY

The troupe find him in the woods, searching for \pgls{griffin} nest he thinks lies nearby.
He moves towards them quietly, hoping to get the jump on them, just to show off his superior stealth \glspl{skill}.

Once out, \gls{dickhead} speaks haughtily of his ability to survive in the forest, and moves with confidence.
He asks the troupe what they've seen, but does not give their stories much importance.

\begin{speechtext}
  So you still have not found the heart of the problem.
  Well keep searching!
  You may not succeed, but it makes for good practice.
\end{speechtext}

\dickhead

\paragraph{Spotting \gls{dickhead}}
as he sneaks up needs a \roll{Wits}{Vigilance} roll
\set{track}{7}%
\addtocounter{track}{\value{Dexterity}}%
\addtocounter{track}{\value{Stealth}}%
at \gls{tn}~\arabic{track}.

\Gls{dickhead} soon leaves, telling everyone not to follow him, as they'll just make noise.

\label{sunwayAirIngredients}
\paragraph{Searching for the \gls{griffin} nest}
requires a \roll{Wits}{Survival} roll at \tn[12].
\ifcase\value{temperature}%
  It's empty, since \glspl{griffin} don't build nests in snow, but \pgls{griffin} remains.

  \griffin[\npc{\A}{Hungry \Glsfmttext{griffin}}]
\or%
  It lies half a mile South, with a single \gls{griffin}.

  \griffin[\npc{\A}{Lonely \Glsfmttext{griffin}}]
\or
  It lies half a mile North, with two \glspl{griffin}.

  \griffin[\npc{\T[2]\A}{\Glsfmtplural{griffin}}]
\else
  It lies half a mile North, with two \glspl{griffin} and three babies.

  \griffin[\npc{\T[2]\A}{\Glsfmtplural{griffin}}]
\fi

\segment{shadepaths}% AREA
{Peeping Woodsman}% NAME
{\Glsfmttext{dickhead} explains his plan to kill \glsfmttext{LifeElder}}% SUMMARY

\Gls{dickhead} has observed the area for some time, noticed the \gls{disgnome} plants, and believes that the giant snails all stem from a single source: a powerful spellcaster.

\begin{speechtext}
  The plan is simple, I find the elf who makes the snails, and \emph{kill him}.
  So I'll set \pgls{ambush} then loose an arrow on whatever gardener grows these giant snails.

  Elves are small.
  I'll just need one arrow.
\end{speechtext}

\Gls{dickhead} will leave the \glspl{pc}, as he does not trust them to stay silent while he plans \pgls{ambush} for \gls{LifeElder}.

\segment[\squash]{plateauGardens}% AREA
{Loose Clothing}% NAME
{\Glsfmttext{dickhead}'s crossbow lies abandoned on the ground}% SUMMARY

\begin{exampletext}
  You don't get to be centuries old without learning how to spot \pgls{ambush}.
  As \gls{LifeElder} performed one of her standard spells to query the living things in the area, she found \gls{dickhead}, and guessed the reason for his hiding.
  Her spell has split his limbs into myriad tentacles, leaving his equipment on the ground.
  He slithered away as the spell took hold, confused and dismayed, dropping pieces of his equipment along the way.
\end{exampletext}

The \glspl{pc} find \gls{dickhead}'s possessions, including his \gls{crossbow} and twelve quarrels on the ground, but carrying it sends a clear signal to the elves that they approve of his methods, and makes them dangerous.
They will suffer a -3~Penalty to social rolls with the elves while the \gls{crossbow} is visible.

\paragraph{If the \glspl{pc} follow the trail,}
have them roll \roll{Wits}{Survival} (\tn[10]).
If they succeed, they can follow him to \gls{ravencops}, and you can jump to the next \gls{segment}, below immediately (skipping any that might have been before it).
If they fail, they lose \pgls{interval} searching, and find nothing (discard the next \gls{segment}).

A tie means they lose \pgls{interval} from constantly missing tracks, but confirm that they go to \gls{ravencops}.

\segment[\squash]{ravencops}% AREA
{Wandering Hood}% NAME
{\Glsfmttext{dickhead}'s clothes lie discarded on the ground}% SUMMARY

The troupe see the last of \glsfmtname{dickhead}'s clothing, discarded just before entering the forest.
Following him further will not be easy; the \gls{tn} rises to~14.

\segment[\squash\R]{oathtower}% AREA
{Retirement}% NAME
{\Glsfmttext{dickhead} now works for \glsfmttext{MindElder} as a mutated servant}% SUMMARY

The next time the troupe enter \gls{oathtower}, they find \gls{dickhead} in his new form -- a twisted creature, with limbs replaced by tentacles, and his neck so shrunk that his shoulder-blades wrap around his ears.

After \gls{LifeElder} twisted his body, \gls{MindElder} twisted his mind.
He now accompanies \gls{MindElder} everywhere, passing him pens, and washing his clothes in the lake outside.
When \gls{dickhead} has nothing to carry, he ascends \gls{oathtower} by grabbing a window from outside, and pulling himself up the wall.
Each time he passes a window, he takes a good look inside to check that nobody inside is breaking any laws, and that everything seems as it should.


\dickheadReborn


\end{multicols}
