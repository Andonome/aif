\playCommentary{
  \begin{description}
    \item[\gls{gm}:]
    The path comes to a cross-roads.
    Do you continue forwards, or go left, or right?
    \item[Player 1:]
    (Sinkmaul)
    Left?
    \item[Player 2:]
    (Mildrain)
    Left\ldots
    \item[\gls{gm}:]
    Now the path has a turn-off to the right\ldots
  \end{description}
}{
  This is awful, and the \gls{gm} should be fed to goblins, feet-first.

  If the players had previously said they want to approach `the tower', then the next part of that process is either arriving at the tower, or failing.
  The \gls{gm} can roll out a long description, to make the length of the march clear, and they can run various distracting \glspl{segment} on the way, but the moment the \glspl{pc} are free to act, the journey to their goal should resume.
}


\chapter{\Glsfmttext{sunderedForest} \& \glsfmttext{enchantedLands}}
\epigraph{
  There was an old lady who swallowed a goat;

  Just opened her throat and swallowed a goat!

  She swallowed the goat to catch the dog,

  She swallowed the dog to catch the cat,

  She swallowed the cat to catch the bird,

  She swallowed the bird to catch the spider

  That wriggled and jiggled and tickled inside her,

  She swallowed the spider to catch the fly;

  I don't know why she swallowed a fly -- perhaps she'll die!
}

\label{elvenForests}

\noindent
The troupe traverse five \glspl{region} in the fey lands.
Every \gls{interval} another \gls{segment} reveals more about the lands, and the people within.
And every \gls{region} has some new places to add to the map \vpageref{feylands}.


