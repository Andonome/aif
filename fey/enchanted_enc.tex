\sidequest[ravencops,oathtower]{An Ordinary Week in the Enchanted Lands}

\sqpart{ravencops}% AREA
{The Square of Life}% NAME
{Giant snail devours \gls{crawler} in acid attack}% SUMMARY
\label{ravencopsIntro}

\begin{exampletext}
  \Gls{MindElder} has taken to twisting the `minds' of snails to make them crave flesh.
  Most snails do not hunt well, but the acidic spray, and disarming appearance, means they regularly consume \glspl{crawler}.
\end{exampletext}

\begin{boxtext}
  The echo of a distant crow's cry reaches you, just before fading to nothing.
  The trees look a kind of uniform-brown, without any mottling or variation.
  And the road feels as smooth as pond-scum.
\end{boxtext}

The troupe should make \pgls{bandAct} action to not fall over while walking on freshly-laid snail tracks.
Failing a \roll{Dexterity}{Athletics} roll (\tn[7]) inflicts 1~\gls{ep}.

\begin{boxtext}
  You find a cross-roads, as the path splits left and right.
  The road to the left looks older, and less slimy, but has \pgls{crawler} running towards you.
  The road to the right also looks old, until the giant snail, approaching silently.
\end{boxtext}

If the \glspl{pc} are near the giant snail when it sprays acid, they can roll \roll{Wits}{Survival} (\tn[7]) to leap into the woods as the snail prepares to spray.

\begin{boxtext}
  As the snail-spittle hits the trees and bushes in a messy gush, they let of a tiny hiss and begin to wilt.
  Leaves wither and bark turns black.
  Behind, the \gls{crawler} turns to flee into the woods with two left-legs melded together.
\end{boxtext}

\giantSnail

\chitincrawler

If the troupe let the situation unfold, the giant snail ignores them, and chases the wounded \gls{crawler} up a tree.

\sqpart[\gls{afternoon}]{ravencops}% AREA
{Got a Permit, Mate?}% NAME
{An oathkeeper goblin wants to check troupe's weapon licences}% SUMMARY

\histEvent{55}{3}{%
  Goblins approached \glsfmttext{enchantedLands}, looking for food.
  However, \gls{MindElder} forced them to swear oaths to uphold the myriad laws of the land.
  They remains, and multiplied, and soon the land held an army of goblins%
}

\begin{exampletext}
  The lack of \glspl{crawler} and plentiful giant snails soon brought a lot of goblins to the area.
  \Gls{MindElder} also turned this problem to his advantage by making the goblins swear to capture or kill anyone disturbing the peace.

  The goblins responded with sarcasm, but the spell worked anyway, soon all the goblins lay dead or agreed to take oaths of good behaviour, which work fine as long as the goblins eat regularly.
\end{exampletext}

\begin{boxtext}
  The sky rumbles with thunder, but the air feels thin.
  In the distance, a small person in a long, green cloak walks towards you, carrying a potato so large that it cannot see you.
  A long nose points up, just above the top of the potato, and pasty-white ears flop at shoulder-length.
\end{boxtext}

If the goblin (Abjad) sees the troop, it switches the potato to a one-arm hold and points accusingly at the troupe, then speaks in an unusually deep voice.

\begin{speechtext}
  I hope you got a licence for those weapons!
  Show me!
  Show me the licence, earless scum!
\end{speechtext}

Abjad (the goblin) observes and insists on the following local laws:

\begin{itemize}
  \item
  No high-pitched noises.
  \item
  No jokes, nor words which move to laughter.
  \item
  No unlicensed weapons.
  \item
  No wandering without clothes on.
  \item
  No singing out-of-lock.
\end{itemize}

\enchantedGoblin[\npc{\M\N}{Abjad}]

\paragraph{Dealing with Abjad}
requires a \roll{Charisma}{Empathy} roll at \tn[9].
Failure means he will insist on them going to \gls{oathtower} to admit their criminal behaviour, while giving him all of their \glspl{weapon}.

If they calm him successfully, then he won't insist on accompanying them to the tower, but they \emph{will} have to promise to go there.

\paragraph{At the slightest hint of aggression,}
Abjad flees and tries to find reinforcements.
The goblins will try to track them down, using their \roll{Intelligence}{Athletics}, so the \glspl{pc} will roll at \tn, however they handle it (perhaps \roll{Speed}{Athletics} to simply run from \gls{ravencops}, or \roll{Intelligence}{Stealth} to hide where goblins won't find them).

\paragraph{At the end of the \gls{interval},}
the troupe receive an additional \gls{mp}, as the thunder above releases more mana.

\sqpart[\gls{afternoon}]{ravencops}% AREA
{Silencing the Starlings}% NAME
{A goblin takes aim at a starling for the crime of high-pitched song}% SUMMARY

\histEvent{39}{3}{%
  Soon \glsfmttext{enchantedLands} filled with sepulchres, and every high-pitched noise woke the sleeping \glsfmtplural{ogre} inside.
  \Glsfmttext{MindElder} banned all high-pitched noises, including most birds.
  Of course, the ravens and crows remain, which started the name `\glsfmttext{ravencops}'%
}

\begin{exampletext}
  When \gls{MindElder} banned high-pitched voices, this included most birds, because bird-song can wake the sleeping \glspl{ogre}.
  The ecosystem has become strange since then, as it has very few birds, except ravens, crows, and magpies.
  This is why people call the local forest `\gls{ravencops}'.
\end{exampletext}

Mora has heard a starling sing, and thought to herself `that's illegal!', then readied her \gls{projectile}.
After a `\emph{thunk}!', the bird lies dead, so she finishes the job with a rock, and sucks out the starling's brains.

Mora will happily speak with the troupe, in a low-pitched voice (to avoid waking any \glspl{ogre}), but if she sees them doing something criminal (like carrying weapons without a licence) she flees to sound the alarm (but quietly).

\enchantedGoblin[\npc{\F\N}{Mora}]

\sqpart[\gls{morning}]{oathtower}% AREA
{Meat Salad}% NAME
{As a snail approaches the tower, \gls{MindElder} turns its mind towards thoughts of meaty salad}% SUMMARY

\histEvent{95}{3}{%
  When giant snails barged into \glsfmttext{enchantedLands}, \glsfmttext{MindElder} decided to kill two birds with one stone, and twisted their little minds to crave flesh.
  The giant snails stalk the woods, looking for some meat to eat with their leaves, and occasionally find \glsfmtplural{crawler}%
}

A giant snail approaches \gls{oathtower}, along a snail road, on a clear day.
The \glspl{pc} probably won't see the snail, unless they're on the road out, but they will certainly see \gls{MindElder} observing the land from the balcony at the top of \gls{oathtower}, and singing a spell to twist the mind of the snail.

\paragraph{Any characters who understand Elvish}
will understand the song relates to a meat-based salad, and that the snail should add meat to its salad.

\spell{Quamahta}% Name
  {Detailed, Distant}% Enhancements
  {Warp}% Action
  {Fate, Water}% Spheres
  {fullness of snail belly}% Resist with
  {The caster describes meat-based salads, and a target snail at \spellRange\ begins to hunt for crawling things in the bushes as much as they do the bushes themselves}% Description
  {They don't eat people who are not surrounded by bushes (because they expect meat only in regular food).}


\sqpart{ravencops}% AREA
{Shoeing a Snail}% NAME
{The goblins hound a snail into a dead-end to kill it}% SUMMARY

Mora shoes a snail into a dead-end road.
She organizes a dozen goblins to spread out, and occasionally stab it (while keeping their distance), in order to hound it to the clearing where six more goblins wait to plant spears on the road.

The scene proceeds just as the game-mechanics suggest.
Goblins throwing javelins deal around $1D6-1$ to $1D6+1$ Damage, and the giant snails have a lot of \gls{dr}, even in their most vulnerable location.
Some javelins hit the shell and shatter, others stick into the snail's `skin' harmlessly, and a few dig into its body enough to inflict a minor wound (perhaps 1 or 2 Damage).

Bringing down the snail will take the entire \gls{interval}, as the goblins run away, some double back to collect javelins which fell on the ground, and others run ahead to climb trees and throw javelins from above.
The goblins have to constantly stop the snail entering the forest.

\paragraph{Once the snail dies,}
the goblins take it apart in three stages.

\begin{enumerate}
  \item
  The barbecue, as every goblin feels famished, and must eat immediately.
  \item
  The dissection, where bloated goblins with pot-bellies cut and cure the snail-meat, then make ropes from its innards.
  \item
  The great smashing, where they use rocks to crack the shell into smaller dishes, then use the dishes to transport remaining meat back to the Icebox House.
\end{enumerate}

Each stage takes another \gls{interval}, so the troupe may see the goblins again if they pass through the area the next day.

\sqpart[\gls{afternoon}]{ravencops}% AREA
{On the Menu}% NAME
{Another giant snail approaches, but ignores anyone on the road}% SUMMARY

Another great snail approaches, with its mind focussed on a meaty-salad.
It ignores anyone on the road.
It only attacks living things in the bushes where it feeds.

The snail stops here and there, inspecting the bushes, and sometimes vomiting on them as part of its external digestion, but never stays for long.

A biting wind blows, bringing one extra \gls{mp} at the end of the \gls{interval}.

\sqpart{ravencops}% AREA
{The Unmerry Band}% NAME
{A dozen goblins walk, every statement brings suspicion of humour}% SUMMARY

Grawl, Majiscule, and Brev patrol the land, looking for trouble-makers, mapping new paths the snails have brought, and noting local monsters.
And as they walk they bicker; Grawl accuses Majiscule of jokes (which \gls{MindElder} banned, so that the high-pitched goblin laughter would not wake any \glspl{ogre}).

\begin{speechtext}

  If we find a snail, we'll need to go back to get the others, and shoe it towards a dead-end.

  What do you mean, `shoe a snail'?

  Is that a joke?
  Are you trying to be funny?

  You said it!
  You said we might shoe a snail!
  You were making the joke!

  Your face is a joke.
  It is a crime, your face.
  You have a funny-looking, criminal face.

\end{speechtext}

Majiscule hides his face until nightfall.
