\thread[shadepaths,plateauGardens]{Events in \glsfmttext{sunderedForest}}

\segment{shadepaths}% AREA
{Light Ascent}% NAME
{Elves fetch water below, and climb up delicate roots}% SUMMARY
\label{shadeAscent}

\Glspl{plateauGardens} can become very dry, due to the lack of rivers.
Elves often have to descend with snail-shell buckets, fetch filthy water from the rivulets in \gls{shadepaths}, then purify it with spells (or just use it as fertilizer for the gardens).

\begin{boxtext}
  Round the next squelching corner, a skinny silhouette collects the mucky water in a bucket.
  A second figure is climbing down roots on the tall walls, completely naked, except for the bucket on a rope.
\end{boxtext}

\paragraph{When the \glspl{pc} approach}
the elves flee up to their plateau, unless the troupe present themselves as \emph{very} non-threatening (\roll{Charisma}{Empathy} at \tn[10]).
Success means that Fanwa and Henton will chat with them, and invite them to discard their possession and climb up the roots.

\stupifiedElf[\npc{\T[2]\F\M\El}{Fanwa \& Henton}%
  \set{spentMP}{6}%
  \renewcommand\Equipment{XYZ}
]

\showStdSpells

\paragraph{Climbing the walls}
means hanging onto the \gls{disgnome} roots, and those spiky roots will inflict a mild poison, removing $1D3$ Wits.

The roots only hold up to \pgls{weight} of~9, so most characters will have to discard their equipment in order to ascend.
Anyone with a Strength~Bonus of +4 would have 10~\glspl{hp}, so their \gls{weight} would be~10, meaning they cannot climb.

\segment{plateauGardens}% AREA
{Mist Below}% NAME
{Snail eyestalks poke above the misty sea in \glsfmttext{shadepaths} below}% SUMMARY

Mist always falls into \gls{shadepaths} by \gls{plateauGardens}.
The disorienting environment is quite lethal to newcomers, as people who don't know where the land lies can easily make a misstep, tumbling down the sides.

\begin{boxtext}
  Mist has risen, but is falling into \gls{shadepaths} beside \gls{plateauGardens}.
  Soon \gls{shadepaths} look like fluffy-white rivers.
  A figure in the distance seems to hover between two plateau gardens, but the walking reveals they must be on a vine-bridge which has fallen just below.
\end{boxtext}

Fast movements require a \roll{Wits}{Survival} roll (\tn[8]) to avoid falling into the misty canals below.
Longer journeys require an \roll{Intelligence}{Survival} roll (\tn[10]).

The mists fade after \pgls{interval}.

\segment[\squash]{shadepaths}% AREA
{Another Treacherous Ascent}% NAME
{More roots on the wall allow light climbers a way up}% SUMMARY

Further \gls{disgnome} roots allow anyone here to climb up to \pgls{plateauGardens} above, just like the previous \gls{segment} (\vpageref{shadeAscent}).
These roots can hold up to \pgls{weight} of~12.

\thread[shadepaths,sunway,plateauGardens]{Grumbling Gardens}

\segment{shadepaths}% AREA
{Excuse Me!}% NAME
{A giant snail blocks the path}% SUMMARY

`Brownie' the snail grew much bigger than the others, and pregnancy did not help.
Now when she moves through \glspl{shadepaths}, she blocks them entirely.
Anyone walking inside \glspl{shadepaths} must simply turn back and find another route.
This may delay the troupe in whatever they wanted to do by \pgls{interval}.

If the troupe try to squeeze past her, have them roll \roll{Dexterity}{Stealth} (\gls{tn}~8 plus the character's Strength Bonus).
Failure inflicts $2D6$ Damage as her massive body smooshes them against \gls{shadepaths} walls.

\segment{sunway}% AREA
{Stampede}% NAME
{An auroch stampede passes through}% SUMMARY

Elves watch from a garden plateau as aurochs stampede through the causeway.
Behind them, \pgls{crawler} chases, but it's losing steam, and soon retreats into the forest.

\segment{plateauGardens}% AREA
{Uninvited Guest}% NAME
{A rare \glsfmttext{woodspy} stalks the garden}% SUMMARY

Most large animals have trouble reaching up the tall walls to \gls{plateauGardens}, but \glspl{woodspy} are clever, and sometimes manage to pull themselves up using a snail, or finding a piece of vine hanging down.
This \gls{woodspy} has reached the \gls{plateauGardens}, and stays close to a vine-bridge, to wait for someone to pass.
Once it grabs someone, it pulls them down onto \gls{shadepaths}.

\paragraph{If any elves are present,}
and still suffer from \gls{disgnome} then
they fail to spot the \gls{woodspy}, and it grabs one.

If the \gls{woodspy} attacks \pgls{npc}, you can resolve the fight by just comparing each \gls{npc}'s \gls{cr} (highest wins), instead of rolling dice alone.

If the \glspl{pc} have uprooted some of the \glspl{disgnome}, any elves spot the \gls{woodspy}.

\segment[\gls{afternoon}]{sunway}% AREA
{Slow Wander Home}% NAME
{The aurochs are returning\ldots slowly}% SUMMARY

Aurochs move slowly through a mild breeze, grazing at \gls{sunway} grass as they go.
Any sudden movement makes the aurochs flee, creating a stampede.

\paragraph{If the troupe wants to pass quietly,}
have them roll \roll{Dexterity}{Empathy} at \tn[8].

The troupe may prefer to wait, but this will take the rest of the \gls{interval}, as the aurochs move slowly, while grazing.

\segment[\gls{evening}]{sunway}% AREA
{Full Moon \& Sacks}% NAME
{Goblins approach the \glsfmttext{plateauGardens}, sniffing loudly for food}% SUMMARY

A guilt of goblins approach \gls{plateauGardens}, two sacks each, looking to steal vegetables.
They spent their time searching the walls before finding roots to climb, slowly up.
Once up they split and search, sniffing so loud it almost sounds like a whistle.
And soon they regroup, as Glottal found some carrots.

\paragraph{If the \glspl{pc} stop them,}
they object.

\begin{speechtext}
  Go fuck a badger, we came for grub!
  It's not `theirs', okay?
  They said the trees belong to the ground, so now they're ours.
  They said all good tea is theft, okay!?
  So we have not broken our oaths.
\end{speechtext}

\enchantedGoblin[\NPC{\F\N}{Glottal}%
  {Long nose, stubby fingers}% DESCRIPTION
  {Fingers her sack}% MANNERISM
  {to find the longest carrot}% WANTS
  \npcQuote{Off and get your own nobody's-carrots}]

\enchantedGoblin[\NPC{\M\N}{Grawl}%
  {Emaciated and agitated}% DESCRIPTION
  {Claws cheeks down till the eye-veins show}% MANNERISM
  {eat or fight, now, now, NOW}% WANTS
  \npcQuote{They said `proper tea', not `good tea', idiot}]

\paragraph{If the \glspl{pc} ask the elves,}
the elves say they don't feel happy with goblins taking their food, but refuse to acknowledge the concept of property.

\segment[\gls{morning}]{shadepaths}% AREA
{The Picnic Choir}% NAME
{Elven songs descend from the plateau above}% SUMMARY

A little group of elves sing a love-song to the clear-blue skies, but don't respond to shouts or calls.
They will, however, respond to someone singing with them.

If someone sings well, they will lower the vines of a broken bridge to help them up.
If they sing poorly, the elves leave silently.

Singing \glspl{pc} should roll \roll{Intelligence}{Performance} (\tn[10]) to understand the elven song, and imitate it.

