(Canals) The Picnic Choir
-----
{Elven songs descend from the plateau above}

A little group of elves sing, but don't respond to shouts or calls.
They will, however, respond to someone singing with them.

If someone sings well, they will lower the vines of a broken bridge to help them up.
If they sing poorly, the elves leave silently.

Singing \glspl{pc} should roll \roll{Intelligence}{Performance} (\tn[10]) to understand the elven song, and imitate it.

(Garden) Interview with the Tao Mistress
-----
{LifeElder wanders the canals, all answers seem strange}

LifeElder wanders, and sometimes sings.
She stops to ponder a flower, then alters a seed so the flower will grow purple.

While the \glspl{pc} are in the garden plateau, they see a small, red-haired elf below, wandering naked and humming to herself.
Other elves may identify her as the source of all the change in the landscape, but will not give her a name (except to say 'hi').

She speaks quickly, and cryptically, and gives deep thought to every word someone says, but quickly tires of conversation.


\begin{speechtext}
  Endings have nothing to do with what happens, an ending is ultimately a manifestation of values.

  If it never rained, the plants wouldn't grow.

  Possessions are just _things_, man.  Don't let your things control you - be free!

  Everything comes, if you wait the right way.

  Order to the woodspy is chaos to the fey.

  You have this obsession with 'good' and 'bad', but where is this 'good'? What colour is 'bad'?

\end{speechtext}

She will not stop mutating snails, or casting spells as she pleases.
But if any \gls{pc} seems upset by anything, she will help them with another spell.

(Causeway) Misty Way
-----
{The mists fill the causeway}

The \glspl{pc} can see nothing in the causeway, as mist hangs low.
They may have to make a navigation roll just to move about, and projectiles suffer double the normal range penalties.

(Garden) Mist Below
-----
{The Canals fill with mist, only eyestalks roam above}

Mist always falls into the canals by the plateau gardens.
The disorienting environment is quite lethal to newcomers, as people who don't know where the land lies can easily make a misstep, tumbling down the sides.

\begin{boxtext}
  Mist has risen, but is falling into the canals beside the plateau gardens.
  Soon the canals look like fluffy-white rivers.
  A figure in the distance seems to hover between two plateau gardens, but the walking reveals they must be on a vine-bridge which has fallen just below.
\end{boxtext}

Fast movements require a \roll{Wits}{Survival} roll (\tn[8]) to avoid falling into the misty canals below.
Longer journeys require an \roll{Intelligence}{Survival} roll (\tn[10]).

The mists fade after \pgls{interval}.

(Causeway) A Voice from on Hi
-----
{LifeElder looks down at the party, ready to converse again}

This time the \gls{pc}S see LifeElder standing far above from her garden plateau.
She asks them about what they've eaten, and what their favourite kind of rain is.

(Garden) Drooping Bean Vines
-----
{Bean vines on a tree provide a route down}

The bridge-vines don't always grow into bridges.
This one just climbed a tree, which is now covered in bean-producing vines.

If the troupe pull the long vines down, they can safely descend to the canal below.
However, leaving the vines there means creatures below can crawl up; so if the troupe use the vines to descend, the next time they arrive here, they see \pgls{\gls{crawler}} ascending to the plateau garden, and assaulting anyone there.

\section{The Ranger}

If the troupe tarry too long, the \gls{jotter} will send a ranger out to find out what's happening.

(Ravencops) An Old Acquaintance
-----
{A ranger arrives to scope out the situation}

The troupe find him in the woods, hunting a griffin nest.
He speaks haughtily of his ability to survive in the forest, and moves with confidence.
He asks the troupe what they've seen, but does not give their stories much importance.


\begin{speechtext}
  So you still have not found the heart of the problem.
  Well keep searching!
  You may not succeed, but it makes for good practice.
\end{speechtext}

(Garden) Peeping Woodsman
-----
{The ranger explains his plan to kill LifeElder}

The ranger has observed the area for some time, noticed the \gls{disgnome} plants, and believes that the giant snails all stem from a single source: a powerful spellcaster.

>>>
The plan is simple, I kill her.
Elves are always wrapped up in their own thing; they never pay attention, and she's probably drowsy from all the \gls{disgnome} in the area.
So I'll set an ambush then loose an arrow on whoever crafts these giant snails, or slit his throat.
>>>

The ranger will leave the \glspl{pc}, as he does not trust them to stay silent while he plans an ambush for the LifeElder.

(Garden) Loose Clothing
-----
{The ranger's crossbow is found on the ground}

>>>
You don't get to be centuries old without learning how to spot an ambush.
As LifeElder performed one of her standard spells to query the living things in the area, she found the ranger, and guessed the reason for his hiding.
Her spell has split his limbs into myriad tentacles, leaving his equipment on the ground.
He slithered away as the spell took hold, confused and dismayed, dropping pieces of his equipment along the way.
>>>

The find his crossbow and twelve quarrels on the ground, but carrying it sends a clear signal to the elves that they approve of his methods, and makes them dangerous.
They will suffer a -3 Penalty to social rolls with the elves while the crossbow is visible.

If the \glspl{pc} follow the trail, have them roll \roll{Wits}{Survival} (\tn[10]).
Success means you can skip to the next Segment, below.
A tie means they succeed, but only after \pgls{interval} (and another Segment).

(Canals) Discarded Clothing
-----
{Rangers clothes lie discarded on the ground}

The troupe see the last of the ranger's clothing, discarded just before entering the forest.
Following him further will not be easy; the \gls{tn} rises to 14.

# The Movements of Monsters

(Canals) Excuse Me!
-----
{A giant snail blocks the path}

'Brownie' the snail grew much bigger than the others, and pregnancy did not help.
Now when the moves through the canals, she blocks them entirely.
Anyone walking inside the canals must simply turn back and find another route.
This may delay the troupe in whatever they wanted to do by \pgls{interval}.

If the troupe try to squeeze past her, have them roll \roll{Dexterity}{Stealth} at \tn[8].
Failure inflicts 2D6 Damage as her massive body smooshes them against the canal walls.

(Causeway) Stampede
-----
{An auroch stampede passes through}

Elves watch from a garden plateau as aurochs stampede through the causeway.
Behind them, \pgls{\gls{crawler}} chases, but it's losing steam, and soon retreats into the forest.

(Garden) Woodspy Spotted
-----
{A woodspy stalks the garden}

Most large animals have trouble reaching up the tall walls to the plateau gardens, but woodspies are clever, and sometimes manage to pull themselves up using a snail, or finding a piece of vine hanging down.

Any elves present will probably spot the creature.
Either way, it slowly meanders towards an exit path, and waits for someone to pass.
Once it grabs someone, it pulls them down into the canyon.

(Causeway) Slow Wander Home
-----
{The aurochs are returning...slowly}

Any sudden movement will prompt the aurochs to begin a stampede away.
The troupe may prefer to wait, but this will take the rest of the \gls{interval}, as the aurochs move slowly, while grazing.

