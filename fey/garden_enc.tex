\sidequest[Shadepaths,Sunway,plateauGardens]{The Enchanted Land}

\sqpart{Shadepaths}% AREA
{The Picnic Choir}% NAME
{Elven songs descend from the plateau above}% SUMMARY

A little group of elves sing, but don't respond to shouts or calls.
They will, however, respond to someone singing with them.

If someone sings well, they will lower the vines of a broken bridge to help them up.
If they sing poorly, the elves leave silently.

Singing \glspl{pc} should roll \roll{Intelligence}{Performance} (\tn[10]) to understand the elven song, and imitate it.

\sqpart{plateauGardens}% AREA
{Interview with the Tao Mistress}% NAME
{\Glsfmttext{LifeElder} wanders the Shadepaths, and her answers seem strange}% SUMMARY

\Gls{LifeElder} wanders, and sometimes sings.
She stops to ponder a flower, then alters a seed so the flower will grow purple.

While the \glspl{pc} are in \gls{plateauGardens}, they see a small, red-haired elf below, wandering naked and humming to herself.
Other elves may identify her as the source of all the change in the landscape, but will not give her a name (except to say `Hi').

She speaks quickly, and cryptically, and gives deep thought to every word someone says, but quickly tires of conversation.

\begin{speechtext}
  Endings have nothing to do with what happens, an ending is ultimately a manifestation of values.

  If it never rained, the plants wouldn't grow.

  Possessions are just \emph{things}, man.
  Don't let your things control you -- be free!

  Everything comes, if you wait the right way.

  Order to the \gls{woodspy} is chaos to the fey.

  You have this obsession with `good' and `bad', but where is this `good'?
  What colour is `bad'?

  You know so much.
  Come back when you know nothing, and I will teach you nothing.

\end{speechtext}

She will not stop mutating snails, or casting spells as she pleases.
But if any \gls{pc} seems upset by anything, she will help them with another spell.

\sqpart{Sunway}% AREA
{Misty Way}% NAME
{The mists fill the causeway}% SUMMARY

The \glspl{pc} can see nothing in the causeway, as mist hangs low.
They may have to make a navigation roll just to move about, and projectiles suffer double the normal range penalties.


\sqpart{plateauGardens}% AREA
{Mist Below}% NAME
{The Shadepaths fill with mist, only eyestalks roam above}% SUMMARY

Mist always falls into the canals by the plateau gardens.
The disorienting environment is quite lethal to newcomers, as people who don't know where the land lies can easily make a misstep, tumbling down the sides.

\begin{boxtext}
  Mist has risen, but is falling into the canals beside the plateau gardens.
  Soon the canals look like fluffy-white rivers.
  A figure in the distance seems to hover between two plateau gardens, but the walking reveals they must be on a vine-bridge which has fallen just below.
\end{boxtext}

Fast movements require a \roll{Wits}{Survival} roll (\tn[8]) to avoid falling into the misty canals below.
Longer journeys require an \roll{Intelligence}{Survival} roll (\tn[10]).

The mists fade after \pgls{interval}.

\sqpart{Sunway}% AREA
{A Voice from on Hi}% NAME
{\Glsfmttext{LifeElder} looks down at the party, ready to converse again}% SUMMARY

This time the \glspl{pc} see \gls{LifeElder} standing far above from her garden plateau.
She asks them about what they've eaten, and what their favourite kind of rain is.

\sqpart{plateauGardens}% AREA
{Drooping Bean Vines}% NAME
{Bean vines on a tree provide a route down}% SUMMARY

The bridge-vines don't always grow into bridges.
This one just climbed a tree, which is now covered in bean-producing vines.

If the troupe pull the long vines down, they can safely descend to the canal below.
However, leaving the vines there means creatures below can crawl up; so if the troupe use the vines to descend, the next time they arrive here, they see \pgls{crawler} ascending to the plateau garden, and assaulting anyone there.

\sidequest[Shadepaths,Sunway,plateauGardens]{Grumbling Gardens}

\sqpart{Shadepaths}% AREA
{Excuse Me!}% NAME
{A giant snail blocks the path}% SUMMARY

`Brownie' the snail grew much bigger than the others, and pregnancy did not help.
Now when she moves through the canals, she blocks them entirely.
Anyone walking inside the canals must simply turn back and find another route.
This may delay the troupe in whatever they wanted to do by \pgls{interval}.

If the troupe try to squeeze past her, have them roll \roll{Dexterity}{Stealth} (\gls{tn}~8 plus the character's Strength Bonus).
Failure inflicts 2D6 Damage as her massive body smooshes them against the canal walls.

\sqpart{Sunway}% AREA
{Stampede}% NAME
{An auroch stampede passes through}% SUMMARY

Elves watch from a garden plateau as aurochs stampede through the causeway.
Behind them, \pgls{crawler} chases, but it's losing steam, and soon retreats into the forest.

\sqpart{plateauGardens}% AREA
{\Glsfmttext{woodspy} Spotted}% NAME
{\Glsfmttext{woodspy} stalks the garden plateaus}% SUMMARY

Most large animals have trouble reaching up the tall walls to the plateau gardens, but woodspies are clever, and sometimes manage to pull themselves up using a snail, or finding a piece of vine hanging down.

Any elves present will probably spot the creature.
Either way, it slowly meanders towards an exit path, and waits for someone to pass.
Once it grabs someone, it pulls them down into the canyon.

\sqpart{Sunway}% AREA
{Slow Wander Home}% NAME
{The aurochs are returning...slowly}% SUMMARY

Any sudden movement will prompt the aurochs to begin a stampede away.
The troupe may prefer to wait, but this will take the rest of the \gls{interval}, as the aurochs move slowly, while grazing.

