\renewcommand\csComments{
  \draw[very thick,gray] (12,0.6) -- (13,0.6) node[anchor=north]{\outline{1 Mile}} -- (17,0.6) node[anchor=north]{\outline{5 Miles}} ;
}

\mapNotes{
  \normalsize Civilization/08/01,
  \Hu~\Glsfmttext{broch}/22/08,
  \Hu~\Glsfmttext{coppernut} \Glsfmttext{village}/20/30,
  \huge\Glsfmttext{sunderedForest}/35/99,
  \huge\Glsfmttext{enchantedLands}/80/99,
  \rotatebox{40}{\Large\nameref{plateauGardens}}/40/70,
  \rotatebox{40}{\El\R}/45/65,
  \rotatebox{40}{\Large\nameref{ravencops}}/55/45,
  \rotatebox{40}{\N}/58/40,
  \nameref{oathtower}/86/75,
  \El/85/65,
  \rotatebox{40}{\Large\nameref{sunway}}/50/60,
}

\widePic{feylands}

\sidequest[coppernut]{Slowburn Snails}

\sqpart{coppernut}% AREA
{The Logistics of \Glsfmtplural{monster}}% NAME
{something approaches a nearby \glsfmttext{village}, prompting arguments over duty vs stupidity}% SUMMARY

\Gls{SnailTamer} has completely lost control over his mount, and now the giant, murderous, snail, is moving towards \gls{coppernut} -- a nearby \gls{village} -- to feed.
However, the \gls{broch} can only see the eyeballs wobbling over the woods, like a pair of bulbous tentacles.

\begin{boxtext}
  Over the sea of dense green, dancing in the heavy wind, something strange and monstrous moves, taller than the canopy, approaching the \gls{village} below.
  The farmers still work the fields below, in the usual loose circle-formation, with archers watching from the walls.
  But they can't see the tentacles writhing above the canopy.
\end{boxtext}

According to \glsfmtname{susjot}:

\begin{speechtext}
  You are the \gls{guard}.
  That is a beast.
  Go guard!
\end{speechtext}

But according to \composeHumanName\ the \gls{soldier}:

\begin{speechtext}
  \Gls{coppernut} lies two miles away, which requires half an hour at a good march.
  The beast looks to be two miles beyond \gls{coppernut}, so it will arrive long before us, and when we arrive, the situation will already have ended\ldots one way or another.

  We should start the pipes, and at least warn the \gls{village}.
\end{speechtext}

And according to \glsfmtname{dickhead}:

\begin{speechtext}
  They won't hear any pipes over this wind -- it's blowing against us!
  You can run two miles in ten minutes, with gear.
  Beasts always take it slow, so we can beat it; but best not.
  We don't know what it is, so we should stay put and watch.
\end{speechtext}

\begin{boxtext}
  \Gls{dickhead} masticates a petulent bone, while \gls{susjot} and the \gls{soldier} argue.
\end{boxtext}

\paragraph{If the troupe watch from the safety of the \gls{broch},}
they will intermittently see the eyestalks rise again above the canopy, and soon notice how slow the thing is (but will not be able to identify it from the two bulbous tentacles) poking into view for those small moments.

\paragraph{If the troupe avoid interference,}
\gls{susjot} orders them down by morning, and they find the farmlands half-destroyed, and see the snail making a well-fed retreat back to \gls{ravencops}.

\paragraph{If the troupe warn the farmers at \gls{coppernut},}
they can arrive on time with a \roll{Speed}{Athletics} roll (\tn[5]).
Once there, the farmers let them in.
Twenty archers stand active on the walls, but they have little effect on the snail.

\paragraph{Once the troupe see the snail,}
\gls{SnailTamer} calls down to them, with his standard sloth and na\"ivet\'e.

\begin{boxtext}
  The wind dies down, leaving a distant crashing noise which becomes louder, and sounds like crushed bushes and snapped branches, mixed with  retching.
  A rock-like surface emerges, the size of a cottage, with a smooth, brown neck hovering above, which tapers to a point above with two tentacles pointed straight in the air.
  An elf sits on the rock-like surface above, waving slowly.
\end{boxtext}

\SnailTamer

If the snail sees the troupe (or anyone) it begins projectile-vomiting acid while \gls{SnailTamer} -- the elf on the snail's enormous shell -- begins to explain himself:

\begin{description}
  \item[round 1]\it So, hey, um\ldots everyone there.
  Can you hear me?

  \item[round 2] I really want to ask you if you might not hurt Nettlerash here.
  She's a peaceful girl usually.

  \item[round 3] Actually, I named Nettles after a human, or more like, humans in general, because, as you can see, she is very tall, like a human.
  Are you sure you need to do that with the sword?
  I think it's upsetting her.

  \item[round 4] So, it all started a while ago, we went a little too close to the tower, and\ldots wait, let me back up a bit.
  It's actually important that you understand some of the earlier events (I suppose you might call these events `history', or is that prejudiced?
  Sorry, I never actually met humans, unless you count dwarves\ldots).

  \item[round 5] Do you count dwarves?
  I mean no judgement, I just wondered, because\ldots wait, Nettlerash really is not looking healthy.
  Okay, this is serious, we really need to discuss the use of swords and how to respect differences, and\ldots
\end{description}

\iftoggle{verbose}{
  The troupe should find this fight strange.
  On the one hand, anyone with a Speed~Bonus at~-3 can only act every second \gls{round}, so the snail will not do very much.
  But on the other hand, the \glspl{pc} can only Damage it by getting \pgls{vitalShot}, and \emph{even then} \gls{dr}~5 remains;
  so the fight will probably last a few \glspl{round}.
}{}

\giantSnail

\paragraph{If the \glspl{pc} attempt a peaceful resolution,}
they can manipulate the snail easily.

\paragraph{If Nettlerash dies,}
\gls{SnailTamer} leaves, saddened, and will not speak with the troupe until he collects his thoughts.

\paragraph{Once the combat dies down,}
\gls{susjot} appears, and asks the troupe what happened to \gls{dickhead}.
Someone needs to find the source of that giant snail, and make sure no more will be coming this way.

\paragraph{If the troupe ask \gls{SnailTamer} about his plans,}
he says he has a letter for \gls{LifeElder}, from \gls{MindElder}, and must deliver it at once.

\begin{speechtext}
  I have to go.
  I have this letter for someone.

  Who is it for?
  She's just a person, like anyone, but she's not into labels.

  I don't call her anything.
  Well, maybe I call her `hi'.
\end{speechtext}

\paragraph{If anyone grabs the letter,}
they discover two unsurprising things:

\begin{itemize}
  \item
  The letter is in Elvish.
  \item
  \Gls{SnailTamer} objects!
\end{itemize}

\talisman{Urgent Letter}% Name
  {Devious, Detailed}% Enhancements
  {Wax}% Action
  {Water, Fate}% Spheres
  {\roll{Wits}{Academics}}% Resistance
  {``\textit{I would like to discuss the matter of the snails.
  At your convenience, reply with a letter via your most reliable courier.  Yours faithfully, \glsentrytext{MindElder}}''}% Summary
  {If the reader ignores the letter, a nagging feeling remains with them, urging them to do as instructed.
  After a few days, they can think of little else, gaining a -2~Penalty to all Mind \glspl{attribute}, then the Penalty grows until it  reaches -\arabic{spellPlusOne} after a week.}% Details

\showTalisman

\Gls{susjot} can read Elvish, so if he receives the letter, he will feel bound to do as it says, and send the \glspl{pc} with a letter, to \gls{MindElder} (though he has no idea who sent the letter, or where they live).

\paragraph{However this ends,}
\gls{susjot} orders the troupe to go along the snail's path, and find out where it came from, and how it came to be, and try to stop any more coming this way.

\paragraph{If the \glspl{pc} ignore the hook,}
(and somehow evade their duties without being prosecuted) then the giant snails continue to wander up to the \gls{village}.
Now that one has eaten a path to the farmlands, the rest will follow that path more quickly.

\sqpart{coppernut}% AREA
{\glsentrysymbol{night}~Midnight Salad}% NAME
{A giant snail gnaws through the crops}% SUMMARY

%! The region should change to some place by the elven lands, full of baileys et c.
A giant snail comes from \gls{plateauGardens}.
Unlike the last one, it only eats vegetation and will not attack anyone on purpose.
It arrives at night, so the troupe have to make sense of the noises they hear.

If \pgls{npc} watchman keeps \pgls{vigil}, they will hear the snail vomit acid onto the \gls{coppernut}'s wheat crops in order to digest them.

\begin{boxtext}
  Someone stands in the darkness, and moves towards a window.
  He asks if you can hear the vomiting.

  The window shutters creaks open, the sound becomes louder.
  It seems to come from beyond the \gls{village}'s wall.
\end{boxtext}

If the troupe do nothing, \gls{coppernut} loses half its crops over the course of the night, as the snail vomits acidic gloop over them, then eats the gloop.

\paragraph{If the snail takes Damage,}
it flees.

\paragraph{If \gls{coppernut}'s farmers lost any crops,}
they will be angry, and demand the \glspl{pc} do something about the snails.

\sqpart{coppernut}% AREA
{\glsentrysymbol{evening}~Eyestalks at Dusk}% NAME
{Another giant snail arrives, this one wants meat}% SUMMARY

The troupe spot yet another snail heading towards \pgls{village}.
The Sun has almost set, and by the time they arrive, they will have no light.
And this one is carnivorous, and aggressive.

\sqpart{coppernut}% AREA
{The Consequences of Inaction}% NAME
{The \glsentrytext{village} lies empty, a giant shell remains outside}% SUMMARY

Giant snails took too many crops, and the farmers had no idea what to do about \pgls{monster} that eats vegetables (they have had no need to protect their crops so far).
The local \glspl{guard} managed to kill a couple of giant snails (their shells remain in the fields), but not fast enough.

Eventually, everyone abandoned their area, and fled to other \glspl{village} where they had family, or had to enter a town to beg for food.

The \gls{village} lies empty, except for \pgls{crawler}, who moved into one of the houses.

\chitincrawler

\playCommentary{
  \begin{description}
    \item[Player 1:]
    (Sinkmaul)
    Let's jump into the gates.
    \item[\gls{gm}:]
    The people usher you in, pushing the thick wooden gate closed behind you.
    The \gls{village} looks dusky-dim, but active.
    People are rushing out with bundles of arrows in hand, and the \gls{village}'s last two cows are mooing loudly in distress; but the children all know to keep quiet until the danger has gone.
    \item[Player 2:]
    (Mildrain)
    Can we get up to the wall?
    \item[\gls{gm}:]
    You go up the stairs, take $2D6+1$ Damage.
    \item[Player 1:]
    (Sinkmaul)
    Wait, are we both up the stairs?
    \item[Player 2:]
    (Mildrain)
    Wait, do I roll the Damage?
    \item[\gls{gm}:]
    No, I can\ldots that's `12 Damage'.
    (Mildrain)
    Okay, so I'm dead\ldots
  \end{description}
}{
  The initial description helps the players picture \pgls{village} under threat, then it all falls apart.
  \begin{itemize}
    \item
    The player wanted to know about how people access the wall, because that part of the arrangement was entirely unclear.
    They didn't mention their \gls{pc} actually going up.
    \item
    Responding with `\textit{yes, can get up the wall; do you?}' isn't much better.
    More description always works better.
    \item
    \Glsentrylongpl{pc} should take \pgls{action} before they die, not because `it's fair', but to make certain they're committed.
    Once they cast the dice, responsibility shifts from the \gls{gm} to the player.
  \end{itemize}
}


\end{multicols}
