\sidequest[plateauGardens,shadepaths]{Places among \glsfmtplural{plateauGardens}}

% Note roads.
\histEvent{100}{5}{%
  To fix the massive snails getting stuck, she gave them acidic vomit, so they could dissolve bushes and trees, burn through \glsentrytext{crawler} webs, and in general move freely.
  Unfortunately, they ate all her vegetable patches%
}

% Bean Vine Bridges
\histEvent{95}{5}{%
  \Glsfmttext{LifeElder} did not like seeing the elves trapped on different plateaus, like some kind of jail.
  She solved the problem by enchanting bean-vines to bridge nearby spaces between the plateaus, creating actual bridges%
}

Each \gls{segment} in this \gls{sq} shows an area within \gls{plateauGardens}, and the \gls{sq} `continues' as long as the troupe remain in the \gls{area}.
You should note each piece on the map, so that when the troupe return, they find the same place again.

\sqpart{shadepaths}% AREA
{\squash~The Watering Hole}% NAME
{Clear pool now an undrinkable snail-bath}% SUMMARY
\label{shadePool}

The troupe see a thin trickle of water running through their mossy path, leading to a little lake where snails bathe.

\begin{boxtext}
  The clean rivulets meet ahead in an opening between the dense rock-walls, where Sunlight falls in.
  They form a little lake, twenty \glspl{step} across, where three giant snails meet, to dip their eyes in the pool, and slide across each other.
\end{boxtext}

The snails leave the pool filthy, and the elves don't like the grime.
The \gls{sunderedForest} does not have many watering holes, so this provides the \glspl{pc} with a potential weakness in the area; if they can remove the  water, the giant snails will all leave the area quickly, and go downhill towards the nearest river, before \gls{MindElder} can make them carnivorous.

\paragraph{Each time the troupe arrive at the watering hole,}
they find 1D6-2 giant snails bathing.

\sqpart[\squash\gls{vlg}]{plateauGardens}% AREA
{Enlightenment}% NAME
{The flowers of the garden of light make you float}% SUMMARY
\label{lightFlowers}

\histEvent{50}{1}{%
  \Glsfmttext{juliet} became bored of trying to manipulate bodies, and focussed herself on the Force \glsfmttext{sphere}%
}

\Gls{juliet} has cultivated these flowers using the Force \gls{sphere}.
The flowers of enlightenment make you light, making it easier to carry things; but first they must wilt and go brown.

\begin{boxtext}
  A bed of bright-red flowers, with long petals like the floppy ears on a dog.
  It looks like someone made space for them, and spent a lot of time on their soil bed.
\end{boxtext}

\label{flowerOfEnlightenment}
\talisman{Flower of Enlightenment}% Name
  {}% Enhancements
  {Wane}% Action
  {Fire, Earth}% Spheres
  {7 + total \glsentrytext{weight} carried}% Resistance
  {Eating the dried-up flowers make people light, and reduces their \gls{weight} by \arabic{spellPlusOne}, which lets them ignore that much \gls{weight} from items, and add a +\arabic{spellPlusOne}~Bonus to \gls{running}}%Summary
  {\par%
    Drying the flowers requires two \glspl{interval} of Sunlight.
  }% Details

\showTalisman

\sqpart[\gls{vlg}]{plateauGardens}% AREA
{Purple, Yellow Beds}% NAME
{Mind-rending plants hide among the carrots}% SUMMARY

Long, green plants spring up, indicating massive carrots below.
The weak elves find pulling them up to be very difficult, and only do so in groups, or by speaking sweetly to the earth, and asking it to let the carrot go.

\Glspl{disgnome} hide among the carrots.
Walking past is harmless, but anyone digging up the carrots will also come into contact with the \gls{disgnome} roots, and start to lose the ability to think clearly.

\sqpart{shadepaths}% AREA
{Guardian Stones}% NAME
{Last hope of the elves: hidden lake uncovered behind seeping-wet wall}% SUMMARY
\label{shadeDamn}

\histEvent{40}{5}{%
  \Glsfmttext{LifeElder} walled off the last clean lake in \glsfmttext{plateauGardens} to stop the giant snails infecting it%
}

With little clean water left, \gls{LifeElder} guarded the last pool of water by summoning stony walls around it.
Water escapes through little holes at the base, which will give the characters a clue about this hidden lake.

\begin{boxtext}
  A shining, tiny, rivulet meanders through the barren, dry canal.
  The water smells fresh!
\end{boxtext}

Nobody can see the lake from the outside.
Trees in \gls{plateauGardens} merge seamlessly with trees around the lake.
From a distance, it all looks like a continuous canopy.

If the elves cannot use this lake -- due to snail-access, or poisoning, or some other catastrophe -- they will find themselves without a clean source of water, and \gls{LifeElder} will have to stop supporting the snails.

\sqpart{shadepaths}% AREA
{The Great Snail Lake}% NAME
{Lake spotted from a garden plateau}% SUMMARY

The canyon widens here, and a barren, slimy land (stripped bare by giant snails) holds a great lake in the centre.
It stretches as far as an arrow's flight, and glistens with a thick film of slime across most of the surface.

Garden plateaus surround the lake, and each one holds a narrow staircase down.
The crack in the plateaus where the stairs descend is very narrow.
Characters with Strength~+1 can only enter the staircase by removing all armour and squeezing through.
Anyone with a higher Strength Bonus cannot enter.

Each time the troupe arrive at the lake,
they see 2D6-2 giant snails bathing, and 1D6-3 elves collecting water.

The elves purify the water with spells when they can, but this requires \glspl{mp}, which are in scarce supply in the area.

