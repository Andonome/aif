\thread[sunway,shadepaths]{Locations in \glsfmttext{sunderedForest}}

\segment[\gls{vlg}]{sunway}% AREA
{Elven Steps}% NAME
{A hidden path leads to \glsfmttext{plateauGardens} above}% SUMMARY
\label{hiddenStairs}

In the causeway between the \gls{ravencops} forest and \gls{plateauGardens}, a single plateau has a hidden stairway, going up.
\Glspl{crawler} cannot make much use of the narrow stairs, with occasional hand-holds for little fingers.
People who don't know about the stairs cannot usually see them, as every step blends into the tall rock-face from below.
But once someone notices the first step, they see the next, and then the next, and so on.

\paragraph{Spotting the rocks}
requires a \roll{Wits}{Vigilance} roll at \tn[12].
The \gls{tn} increases by~+1 in the rain, and by~+3 at night.

\segment[\squash\gls{vlg}]{shadepaths}% AREA
{The Watering Hole}% NAME
{Clear pool now an undrinkable snail-bath}% SUMMARY
\label{shadePool}

The troupe see a thin trickle of water running through their mossy path, leading to a little lake where snails bathe.

\begin{boxtext}
  The clean rivulets meet ahead in an opening between the dense rock-walls, where Sunlight falls in.
  They form a little lake, twenty \glspl{step} across, where three giant snails meet, to dip their eyes in the pool, and slide across each other.
\end{boxtext}

The snails leave the pool filthy, and the elves don't like the grime.
\Gls{sunderedForest} does not have many watering holes, so this provides the \glspl{pc} with a potential weakness in the area; if they can remove the  water, the giant snails will all leave the area quickly, and go downhill towards the nearest river, before \gls{MindElder} can make them carnivorous.

\paragraph{Each time the troupe arrive at the watering hole,}
they find 1D6-2 giant snails bathing.

\segment[\gls{vlg}]{shadepaths}% AREA
{Guardian Stones}% NAME
{Last hope of the elves: hidden lake uncovered behind seeping-wet wall}% SUMMARY
\label{shadeDamn}

\histEvent{40}{5}{%
  \Glsfmttext{LifeElder} walled off the last clean lake in \glsfmttext{plateauGardens} to stop the giant snails infecting it%
}

With little clean water left, \gls{LifeElder} guarded the last pool of water by summoning stony walls around it.
Water escapes through little holes at the base, which will give the characters a clue about this hidden lake.

\begin{boxtext}
  A shining, tiny, rivulet meanders through the barren, dry canal.
  The water smells fresh!
\end{boxtext}

Nobody can see the lake from the outside.
Trees in \gls{plateauGardens} merge seamlessly with trees around the lake.
From a distance, it all looks like a continuous canopy.

\paragraph{If the walls break,}
then the lake spills, creating \pgls{flood}.
With the last clean source of water gone, \gls{LifeElder} will have to stop supporting the snails.

\segment[\gls{vlg}]{shadepaths}% AREA
{The Great Snail Lake}% NAME
{Lake spotted from a garden plateau}% SUMMARY

The canyon widens here, and a barren, slimy land (stripped bare by giant snails) holds a great lake in the centre.
It stretches as far as an arrow's flight, and glistens with a thick film of slime across most of the surface.

Garden plateaus surround the lake, and each one holds a narrow staircase down.
The crack in the plateaus where the stairs descend is very narrow.
Characters with Strength~+1 can only enter the staircase by removing all armour and squeezing through.
Anyone with a higher Strength Bonus cannot enter.

Each time the troupe arrive at the lake,
they see 2D6-2 giant snails bathing, and 1D6-3 elves collecting water.

The elves purify the water with spells when they can, but this requires \glspl{mp}, which are in scarce supply in the area.

