\section{\Glsfmttext{plateauGardens}}
\label{plateauGardens}

Elves live along the plentiful gardens, where \gls{LifeElder} uses her spells to encourage vegetables to grow massive in half the usual time, or even less.
Most of the elves have almost no chance to practice any spells, as the \gls{region} has almost no \glspl{mp}; but they don't care.
They learn to sing, construct and deconstruct new ideas, then gossip about infidelities among local robins and kestrels.

\Gls{LifeElder} casts spells every \gls{interval}, so she consumes most of the \glsentrylongpl{mp} in the \gls{region} before anyone else can.
When she's far away, the troupe can divide 2~\glspl{mp} among themselves, and when she's nearer, they only regain 1~\gls{mp} each \gls{interval}.
And, of course, once the troupe regain 0~\glspl{mp}, they should understand, that \gls{LifeElder} must be close.

\printThreadsInRegion{plateauGardens}

\begin{multicols}{2}

\subsection{What's Up There?}

Each of the \glspl{plateauGardens} is around 250~\glspl{step} across, and holds an abundance of plants.
As the \glspl{pc} step onto \pgls{plateauGardens}, roll three dice, and apply every result.

\begin{dlist}
  \item
  A vine-bridge connects to another plateau.
  It can hold up to \pgls{weight} of 20 before collapsing.
  
  The next plateau holds a snail-shell house, with tools to make snail-saddles below, and sleeping quarters above.
  \item
  A ripened vine-bridge with plenty of beans stretches to another plateau.
  It holds up to \pgls{weight} of 10 before collapsing.
  \item
  \Glspl{disgnome} hide among the flowers.
  Digging up vegetables results in a sting from its roots which saps $1D3$~\glspl{ep}.
  \item
  Fresh vegetables -- tomatoes, potatoes, and carrots -- grow in abundance.
  The garden holds $1D6 \times 10$ days' \glspl{ration}.
  \item
  Fruit trees galore!
  Figs, plums, hazelnuts, and peaches, all ripe.
  The garden holds $1D6 \times 5$ days' \glspl{ration}.
  \item
  Elven song from the next plateau, where $1D3$ elves sing together.
\end{dlist}

Not every plateau has a vine-bridge, so if the troupe manage to ascend, they will not necessarily be able to journey across the \glspl{plateauGardens}.

\thread[shadepaths,plateauGardens]{The Tao Mistress}

\segment{plateauGardens}% AREA
{Lazing Elves}% NAME
{A group of elves discuss Philosophy}% SUMMARY

Helin, Huon, and Lav\"e are discussing Philosophy (in Elvish).
They take a brief interest in the \glspl{pc}, the return to their conversation.
They answer questions with more questions, then argue both sides of their own question.

To avoid vexing the players (while still vexing the \glspl{pc}) you might want to abstract this process with a \roll{Wits}{Empathy} (\tn[10], or 7 for characters who speak Elvish).
A tie results in one proper answer in exchange for the \glspl{pc} leaving the elves in peace, and each Margin grants another answer.

\begin{itemize}
  \item\it
  Do you live here?
  \item[\adforn{51}]\sl
  Ma quetill\"e firyon lamb\"e?
  \item[\adforn{53}]\bf
  I am here and I live.
  I suppose you live here too, for now.
  \item\it
  Can you tell us\ldots
  \item[\adforn{53}]\bf
  Are you `sea humans'?
  \item[\adforn{52}]\bf\sl
  No, sea humans wear sails as clothes. 
  These are `high humans', who live in stone constructions, and make barrels of rotten auroch milk.
  \item\it
  Right, we live in houses, so can you tell us\ldots
  \item[\adforn{53}]\bf
  You're thinking of dwarves.
  These are `forest humans', look at their hunting tools.
  They use these to stab aurochs, and then have to eat the auroch before it rots.
  \item\it
  Who made the snails?
  \item[\adforn{52}]\bf\sl
  Nobody can make a snail except another snail.
  But it's a bit of a snail-and-egg problem.
  Was there a first snail, or a first egg?
  \item\it
  Okay, but who made the snails \emph{big}?
  \item[\adforn{51}]\sl
  Ma firyar nar orqui?
  \item\it
  Sorry, I don't speak Elvish.
  \item[\adforn{53}]\bf
  Sancossi nar firyar, nan sin\"e firyali umir sancossi.
\end{itemize}

\elf[\npc{\T[2]\M\M\El}{Helin \& Huon}]

\elf[\npc{\F\El}{Lav\"e}]

\showStdSpells

\segment{plateauGardens}% AREA
{Interview with the Tao Mistress}% NAME
{\Glsfmttext{LifeElder} wanders \glsfmttext{shadepaths}, and her answers seem strange}% SUMMARY

While the \glspl{pc} are in \gls{plateauGardens}, they see a small, red-haired elf below, wandering naked and humming to herself.
Other elves may identify her as the source of all the change in the landscape, but will not give her a name (except to say `Hi').

\Gls{LifeElder} wanders, and sometimes sings in a way that somehow matches the breeze.
She stops to ponder a flower, then alters a seed so the flower will grow purple.

She speaks quickly, and cryptically, and gives deep thought to every word someone says, but quickly tires of conversation.

\begin{itemize}
  \item\it
  What's your name?
  \item[\adforn{54}]\bf
  I'm not really into labels.
  \item\it
  Did you make the snails?
  \item[\adforn{54}]\bf
  Nobody can make a snail.
  Snails are completely impossible, unless you have a snail.
  Is that a paradox?
  \item\it
  Did you make the snails grow big?
  \item[\adforn{54}]\bf
  No, but I suggested it.
  Making someone do something sounds violent.

  Have you had a ride on a snail yet?
  \item\it
  Could you stop making large snails?
  \item[\adforn{54}]\bf
  Certainly.
  I've not been doing a lot of things, what's one more thing to not do?

  But who will all the little biters to the East eat?
  \item\it
  I don't care what the goblins eat, could you just stop\ldots wait did you say `who'?
  \item[\adforn{54}]\bf
  No, wrong word.
  I haven't spoken Gnomish in a while.
  It should be `whom', shouldn't it?
  \item\it
  We're human.
  Wait! what do you mean `whom the goblins will eat'?
  \item[\adforn{54}]\bf
  You speak Gnomish well for a human.
  You know so much!
  \item\it
  Thanks.
  Listen.
  I know you're busy, but\ldots
  \item[\adforn{54}]\bf
  Come back when you know nothing, and I will teach you nothing.
\end{itemize}

\LifeElder

\showStdSpells

\segment[\squash]{shadepaths}% AREA
{Misty Way}% NAME
{The troupe must coordinate by sound and deduction}% SUMMARY

The \glspl{pc} can see nothing in the causeway, as mist hangs low.
They may have to make a navigation roll just to move about, and projectiles suffer double the normal range penalties.

\segment{shadepaths}% AREA
{A Voice from on Hi}% NAME
{\Glsfmttext{LifeElder} looks down at the party, ready to converse again}% SUMMARY

While the troupe wander through \gls{shadepaths}, \gls{LifeElder} has been searching \glspl{plateauGardens} for \glspl{ingredient} to make \glspl{boon} so she can make more giant snails.

\begin{boxtext}
  A voice from above floats down into the shadows.
  The grey-haired elf looks down, eyes wide-open, and she says ``\textit{I wonder if you found enough food}''.
\end{boxtext}

As before, she remains distracted and indifferent, but potentially helpful if the \glspl{pc} seem peaceful.

Here are the kinds of things she says:

\begin{itemize}
  \item[\adforn{54}]\bf
  What is a human's favourite type of rain?

  Or do humans like all rain equally?
  \item[\adforn{54}]\bf
  Endings have nothing to do with what happens.
  An ending is ultimately a manifestation of values.
  \item[\adforn{54}]\bf
  If it never rained, the plants wouldn't grow.
  \item[\adforn{54}]\bf
  Possessions are just \emph{things}.
  Don't let your things control you -- be free!
\end{itemize}

\paragraph{She leaves soon after,}
hoping to gather \glspl{ingredient} to make 2~\glspl{boon} so she can make more giant snails.

\spell{Hyalmahta}% Name
  {Detailed, Devious, Duplicated}% Enhancements
  {Wax}% Action
  {Earth, Water}% Spheres
  {available plant quantities}% Resist with
  {The caster thinks about past dreams of salad and \arabic{spellTargets} snails in the vicinity begin to eat and grow.
  Within a few days, they reach the size of a human, and after one \showOnset\ reach the size of a house.
  Each snail has Strength~+5, Speed~-3, \gls{dr}~5 (or 10 on the shell), along with the abilities Viscid and Acid Spray}% Description
  {}


\segment{plateauGardens}% AREA
{Sudden Chills}% NAME
{All the mana vanishes, as \glsfmttext{LifeElder} wanders nearby}% SUMMARY

\Gls{LifeElder} wanders nearby, so all the \glspl{mp} in the area vanish.
For the next two \glspl{interval}, nobody in the \gls{area} receives a single drop.
The \glspl{pc} (or friendly elves) may be able to use this to locate \gls{LifeElder} with spells.%
\exRef{core}{Core Rules}{ManaVoidSpell}

\end{multicols}
