\chapter{Call to Adventure}
\epigraph{
  There was an old lady who swallowed a fly,

  I don't know why she swallowed a fly – perhaps she'll die!
}


\label{callToAdventure}

\section{Introduction}

\begin{multicols}{2}

\begin{exampletext}
  There was an old lady who swallowed a fly,
  I don't know why she swallowed a fly – perhaps she'll die!
\end{exampletext}

Elves move slowly through the dense forest.
They don't have the numbers or strength to build a proper road -- but they their ways.

\begin{exampletext}
  Snails leave little roads\ldots
  If I grew giant snails, they could forge a road as they move.
\end{exampletext}

Unfortunately, the giant snails could not move through the dense trees.

\begin{exampletext}
  I'll grow them differently next time, with acidic vomit to burn away the trees.
  They only eat plants, and they move slowly, so they won't hurt anyone.
\end{exampletext}

However, the snails proceeded to eat all the food in every garden, leaving the elves with nothing.

\begin{exampletext}
  I will ask the earth to rise up around the gardens, creating raised platforms.
  The giant snails can wander through the little canyons between the gardens, keeping the paths clear, while we enjoy the gardens.
  Finally, the plan is perfect.
\end{exampletext}

So the elves cultivated their raised gardens, though they had trouble getting down, and soon made a new plan\ldots

\begin{exampletext}
  \noindent
  There was an old lady who swallowed a goat; \\
  Just opened her throat and swallowed a goat! \\
  She swallowed the goat to catch the dog, \\
  She swallowed the dog to catch the cat, \\
  She swallowed the cat to catch the bird, \\
  She swallowed the bird to catch the spider \\
  That wriggled and jiggled and tickled inside her, \\
  She swallowed the spider to catch the fly; \\
  I don't know why she swallowed a fly – perhaps she'll die! \\
\end{exampletext}

But what happens when the cat dies?
What does the dog chase, and where does the bird go?

Every \gls{spell} adds a new problem, and every problem can be fixed with more magic.
And eventually a giant snail escapes to the human lands, where it becomes the players' problem.
They journey back to the source, going back down the chain of magical fixes until they reach the source.

And after that, I have no idea what happens.
Hopefully they won't just fix one random problem, creating a domino-effect throughout the land.

The chain of \glspl{spell} follows the pattern of the children's song, \textit{There Was an Old Lady Who Swallowed a Fly}.
Or perhaps it's more like a bad programmer, who patches bugs with dirty fixes, and patches those with more fixes, until nothing can be changed without the entire program collapsing.


\spell{Hyalmahta}% Name
  {Detailed, Devious, Duplicated}% Enhancements
  {Wax}% Action
  {Earth, Water}% Spheres
  {available plant quantities}% Resist with
  {The caster thinks about past dreams of salad and \arabic{spellTargets} snails in the vicinity begin to eat and grow.
  Within a few days, they reach the size of a human, and after one \showOnset\ reach the size of a house.
  Each snail has Strength~+5, Speed~-3, \gls{dr}~5 (or 10 on the shell), along with the abilities Viscid and Acid Spray}% Description
  {}

\spell{Quamahta}% Name
  {Detailed, Duplicated}% Enhancements
  {Warp}% Action
  {Fate, Water}% Spheres
  {fullness of snail belly}% Resist with
  {The caster describes meat-based salads, and \arabic{spellTargets} snails begin to hunt for crawling things in the bushes as much as they do the bushes themselves}% Description
  {They don't eat people who are not surrounded by bushes (because they expect meat only in regular food).}


\subsection{\Glsfmtname{coppernut}}

\histEvent{30}{5}{%
  The river stretching through \glsfmttext{shadepaths} becomes so polluted with snail-muck that the residents of \glsfmttext{coppernut} contract diseases%
}

Named after the crown of copper spikes along its outer wall, and the ring of walnut trees around the farmland perimeter, this little \gls{village} holds 100 humans, out here at the \gls{edge}.
They struggle to survive, as \glspl{monster} attack while they farm, and crawl over the houses at night, trying to find an opening.
And over the last few decades, many have contracted Mindflash Syndrome, or Guardbane.%
\exRef{judgement}{Judgement}{diseases}

However, they do benefit from being close to \gls{ravencops}, as no \glspl{monster} come from that forest.
Nothing ever came from that forest, until today.


%\begin{multicols}{2}

\sidequest[Broch]{Slowburn Snails}

\sqpart{Broch}% AREA
{A Logistical Discussion}% NAME
{A monster approaches a distant \glsfmttext{village}, prompting arguments over duty vs stupidity}% SUMMARY

A giant, out-of-control snail, is moving towards a nearby \gls{village} to feed.
Its rider -- \gls{SnailTamer} -- has completely lost control of it.

However, the \gls{broch} can only see the eyeballs wobbling over the woods, like a pair of bulbous tentacles.

\begin{boxtext}
  Over the sea of dense green, dancing in the heavy wind, something strange and monstrous moves, taller than the canopy, approaching the \gls{village} below.
\end{boxtext}

According to the \gls{jotter}:

\begin{speechtext}
  You are the \gls{guard}.
  That is a beast.
  Go guard!
\end{speechtext}

But according to the \gls{soldier}:

\begin{speechtext}
  The \gls{village} lies two miles away, which requires half an hour at a good march.
  The beast looks to be two miles beyond the \gls{village}, so it will arrive long before us, and when we arrive, the situation will already have ended\ldots one way or another.

  We should start the pipes, and at least warn the \gls{village}.
\end{speechtext}

And according to \gls{dickhead}:

\begin{speechtext}
  The \gls{village} won't hear any pipes over this wind -- it's blowing against us.
  You can run two miles in ten minutes, with gear.
  Beasts always take it slow, so we can beat it; but best not.
  We don't know what it is, so we should stay put.
\end{speechtext}

In truth, \gls{SnailTamer} has lost control of his mount, `Nettlerash', who recently acquired a taste for living flesh in the Kingdom of Oaths.

If the troupe watch from the safety of the \gls{broch},
they will intermittently see the eyestalks rise again above the canopy, and soon notice how slow the thing is (but will not be able to identify it from the two bulbous tentacles) poking into view for those small moments.

\paragraph{If the troupe warn the \gls{village},}
the farmers let them in, then cover the wall in archers -- 20 in total.

\paragraph{Once the troupe see the snail,}
\gls{SnailTamer} calls down to them, with his standard sloth and na\"ivet\'e.

\begin{boxtext}
  The wind dies down, leaving a distant crashing noise which becomes louder, and sounds like crushed bushes and snapped branches, mixed with  retching.
  A rock-like surface emerges, the size of a cottage, with a smooth, brown neck hovering above, which tapers to a point above with two tentacles pointed straight in the air.
  An elf sits on the rock-like surface above, waving slowly.
\end{boxtext}

\SnailTamer

If the snail sees the troupe (or anyone) it begins projectile-vomiting acid while \gls{SnailTamer} -- the elf on the snail's enormous shell -- begins to explain himself:

\begin{description}
  \item[round 1] So, hey, um\ldots everyone there.
  Can you hear me?

  \item[round 2] That's good, good that you can hear me, because I really want to ask you if you might not hurt Nettlerash here.
  She's a good girl usually.

  \item[round 3] Actually, I named Nettles after a human, or more like, humans in general, because, as you can see, she is very tall, like a human.
  Are you sure you need to do that with the sword?
  I think it's upsetting her.

  \item[round 4] So, it all started a while ago, we went a little too close to the tower, and\ldots wait, let me back up a bit.
  It's actually important that you understand some of the earlier events (I suppose you might call these events `history', or is that prejudiced?
  Sorry, I never actually met humans, unless you count dwarves\ldots).

  \item[round 5] Do you count dwarves?
  I mean no judgement, I just wondered, because\ldots wait, Nettlerash really is not looking good.
  Okay, this is serious, we really need to discuss the use of swords and how to respect differences, and\ldots
\end{description}

\iftoggle{verbose}{
  The troupe should find this fight strange.
  On the one hand, anyone with a Speed~Bonus at~-3 can only act every second \gls{round}.
  But on the other hand, the \glspl{pc} can only Damage it by getting \pgls{vitalShot}, and \emph{even then} \gls{dr}~5 remains;
  so if they can't deal massive Damage, the fight will take a few \glspl{round}.
}{}

\giantSnail

\paragraph{If the \glspl{pc} attempt a peaceful resolution,}
they can manipulate the snail easily.

\paragraph{If Nettlerash dies,}
\gls{SnailTamer} leaves, saddened, and will not speak with the troupe until he collects his thoughts.

\paragraph{Once the combat dies down,}
the \gls{jotter} appears, and asks the troupe what happened to \gls{dickhead}.
Someone needs to find the source of that giant snail, and make sure no more will be coming this way.

\paragraph{If the troupe ask \gls{SnailTamer} about his plans,}
he says he has a letter for \gls{LifeElder}, from \gls{MindElder}, and must deliver it at once.

\begin{speechtext}
  I have to go. \\
  I have this letter for someone.

  Who is it? \\
  She's just a person, like anyone, but she's not into labels.

  I don't call her anything. \\
  Well, maybe I call her `hi'.
\end{speechtext}

\paragraph{If the \glspl{pc} ignore the hook,}
(and somehow evade their duties without being prosecuted) then the giant snails continue to attack.

\paragraph{At the end of the \gls{interval},}
the wind glow stronger, becoming \pgls{hurricane}.

\sqpart{Broch}% AREA
{\glsentrysymbol{night}~Midnight Salad}% NAME
{A giant snail gnaws through the crops}% SUMMARY

%! The region should change to some place by the elven lands, full of baileys et c.
A giant snail comes from \gls{plateauGardens}.
Unlike the last one, it only eats vegetation and will not attack anyone on purpose.
It arrives at night, so the troupe have to make sense of the noises they hear.

If \pgls{npc} watchman keeps \pgls{vigil}, they will hear the snail vomit acid onto the \gls{village}'s wheat crops in order to digest them.

\begin{boxtext}
  Someone stands in the darkness, and moves towards a window.
  He asks if you can hear the vomiting.

  The window shutters creaks open, the sound becomes louder.
  It seems to come from beyond the \gls{village}'s wall.
\end{boxtext}

If the troupe do nothing, the \gls{village} loses half its crops over the course of the night, as the snail vomits acidic gloop over them, then eats the gloop.

\paragraph{If the snail takes Damage,}
it flees.

\paragraph{If the \gls{village} lost any crops,}
the farmers will be angry, and demand the \glspl{pc} do something about the snails.

\sqpart{Broch}% AREA
{\glsentrysymbol{evening}~Eyestalks at Dusk}% NAME
{Another giant snail arrives, this one wants meat}% SUMMARY

The troupe spot yet another snail heading towards \pgls{village}.
The Sun has almost set, and by the time they arrive, they will have no light.

\sqpart{Broch}% AREA
{The Consequences of Inaction}% NAME
{The \glsentrytext{village} lies empty, a giant shell remains outside}% SUMMARY

Giant snails took too many crops, and the farmers had no idea what to do about \pgls{monster} that eats vegetables (they have had no need to protect their crops so far).
The local \glspl{guard} managed to kill a couple of giant snails (their shells remain in the fields), but not fast enough.

Eventually, everyone abandoned their area, and flet to \glspl{village} where they had family, or had to enter a town to beg for food.

The \gls{village} lies empty, except for \pgls{crawler}, who moved into one of the houses.

\end{multicols}


\commentary{
  \begin{description}
    \item[\gls{gm}:]
    The path comes to a cross-roads.
    Do you continue forwards, or go left, or right?
    \item[Player 1:]
    (Sinkmaul)
    Left?
    \item[Player 2:]
    (Mildrain)
    Left\ldots
    \item[\gls{gm}:]
    Now the path has a turn-off to the right\ldots
  \end{description}
}{
  This is awful, and the \gls{gm} should be fed to goblins, feet-first.

  If the players had previously said they want to approach `the tower', then the next part of that process is either arriving at the tower, or failing.
  The \gls{gm} can roll out a long description, to make the length of the march clear, and they can run various distracting \glspl{segment} on the way, but the moment the \glspl{pc} are free to act, the journey to their goal should resume.
}

\chapter{Elven Forests}
\epigraph{
  There was an old lady who swallowed a goat;

  Just opened her throat and swallowed a goat!

  She swallowed the goat to catch the dog,

  She swallowed the dog to catch the cat,

  She swallowed the cat to catch the bird,

  She swallowed the bird to catch the spider

  That wriggled and jiggled and tickled inside her,

  She swallowed the spider to catch the fly;

  I don't know why she swallowed a fly – perhaps she'll die!
}


\label{elvenForests}

\section{\Glsfmttext{ravencops}}
\label{ravencops}

\Gls{ravencops} received its name from the guttural bird-calls in the area.
Despite being a verdant forest, it feels bleak, and has little edible food.
All attempts at \gls{foraging} are at \gls{tn}~14.

\Gls{oathtower} sucks up most of the \glspl{mp} in the surrounding \gls{region}, which makes the air feel thin.
Unless \pgls{witch} is missing 6~\glspl{mp} or more, they can only receive 2~\glspl{mp} at the end of \pgls{interval}.%
\exRef{core}{Core Rules}{manaVacuum}

\printSideQuestsInRegion{ravencops}

\histEvent{105}{4}{%
  \Gls{LifeElder} loved the little paths snails make, and wanted to walk across them, but she was too big.
  To create her roads, she used Life \glspl{spell} to grow the snails to monstrous proportions.
  But once they were as big as a house, they just got stuck in the tall trees%
}

\begin{multicols}{2}

\iftoggle{verbose}{
  When the eye of the story moves North, it finds a stone sepulchre.
  Then it moves West and finds a quarry, surrounded by giant snails.
  The land seems full of life, because the \glspl{sq} below play a trick.
  The elven lands have only five locations, but every time the \glspl{pc} move, another \gls{segment} in the story places another location on the map.

  Every \gls{interval}, scan down the list of readied \glspl{segment} in the \gls{region} (marked `\gls{sqr}'), and make it happen.
  Once the \gls{segment} ends, mark the \emph{next \gls{segment} in the \gls{sq} as ready} (not the next \gls{segment} in the list).

  The order of the \glspl{segment} depends a lot on the players.
  As an example:

  \begin{itemize}
    \item
    The players enter \gls{ravencops}, and the \gls{gm} runs the first two available \glspl{segment} (check them \vpageref{ravencops}).
    \item
    The players decide to investigate \gls{oathtower}, so the \gls{gm} finds that location \vpageref{oathtower}\ldots
    \item
    As the \gls{gm} runs \gls{oathtower}'s only available \gls{segment}, the \glspl{pc} get into a lot of trouble and have to flee.
    The \gls{gm} finds no more \glspl{segment} marked as ready in that \gls{region}, so nothing more happens in that \gls{region}, until more \glspl{segment} become ready.
    \item
    The \glspl{pc} want to investigate \gls{plateauGardens}, but they have to journey through \gls{ravencops} again to reach it.
    A new \gls{segment} has been made available after journeying to the \gls{oathtower}.
  \end{itemize}

  Try jumping through the \glspl{region} quickly, and checking off two \glspl{segment} in each one.
  What happens if the players decide to ignore \gls{oathtower} and go straight from \gls{ravencops} to \gls{sunway} (\vpageref{sunway}) then \gls{plateauGardens} (\vpageref{plateauGardens})?
}{}

\sidequest[ravencops,oathtower,sunway]{\Glsfmttext{enchantedLands}}

\sqpart{oathtower}% AREA
{Groaning Sepulchres}% NAME
{The troupe must walk quietly and avoid the groaning sepulchres}% SUMMARY

\histEvent{20}{2}{%
  \Glsentrytext{MindElder} finds the goblins overpopulating the area, and fears the day they run out of food; even the most powerful enchantments cannot withstand goblin hunger.
  He rewards loyal goblins with obscene amounts of food, which lets them grow and grow, into hobgoblins, and eventually into \glsfmtplural{ogre}.
  Trials include service in \glsfmttext{oathtower}, hunting dangerous creatures, and plenty of duels (which really helps reduce the population).
  Once the goblins ascend, \glsentrytext{MindElder} places them in an enchanted sleep inside a stone sepulchre (where the goblins can check on them)}

\begin{exampletext}
  Goblins have no natural height limits, so when they eat too much, they just grow and grow, until one day, without any clear cut-off point, people call them a `hobgoblin', and soon after, `\gls{ogre}'.

  When the goblins become \glspl{ogre}, \gls{MindElder} puts them into an enchanted sleep, and tells the goblins their big brothers will awaken when the time of grand feasting comes.
  The goblins must see the sleeping \glspl{ogre} from time to time, or they will suspect murder and betrayal, and even their oaths will not keep them passive.

  So the forest around \gls{oathtower} has slowly filled up with snoring sepulchres, and everyone must tread quietly, lest they wake and ask for breakfast\ldots
\end{exampletext}

\begin{boxtext}
  Past the trees, an arrow's flight away, a mossy tower stands as tall as a feasting hall turns on its end.
  A low groaning noise, like a distant earthquake, floods through the trees, surrounding you.
\end{boxtext}

\Gls{oathtower} has a lot of sepulchres dotted around it, often hidden by trees, and always with little paths leading towards them.
Each one has three \glspl{ogre}, cramped in together.
They stand as wide as a cottage, but not as tall, and the sound of snoring emanates for a few hours each day.

\paragraph{High-pitched noises near \gls{oathtower}}
have a 1 in 6 chance of waking \pgls{ogre}.
The chances increase by 1 for loud noises, or noises closer to the sepulchres.

In order to avoid waking the \glspl{ogre}, \gls{MindElder} has told the goblins to slay anyone making high-pitched noises, such as whistling or laughing.
Farting at any pitch is also banned, as it makes the goblins giggle, which then wakes the \glspl{ogre}.

\enchantedOgre[\NPC{\M\N}{`The Grave'}%
  {slate-coloured skin, with bright-blue eyes}%
  {stretches calves}%
  {deer with cheese}%
  \npcQuote{only asking, only asking\ldots}]

\enchantedOgre[\NPC{\F\N}{Kerning}%
  {bra made from human faces (it helps with running, not modesty)}%
  {chews leaves, then spits them out}%
  {\gls{crawler} soup}%
  \npcQuote{the road goes ever on, until it doesn't.
  `Dead end', they call it}]

\sqpart{ravencops}% AREA
{Goblins in the Quarry}% NAME
{\Glsfmttext{romeo} should be working, but needs to complete the perfect poem}% SUMMARY
\label{goblinQuarry}

\Gls{MindElder} has sent his son \gls{romeo} to oversee the goblins, excavating rock at the quarry, and cutting long slabs to construct more sepulchres.
But \gls{romeo} can't think of anything but the poem he needs to write, to tell his beloved how how he feels.
Unfortunately, his father raised him to be a perfectionist, which means he can't write perfect poetry, or good poetry, or bad poetry, or any poetry at all.

\begin{speechtext}
  What rhymes with snail?
  Mail, sail, bail\ldots hay-bail?
  Are hey-bails a thing?
  But `hey' is too informal.
  Better to say `hello'.
  `Hello-bail'\ldots no it sounds non-committal.
\end{speechtext}

So he stands looking at a blank sheet of paper, while twenty goblins ignore him, and bicker about pick-axes and the proper way to use a cart.

\begin{boxtext}
  In the near-distance, around this corner (or possibly two), someone, or something, is hitting metal on rocks.
  The metal sounds strange, butt probably iron.
  The rocks give that satisfying crack that rocks give with a long, clean cut.
\end{boxtext}

\paragraph{If the \glspl{pc} ask about the poem's recipient,}
\gls{romeo} explains he has no idea whom he loves, so he can only describe their mind.

\begin{speechtext}
  A quick wit, and very insightful in material matters -- able to tell the weight of a stone, bird, or an entire tree just by looking at it.
  And a deep critical thinker, not in any malicious sense, but nevertheless with cutting questions, whenever the need arises.
  This someone has wisdom beyond their years, though I don't know how old they might be, but still I'm sure of it\ldots

  I read their writing, and learned so much.
  They taught me how to move, and how things move.
  We write back and forth, we know each other so well.

  \ldots and yet, I cannot describe a face.
  But what's a face?
  Who cares?
  I just want to explain how I feel, and marriage to seal the deal.

  `Seal the deal'\\
  `An oath would make me less morose\ldots'

\end{speechtext}

\gls{romeo} does not know whom it's for, and explains he learned from his teacher by reading, and fell in love utterly.
His father doesn't approve of the `oathless' types, and he feels ashamed of loving such an `air-headed' person, despite all she's taught him.

\paragraph{If the \glspl{pc} help him with the poem,}
then he perks up and quickly finishes it, then asks them if they might try to find the recipient.

\paragraph{If the \glspl{pc} do nothing,}
\gls{romeo} remains at the quarry, thinking of the perfect words.

\paragraph{If the troupe commit crimes,}
the goblins will ignore them as long as they can.
They have taken an oath to dig rocks, and they will continue to dig until something shakes them from their oath.

\romeo

\showStdSpells

\paragraph{As the troupe leave,}
they notice \gls{romeo} using the Force \gls{sphere} to make the goblins' rocks lighter.%
\footnote{This tells the \glspl{pc} that \gls{romeo} understands the Force \gls{sphere}, which indicates a link to \gls{juliet}.
This becomes important later, in \nameref{oathlessLovers}, \vpageref{oathlessLovers}.}

\sqpart{sunway}% AREA
{Elven Steps}% NAME
{A hidden path leads to \glsfmttext{plateauGardens} above}% SUMMARY
\label{hiddenStairs}

In the causeway between the \gls{ravencops} forest and the \gls{plateauGardens}, a single plateau has a hidden stairway, going up.
\Glspl{crawler} cannot make much use of the narrow stairs, with occasional hand-holds for little fingers.
People who don't know about the stairs cannot usually see them, as every step blends into the tall rock-face from below.
But once someone notices the first step, they see the next, and then the next, and so on.

\paragraph{Spotting the rocks}
requires a \roll{Wits}{Vigilance} roll at \tn[12].
The \gls{tn} increases by~+1 in the rain, and by~+3 at night.

\sqpart{ravencops}% AREA
{The Icebox House}% NAME
{Underground elves live to guard food packed in ice}% SUMMARY
\label{iceboxHouse}

\histEvent{130}{3}{%
  With the old lich killed, \glsfmttext{MindElder} decided to settle down, build the perfect tower, and raise perfect children in a perfect land.
  Unfortunately, the children stole, fought, and disobeyed his orders to stay at home and keep safe.
  He made them all swear oaths to uphold the law, never harming any elf, nor taking property, nor singing out of key.
  Without the ability to sing out of key, none have learnt to sing, but this turned out to be an improvement, as \glsfmttext{MindElder} always enjoyed silence more than song%
}

Thick, glass tiles, a full step wide, pepper the land; these tiles are the roof-windows of elven houses.%
\exRef{stories}{Stories}{elvenGlades}
Smoke rises from a chimney, which juts out through a tree.
Three tall trees surround and hide a stone stairway, leading down to a little door.
Three short taps permits entry.

\begin{boxtext}
  One elf takes water, and whispers gently until the water sleeps, and turns to ice.
  Another prepares a little food, using a rapier's broken-off tip as a knife.
  The rest of the rapier remains mounted on the wall, above the fire; but these elves have no use for weapons.
  Harming people causes pain, and they have promised not to harm anyone.
\end{boxtext}

A little goblin sleeps in a hammock, muttering in his sleep.
`\textit{Carapace pies, tentacle-fry\ldots}'
The elves will have something cooked for him by the time he wakes, and then he must fetch more water, using the ornate bucket, carved from carapace, with an abstract map of the land chiselled around its side.

\elf

\paragraph{If the \glspl{pc} ignore the smoking chimney,}
that's fine.
This \gls{segment} does not advance any plot, and there is nothing the \glspl{pc} need to do.
This \gls{segment} exists simply because elves live in \gls{enchantedLands}, and they will greet guests who knock on their door in a friendly manner, and start telling long, boring stories.

\paragraph{If the \glspl{pc} inspect the bucket-map}
they receive a +2~Bonus to all \gls{navigation} checks within the surrounding \glspl{area}.

\paragraph{If they ask for help,}
they receive it, as long as they make small, reasonable requests.

\paragraph{If they ask questions,}
the elves answer happily.
They know most of the history of the area (find the summary \vpageref{chronologicalEvents}).

\sqpart[\squash]{ravencops}% AREA
{Subtle Sepulchres}% NAME
{Most goblins have forgotten about these \glsfmtplural{ogre}}% SUMMARY

\Gls{MindElder} ordered this sepulchre made before he understood how easily the dreams of \glspl{ogre} break.
It houses three, who snore quietly.

\sqpart{ravencops}% AREA
{The House of Grand Stories}% NAME
{Underground elves tell painless stories in perfect rhythm}% SUMMARY

Beehives buzz around a flowery garden.
Three large boulders (which look quite out of place) hide stairs down to a long hall, where perfect elves tell perfect stories of perfect people.
The stories rhyme in a precise pattern, and follow the hero's journey exactly.
The characters in the stories do no wrong, and have no fights, because fighting hurts people, and the storytellers never think about hurting people.

The elven home has three chambers, for three elves, and a central area for cooking.
Goblins occasionally visit, bringing supplies of snail-meat and stolen vegetables.
The various cupboards also have $1D6-3$ of the following items:

\begin{itemize}
  \item
  Smoked meats (usable as a day's \glspl{ration}).
  \item
  Stormy moonlight from a storm, captured in a large, glass, phial (usable as a Water \gls{ingredient}).
  \item
  Auroch hooves (usable as an Earth \gls{ingredient}).
\end{itemize}

\elf

\end{multicols}

\stopcontents[sq]


\commentary{
  \begin{description}
  \item[\gls{gm}:]
  Some miles on, and the Sun's high.
  You keep walking with that `squelch, squelch, squelch' along the snail-path, and spot a potato on the path ahead, held in someone's arms.
  \item[Player 3:]
  (Cleftbarb)
  Hide!
  \item[\gls{gm}:]
  That's \roll{Wits}{Stealth}, \glsentrylong{tn}~6.
  \item[Player 2:]
  (Mildrain)
  Scared of a potato-wielding marauder?
  Okay then\ldots\dicef{6}.
  \item[\gls{gm}:]
  Hiding in the thick darkness which surrounds the road, you wait until the small figure passes, grumbling to himself in the \gls{tradeTongue} about `the sleeping ones'.
  Once silence returns, the march continues, through hours of nearly-identical woodland.
  By the time you approach the tower, it's nearly night, but the dusk's light highlights a stone structure in the forest.
  A long cube, a rumbling stone box.
  \item[Player 2:]
  (Mildrain)
  Ignore!
  \item[Player 1:]
  (Sinkmaul)
  Right, `tower'.
  \item[Player 3:]
  What if it's treasure?
  \item[\gls{gm}:]
  The road opens, revealing a shining lake, with the tower in the centre\ldots
  \end{description}
}{
  The pacing sped up suddenly as the players hopped through three \glspl{segment} within a couple of minutes.
  The players still received information (giant snails reduce the \gls{monster} population), gained questions (though nobody has actively asked about the source of the giant potato), and have a new location on the map which they might return to.

  That last \gls{segment} with the `stone box' (\gls{sepulchre}) has the `\gls{vlg}' symbol listed next to it.
  This indicates it should go onto the map, with the assumption it's always been there.
}

\section{\Glsfmttext{oathtower}}
\label{oathtower}

\Gls{oathtower} stands four storeys tall in the centre of a shining lake.
People can see the tower's top for miles around wherever the tree coverage is not too thick.

Whenever the \glspl{pc} approach, \pgls{segment} activates; they do not have to \emph{enter} \gls{oathtower}, most \glspl{segment} describe events in the nearby forest, or at the edge of the lake.

Nobody near \gls{oathtower} can breathe a wisp of mana, because mana always heads towards the largest vacuum.
And since \gls{MindElder} spends so many \glsentrylongpl{mp} each day, he usually has the most missing by the end; so everything flocks towards him, leaving the air feeling stagnant even in a storm.

\printSideQuestsInRegion{oathtower}

\begin{multicols}{2}

\subsection{The Lake}

\noindent
Whistling cane grows around the South side of the lake, giving it an eerie sound whenever the wind blows.%
\exRef{judgement}{Judgement}{whistlingCane}
The elves use it to make paper, while the hobgoblins use it as an instrument.

\begin{boxtext}
  The \gls{oathtower}'s tall, wooden, door has an identical shade of grim-brown to its stone walls.
  The tower stands a stone's throw away, at the centre of a shining lake.
  A single, short, figure in a green tunic stands in a boat by the tower.
  Her ears are long, her skin maggoty-white, and her eyes full of suspicion.
\end{boxtext}

\subsubsection{Parley with the Boat Goblin}
is the only means of communication with \gls{oathtower}, and takes \pgls{interval}.
She will row across, listen to the \glspl{pc}, and then respond to \emph{every} question with ``\textit{I dunno\ldots I'll ask}''; then she rows back to \gls{oathtower}, walks up the stairs (the \glspl{pc} see her passing by a window), asks \gls{MindElder}, then returns with the reply.

\paragraph{If the \glspl{pc} want information}
then \gls{MindElder}, or the elves in room \vref{towerKitchen} furnish Ha\^{c}ek with replies.
The replies will be accurate, but curt, with no additional information.

\paragraph{If the \glspl{pc} ask for anything more}
(such as weapons' licences or entry to \gls{oathtower})
\Gls{MindElder} stands on his balcony (noted \vpageref{MindElderRoom}) and calls down to them, asking them about their own intentions, and asks them to promise to obey the local laws.
The \glspl{pc} should suspect his request has cast \pgls{spell} on them, and you should ask them to roll \roll{Wits}{Academics} at \tn[13].
However, a successful roll lets the character know that \gls{MindElder} has cast nothing at all (he's nearly out of \glspl{mp}, and too busy to bother with the \glspl{pc}).

\enchantedGoblin[\npc{\F\N}{Ha\^{c}ek the Boat Goblin}]

\paragraph{If the \glspl{pc} break the local laws here,}
\gls{MindElder} will step onto his balcony, and begin casting offensive spells.

\subsection{Stony Spiral}

The tall structure's stone floor, wooden doors, pitch torches, and loitering hobgoblins give the false impression of a classic `dungeon'.
However, everyone in \gls{enchantedLands} keeps their promises, so they have no need for currency, or locks.
Every door opens at a push, but the only chests contain bottles of brandy or hobgoblin underwear, waiting to be washed.

\subsubsection{The entrance door}
opens into a hall with a staircase and three doors.
If the troupe enter, Majuscule the hobgoblin asks them to place their weapons in room \vref{towerWeaponStorage}.

\begin{description}
  \item[The ground floor]
  has a grand hall with two doors and a spiral staircase.

  In the centre of the hall, $1D6$ hobgoblins sit with a deck of cards, debating the importance of rules in games.
  The hobgoblin guards have \gls{armour} made from \pgls{basilisk}'s hide \gls{covering} their torso, and a mottled-brown helmet made from the shell of a giant snail.
  \begin{enumerate}
    \item
    \textbf{Banging noises} echo from $1D6$ hobgoblins mending a boat for the lake around \glsfmttext{oathtower}.
    \item
    \textbf{Loud snoring} emanates from a sleeping hobgoblin.
    The room has a dozen hammocks made from cured snail-skin.
    \item
    Hobgoblin weapon storage, with $1D6 \times 2$ shortswords mounted on the walls.
    \label{towerWeaponStorage}
  \end{enumerate}
  \item[The first floor]\label{towerKitchen}
  has a massive kitchen with two storage rooms at opposite sides.
  $1D6$ elves debate politely, but their words hide fierce insinuations about how the others once burnt an egg, or used the wrong type of wood for cooking snail meat.
  \begin{enumerate}
    \item
    Light supplies, with $1D6 \times 3$ torches, and $1D6 \times 4$ candles.
    \item
    Meat storage, with $1D6$ meals' of \gls{crawler} eggs in salt, $1D6$ meals of snail-meat, $1D6$ meals' worth of vegetables (half of them gigantic and stolen by goblins from the garden plateaus), and $1D6$ meals of auroch meat.
  \end{enumerate}
  \item[The second floor]
  has an open room, full of cushions, with $1D6-3$ elves lounging.
  These elves have taken so many oaths that they are practically incapable of action, and need goblins to tend to them daily.

  The spiral staircase ends here.
  One door and two entrances sit behind the elves -- both open but gloomy.
  \begin{enumerate}
    \item
    Behind the door, an in-house outhouse which empties into the lake below.
    \item
    The first alcove houses a small study, with books on poetry, and fantastical erotica where rabbits ride foxes to battle snakes.

    The bookcase serves as a ladder to room \ref{MindElderRoom}, above.
    \item
    A cupboard of \glspl{ingredient}, including $1D6-3$ \gls{woodspy} beaks (Water \glspl{ingredient}), $1D6-3$ \gls{crawler} spinnerets (Fate \glspl{ingredient}), and $1D6-3$ phials of human blood (also Fate \glspl{ingredient}).

    The cupboard lifts back to reveal stairs up to room \ref{RomeoRoom}, above.
  \end{enumerate}
  \item[The third floor]
  has two rooms, accessible only from hidden entrances, below.
  \begin{enumerate}
    \item
    \Gls{MindElder}'s bedroom has a balcony, a bed with white sheets, and nothing else.\label{MindElderRoom}
    \item
    \Gls{romeo}'s bedroom has three books on poetry placed randomly around the room; each of them discuss the importance of a Mathematical underpinning.\label{RomeoRoom}
  \end{enumerate}
\end{description}

\enchantedHobgoblin[\npc{\T[3]\N}{Hobgoblins}]

\paragraph{Names:}
Glyph, Sillabery, and Majiscule.

\enchantedHobgoblin[\npc{\T[3]\N}{Hobgoblins}]

\subsubsection{An Interview with \glsfmtname{MindElder}}
will not be easy, because \Gls{MindElder} does not trust unknown travellers who carry weapons.
If the \glspl{pc} give him a good reason to speak, he speaks with one alone, in \gls{oathtower}, after they have left any weapons in with the hobgoblins on the ground floor, in room \vref{towerWeaponStorage}.

\begin{itemize}
  \item\it
  What's your name?
  \item[\adforn{54}]\bf
  `\Glsfmtname{MindElder}'.
  How can I help?
  \item\it
  Can you stop forcing people into oaths?
  \item[\adforn{54}]\bf
  Which oaths in particular?
  Do you want people to break their promises?
  Are you here to attack the peaceful people here in \gls{ravencops}?
  \item\it
  No, we won't attack anyone, but\ldots
  \item[\adforn{54}]\bf
  I thank you for your oath of peace.
  \item\it
  That wasn't an oath!
  \item[\adforn{54}]\bf
  In fact, it was.
  \item\it
  People should be able to sing, even if it's not in key!
  \item[\adforn{54}]\bf
  If you enjoy that sort of thing, do it elsewhere.
  I don't want to hear any more tuneless goblin-songs.
  \item\it
  Can you stop making snails attack people?
  \item[\adforn{54}]\bf
  Aggressive snails eat \glspl{crawler}, and even attack \glspl{basilisk}.
  They make the land a lot safer on average.
  \item\it
  Well we're going to stop the snails being made in the first place!
  \item[\adforn{54}]\bf
  That saddens me, but I shall not stop you.
  I don't own the snails until they enter my land -- they come from the West, where lawless elves live.
  What will you do after you leave my tower?
  \item\it
  I suppose we'll go West, and then\ldots
  \item[\adforn{54}]\bf
  I accept your oath to travel West, and wish you well on the journey.
  Travel well!
  \item\it
  Thanks!
  I'm really excited to get going!
\end{itemize}

\MindElder

\showStdSpells[
  \input{config/spells/Water4.tex}
  \input{config/spells/Mind3.tex}
  \input{config/spells/Mind3.tex}
  \input{config/spells/Mind2.tex}
]

\sidequest[plateauGardens,ravencops,oathtower,sunway]{Oathless Lovers}
\label{oathlessLovers}

\noindent
\Gls{romeo} and \gls{juliet} have never met, but still managed to fall in love through a series of song-spells carved into trees in \gls{sunway}.
Each one spends months composing a new verse, then commits to the dangerous journey to carve more glyphs in their secret, shared spot.%
\footnote{Find that spot \vpageref{sunwayGlyphs}.}

Unfortunately, \gls{romeo} only understands love as an oath to be kept, while \gls{juliet} only understands oaths as an insult and a violation.
If the \glspl{pc} nudge them together, the two soon form a plan to meld the two elven lands into one, with \pgls{spell} so powerful it will destabilize the land, and bring all other threads to a dramatic and confusing conclusion.

\paragraph{If the couple unite,}
they discuss the grand spell, and request the \glspl{pc} help them gather the necessary \glspl{ingredient}.
They need sixteen in total:

%
\null
\begin{itemize}
  \item
  \Repeat{4}{\sqn}\quad 4 Earth \glspl{ingredient}
  \item
  \Repeat{4}{\sqn}\quad 4 Fire \glspl{ingredient}
  \item
  \Repeat{4}{\sqn}\quad 4 Water \glspl{ingredient}
  \item
  \Repeat{4}{\sqn}\quad 4 Fate \glspl{ingredient}
\end{itemize}

\begin{exampletext}
  We want to unite the lands, making all one, a single people and place.
  This \gls{spell} will reframe everyone's problems into a memory.
\end{exampletext}

The couple won't be able to explain the \gls{spell}, except in abstract terms (`united, entirely', `the grand crossing of perspective').
See `\nameref{grandSpell}' \vpageref{grandSpell} for the complete description.

\sqpart[\gls{vlg}]{oathtower}% AREA
{Groaning \Glsfmtplural{sepulchre}}% NAME
{The troupe must walk quietly and avoid the groaning sepulchres}% SUMMARY

\histEvent{20}{2}{%
  \Glsentrytext{MindElder} finds the goblins overpopulating the area, and fears the day they run out of food; even the most powerful enchantments cannot withstand goblin hunger.
  He rewards loyal goblins with obscene amounts of food, which lets them grow and grow, into hobgoblins, and eventually into \glsfmtplural{ogre}.
  Trials include service in \glsfmttext{oathtower}, hunting dangerous creatures, and plenty of duels (which really helps reduce the population).
  Once the goblins ascend, \glsentrytext{MindElder} places them in an enchanted sleep inside a stone sepulchre (where the goblins can check on them)}

\begin{exampletext}
  Goblins have no natural height limits, so when they eat too much, they just grow and grow, until one day, without any clear cut-off point, people call them a `hobgoblin', and soon after, `\gls{ogre}'.

  When the goblins become \glspl{ogre}, \gls{MindElder} puts them into an enchanted sleep, and tells the goblins their big brothers will awaken when the time of grand feasting comes.
  The goblins must see the sleeping \glspl{ogre} from time to time, or they will suspect murder and betrayal, and even their oaths will not keep them passive.

  So the forest around \gls{oathtower} has slowly filled up with snoring sepulchres, and everyone must tread quietly, lest they wake and ask for breakfast\ldots
\end{exampletext}

\begin{boxtext}
  Past the trees, an arrow's flight away, a mossy tower stands as tall as a feasting hall turns on its end.
  A low groaning noise, like a distant earthquake, floods through the trees, surrounding you.
\end{boxtext}

\Gls{oathtower} has a lot of \glspl{sepulchre} dotted around it, often hidden by trees, and always with little paths leading towards them.
Each one has three \glspl{ogre}, cramped in together.
The sound of snoring emanates for a few hours each day.

\iftoggle{verbose}{
  \paragraph{Once the \glspl{pc} approach \gls{oathtower},}
  place \pgls{sepulchre} on the map \vpageref{extracted/enchanted}.
  It should be inside the forest, near the \glspl{pc}' current location.
}{}

\paragraph{High-pitched noises near \gls{oathtower}}
have a 1 in 6 chance of waking \pgls{ogre}.
The chances increase by~1 for loud noises, or noises closer to the \glspl{sepulchre}.

In order to avoid waking the \glspl{ogre}, \gls{MindElder} has told the goblins to slay anyone making high-pitched noises, such as whistling or laughing.
Farting is also banned, as it makes the goblins giggle, which then wakes the \glspl{ogre}.

\enchantedOgre[\NPC{\M\N}{`The Grave'}%
  {slate-coloured skin, with bright-blue eyes}% DESCRIPTION
  {stretches calves}% MANNERISM
  {deer with cheese}% WANTS
  \npcQuote{only asking, only asking\ldots}]

\enchantedOgre[\NPC{\F\N}{Kerning}%
  {bra made from human faces (it helps with running, not modesty)}% DESCRIPTION
  {chews leaves, then spits them out}% MANNERISM
  {\gls{crawler} soup}% WANTS
  \npcQuote{the road goes ever on, unless it doesn't.
  `Dead end', they call it}]

\sqpart[\gls{vlg}]{ravencops}% AREA
{Goblins in the Quarry}% NAME
{\Glsfmttext{romeo} should be working, but needs to complete the perfect poem}% SUMMARY
\label{goblinQuarry}

\Gls{MindElder} has sent his son \gls{romeo} to oversee the goblins, excavating rock at the quarry, and cutting long slabs to construct more sepulchres.
But \gls{romeo} can't think of anything but the poem he needs to write, to tell his beloved how he feels.
Unfortunately, his father raised him to be a perfectionist, which means he can't write perfect poetry, or good poetry, or bad poetry, or any poetry at all.

\begin{speechtext}
  What rhymes with snail?
  Mail, sail, bail\ldots hay-bail?
  Are hey-bails a thing?
  But `hey' is too informal.
  Better to say `hello'.
  `Hello-bail'\ldots no it sounds non-committal.
\end{speechtext}

So he stands looking at a blank sheet of paper, while twenty goblins ignore him, and bicker about pick-axes and the proper way to use a cart.

\begin{boxtext}
  In the near-distance, around this corner (or possibly two), someone, or something, is hitting metal on rocks.
  The metal sounds strange, butt probably iron.
  The rocks give that satisfying crack that rocks give with a long, clean cut.
\end{boxtext}

\paragraph{If the \glspl{pc} ask about the poem's recipient,}
\gls{romeo} explains he has no idea whom he loves, so he can only describe their mind.

\begin{speechtext}
  A quick wit, and very insightful in material matters -- able to tell the weight of a stone, bird, or an entire tree just by looking at it.
  And a deep critical thinker, not in any malicious sense, but nevertheless with cutting questions, whenever the need arises.
  This someone has wisdom beyond their years, though I don't know how old they might be, but still I'm sure of it\ldots

  I read their writing, and learned so much.
  They taught me how to move, and how things move.
  We write back and forth, we know each other so well.

  \ldots and yet, I cannot describe a face.
  But what's a face?
  Who cares?
  I just want to explain how I feel, and marriage to seal the deal.

  `Seal the deal'\\
  `An oath would make me less morose\ldots'

\end{speechtext}

\Gls{romeo} does not know whom it's for, and explains he learned from his teacher by reading, and fell in love utterly.
His father doesn't approve of the `oathless' types, and he feels ashamed of loving such a lawless person, despite all she's taught him.

\paragraph{If the \glspl{pc} help him with the poem,}
then he perks up and quickly finishes it, then asks them if they might try to find the recipient.

\paragraph{If the \glspl{pc} do nothing,}
\gls{romeo} remains at the quarry, thinking of the perfect words.

\paragraph{If the troupe commit crimes,}
the goblins will ignore them as long as they can.
They have taken an oath to dig rocks, and they will continue to dig until something shakes them from their oath.

\romeo

\showStdSpells

\paragraph{As the troupe leave,}
they notice \gls{romeo} using the Force \gls{sphere} to make the goblins' rocks lighter.%
\footnote{This tells the \glspl{pc} that \gls{romeo} understands the Force \gls{sphere}, which indicates a link to \gls{juliet}.}

\sqpart[\gls{vlg}]{sunway}% AREA
{Sunlit Glyphs}% NAME
{Elven glyphs, carved in wood, describe a song of ritual magic}% SUMMARY
\label{sunwayGlyphs}

\begin{exampletext}
  \Gls{juliet} began coming here to practice using the Force \gls{sphere} through ritual songs, and carved the notes into local trees to remember them.
  Years later, \gls{romeo} found the patch of glyphs and felt fascinated.
  So little by little, he taught himself the songs, and the rituals of the Force \gls{sphere}.

  And eventually, he began to carve his own glyphs, adding to her songs, or copying them with variation.
\end{exampletext}

\histEvent{45}{4}{%
  \Glsfmttext{juliet} begins carving song-rituals of the Force \gls{sphere} into a patch of trees in \glsfmttext{sunway}%
}

\histEvent{40}{3}{%
  \Glsfmttext{romeo} finds \glsfmttext{juliet}'s song-ritual glyphs, and slowly learns the Force \glsentrytext{sphere} from them%
}

\begin{boxtext}
  The dusky Sunlight makes little rune-shadows across the glyphs carved into all the surrounding trees.
\end{boxtext}

The elves who pass through here don't notice the glyphs because they normally ride giant snails.
The goblins who pass through here don't notice the glyphs, because they don't care.

\paragraph{Understanding the glyphs}
requires an \roll{Intelligence}{Academics} roll.

\begin{boxtable}

  \textbf{Roll} & \textbf{Result} \\\hline

   6 & These glyphs mean musical notes.  \\

   7 & \ldots and a few words in Elvish.  \\

   8 & Two people carved them.  \\

   9 & The `lyrics' pertain to Fire.  \\

  10 & \ldots and Earth.  \\

  11 & They form a magical ritual.  \\

  12 & Actually, they make many rituals.  \\

  13 & The rituals are of the Force \gls{sphere}.  \\

  14 & The ritual won't work without the missing words.  \\

\end{boxtable}


\sqpart{plateauGardens}% AREA
{\Glsfmttext{juliet}'s Flowers}% NAME
{Once dry, each works as a Force \glsfmttext{talisman}}% SUMMARY

A dozen elves spot clouds, describing what the clouds look like, and telling stories about the exploits of the `cloud-willow', and the `cloud-river'.
They think in terms of moving forests, and complete species, rather than individuals.

\Gls{juliet} takes no interest, as she is busy drying her flowers by pressing them onto a rock, then hanging them from a bush.
These are the `flowers of enlightenment', which she grows to make things float.
Check the details \vpageref{flowerOfEnlightenment}.

\juliet

\paragraph{If the \glspl{pc} look tired,}
\gls{juliet} hands them a flower or two, and just says `\textit{flower of enlightenment, for the weight of the world}' (she doesn't speak the \gls{tradeTongue} well, but anyone who speaks Elvish can clarify).

\paragraph{If the \glspl{pc} identify \gls{juliet}}
(and potentially deliver a poem, written \vpageref{goblinQuarry}) she asks them to deliver a message back to \gls{romeo} (although she does not know his name).

\begin{speechtext}
  You met the author?
  Can you deliver a message, saying to meet where we write?

  What eye-colour do they have?
\end{speechtext}

\sqpart{oathtower}% AREA
{Wherefore Not?}% NAME
{\Glsfmttext{MindElder} tells \glsentrytext{romeo}, `there is no love without an oath'}% SUMMARY

\Gls{MindElder} has no respect for anyone's privacy, and has been \gls{casting} \textit{Witness Mind} to figure out why his son, \gls{romeo}, seems so detached lately.
Once he discovered that \gls{romeo} has become infatuated with an unknown elf from \gls{plateauGardens}, he felt enraged, and began to argue with \gls{romeo} about the character of the `unruly' elves who live over there.

\begin{boxtext}
  An angry voice echoes from \gls{oathtower}, then another.
  Fast, bitter words in Elvish ring out, then the tower goes silent.
\end{boxtext}

\Gls{romeo} eventually leaves through the tower's front door.
He feels angry, but has accepted his father's idea: ``love means nothing without an oath''.

The \glspl{pc} may try to change his mind, but it won't be easy.

\begin{speechtext}
  Why is she so committed to avoiding an oath of love?

  Why would you refuse an oath of love if you don't plan on violating the oath?

  If I can promise to love her, shouldn't she reciprocate?
  Do I have more duties than she?

  Why does she never wear clothes?
  What's wrong with those people?
\end{speechtext}

\paragraph{If the \glspl{pc} can provide rational answers,}
\gls{romeo} calms down and starts thinking clearly.

\paragraph{If the grand plan has already begun,}
\Gls{romeo} reveals he has collected one \gls{ingredient} of each type required (Earth, Fire, Water, and Fate) from a storage room in \gls{oathtower}.
The plan requires twelve more \glspl{ingredient} in total.

\sqpart{sunway}% AREA
{Insider Knowledge}% NAME
{\Glsfmttext{romeo} gathers \glsfmtplural{ingredient}, while talking about the problems with goblins}% SUMMARY

\Gls{romeo} has found a spot of \glspl{marchingMushroom}, which he puts in his snail-gut satchel.
He has enough for two Earth \glspl{ingredient} so far, and hopes to find more.

\paragraph{If he trusts the troupe,}
he speaks openly to them about the problems in the \gls{enchantedLands}.
Otherwise, he hides.

\begin{speechtext}
  My father has played a dangerous game with all these goblins.
  He controls them only as long as they remain fed, but if they feed, they will breed.
  He sends them to kill dangerous creatures, or bring news from far away, but most survive.
  Every \gls{sepulchre} with sleeping \glspl{ogre} marks another failure to control the population.%
  \footnote{\Gls{romeo} won't say so, but he holds the same view of local humans, and would destroy \gls{coppernut} and everyone inside if they looked threatening.}

  If the goblins suspected that \gls{MindElder} wants to reduce them, they would begin to push back against their oaths, or simply leave the area, past where he can reach them.

  We do not want that.
  We do not want to see the goblins abandoned, in great number, to their own hunger.
\end{speechtext}

\sqpart{sunway}% AREA
{Change of Plan}% NAME
{\Glsfmttext{juliet} has figured out the spell will not work unless the troupe engineer an artificial \gls{flood}}% SUMMARY

\Gls{juliet} enters \gls{sunway}, studying her old notes, and realizes the plan for the great spell to unite everyone's perspectives across the two lands will not work; she needs \pgls{flood}.
The troupe will have to scout out \gls{shadepaths} to find where the water comes from,%
\footnote{Check the river locations \vpagerefrange{shadePool}{shadeDamn}.}
and engineer \pgls{flood}, perhaps with a damn.

How the \glspl{pc} engineer \pgls{flood} depends on them, but \gls{shadepaths}'s high walls, and secret streams allows them more opportunities than most locations.
Whatever their plan, it should be abstracted to \pgls{natural}; and if the roll fails, they will simply have to change their methods and tools to provide bigger bonuses.

\sqpart[\squash]{ravencops}% AREA
{The Goblin Hunting Party}% NAME
{A dozen goblins hunt for \glsfmttext{romeo}}% SUMMARY

\Gls{romeo} left his father's lands, to search among the \gls{plateauGardens} for \gls{juliet}.
\Gls{MindElder} has sent out bands of goblins to find his location, and report back.

Majiscule leads six other goblins through the woods.
Soon they will reach \gls{plateauGardens}.
The goblins speak frankly with the troupe about their mission, and will not deviate from it, even for a moment, unless their lives are in danger.

\paragraph{If the grand plan has begun,}
then \gls{romeo} will have used this time to collect one more \gls{ingredient} of each type required.

\sqpart{sunway}% AREA
{The Grand Spell}% NAME
{The two unite the elves, goblins, \glsfmtplural{pc}, and snails as one}% SUMMARY
\label{grandSpell}

This climactic spell, planned by \gls{romeo} and \gls{juliet} will require 16~\glspl{ingredient} in total, plus \pgls{flood} and \pgls{earthquake}.
The \glspl{pc} will have to engineer the \gls{flood} inside \gls{shadepaths}, where rivers have been artificially diverted.
The \gls{earthquake} will occur naturally, as it always does, at the end of each \gls{cycle}.
This means the plan must wait until the end of \showCycle, so it can only occur at the end of the session.

Once the couple have all they need, they make each one into \pgls{boon}, and begin a grand ritual of song-magic, harmonizing together.

\spell{Shattered Identity in G Minor}% Name
  {Detailed, Distant, Divergent, Duplicated}% Enhancements
  {Warp}% Action
  {Water, Fate, Earth, Fire}% Spheres
  {\roll{Wits}{Survival}}% Resist with
  {The chorus alone is twenty minutes long, but once the spell really gets going, it binds everyone and everything at \spellRange\ into a self-jealous, psychic lump of non-space.}% Description
  {Everyone involved drops pieces of their memories, desires, and their perception of time into the fractured space, and each space involved loops together.

  Everyone involved gains a +\arabic{spellPlusOne}~Bonus to Empathy tasks, and a -\arabic{spellPlusOne}~Penalty to all other tasks.
  Both the Bonus and the Penalty reduce by one each time someone travels between one of the connected area, and finds another part of themselves.}

\begin{boxtext}
  Elves from \gls{plateauGardens} look down at you, scared and confused.
  Their usual blank-eyed cool has gone and you remember kissing \pgls{ogre} good night, and hoping to become like him one day.

  \Gls{oathtower}'s stairs lead up to the \gls{plateauGardens}, or you can go into the elven home through the trees.
  Two goblins stand on the cooker holding your memory of learning the Air \gls{sphere}.

    ``\textit{Sorry, wrong person!}'', the little goblin says in Elvish.

    Up the stairs, and out of the elvish house, \gls{oathtower}'s library is full of goblins, all of them are \gls{MindElder}, shouting for everyone to leave his home.
    Then they turn and point at you in unison, and begin to shriek, ``\textit{Identify! Identify}''.

    An eyeball reaches in, to return your name, if you want it\ldots ?
    The other eyeball searches the room upstairs.

    The name feels right, but the snail has another, and that one feels right too.
    The snail wonders how you know which name is your name, and wonders if you would share.
    The snail is ashamed of its nakedness, so the eyeball slithers back out of the window.

    The goblin hunger was deep, now infectious.
    You could enjoy the taste of your own arm, and it wouldn't actually be \emph{yours}, so it would be fine, and you would get a whole arm to yourself.

    \Gls{SnailTamer} arrives, telling you he loves all of you, and you're all doing really well.
    He's not really there, as he was taking a nap, and just decided to have a little dream with you guys to help everyone out.
\end{boxtext}

The spell ends before long, or it should if the players aren't into Dada-Taoism.
Or if they're getting the vibe, continue handing out memories until they've pieced their characters together.

After the experience ends, \gls{LifeElder} wanders past, stops to observe a spider-web, then continues.

\paragraph{One the spell wears off}
the two groups of elves and the goblins will begin to work together, planning a route forward, each more fully aware of the others points of view.


\end{multicols}


\commentary{
  \begin{description}
  \item[Player 1:]
  Okay, so the elf at the quarry asked us to deliver this letter, and showed he knows some kind of levitation spell, now this elf is growing floating plants.
  Is she the one?
  \end{description}
}{
  The eye of the story moves North, and finds a stone \gls{sepulchre}.
  Then it moves West and finds a quarry, surrounded by giant snails.
  The land seems full of life, because the \glspl{sq} from \autoref{elvenForests} place the living things in front of the players.
}

\section{\Glsfmttext{sunway}}
\label{sunway}

At the border between the two lands the trees grow thin.
Bushes dominate, grasses grow, and small circles of Sunlight form where the canopy breaks.
The \gls{sunway} cuts a line across many miles, providing easy passage for aurochs, who help widen the passage every time they pass, by eating young trees and stamping down everything they can.

Here at last, \pgls{witch} can breathe easily.
\Glspl{mp} regenerate at the normal rate.

\printSideQuestsInRegion{sunway}

\begin{multicols}{2}

\subsection{\Glsfmttext{plateauGardens} Walls}

\begin{boxtext}
  \Gls{sunway} comes to a harsh stop at a wall of perfectly vertical, and strangely solid, earth.
  It stands as tall as \pgls{broch}, and tree-branches sway even higher up, although no roots stick out of the earth anywhere.

  The great wall continues down this Sunlit part of the sparse forest farther than you can see, in both directions.
  But it also has crevices, or cracks, or some darkness dotted along it.
\end{boxtext}

Travelling along the great wall reveals low paths which go between the high places, and into \gls{shadepaths}.

Half a dozen goblins occasionally arrive with ropes, or climbing equipment, to enter \gls{plateauGardens} above and take some food.
Initially, they asked for the food, but \gls{LifeElder} explained that `property is a construct of your mind'.
Some thought the carrots might be illusions, but they still tasted great, and nobody owned them, so the goblins began to take them.

\end{multicols}


\section{\Glsfmttext{shadepaths}}
\label{shadepaths}

Between \glspl{plateauGardens}, great snails wander through shadowy canals.
The earth walls creak like the sound of ripping metal, as if they were about to collapse; but they never do.
Sunlight only falls at noon, so almost nothing grows.

These deep troughs receive few \glspl{mp}, so the atmosphere feels stagnant.
The troupe can only receive 1~\glspl{mp} at the end of each \gls{interval}, or 0~\glspl{mp} if \gls{LifeElder} wanders nearby.

\printThreadsInRegion{shadepaths}

\histEvent{100}{5}{%
  To stop the giant snails eating all the vegetables, \gls{LifeElder} cracked the land, sundering the soil and creating raised plateaus, where she and the other elves could live, cultivating plants.
  Meanwhile, the snails remained in the lower regions.
  Unfortunately, the elves could not get from one plateau to the other due the tall, sheer walls%
}



\begin{multicols}{2}
(Some Broch) A Logistical Discussion
-----
{A monster approaches a distant bailey, will the troupe make it?}

Over the sea of dense green, something strange and monstrous moves, taller than the canopy, approaching the bailey below.

According to the \gls{jotter}:

> You are the night guard.  That is a beast.  Go guard!

But according to the archer:

> The bailey lies two miles away, which requires half an hour at a good march.  The beast looks to be two miles beyond the bailey, so it will arrive long before us, and when we arrive, the situation will already have ended...one way or another.

And according to the ranger:

> You can run two miles in ten minutes, with gear.  Beasts always take it slow, so we can beat it; but best not.  We don't know what it is.

In truth, SnailTamer has lost control of his mount, 'Nettlerash', who recently acquired a taste for living flesh in the Kingdom of Oaths.

If the troupe watch from the safety of the broch,
they will intermittently see the eyestalks rise again above the canopy, and soon notice how slow the thing is (but will not be able to identify it from the two bulbous tentacles) poking into view for those small moments.

If the troupe warn the bailey,
the farmers let them in, then cover the wall in archers - 20 in total.

Once the troupe see the snail,
SnailTamer call down to them, with his standard sloth and naïveté.

> The wind dies down, leaving a distant crashing noise which becomes louder, and sounds like crushed bushes and snapped branches, mixed with  retching.  A rock-like surface emerges, the size of a cottage, with a smooth, brown neck hovering above, which tapers to a point above with two tentacles pointed straight in the air.  An elf sits on the rock-like surface above, waving slowly.

If the snail sees the troupe (or anyone) it begins projectile-vomiting acid while SnailTamer - the elf on the snail's enormous shell - begins to explain himself:

> (round 1) So, hey, um...everyone there.  Can you hear me?

> (round 2) That's good, good that you can hear me, because I really want to ask you if you might not hurt Nettlerash here.  She's a good girl usually.

> (round 3) Actually, I named Nettles after a human, or more like, humans in general, because, as you can see, she is very tall, like a human.  Are you sure you need to do that with the sword?  I think it's upsetting her.

> (round 4) So, it all started a while ago, we went a little too close to the tower, and...wait, let me back up a bit.  It's actually important that you understand some of the earlier events (I suppose you might call these events 'history', or is that prejudiced?  Sorry, I never actually met humans, unless you count dwarves...).

> (round 5) Do you count dwarves?  I mean no judgement, I just wondered, because...wait, Nettlerash really is not looking good.  Okay, this is serious, we really need to discuss the use of swords and how to respect differences, and...

If the \glspl{pc} attempt a peaceful resolution,
they can manipulate the snail easily.

If Nettlerash dies,
SnailTamer leaves, saddened, and will not speak with the troupe until he collects his thoughts.

Once the combat dies down,
the \gls{jotter} appears, and asks the troupe what happened to the ranger.
Someone needs to find the source of that giant snail, and make sure no more will be coming this way.

If the troupe ask SnailTamer about his plans,
he says he has a letter for LifeQueen, from MindElder, and must deliver it at once.

> I have to go.  I have this letter for someone.

> Who is it?  She's just a person, like anyone, but she's not into labels.

> I don't call her anything.  Well, maybe I call her 'hi'.

(Canals) Debating Parsnips
-----
{SnailTamer meanders home, wondering if he should make parsnip or potato soup}

SnailTamer has missed a turn-off in the maze of canyons, and can't remember which direction he's going.
However, he's more concerned with the question of what kind of soup to make.

>>>
A voice in the distance sounds like it's practising Elvish vocabulary, and you recognize the word for 'parsnip'.
Around the corner, SnailTamer appears, as naked as the Sun, then stops and stares, as if trying to recognize you.
>>>

> So, like.  Guys?  Yea, you guys.  Okay then.  So we don't make people do things here, no laws about fabrics over your body - you just walk about however you like, okay?

> Potatoes are always great, but you can't eat potatoes all day.  It's been ages since I had parsnip soup actually.  But the parsnips are fresher...I don't want the potatoes to go bad.  That would be a waste.

SnailTamer continues meandering and mumbling until someone snaps him out of it.
If \pgls{pc} makes a clear request help getting out of the canyon and onto the plateau, he agrees, says '*let's go then*', and starts walking while whistling a sad song.
Soon enough, an elf in one of the raised gardens will hear him, and lower a few ripe bean-vines.

The bean-vines can take a total Weight of 10.
The \glspl{pc} can gauge the weight limit with an Intelligence + Survival roll (TN 8).

(Gardens) Shell Escape
-----
{Giant potato sprouts vomit from SnailTamer's house, it may fall any moment!}

If you leave potatoes in a bag with a little Sunlight, they will grow arms and try to find earth to implant themselves into.
Giant, magical, elf-potatoes are entirely normal in this regard, but much bigger.

>>>
Across the next bean-vine bridge, a giant snail-shell stands, dangerously close to falling into the canyon.
Sunlight shines through the shell, displaying an upper floor with a desk, hammock, some cloaks, and jars.

Pale tendrils push out the door, then have a hard twist down into the earth, like a limb trying to grasp the ground.
>>>

- Get an item from the house: Dexterity + Stealth, TN 12 (+1 per item)
- Righting the house, so it will not fall over: Strength + Cultivation, TN 12.
- Accepting the situation, securing the potato inside the house, and directing the vines to keep the house safe: Intelligence + Cultivation, TN 10.


(Gardens) Choosing a Shell
-----
{SnailTamer to pick a new snail}

SnailTamer needs to pick a new snail so he can get about safely, but he can't decide which snail is best.
One seems to unbalanced for him to put the wooden carrot-rod onto, another looks 'too old', then the next is pregnant (which will cause problems before long).

If SnailTamer still suffers from the dignome stings,
the \glspl{pc} can usher him along with a Charisma + Cultivation roll (TN 10).
Otherwise, he'll pick a snail within the Interval.

(Causeway) Urgent Delivery
-----
{SnailTamer races to deliver latter on back of snail}

LifeQueen was bound to send 'her most reliable messenger', and that's SnailTamer.
He's not the fastest, or well-spoken, nor is he routinely sure of where he is.
However, he has delivered two messages, which means his success rate is $2/2$, and 100\%.

So she charged him with delivering the letter to the MindElder, and he's finally en route.

(Oath Tower) The Letter of the Law
-----
{SnailTamer wants someone else to deliver the letter, so he can avoid any binding enchantments}

At the tower, SnailTamer has changed his mind about entering, and decided to hide behind his new mount.
This plan would have worked well, except for the large wooden pole mounted on the snail's back, indicating that it has a rider.

If the \glspl{pc} agree to deliver the letter,
SnailTamer hands his satchel, which feels dank and smells musty, as it's full of moss.
Once opened, the moss is in the shape of the letter 'F' (which is the letter LifeQueen wanted to send).


\end{multicols}

\section{\Glsfmttext{plateauGardens}}
\label{plateauGardens}

\Gls{LifeElder} casts spells every \gls{interval}, so she takes all the \glspl{mp} in the area before anyone else has a chance.
Normally, people in the \gls{plateauGardens} who need 9~\glspl{mp} can regenerate 1~\gls{mp} each \gls{interval}.
Those will a vacuum of less than 9~\glspl{mp} receive nothing.

However, when \gls{LifeElder} is nearby, the troupe will regenerate nothing, unless they have a larger vacuum than her.
This allows the \glspl{pc} a chance to track \gls{LifeElder}, as they will know when she wanders nearby (although guessing where exactly will not be easy).

\printSideQuestsInRegion{plateauGardens}

\begin{multicols}{2}

\sidequest[plateauGardens,shadepaths]{Places among \Glsfmtplural{plateauGardens}}

% Note roads.
\histEvent{100}{5}{%
  To fix the massive snails getting stuck, she gave them acidic vomit, so they could dissolve bushes and trees, burn through \glsentrytext{crawler} webs, and in general move freely.
  Unfortunately, they ate all her vegetable patches%
}

% Bean Vine Bridges
\histEvent{95}{5}{%
  \Glsfmttext{LifeElder} did not like seeing the elves trapped on different plateaus, like some kind of jail.
  She solved the problem by enchanting bean-vines to bridge nearby spaces between the plateaus, creating actual bridges%
}

Each \gls{segment} in this \gls{sq} shows an area within \gls{plateauGardens}, and the \gls{sq} `continues' as long as the troupe remain in the \gls{area}.
You should note each piece on the map, so that when the troupe return, they find the same place again.

\sqpart{plateauGardens}% AREA
{\Glsfmttext{disgnome} Thickets}% NAME
{The mind-rending yellow thickets hides within the tomatoes}% SUMMARY

A tiny prick from the needle-tips can slow someone's mental capacity, making them feel like everyone around them is babbling fast-paced nonsense, and makes the day seem to last an hour.
A number of the elves have been affected, but they blame their headaches and confusion on enchantments or don't even consider the cause.

The yellow, spiky plants hide among a bed of tomato plants, ready to prick anyone who wanders by.

\paragraph{If the troupe destroy the \glspl{disgnome},}
\gls{SnailTamer} soon returns to his natural human-paced thinking (about the same as the average human).

\sqpart{shadepaths}% AREA
{The Watering Hole}% NAME
{Clear pool now an undrinkable snail-bath}% SUMMARY

Little rivers gather into a little pool.
The shallow basin stretches only twenty steps wide, just enough for a few snails at a time.
They slither around and drink, replenishing their slime.

This leaves the pool filthy, and the elves don't like the grime.
Carrots, on the other hand, love the water's brine.

The troupe see a thin trickle of water running through their mossy path.
Following it leads to the watering hole.

\paragraph{Each time the troupe arrive at the watering hole,}
they find 1D6-2 giant snails bathing.

\sqpart{plateauGardens}% AREA
{Enlightenment}% NAME
{The flowers of the garden of light make you float}% SUMMARY

\histEvent{50}{1}{%
  \Glsfmttext{juliet} became bored of trying to manipulate bodies, and focussed herself on the Force \glsfmttext{sphere}%
}

\Gls{juliet} has cultivated these flowers using the Force Sphere.
The flowers of enlightenment make you light, and let you float; but first they must wilt and go brown.
The decaying process requires two \glspl{interval} of Sunlight, after which someone can eat the plants.
Doing so reduces their \gls{weight} by~5.

\begin{boxtext}
  A bed of bright-red flowers, with long petals like the floppy ears on a dog.
  It looks like someone made space for them, and spent a lot of time on their soil bed.
\end{boxtext}

\sqpart{plateauGardens}% AREA
{Purple, Yellow Beds}% NAME
{More mind-rending plants hide by a carrot-patch}% SUMMARY

Long, green plants spring up, indicating massive carrots below.
The weak elves find pulling them up to be very difficult, and only do so in groups, or by speaking sweetly to the earth, and asking it to let the carrot go.

\Glspl{disgnome} hide around the side of the carrot beds.
Anyone passing through must roll to notice them, or suffer the usual consequences.

\sqpart{shadepaths}% AREA
{Guardian Stones}% NAME
{Last hope of the elves: hidden lake uncovered behind seeping-wet wall}% SUMMARY

\histEvent{40}{5}{%
  \Glsfmttext{LifeElder} walled off the last clean lake in \glsfmttext{plateauGardens} to stop the giant snails infecting it%
}

With little clean water left, \gls{LifeElder} guarded the last pool of water by summoning stony walls around it.
Water escapes through little holes at the base, which will give the characters a clue about this hidden lake.

\begin{boxtext}
  A shining, tiny, rivulet meanders through the barren, dry canal.
  It smells and tastes fresh!
\end{boxtext}

Nobody can see the lake from the outside.
Trees in the plateau gardens merge seamlessly with trees around the lake.
From a distance, it all looks like a continuous canopy.

If the elves cannot use this lake -- due to snail-access, or poisoning, or some other catastrophe -- they will find themselves without a good source of water, and \gls{LifeElder} will have to stop supporting the snails.

\sqpart{plateauGardens}% AREA
{The Great Snail Lake}% NAME
{Lake spotted from a garden plateau}% SUMMARY

The canyon widens here, and a barren, slimy land (stripped bare by giant snails) holds a great lake in the centre.
It stretches as far as an arrow's flight, and glistens with a thick film of slime across most of the surface.

Garden plateaus surround the lake, and each one holds a narrow staircase down.
The crack in the plateaus where the stairs descend is very narrow.
Characters with Strength~+1 can only enter the staircase by removing all armour and squeezing through.
Anyone with a higher Strength Bonus cannot enter.

Each time the troupe arrive at the lake,
they see 2D6-2 giant snails bathing, and 1D6-3 elves collecting water.

The elves purify the water with spells when they can, but this requires \glspl{mp}, which are in scarce supply in the area.

\end{multicols}


\chapter{Loose Threads}
\epigraph{
  Can you name the nameless one?

  Can you shoe a snail?
}

\label{looseThreads}


\section{Meandering Tails}

\begin{multicols}{2}

\sidequest[plateauGardens,ravencops,oathtower,sunway]{Oathless Lovers}
\label{oathlessLovers}

\noindent
\Gls{romeo} and \gls{juliet} have never met, but still managed to fall in love through a series of song-spells carved into trees in \gls{sunway}.
Each one spends months composing a new verse, then commits to the dangerous journey to carve more glyphs in their secret, shared spot.%
\footnote{Find that spot \vpageref{sunwayGlyphs}.}

Unfortunately, \gls{romeo} only understands love as an oath to be kept, while \gls{juliet} only understands oaths as an insult and a violation.
If the \glspl{pc} nudge them together, the two soon form a plan to meld the two elven lands into one, with \pgls{spell} so powerful it will destabilize the land, and bring all other threads to a dramatic and confusing conclusion.

\paragraph{If the couple unite,}
they discuss the grand spell, and request the \glspl{pc} help them gather the necessary \glspl{ingredient}.
They need sixteen in total:

%
\null
\begin{itemize}
  \item
  \Repeat{4}{\sqn}\quad 4 Earth \glspl{ingredient}
  \item
  \Repeat{4}{\sqn}\quad 4 Fire \glspl{ingredient}
  \item
  \Repeat{4}{\sqn}\quad 4 Water \glspl{ingredient}
  \item
  \Repeat{4}{\sqn}\quad 4 Fate \glspl{ingredient}
\end{itemize}

\begin{exampletext}
  We want to unite the lands, making all one, a single people and place.
  This \gls{spell} will reframe everyone's problems into a memory.
\end{exampletext}

The couple won't be able to explain the \gls{spell}, except in abstract terms (`united, entirely', `the grand crossing of perspective').
See `\nameref{grandSpell}' \vpageref{grandSpell} for the complete description.

\sqpart{plateauGardens}% AREA
{\Glsfmttext{juliet}'s Flowers}% NAME
{Once dry, each works as a Force \glsfmttext{talisman}}% SUMMARY

A dozen elves spot clouds, describing what the clouds look like, and telling stories about the exploits of the `cloud-willow', and the `cloud-river'.
They think in terms of moving forests, and complete species, rather than individuals.

\Gls{juliet} takes no interest, as she is busy drying her flowers by pressing them onto a rock, then hanging them from a bush.
These are the `flowers of enlightenment', which she grows to make things float.
Check the details \vpageref{flowerOfEnlightenment}.

\juliet

\paragraph{If the \glspl{pc} look tired,}
\gls{juliet} hands them a flower or two, and just says `\textit{flower of enlightenment, for the weight of the world}' (she doesn't speak the \gls{tradeTongue} well, but anyone who speaks Elvish can clarify).

\paragraph{If the \glspl{pc} identify \gls{juliet}}
(and potentially deliver a poem, written \vpageref{goblinQuarry}) she asks them to deliver a message back to \gls{romeo} (although she does not know his name).

\begin{speechtext}
  You met the author?
  Can you deliver a message back for me?

  Tell them to meet at our special place.
  They'll know what that means.

  What eye-colour do they have?
  Are they a man or woman?
\end{speechtext}

\paragraph{If the troupe mention that \gls{romeo} lives in \gls{ravencops},}
she curls her nose in disgust, and seems sad, but perseveres in trying to meet with him.%
\footnote{The two elven groups have incompatible values.}

\sqpart{oathtower}% AREA
{Wherefore Not?}% NAME
{\Glsfmttext{MindElder} tells \glsentrytext{romeo}, `there is no love without an oath'}% SUMMARY

\Gls{MindElder} has no respect for anyone's privacy, and has been \gls{casting} \textit{Witness Mind} to figure out why his son, \gls{romeo}, seems so detached lately.
Once he discovered that \gls{romeo} has become infatuated with an unknown elf from \gls{plateauGardens}, he felt enraged, and began to argue with \gls{romeo} about the character of the `unruly' elves who live over there.

\begin{boxtext}
  An angry voice echoes from \gls{oathtower}, then another.
  Fast, bitter words in Elvish ring out, then the tower goes silent.
\end{boxtext}

\Gls{romeo} eventually leaves through the tower's front door.
He feels angry, but has accepted his father's idea: ``love means nothing without an oath''.

The \glspl{pc} may try to change his mind, but it won't be easy.

\begin{speechtext}
  Why is she so committed to avoiding an oath of love?

  Why would you refuse an oath of love if you don't plan on violating the oath?

  If I can promise to love her, shouldn't she reciprocate?
  Do I have more duties than she?

  Why does she never wear clothes?
  What's wrong with those people?
\end{speechtext}

\paragraph{If the \glspl{pc} can provide rational answers,}
\gls{romeo} calms down and starts thinking clearly.

\paragraph{If the grand plan has already begun,}
\Gls{romeo} reveals he has collected one \gls{ingredient} of each type required (Earth, Fire, Water, and Fate) from a storage room in \gls{oathtower}.
The plan requires twelve more \glspl{ingredient} in total.

\sqpart{sunway}% AREA
{Insider Knowledge}% NAME
{\Glsfmttext{romeo} gathers \glsfmtplural{ingredient}, while talking about the problems with goblins}% SUMMARY

\Gls{romeo} has found a spot of \glspl{marchingMushroom}, which he puts in his snail-gut satchel.
He has enough for two Earth \glspl{ingredient} so far, and hopes to find more.

\paragraph{If he trusts the troupe,}
he speaks openly to them about the problems in the \gls{enchantedLands}.
Otherwise, he hides.

\begin{speechtext}
  My father has played a dangerous game with all these goblins.
  He controls them only as long as they remain fed, but if they feed, they will breed.
  He sends them to kill dangerous creatures, or bring news from far away, but most survive.
  Every \gls{sepulchre} with sleeping \glspl{ogre} marks another failure to control the population.%
  \footnote{\Gls{romeo} won't say so, but he holds the same view of local humans, and would destroy \gls{coppernut} and everyone inside if they looked threatening.}

  If the goblins suspected that \gls{MindElder} wants to reduce them, they would begin to push back against their oaths, or simply leave the area, past where he can reach them.

  We do not want that.
  We do not want to see the goblins abandoned, in great number, to their own hunger.
\end{speechtext}

\sqpart{sunway}% AREA
{Change of Plan}% NAME
{\Glsfmttext{juliet} has figured out the spell will not work unless the troupe engineer an artificial \gls{flood}}% SUMMARY

\Gls{juliet} enters \gls{sunway}, studying her old notes, and realizes the plan for the great spell to unite everyone's perspectives across the two lands will not work; she needs \pgls{flood}.
The troupe will have to scout out \gls{shadepaths} to find where the water comes from,%
\footnote{Check the river locations \vpagerefrange{shadePool}{shadeDamn}.}
and engineer \pgls{flood}, perhaps with a damn.

How the \glspl{pc} engineer \pgls{flood} depends on them, but \gls{shadepaths}'s high walls, and secret streams allows them more opportunities than most locations.
Whatever their plan, it should be abstracted to \pgls{natural}; and if the roll fails, they will simply have to change their methods and tools to provide bigger bonuses.

\sqpart[\squash]{ravencops}% AREA
{The Goblin Hunting Party}% NAME
{A dozen goblins hunt for \glsfmttext{romeo}}% SUMMARY

\Gls{romeo} left his father's lands, to search among the \gls{plateauGardens} for \gls{juliet}.
\Gls{MindElder} has sent out bands of goblins to find his location, and report back.

Majiscule leads six other goblins through the woods.
Soon they will reach \gls{plateauGardens}.
The goblins speak frankly with the troupe about their mission, and will not deviate from it, even for a moment, unless their lives are in danger.

\paragraph{If the grand plan has begun,}
then \gls{romeo} will have used this time to collect one more \gls{ingredient} of each type required.



\sidequest[oathtower,sunway,shadepaths,plateauGardens,ravencops]{The \Glsfmttext{ranger}}

\noindent
Soon after the troupe left their \gls{broch} to find the source of the giant snail, \gls{susjot} sent \gls{dickhead} out on the same mission, alone, because nobody at the \gls{broch} likes \gls{dickhead}.

\sqpart{sunway}% AREA
{An Old Acquaintance}% NAME
{\Glsfmttext{dickhead} arrives to scope out the situation}% SUMMARY

The troupe find him in the woods, searching for \pgls{griffin} nest he thinks lies nearby.
He moves towards them quietly, hoping to get the jump on them, just to show off his superior stealth \glspl{skill}.

Once out, \gls{dickhead} speaks haughtily of his ability to survive in the forest, and moves with confidence.
He asks the troupe what they've seen, but does not give their stories much importance.

\begin{speechtext}
  So you still have not found the heart of the problem.
  Well keep searching!
  You may not succeed, but it makes for good practice.
\end{speechtext}

\dickhead

\paragraph{Spotting \gls{dickhead}}
as he sneaks up needs a \roll{Wits}{Vigilance} roll
\set{track}{7}%
\addtocounter{track}{\value{Dexterity}}%
\addtocounter{track}{\value{Stealth}}%
at \gls{tn}~\arabic{track}.

\Gls{dickhead} soon leaves, telling everyone not to follow him, as they'll just make noise.

\sqpart{shadepaths}% AREA
{Peeping Woodsman}% NAME
{\Glsfmttext{dickhead} explains his plan to kill \glsfmttext{LifeElder}}% SUMMARY

\Gls{dickhead} has observed the area for some time, noticed the \gls{disgnome} plants, and believes that the giant snails all stem from a single source: a powerful spellcaster.

\begin{speechtext}
  The plan is simple, I find the elf who makes the snails, and \emph{kill him}.
  So I'll set \pgls{ambush} then loose an arrow on whatever gardener grows these giant snails.

  Elves are small.
  I'll just need one arrow.
\end{speechtext}

\Gls{dickhead} will leave the \glspl{pc}, as he does not trust them to stay silent while he plans \pgls{ambush} for \gls{LifeElder}.

\sqpart[\squash]{plateauGardens}% AREA
{Loose Clothing}% NAME
{\Glsfmttext{dickhead}'s crossbow lies abandoned on the ground}% SUMMARY

\begin{exampletext}
  You don't get to be centuries old without learning how to spot \pgls{ambush}.
  As \gls{LifeElder} performed one of her standard spells to query the living things in the area, she found \gls{dickhead}, and guessed the reason for his hiding.
  Her spell has split his limbs into myriad tentacles, leaving his equipment on the ground.
  He slithered away as the spell took hold, confused and dismayed, dropping pieces of his equipment along the way.
\end{exampletext}

The \glspl{pc} find \gls{dickhead}'s possessions, including his \gls{crossbow} and twelve quarrels on the ground, but carrying it sends a clear signal to the elves that they approve of his methods, and makes them dangerous.
They will suffer a -3~Penalty to social rolls with the elves while the \gls{crossbow} is visible.

\paragraph{If the \glspl{pc} follow the trail,}
have them roll \roll{Wits}{Survival} (\tn[10]).
If they succeed, they can follow him to \gls{ravencops}, and you can jump to the next \gls{segment}, below immediately (skipping any that might have been before it).
If they fail, they lose \pgls{interval} searching, and find nothing (discard the next \gls{segment}).

A tie means they lose \pgls{interval} from constantly missing tracks, but confirm that they go to \gls{ravencops}.

\sqpart[\squash]{ravencops}% AREA
{Wandering Hood}% NAME
{\Glsfmttext{dickhead}'s clothes lie discarded on the ground}% SUMMARY

The troupe see the last of \glsfmtname{dickhead}'s clothing, discarded just before entering the forest.
Following him further will not be easy; the \gls{tn} rises to~14.

\sqpart[\squash\R]{oathtower}% AREA
{Retirement}% NAME
{\Glsfmttext{dickhead} now works for \glsfmttext{MindElder} as a mutated servant}% SUMMARY

The next time the troupe enter \gls{oathtower}, they find \gls{dickhead} in his new form -- a twisted creature, with limbs replaced by tentacles, and his neck so shrunk that his shoulder-blades wrap around his ears.

After \gls{LifeElder} twisted his body, \gls{MindElder} twisted his mind.
He now accompanies \gls{MindElder} everywhere, passing him pens, and washing his clothes in the lake outside.
When \gls{dickhead} has nothing to carry, he ascends \gls{oathtower} by grabbing a window from outside, and pulling himself up the wall.
Each time he passes a window, he takes a good look inside to check that nobody inside is breaking any laws, and that everything seems as it should.


\dickheadReborn



\sidequest[ravencops,oathtower]{An Ordinary Week in the Enchanted Lands}

\sqpart{ravencops}% AREA
{The Square of Life}% NAME
{Giant snail devours \gls{crawler} in acid attack}% SUMMARY

\begin{exampletext}
  \Gls{MindElder} has taken to twisting the `minds' of snails to make them crave flesh.
  Most snails do not hunt well, but the acidic spray, and disarming appearance, means they regularly consume \glspl{crawler}.
\end{exampletext}

\begin{boxtext}
  The echo of a distant crow's cry reaches you, just before dying.
  The trees look a kind of uniform-brown, without any mottling or variation.
  And the road feels as smooth as pond-scum.
\end{boxtext}

The troupe should make \pgls{bandAct} action to not fall over while walking on freshly-laid snail tracks.
Failing a \roll{Dexterity}{Athletics} roll (\tn[7]) inflicts 1~\gls{ep}.

\begin{boxtext}
  You find a cross-roads, as the path splits left and right.
  The road to the left looks older, and less slimy, but has \pgls{crawler} running towards you.
  The road to the right also looks old, until the giant snail, approaching silently.
\end{boxtext}

If the \glspl{pc} are near the giant snail when it sprays acid, they can roll \roll{Wits}{Survival} (\tn[7]) to leap into the woods as the snail prepares to spray.

\begin{boxtext}
  As the snail-spittle hits the trees and bushes in a messy gush, they let of a tiny hiss and begin to wilt.
  Leaves wither and bark turns black.
  Behind, the \gls{crawler} turns to flee into the woods with two left-legs melded together.
\end{boxtext}

\giantSnail

\chitincrawler

If the troupe let the situation unfold, the giant snail ignores them, and chases the wounded \gls{crawler} up a tree.

\sqpart{ravencops}% AREA
{Got a Permit, Mate?}% NAME
{\glsentrysymbol{afternoon}~An oathkeeper goblin wants to check troupe's weapon licences}% SUMMARY

\histEvent{55}{3}{%
  Goblins approached \glsfmttext{enchantedLands}, looking for food.
  However, \gls{MindElder} forced them to swear oaths to uphold the myriad laws of the land.
  They remains, and multiplied, and soon the land held an army of goblins%
}

\begin{exampletext}
  The lack of \glspl{crawler} and plentiful giant snails soon brought a lot of goblins to the area.
  \Gls{MindElder} also turned this problem to his advantage by making the goblins swear to capture or kill anyone disturbing the peace.

  The goblins responded with sarcasm, but the spell worked anyway, soon all the goblins lay dead or agreed to take oaths of good behaviour, which work fine as long as the goblins eat regularly.
\end{exampletext}

\begin{boxtext}
  The sky rumbles with thunder, but the air feels thin.
  In the distance, a small person in a long, green cloak walks towards you, carrying a potato so large that it cannot see you.
  A long nose points up, just above the top of the potato, and pasty-white ears flop at shoulder-length.
\end{boxtext}

If the goblin (Abjad) sees the troop, it switches the potato to a one-arm hold and points accusingly at the troupe, then speaks in an unusually deep voice.

\begin{speechtext}
  I hope you got a licence for those weapons!
  Show me!
  Show me the licence, earless scum!
\end{speechtext}

Abjad (the goblin) observes and insists on the following local laws:

\begin{itemize}
  \item
  No high-pitched noises.
  \item
  No jokes, nor words which move to laughter.
  \item
  No unlicensed weapons.
  \item
  No wandering without clothes on.
  \item
  No singing out-of-lock.
\end{itemize}

\enchantedGoblin[\npc{\M\N}{Abjad}]

\paragraph{Dealing with Abjad}
requires a \roll{Charisma}{Empathy} roll at \tn[9].
Failure means he will insist on them going to \gls{oathtower} to admit their criminal behaviour, while giving him all of their \glspl{weapon}.

If they calm him successfully, then he won't insist on accompanying them to the tower, but they \emph{will} have to promise to go there.

\paragraph{At the slightest hint of aggression,}
Abjad flees and tries to find reinforcements.
The goblins will try to track them down, using their \roll{Intelligence}{Athletics}, so the \glspl{pc} will roll at \tn, however they handle it (perhaps \roll{Speed}{Athletics} to simply run from \gls{ravencops}, or \roll{Intelligence}{Stealth} to hide where goblins won't find them).

\paragraph{At the end of the \gls{interval},}
the troupe receive an additional \gls{mp}, as the thunder above releases more mana.

\sqpart{ravencops}% AREA
{Silencing the Starlings}% NAME
{A goblin takes aim at a starling for the crime of high-pitched song}% SUMMARY

\histEvent{39}{3}{%
  Soon \glsfmttext{enchantedLands} filled with sepulchres, and every high-pitched noise woke the sleeping \glsfmtplural{ogre} inside.
  \Glsfmttext{MindElder} banned all high-pitched noises, including most birds.
  Of course, the ravens and crows remain, which started the name `\glsfmttext{ravencops}'%
}

\begin{exampletext}
  When \gls{MindElder} banned high-pitched voices, this included most birds, because bird-song can wake the sleeping \glspl{ogre}.
  The ecosystem has become strange since then, as it has very few birds, except ravens, crows, and magpies.
  This is why people call the local forest `\gls{ravencops}'.
\end{exampletext}

Mora has heard a starling sing, and thought to herself `that's illegal!', then readied her \gls{projectile}.
After a `\emph{thunk}!', the bird lies dead, so she finishes the job with a rock, and sucks out the starling's brains.

Mora will happily speak with the troupe, in a low-pitched voice (to avoid waking any \glspl{ogre}), but if she sees them doing something criminal (like carrying weapons without a licence) she flees to sound the alarm (but quietly).

\enchantedGoblin[\npc{\F\N}{Mora}]

\sqpart{oathtower}% AREA
{Meat Salad}% NAME
{As a snail approaches the tower, \gls{MindElder} turns its mind towards thoughts of meaty salad}% SUMMARY

\histEvent{95}{3}{%
  When giant snails barged into \glsfmttext{enchantedLands}, \glsfmttext{MindElder} decided to kill two birds with one stone, and twisted their little minds to crave flesh.
  The giant snails stalk the woods, looking for some meat to eat with their leaves, and occasionally find \glsfmtplural{crawler}%
}

A giant snail approaches \gls{oathtower}, along a snail road, on a clear day.
The \glspl{pc} probably won't see the snail, unless they're on the road out, but they will certainly see \gls{MindElder} observing the land from the balcony at the top of \gls{oathtower}, and singing a spell to twist the mind of the snail.

\paragraph{Any characters who understand Elvish}
will understand the song relates to a meat-based salad, and that the snail should add meat to its salad.

\sqpart{ravencops}% AREA
{Shoeing a Snail}% NAME
{The goblins hound a snail into a dead-end to kill it}% SUMMARY

Mora shoes a snail into a dead-end road.
She organizes a dozen goblins to spread out, and occasionally stab it (while keeping their distance), in order to hound it to the clearing where six more goblins wait to plant spears on the road.

The scene proceeds just as the game-mechanics suggest.
Goblins throwing javelins deal around $1D6-1$ to $1D6+1$ Damage, and the giant snails have a lot of \gls{dr}, even in their most vulnerable location.
Some javelins hit the shell and shatter, others stick into the snail's `skin' harmlessly, and a few dig into its body enough to inflict a minor wound (perhaps 1 or 2 Damage).

Bringing down the snail will take the entire \gls{interval}, as the goblins run away, some double back to collect javelins which fell on the ground, and others run ahead to climb trees and throw javelins from above.
The goblins have to constantly stop the snail entering the forest.

\paragraph{Once the snail dies,}
the goblins take it apart in three stages.

\begin{enumerate}
  \item
  The barbecue, as every goblin feels famished, and must eat immediately.
  \item
  The dissection, where bloated goblins with pot-bellies cut and cure the snail-meat, then make ropes from its innards.
  \item
  The great smashing, where they use rocks to crack the shell into smaller dishes, then use the dishes to transport remaining meat back to the Icebox House.
\end{enumerate}

Each stage takes another \gls{interval}, so the troupe may see the goblins again if they pass through the area the next day.

\sqpart{ravencops}% AREA
{On the Menu}% NAME
{Another giant snail approaches, but ignores anyone on the road}% SUMMARY

Another great snail approaches, with its mind focussed on a meaty-salad.
It ignores anyone on the road.
It only attacks living things in the bushes where it feeds.

The snail stops here and there, inspecting the bushes, and sometimes vomiting on them as part of its external digestion, but never stays for long.

A biting wind blows, bringing one extra \gls{mp} at the end of the \gls{interval}.

\sqpart{ravencops}% AREA
{The Unmerry Band}% NAME
{A dozen goblins walk, every statement brings suspicion of humour}% SUMMARY

Grawl, Majiscule, and Brev patrol the land, looking for trouble-makers, mapping new paths the snails have brought, and noting local monsters.
And as they walk they bicker; Grawl accuses Majiscule of jokes (which \gls{MindElder} banned, so that the high-pitched goblin laughter would not wake any \glspl{ogre}).

\begin{speechtext}

  If we find a snail, we'll need to go back to get the others, and shoe it towards a dead-end.

  What do you mean, `shoe a snail'?

  Is that a joke?
  Are you trying to be funny?

  You said it!
  You said we might shoe a snail!
  You were making the joke!

  Your face is a joke.
  It is a crime, your face.
  You have a funny-looking, criminal face.

\end{speechtext}

Majiscule hides his face until nightfall.


(Canals) The Picnic Choir
-----
{Elven songs descend from the plateau above}

A little group of elves sing, but don't respond to shouts or calls.
They will, however, respond to someone singing with them.

If someone sings well, they will lower the vines of a broken bridge to help them up.
If they sing poorly, the elves leave silently.

Singing \glspl{pc} should roll Intelligence + Performance to understand the elven song, and imitate it.

(Garden) Interview with the Tao Mistress
-----
{LifeQueen wanders the canals, all answers seem strange}

LifeQueen wanders, and sometimes sings.
She stops to ponder a flower, then alters a seed so the flower will grow purple.

While the \glspl{pc} are in the garden plateau, they see a small, red-haired elf below, wandering naked and humming to herself.
Other elves may identify her as the source of all the change in the landscape, but will not give her a name (except to say 'hi').

She speaks quickly, and cryptically, and gives deep thought to every word someone says, but quickly tires of conversation.


\begin{speechtext}
  Endings have nothing to do with what happens, an ending is ultimately a manifestation of values.

  If it never rained, the plants wouldn't grow.

  Possessions are just _things_, man.  Don't let your things control you - be free!

  Everything comes, if you wait the right way.

  Order to the woodspy is chaos to the fey.

  You have this obsession with 'good' and 'bad', but where is this 'good'? What colour is 'bad'?

\end{speechtext}

She will not stop mutating snails, or casting spells as she pleases.
But if any \gls{pc} seems upset by anything, she will help them with another spell.

(Causeway) Misty Way
-----
{The mists fill the causeway}

The \glspl{pc} can see nothing in the causeway, as mist hangs low.
They may have to make a navigation roll just to move about, and projectiles suffer double the normal range penalties.

(Garden) Mist Below
-----
{The Canals fill with mist, only eyestalks roam above}

Mist always falls into the canals by the plateau gardens.
The disorienting environment is quite lethal to newcomers, as people who don't know where the land lies can easily make a misstep, tumbling down the sides.

\begin{boxtext}
  Mist has risen, but is falling into the canals beside the plateau gardens.
  Soon the canals look like fluffy-white rivers.
  A figure in the distance seems to hover between two plateau gardens, but the walking reveals they must be on a vine-bridge which has fallen just below.
\end{boxtext}

Fast movements require a Wits + Survival roll (TN 8) to avoid falling into the misty canals below.
Longer journeys require an Intelligence + Survival roll (TN 10).

The mists fade after an Interval.

(Causeway) A Voice from on Hi
-----
{LifeQueen looks down at the party, ready to converse again}

This time the \gls{pc}S see LifeQueen standing far above from her garden plateau.
She asks them about what they've eaten, and what their favourite kind of rain is.

(Garden) Drooping Bean Vines
-----
{Bean vines on a tree provide a route down}

The bridge-vines don't always grow into bridges.
This one just climbed a tree, which is now covered in bean-producing vines.

If the troupe pull the long vines down, they can safely descend to the canal below.
However, leaving the vines there means creatures below can crawl up; so if the troupe use the vines to descend, the next time they arrive here, they see \pgls{\gls{chitincrawler}} ascending to the plateau garden, and assaulting anyone there.

\section{The Ranger}

If the troupe tarry too long, the \gls{jotter} will send a ranger out to find out what's happening.

(Ravencops) An Old Acquaintance
-----
{A ranger arrives to scope out the situation}

The troupe find him in the woods, hunting a griffin nest.
He speaks haughtily of his ability to survive in the forest, and moves with confidence.
He asks the troupe what they've seen, but does not give their stories much importance.


\begin{speechtext}
  So you still have not found the heart of the problem.
  Well keep searching!
  You may not succeed, but it makes for good practice.
\end{speechtext}

(Garden) Peeping Woodsman
-----
{The ranger explains his plan to kill LifeQueen}

The ranger has observed the area for some time, noticed the disgnome plants, and believes that the giant snails all stem from a single source: a powerful spellcaster.

>>>
The plan is simple, I kill her.
Elves are always wrapped up in their own thing; they never pay attention, and she's probably drowsy from all the disgnome in the area.
So I'll set an ambush then loose an arrow on whoever crafts these giant snails, or slit his throat.
>>>

The ranger will leave the \glspl{pc}, as he does not trust them to stay silent while he plans an ambush for the LifeQueen.

(Garden) Loose Clothing
-----
{The ranger's crossbow is found on the ground}

>>>
You don't get to be centuries old without learning how to spot an ambush.
As LifeQueen performed one of her standard spells to query the living things in the area, she found the ranger, and guessed the reason for his hiding.
Her spell has split his limbs into myriad tentacles, leaving his equipment on the ground.
He slithered away as the spell took hold, confused and dismayed, dropping pieces of his equipment along the way.
>>>

The find his crossbow and twelve quarrels on the ground, but carrying it sends a clear signal to the elves that they approve of his methods, and makes them dangerous.
They will suffer a -3 Penalty to social rolls with the elves while the crossbow is visible.

If the \glspl{pc} follow the trail, have them roll Wits + Survival (TN 10).
Success means you can skip to the next Segment, below.
A tie means they succeed, but only after an Interval (and another Segment).

(Canals) Discarded Clothing
-----
{Rangers clothes lie discarded on the ground}

The troupe see the last of the ranger's clothing, discarded just before entering the forest.
Following him further will not be easy; the TN rises to 14.

# The Movements of Monsters

(Canals) Excuse Me!
-----
{A giant snail blocks the path}

'Brownie' the snail grew much bigger than the others, and pregnancy did not help.
Now when the moves through the canals, she blocks them entirely.
Anyone walking inside the canals must simply turn back and find another route.
This may delay the troupe in whatever they wanted to do by an Interval.

If the troupe try to squeeze past her, have them roll Dexterity + Stealth at TN 8.
Failure inflicts 2D6 Damage as her massive body smooshes them against the canal walls.

(Causeway) Stampede
-----
{An auroch stampede passes through}

Elves watch from a garden plateau as aurochs stampede through the causeway.
Behind them, \pgls{\gls{chitincrawler}} chases, but it's losing steam, and soon retreats into the forest.

(Garden) Woodspy Spotted
-----
{A woodspy stalks the garden}

Most large animals have trouble reaching up the tall walls to the plateau gardens, but woodspies are clever, and sometimes manage to pull themselves up using a snail, or finding a piece of vine hanging down.

Any elves present will probably spot the creature.
Either way, it slowly meanders towards an exit path, and waits for someone to pass.
Once it grabs someone, it pulls them down into the canyon.

(Causeway) Slow Wander Home
-----
{The aurochs are returning...slowly}

Any sudden movement will prompt the aurochs to begin a stampede away.
The troupe may prefer to wait, but this will take the rest of the Interval, as the aurochs move slowly, while grazing.



\stopcontents[sq]

\end{multicols}

%\section{Mound}

So we have a tunnel, which takes you right there, although it did have a problem recently\ldots

- We had a rat problem,
- so we put \pgls{crawler} down there,
- but it didn't come out, so we lured this basilisk to use its poisonous breath to kill anything in the hole,
- but the basilisk crawled in once it finished.

So...do you think you can help?


\stopcontents[sq]

\section{In Closing}
\label{feyClosing}

\begin{multicols}{2}

\subsection{The Controlled Collapse of an Ecosystem}

The grotesque stack of \glspl{spell} and plans infecting the Elven lands have all the balance of a two-legged spider.
And like any unbalanced thing, if it falls wrong, then it can fall on people nearby.
The players will need to understand the ecosystem and all its parts in order to make it fall down in the right way.

The following \glspl{sq} do not begin ready-to-go.
You can mark them ready (`\sqr') as the eco-system begins to collapse.

\subsubsection{Messing with the Goblins}
proves easy.
Goblins tend not to follow rules and laws smoothly,%
\exRef{judgement}{Judgement}{goblin}
and don't fully grasp the rules of \gls{enchantedLands}.

If the \glspl{pc} steal a goblin's weapon's licence, the goblin drops the weapon.
If they tell a goblin that she just made a joke, she will believe them, and hand herself in to \gls{oathtower}.

\paragraph{With enough in-fighting,}
the goblin population will decrease, and you can mark \pgls{segment} from \nameref{goblinsRise} as not ready (`\sqn').

\sidequest[ravencops,oathtower]{Waking the \Glsfmtplural{ogre}}

\noindent
Every high-pitched noise around \gls{enchantedLands} activates another \gls{segment}, below.
This may occur, for example, if goblins raid \gls{plateauGardens} and take some of the beans from the bean-vine bridges, which makes them fart (which then makes other goblins giggle like a rusty gate).

The \glspl{pc} might try to kill the sleeping \glspl{ogre} quietly, but the \glspl{sepulchre} have only tiny grates at the side, enough for a goblin to hear their breathing but not enough to crawl through, or even extend an arm into comfortably.
The \glspl{pc} also don't know where \gls{MindElder} placed all of them.
In total, eight \glspl{sepulchre} remain in \gls{ravencops}, mostly near \gls{oathtower}.

\paragraph{Every time the \glspl{pc} destroy \pgls{sepulchre},}
the goblins notice within a day, and become enraged.
Mark one \gls{segment} below as ready (as they begin to wake \glspl{ogre} to protect them), and make another \gls{segment} ready in \nameref{goblinsRise} (\vpageref{goblinsRise}).

Once the \glspl{pc} destroy \ref{lastOgreSegment} \glspl{sepulchre}, you can start marking the \glspl{segment} below as unavailable, scoring them out (starting with the last).
However, \glspl{segment} from \nameref{goblinsRise} remain active.

\sqpart[\squash]{ravencops}% AREA
{Breakfast to Go}% NAME
{Three \glsfmtplural{ogre} awaken and need food}% SUMMARY

When the \glspl{ogre} awaken initially, they feel hungry, but still bound by their oaths.
They ask the troupe for food, and will follow them anywhere, or go anywhere they point to.

\enchantedOgre[\npc{\T[3]\M\F\N}{\Glsfmtplural{ogre}}]

\paragraph{Any goblin present}
leads them to food (perhaps in \gls{oathtower}, or the Icebox House \vpageref{iceboxHouse}).
\Gls{MindElder} soon puts them back to sleep, and inquires about who might have been making high-pitches noises.

\sqpart[\squash]{ravencops}% AREA
{Hunting for the Ice Box}% NAME
{Three \glsfmtplural{ogre} hunt for the ice box house}% SUMMARY
\label{ogresEatIcebox}

Three \glspl{ogre} have awakened, and hunt for the Icebox House, where \gls{MindElder} has food packed in ice, below the ground.
If you have that house on the map already, the \glspl{ogre} head there, quickly.
If not, the \glspl{ogre} are trying to find it, but can't remember where it is.

Find the Icebox house \vpageref{iceboxHouse}.

\enchantedOgre[\npc{\T[3]\M\F\N}{\Glsfmtplural{ogre}}]

\paragraph{If the \glspl{pc} speak with the \glspl{ogre},}
they should roll \roll{Wits}{Empathy} (\tn[10]) to avoid agitating the \glspl{ogre}.
Failure means the \glspl{pc} are lunch, and a tie means the \glspl{ogre} discuss eating them for a few \glspl{round} before attacking.

\paragraph{If the \glspl{ogre} arrive at the Icebox house,}
they find it empty -- other \glspl{ogre} have already eaten everything inside.
At that point, nothing can stop them eating everyone in sight.

The \glspl{pc} may simply flee, and let the \glspl{ogre} eat the elves who live there.

\sqpart[\squash]{ravencops}% REGION
{Elf Smash}% NAME
{\Glsfmtplural{ogre} eat through an elven house}% SUMMARY

\begin{exampletext}
  Three \glspl{ogre} entered an elven home, famished and angry.
  The elves were not prepared for violence, due to all the oaths of peace.
  The \glspl{ogre} killed the lot, and searched for food and \glspl{weapon}, before taking the elves up to eat them.
\end{exampletext}

The \glspl{pc} find three \glspl{ogre} gnawing on five elf corpses.

\paragraph{If the \glspl{pc} leave quickly,}
the \glspl{ogre} leave them alone.

\ogre[\npc{\T[3]\M\F\N}{\Glsfmtplural{ogre}}]

\sqpart[\squash]{ravencops}% REGION
{The Questing Oath}% NAME
{\Glsfmtname{MindElder} requests the troupe kill rogue \glsfmtplural{ogre}}% SUMMARY

At \gls{oathtower}, \gls{MindElder} spots the \glspl{pc} from his balcony, and asks them to come up to speak.
Once Ha\^{c}ek, the boat goblin, has ferried them into the tower, an elf leads them up for a personal audience with \gls{MindElder} where he requests the \glspl{pc} hunt down and kill the \glspl{ogre}.

\Gls{MindElder} has no money, but he will bargain with anything else he might have in the tower -- \glspl{spell}, \glspl{weapon}, \glspl{ration}, or just `a very big favour'.

\sqpart[\squash]{ravencops}% REGION
{Everyone Up!}% NAME
{All the \glsfmtplural{ogre} have awakened, and formed a war band}% SUMMARY

The band of eight \glspl{ogre} have eaten through most of the giant snails in \gls{enchantedLands}, and hunt for what remains.

\ogre[\npc{\T[4]\M\F\N}{\Glsfmtplural{ogre}}]

\enchantedOgre[\npc{\T[4]\M\F\N}{\Glsfmtplural{ogre}}]

The \glspl{pc} will probably spot them in the distance.
Hiding requires a \roll{Wits}{Stealth} roll at
\setTN{Wits}{Vigilance}
\tn.

\label{lastOgreSegment}

\stopcontents[sq]

\subsubsection{\Glsfmtname{MindElder}'s Death}
would be disastrous.
Goblin oaths would start to break, little by little, while the elves in \gls{enchantedLands} would lose their enchantments much more slowly.
The goblins would awaken the \glspl{ogre} with screeches and wails held in for half of their lives.

\index{\expandafter\Glsfmtname{MindElder}'s Death}

\begin{itemize}
  \item
  Within a week, they would eat through the giant snails in \gls{ravencops} (while many continue to uphold bits and pieces of their various oaths).
  \item
  Within a week and one day, the goblin horde would eat through every elf in \gls{enchantedLands}.
  \item
  The ninth day would begin with a siege upon \gls{coppernut}.
\end{itemize}

\sidequest[ravencops,plateauGardens]{Goblins Rise}
\label{goblinsRise}

\renewcommand\enchantedRations{empty bowl}

If the giant snail population decreases, mark one of the \glspl{segment} as ready (`\sqr') after a week.
If the goblins become agitated by finding \pgls{sepulchre} destroyed, mark \pgls{segment} as ready immediately.

\sqpart[\squash]{ravencops}% AREA
{Goblins Don't Share}% NAME
{Goblins eat animal, root, and berry in \glsfmttext{ravencops}}% SUMMARY

\Gls{foraging} in \gls{ravencops} rises to \tn[16] as the goblins pick the forest bare and hunt everything that moves.

\sqpart[\squash]{ravencops}% AREA
{Husk}% NAME
{A starving goblin demands food}% SUMMARY

Pica hasn't eaten in two days, and hasn't eaten well in four.

\begin{boxtext}
  A goblin approaches ahead.
  Its robes are slipping from the shoulders
\end{boxtext}

\begin{speechtext}
  I know you got something nice.
  I smell it.
  Give me a bite.

  Make it legal -- make it a gift!
\end{speechtext}

\set{r3}{3}
\enchantedGoblin[\npc{\F\N}{Pica}]%

If the \glspl{pc} don't `gift' her some food, her strained oath snaps, and she becomes stupefied.
She attacks with the \weaponName, but the first attack automatically fails, as she receives the crushing weight of a -5~Penalty to thinking straight.

\sqpart{ravencops}% AREA
{Rebalancing Grass}% NAME
{Growths along the snail paths shows the declining snail population}% SUMMARY

Without the snails, grass and bushes begin to grow across the snail paths.
In the first week, the new growth is subtle, but still possible to notice with a \roll{Wits}{Survival} roll (\tn[12]).

\sqpart[\squash]{shadepaths}% AREA
{Hunting by Shadow}% NAME
{Hungry goblins hunt for a snail, then eat until they become hobgoblins}% SUMMARY

This band feel starving, so they hunt through \gls{shadepaths} for a snail to feast on.
The snails here may have acidic vomit, but they don't attack people.

Once a snail is found (which won't be long) they kill it, and take two \glspl{interval} to eat through the corpse, belching and groaning loudly the entire time.

If the elves remain affected by the \glspl{disgnome}, they watch and slowly talk about potential solutions.
But any who awakened will think about safe methods of attack, from the high vantage point of \gls{plateauGardens}.

\sqpart{plateauGardens}% AREA
{Night Raids}% NAME
{Goblins come to remove the foods of \glsfmttext{plateauGardens}}% SUMMARY

The goblins arrive and try to take food.
If the elves have shaken off their sleep from the \gls{disgnome}, they object to the goblins taking food and may turn violent.
The situation is tense, so however the \glspl{pc} approach it, set the \gls{tn} to 10 or more.

\enchantedGoblin[\npc{\T[12]\M\F\N}{Grotesk, Gadzook, \& c.}]

\begin{boxtext}
  Put some clothes on, filthy elf!
  Take a hike!
\end{boxtext}

\sqpart{ravencops}% AREA
{Chewing Ice}% NAME
{Goblins raid the Icebox house}% SUMMARY

A dozen famished goblins try to find the Icebox house%
\footnote{Find the location \vpageref{iceboxHouse}.}
to take the food.
However, this breaks their oaths to uphold the law, which gives them a -4~Penalty to all Mental~\glspl{attribute}.

\begin{boxtext}
  A quiet \textit{thck-thck-thck} sound signals feed moving across the snail path.
  A dozen goblins approach, with an animalistic look to them -- vacant, unfocussed, but full of purpose.
\end{boxtext}

The goblins do not speak, except to say `Icebox elves', and `food'.

\paragraph{If the \glspl{pc} lead them to the Icebox house,}
they follow.
However, they remember enough to know if they're going the wrong way.

Of course, once they arrive they must eat.
If \glspl{ogre} ate through the Icebox food stores (\gls{segment}~\vref{ogresEatIcebox}) then the goblins become feral instantly.

\enchantedGoblin[\npc{\T[12]\M\F\N}{Ligature, Spur, \& c.}]

\paragraph{If they don't arrive within \pgls{interval},}
they attack.

\paragraph{If the goblins receive \pgls{ration} each}
they calm down, get back to \gls{oathtower}, and renew their oaths.


\sqpart{ravencops}% AREA
{The Horde Rises}% NAME
{Everyone bands together to feed}% SUMMARY

The goblins have broken their oaths, and begun to wake the \glspl{ogre} who sleep an enchanted sleep in the \glspl{sepulchre} near \gls{oathtower}.

\goblin

\sidequest[plateauGardens]{Removing the \Glsfmtplural{disgnome}}

\noindent
If the elves remain affected by all the \gls{disgnome}, they will be far less able to spot attacks, and use magic to defend themselves.
But if the \glspl{pc} remove the \glspl{disgnome},
many of the elves of \gls{plateauGardens} wake up and decide to get proactive.

Removing the \gls{disgnome} around \glspl{plateauGardens} takes time -- at least \pgls{interval} -- and a \roll{Strength}{Cultivation} roll at \tn[10].

\setcounter{sqNo}{-1}

\sqpart[\squash]{plateauGardens}% AREA
{Eyes Open}% NAME
{Orv\"e is awake and irate}% SUMMARY

Orv\"e has recently shaken off the effects of repeated \gls{disgnome} contact.
She mutters about how difficult finding water has become, since the giant snails make nearby rivers and lakes disgusting, and unusable.

Orv\"e mentions feeling better since avoiding the \gls{disgnome}, and asks the \glspl{pc} if they could remove any they see.

\elf[\NPC{\F\El}{Orv\"e}% Name
  {skeletal face, purple eyes}% Description
  {wrinkles nose}% Mannerism
  {to take a bath}]% Wants

\paragraph{If the \glspl{pc} encourage her irritation with the snails,}
she asks \gls{LifeElder} to stop making the giant snails.

\stepcounter{sqNo}

\sqpart[\squash\gls{vlg}]{plateauGardens}% AREA
{Defences Up}% NAME
{The elves fortify \glsfmtplural{plateauGardens}}% SUMMARY

Ontamon does not like the goblins stealing food from the gardens.
A few of the elves have decided to fortify \gls{plateauGardens}, so now he's double-checking the fortifications, to give them some feedback.

Along the nearest \glspl{plateauGardens}, thorny bushes grow along the edges, encouraged by \glspl{spell}.
Most just provide an irritating barrier (and +2 to any \gls{tn} to climb past them), but 1 in 6 have a venom, which inflicts $1D6-1$~\glspl{ep}.

\paragraph{If goblins raid \gls{plateauGardens},}
they die, and you can remove a goblin encounter (\vpageref{goblinsRise}).

\elf[\NPC{\M\El}{Ontamon}% Name
  {Elaborate, brown, braids}% Description
  {pleats hair}% Mannerism
  {to defend against those goblins}]% Wants
\label{ontamon}

{\small
  \showStdSpells
}

\sqpart[\squash]{plateauGardens}% AREA
{Death to Sleep}% NAME
{An elf removes the last of the \glsfmttext{disgnome}}% SUMMARY

Ontamon (\vpageref{ontamon}) has decided to remove the last of the \gls{disgnome} himself.
The \glspl{pc} find him pouring a plant-poison onto a patch.

The last of the \gls{disgnome} will soon disappear, and the elves will lose all Wits~Penalties.

\sqpart[\squash]{plateauGardens}% AREA
{Ready for War}% NAME
{The elves discuss the ramifications of snails dying}% SUMMARY

If the giant snails all die, the goblins will go hungry, and the enchanted oaths which keep them peaceful will break.
The elves don't want that, and they also don't want the elves of \gls{enchantedLands} to die.

\stopcontents[sq]

\subsubsection{Making the Elders Talk}
will take perseverance, but it can work.
If \gls{LifeElder} and \gls{MindElder} work together, they can  reduce the giant snails in the area slowly, while \gls{LifeElder} introduces a few, subtle, `goblin-traps'.

\subsubsection{Elven Death by Human Hands}
will prompt a vicious reaction from \gls{romeo} and \gls{juliet}, along with a couple of other elves from \gls{plateauGardens} if the \gls{disgnome} has been quelled.
The elves will begin planning to wipe out \gls{coppernut} in order to remove all humans from the area.
The plot will begin by casting strange spells on the road out, so they can ensure complete destruction.%
\footnote{If ten refugees return, a battalion of humans may return.
But if nobody returns from the road for a month, then people will simply decide to not travel along that road rather than investigate.}

\subsubsection{The Grand Spell}
\label{grandSpell}
planned by \gls{romeo} and \gls{juliet} will require 16~\glspl{ingredient} in total, plus \pgls{flood} and \pgls{earthquake}.
The \glspl{pc} will have to engineer the \gls{flood} inside \gls{shadepaths}, where rivers have been artificially diverted.
The \gls{earthquake} will occur naturally, as it always does, at the end of each \gls{cycle}.
This means the plan must wait until the end of \showCycle, so it can only occur at the end of the session.

Once the couple have all they need, they make each one into \pgls{boon}, and begin a grand ritual of song-magic, harmonizing together.

\spell{Shattered Identity in G Minor}% Name
  {Detailed, Distant, Divergent, Duplicated}% Enhancements
  {Warp}% Action
  {Water, Fate, Earth, Fire}% Spheres
  {\roll{Wits}{Survival}}% Resist with
  {The chorus alone is twenty minutes long, but once the spell really gets going, it binds everyone and everything at \spellRange\ into a self-jealous, psychic lump of non-space.}% Description
  {Everyone involved drops pieces of their memories, desires, and their perception of time into the fractured space, and each space involved loops together.

  Everyone involved gains a +\arabic{spellPlusOne}~Bonus to Empathy tasks, and a -\arabic{spellPlusOne}~Penalty to all other tasks.
  Both the Bonus and the Penalty reduce by one each time someone travels between one of the connected area, and finds another part of themselves.}

\begin{boxtext}
  Elves from \gls{plateauGardens} look down at you, scared and confused.
  Their usual blank-eyed cool has gone and you remember kissing \pgls{ogre} good night, and hoping to become like him one day.

  \Gls{oathtower}'s stairs lead up to the \gls{plateauGardens}, or you can go into the elven home through the trees.
  Two goblins stand on the cooker holding your memory of learning the Air \gls{sphere}.

    ``\textit{Sorry, wrong person!}'', the little goblin says in Elvish.

    Up the stairs, and out of the elvish house, \gls{oathtower}'s library is full of goblins, all of them are \gls{MindElder}, shouting for everyone to leave his home.
    Then they turn and point at you in unison, and begin to shriek, ``\textit{Identify! Identify}''.

    An eyeball reaches in, to return your name, if you want it\ldots ?
    The other eyeball searches the room upstairs.

    The name feels right, but the snail has another, and that one feels right too.
    The snail wonders how you know which name is your name, and wonders if you would share.
    The snail is ashamed of its nakedness, so the eyeball slithers back out of the window.

    The goblin hunger was deep, now infectious.
    You could enjoy the taste of your own arm, and it wouldn't actually be \emph{yours}, so it would be fine, and you would get a whole arm to yourself.

    \Gls{SnailTamer} arrives, telling you he loves all of you, and you're all doing really well.
    He's not really there, as he was taking a nap, and just decided to have a little dream with you guys to help everyone out.

\end{boxtext}

The spell ends before long, or it should if the players aren't into Dada-Taoism.
Or if they're getting the vibe, continue handing out memories until they've pieced their characters together.

After the experience ends, \gls{LifeElder} wanders past, stops to observe a spider-web, then continues.

\paragraph{One the spell wears off}
the two groups of elves and the goblins will begin to work together, planning a route forward, each more fully aware of the others points of view.

\iftoggle{verbose}{
  \subsection{Broken Resolutions}

  If your table runs out of time for the night, and you need to wrap everything up quickly, switch to \gls{downtime} plans, and give each \gls{pc} \pgls{action}.
  If they want to negotiate with \gls{MindElder}, they might roll \roll{Intelligence}{Empathy}, or if they think it's time to kill those \glspl{ogre}, sleeping in their \glspl{sepulchre}.

  For some added drama, ask each player to roll under a cup, napkin, or sleeve.
  Leave the \gls{natural} hidden until everyone has decided what they want to do.
  One all results are in, determine the outcomes based entirely on the rolls.
  If \pgls{pc} spent the entire \gls{downtime} trying to redirect the river through \gls{shadepaths}, it could \gls{flood} the \gls{sunway} and block snails from accessing \gls{ravencops}.
  Or if they spent their time trying to kill \glspl{ogre}, a failed roll could mean that \pgls{ogre} eats the \gls{pc} alive before marauding around \gls{ravencops}.
}{

  \subsection{No Way Home}

  If you want to run this arc over multiple sessions, or if the troupe find themselves stuck here over \gls{downtime}, the elven lands have plenty of reasons for characters to leave, disappear or get distracted.

  \subsubsection{In \glsfmttext{plateauGardens}}

  Characters leave because,

  \begin{enumerate}
    \item
    they ate the wrong thing, and their limbs shrivelled.
    The elves hope to fix them `soon' (and will, once the player returns).
    \item
    they wandered off with an elf (leading to gossip and laughter), and have not returned.
    The next session resumes once all hope is lost for finding them.
    \item
    a clerical error resulted in an immediate summoning from an overseer.
    A new \gls{pc} arrived to deliver the letter.
  \end{enumerate}

  Characters arrive because,

  \begin{enumerate}
    \item
    the \gls{jotter} wants to know what the hold-up is.
    \item
    the \gls{pc} was lost, some time ago.
    Hunger and monsters killed the rest of their troupe.
  \end{enumerate}

  \subsubsection{In the pale forest}

  Characters leave because,

  \begin{enumerate}
    \item
    they whistled out of key, and the goblins told them it was time to leave.
    \item
    \pgls{ogre} woke up, her hand reached out, and she grabbed the character.
    \gls{MindElder} reacted quickly, putting everyone in enchanted sleep, but she holds the character like a child with a doll; best not to wake them.
    \item
    goblins prepared food for Winter, grabbing all that was edible, and stuffing it in ice.
    The `it' included the character, who is `edible'; but the \gls{MindElder}'s enchanted sleep spell should deep them safe.
  \end{enumerate}

  Characters arrive because,

  \begin{enumerate}
    \item
    twelve oathkeeper goblins caught them in the forest, along with a bear and \pgls{griffin}.
    The goblins take their prizes -- all tied to poles -- back to share with the rest.
    \item
    the local overseer sent the character with a letter for \gls{MindElder}.
    It contains a generic proposal for mutual aid in killing beasts.
  \end{enumerate}

}

\end{multicols}




