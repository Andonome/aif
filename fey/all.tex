\chapter{Call to Adventure}
\epigraph{
  There was an old lady who swallowed a fly,

  I don't know why she swallowed a fly – perhaps she'll die!
}


\label{callToAdventure}

\enchantedMap

\section*{The Picture on the Tapestry}

\begin{multicols}{2}

\begin{description}
  \item[The tail begins]
  when a giant snail emerges from \gls{ravencops} and tries to eat \pgls{village}'s crops.

  \textit{If nobody stops the snails, more will come, destroying the \gls{village}'s crops.}

  \item[In \autoref{elvenForests}]
  the troupe enter \gls{ravencops} to find out where that giant snail came from.
  They find a labyrinth of wide roads, all created by giant snail-trails.

  Within a day, a giant snail approaches, and attacks \pgls{crawler}, which explains why \gls{ravencops} has so few monsters.

  \textit{And if the snails die, \glspl{monster} will populate \gls{ravencops} again, attacking the nearby \gls{village}.}

  \item[Closer to \gls{oathtower}]
  a goblin approaches, demanding to see the troupe's weapons licences.
  If they speak politely, he leads them to \gls{oathtower} to obtain licences.

  \textit{The lord of \gls{oathtower} keeps the goblins under enchanted oaths, so and they enforce the local laws.}

  \item[Near \gls{oathtower}'s shining lake]
  the \glspl{pc} find \pgls{sepulchre} that snores like an avalanche.
  Their goblin-guide explains that three \glspl{ogre} rest inside, in an enchanted sleep, and they will awaken during the time of feasting.

  \textit{High-pitched noises in \gls{ravencops} will awaken a horde of hungry \glspl{ogre}.}

  \item[The boatman by \gls{oathtower}]
  can give the \glspl{pc} weapons licences, and tell about the snails.

  \textit{The enchanter at the top of \gls{oathtower} makes the snails attack \glspl{monster}, but they come from the lawless elves in the West.}

  \item[To the West]
  they find the earth juts upwards, in tall pillars.
  Climbing the roots stings with a venom, that dulls the senses.
  At the top, elves languor in misty gardens, their thoughts slowed by the \gls{disgnome}.

  \textit{Without the poisonous roots, the local elves will become less useless and dull, and may help the \glspl{pc}.}
  \item[In shadows below,]
  giant snails roam in \gls{shadepaths}.

  \textit{A nameless elf wanders below.
  She makes the snails grow.}

  \item[The crux]
  arrives when the players consider what levers they have.
  \begin{itemize}
    \item
    Removing the snails means ravenous goblins.
    \item
    Removing the goblins means more snails, and more monsters in \gls{ravencops}, which will travel to human lands.
    \item
    Removing the \glspl{ogre} requires delicacy.
  \end{itemize}
  \item[In \autoref{looseThreads}]
  \autoref{feyClosing}'s little \glspl{thread} cover the dangers of ecological collapse.

  Finally, \autoref{feyClosingThreads} has fall-back \glspl{segment} which place locations and show the daily life of each \gls{region}.

  \item[The end]
  depends on your group.
  Perhaps they kill the giant snails.
  Perhaps they balance another fix on top of the problems, and the world wobbles on.
\end{description}


\end{multicols}


\renewcommand\csComments{
  \draw[very thick,gray] (12,0.6) -- (13,0.6) node[anchor=north]{\outline{1 Mile}} -- (17,0.6) node[anchor=north]{\outline{5 Miles}} ;
}

\mapNotes{
  \normalsize Civilization/08/01,
  \Hu~\Glsfmttext{broch}/22/08,
  \Hu~\Glsfmttext{coppernut} \Glsfmttext{village}/20/30,
  \huge\Glsfmttext{sunderedForest}/35/99,
  \huge\Glsfmttext{enchantedLands}/80/99,
  \rotatebox{40}{\Large\nameref{plateauGardens}}/40/70,
  \rotatebox{40}{\El\R}/45/65,
  \rotatebox{40}{\Large\nameref{ravencops}}/55/45,
  \rotatebox{40}{\N}/58/40,
  \nameref{oathtower}/86/75,
  \El/85/65,
  \rotatebox{40}{\Large\nameref{sunway}}/50/60,
}

\widePic{feylands}

\sidequest[coppernut]{Slowburn Snails}

\sqpart{coppernut}% AREA
{The Logistics of \Glsfmtplural{monster}}% NAME
{something approaches a nearby \glsfmttext{village}, prompting arguments over duty vs stupidity}% SUMMARY

\Gls{SnailTamer} has completely lost control over his mount, and now the giant, murderous, snail, is moving towards \gls{coppernut} -- a nearby \gls{village} -- to feed.
However, the \gls{broch} can only see the eyeballs wobbling over the woods, like a pair of bulbous tentacles.

\begin{boxtext}
  Over the sea of dense green, dancing in the heavy wind, something strange and monstrous moves, taller than the canopy, approaching the \gls{village} below.
  The farmers still work the fields below, in the usual loose circle-formation, with archers watching from the walls.
  But they can't see the tentacles writhing above the canopy.
\end{boxtext}

According to \glsfmtname{susjot}:

\begin{speechtext}
  You are the \gls{guard}.
  That is a beast.
  Go guard!
\end{speechtext}

But according to \composeHumanName\ the \gls{soldier}:

\begin{speechtext}
  \Gls{coppernut} lies two miles away, which requires half an hour at a good march.
  The beast looks to be two miles beyond \gls{coppernut}, so it will arrive long before us, and when we arrive, the situation will already have ended\ldots one way or another.

  We should start the pipes, and at least warn the \gls{village}.
\end{speechtext}

And according to \glsfmtname{dickhead}:

\begin{speechtext}
  They won't hear any pipes over this wind -- it's blowing against us!
  You can run two miles in ten minutes, with gear.
  Beasts always take it slow, so we can beat it; but best not.
  We don't know what it is, so we should stay put and watch.
\end{speechtext}

\begin{boxtext}
  \Gls{dickhead} masticates a petulent bone, while \gls{susjot} and the \gls{soldier} argue.
\end{boxtext}

\paragraph{If the troupe watch from the safety of the \gls{broch},}
they will intermittently see the eyestalks rise again above the canopy, and soon notice how slow the thing is (but will not be able to identify it from the two bulbous tentacles) poking into view for those small moments.

\paragraph{If the troupe avoid interference,}
\gls{susjot} orders them down by morning, and they find the farmlands half-destroyed, and see the snail making a well-fed retreat back to \gls{ravencops}.

\paragraph{If the troupe warn the farmers at \gls{coppernut},}
they can arrive on time with a \roll{Speed}{Athletics} roll (\tn[5]).
Once there, the farmers let them in.
Twenty archers stand active on the walls, but they have little effect on the snail.

\paragraph{Once the troupe see the snail,}
\gls{SnailTamer} calls down to them, with his standard sloth and na\"ivet\'e.

\begin{boxtext}
  The wind dies down, leaving a distant crashing noise which becomes louder, and sounds like crushed bushes and snapped branches, mixed with  retching.
  A rock-like surface emerges, the size of a cottage, with a smooth, brown neck hovering above, which tapers to a point above with two tentacles pointed straight in the air.
  An elf sits on the rock-like surface above, waving slowly.
\end{boxtext}

\SnailTamer

If the snail sees the troupe (or anyone) it begins projectile-vomiting acid while \gls{SnailTamer} -- the elf on the snail's enormous shell -- begins to explain himself:

\begin{description}
  \item[round 1]\it So, hey, um\ldots everyone there.
  Can you hear me?

  \item[round 2] I really want to ask you if you might not hurt Nettlerash here.
  She's a peaceful girl usually.

  \item[round 3] Actually, I named Nettles after a human, or more like, humans in general, because, as you can see, she is very tall, like a human.
  Are you sure you need to do that with the sword?
  I think it's upsetting her.

  \item[round 4] So, it all started a while ago, we went a little too close to the tower, and\ldots wait, let me back up a bit.
  It's actually important that you understand some of the earlier events (I suppose you might call these events `history', or is that prejudiced?
  Sorry, I never actually met humans, unless you count dwarves\ldots).

  \item[round 5] Do you count dwarves?
  I mean no judgement, I just wondered, because\ldots wait, Nettlerash really is not looking healthy.
  Okay, this is serious, we really need to discuss the use of swords and how to respect differences, and\ldots
\end{description}

\iftoggle{verbose}{
  The troupe should find this fight strange.
  On the one hand, anyone with a Speed~Bonus at~-3 can only act every second \gls{round}, so the snail will not do very much.
  But on the other hand, the \glspl{pc} can only Damage it by getting \pgls{vitalShot}, and \emph{even then} \gls{dr}~5 remains;
  so the fight will probably last a few \glspl{round}.
}{}

\giantSnail

\paragraph{If the \glspl{pc} attempt a peaceful resolution,}
they can manipulate the snail easily.

\paragraph{If Nettlerash dies,}
\gls{SnailTamer} leaves, saddened, and will not speak with the troupe until he collects his thoughts.

\paragraph{Once the combat dies down,}
\gls{susjot} appears, and asks the troupe what happened to \gls{dickhead}.
Someone needs to find the source of that giant snail, and make sure no more will be coming this way.

\paragraph{If the troupe ask \gls{SnailTamer} about his plans,}
he says he has a letter for \gls{LifeElder}, from \gls{MindElder}, and must deliver it at once.

\begin{speechtext}
  I have to go.
  I have this letter for someone.

  Who is it for?
  She's just a person, like anyone, but she's not into labels.

  I don't call her anything.
  Well, maybe I call her `hi'.
\end{speechtext}

\paragraph{If anyone grabs the letter,}
they discover two unsurprising things:

\begin{itemize}
  \item
  The letter is in Elvish.
  \item
  \Gls{SnailTamer} objects!
\end{itemize}

\talisman{Urgent Letter}% Name
  {Devious, Detailed}% Enhancements
  {Wax}% Action
  {Water, Fate}% Spheres
  {\roll{Wits}{Academics}}% Resistance
  {``\textit{I would like to discuss the matter of the snails.
  At your convenience, reply with a letter via your most reliable courier.  Yours faithfully, \glsentrytext{MindElder}}''}% Summary
  {If the reader ignores the letter, a nagging feeling remains with them, urging them to do as instructed.
  After a few days, they can think of little else, gaining a -2~Penalty to all Mind \glspl{attribute}, then the Penalty grows until it  reaches -\arabic{spellPlusOne} after a week.}% Details

\showTalisman

\Gls{susjot} can read Elvish, so if he receives the letter, he will feel bound to do as it says, and send the \glspl{pc} with a letter, to \gls{MindElder} (though he has no idea who sent the letter, or where they live).

\paragraph{However this ends,}
\gls{susjot} orders the troupe to go along the snail's path, and find out where it came from, and how it came to be, and try to stop any more coming this way.

\paragraph{If the \glspl{pc} ignore the hook,}
(and somehow evade their duties without being prosecuted) then the giant snails continue to wander up to the \gls{village}.
Now that one has eaten a path to the farmlands, the rest will follow that path more quickly.

\sqpart{coppernut}% AREA
{\glsentrysymbol{night}~Midnight Salad}% NAME
{A giant snail gnaws through the crops}% SUMMARY

%! The region should change to some place by the elven lands, full of baileys et c.
A giant snail comes from \gls{plateauGardens}.
Unlike the last one, it only eats vegetation and will not attack anyone on purpose.
It arrives at night, so the troupe have to make sense of the noises they hear.

If \pgls{npc} watchman keeps \pgls{vigil}, they will hear the snail vomit acid onto the \gls{coppernut}'s wheat crops in order to digest them.

\begin{boxtext}
  Someone stands in the darkness, and moves towards a window.
  He asks if you can hear the vomiting.

  The window shutters creaks open, the sound becomes louder.
  It seems to come from beyond the \gls{village}'s wall.
\end{boxtext}

If the troupe do nothing, \gls{coppernut} loses half its crops over the course of the night, as the snail vomits acidic gloop over them, then eats the gloop.

\paragraph{If the snail takes Damage,}
it flees.

\paragraph{If \gls{coppernut}'s farmers lost any crops,}
they will be angry, and demand the \glspl{pc} do something about the snails.

\sqpart{coppernut}% AREA
{\glsentrysymbol{evening}~Eyestalks at Dusk}% NAME
{Another giant snail arrives, this one wants meat}% SUMMARY

The troupe spot yet another snail heading towards \pgls{village}.
The Sun has almost set, and by the time they arrive, they will have no light.
And this one is carnivorous, and aggressive.

\sqpart{coppernut}% AREA
{The Consequences of Inaction}% NAME
{The \glsentrytext{village} lies empty, a giant shell remains outside}% SUMMARY

Giant snails took too many crops, and the farmers had no idea what to do about \pgls{monster} that eats vegetables (they have had no need to protect their crops so far).
The local \glspl{guard} managed to kill a couple of giant snails (their shells remain in the fields), but not fast enough.

Eventually, everyone abandoned their area, and fled to other \glspl{village} where they had family, or had to enter a town to beg for food.

The \gls{village} lies empty, except for \pgls{crawler}, who moved into one of the houses.

\chitincrawler

\playCommentary{
  \begin{description}
    \item[Player 1:]
    (Sinkmaul)
    Let's jump into the gates.
    \item[\gls{gm}:]
    The people usher you in, pushing the thick wooden gate closed behind you.
    The \gls{village} looks dusky-dim, but active.
    People are rushing out with bundles of arrows in hand, and the \gls{village}'s last two cows are mooing loudly in distress; but the children all know to keep quiet until the danger has gone.
    \item[Player 2:]
    (Mildrain)
    Can we get up to the wall?
    \item[\gls{gm}:]
    You go up the stairs, take $2D6+1$ Damage.
    \item[Player 1:]
    (Sinkmaul)
    Wait, are we both up the stairs?
    \item[Player 2:]
    (Mildrain)
    Wait, do I roll the Damage?
    \item[\gls{gm}:]
    No, I can\ldots that's `12 Damage'.
    (Mildrain)
    Okay, so I'm dead\ldots
  \end{description}
}{
  The initial description helps the players picture \pgls{village} under threat, then it all falls apart.
  \begin{itemize}
    \item
    The player wanted to know about how people access the wall, because that part of the arrangement was entirely unclear.
    They didn't mention their \gls{pc} actually going up.
    \item
    Responding with `\textit{yes, can get up the wall; do you?}' isn't much better.
    More description always works better.
    \item
    \Glsentrylongpl{pc} should take \pgls{action} before they die, not because `it's fair', but to make certain they're committed.
    Once they cast the dice, responsibility shifts from the \gls{gm} to the player.
  \end{itemize}
}


\end{multicols}


\commentary{
  \begin{description}
    \item[\gls{gm}:]
    The path comes to a cross-roads.
    Do you continue forwards, or go left, or right?
    \item[Player 1:]
    (Sinkmaul)
    Left?
    \item[Player 2:]
    (Mildrain)
    Left\ldots
    \item[\gls{gm}:]
    Now the path has a turn-off to the right\ldots
  \end{description}
}{
  This is awful, and the \gls{gm} should be fed to goblins, feet-first.

  If the players had previously said they want to approach `the tower', then the next part of that process is either arriving at the tower, or failing.
  The \gls{gm} can roll out a long description, to make the length of the march clear, and they can run various distracting \glspl{segment} on the way, but the moment the \glspl{pc} are free to act, the journey to their goal should resume.
}

\chapter{Elven Forests}
\epigraph{
  There was an old lady who swallowed a goat;

  Just opened her throat and swallowed a goat!

  She swallowed the goat to catch the dog,

  She swallowed the dog to catch the cat,

  She swallowed the cat to catch the bird,

  She swallowed the bird to catch the spider

  That wriggled and jiggled and tickled inside her,

  She swallowed the spider to catch the fly;

  I don't know why she swallowed a fly – perhaps she'll die!
}


\label{elvenForests}

\section{\Glsfmttext{ravencops}}
\label{ravencops}

\Gls{ravencops} received its name from the guttural bird-calls in the area.
Despite being a verdant forest, it feels bleak, and has little edible food.
All attempts at \gls{foraging} are at \gls{tn}~14.

\Gls{oathtower} sucks up most of the \glspl{mp} in the surrounding \gls{region}, which makes the air feel thin.
Unless \pgls{witch} is missing 6~\glspl{mp} or more, they can only receive 2~\glspl{mp} at the end of \pgls{interval}.%
\exRef{core}{Core Rules}{manaVacuum}

\printSideQuestsInRegion{ravencops}

\histEvent{105}{4}{%
  \Gls{LifeElder} loved the little paths snails make, and wanted to walk across them, but she was too big.
  To create her roads, she used Life \glspl{spell} to grow the snails to monstrous proportions.
  But once they were as big as a house, they just got stuck in the tall trees%
}

\begin{multicols}{2}

\commentary[t]{
  \begin{description}
    \item[\gls{gm}:]
    Ahead on the road\ldots one moment, \vpageref{ravencopsIntro}\ldots
    ``The echo of a distant crow's cry reaches you, just before fading.
    The trees look a kind of uniform-brown, without any mottling or variation.
    And the road feels as smooth as pond-scum.''
    \item[Player 1:]
    (Sinkmaul)
    Can we stand on that?
    \item[\gls{gm}:]
    Take it easy with \pgls{restingaction}, and set a die to six, now roll the other one.
    Since the \glsentrylong{tn} is 7, you'll pass this automatically, as long as you're walking with care.
    \item[Player 1:]
    (Sinkmaul)
    So the road enforces chill\ldots
    \item[\gls{gm}:]
    The trees obscure the tower, so you can't see which way gets you closest to the tower right now, but you can see a giant arachnid crawling forward as deep-brown as the trees, it looks nearly as tough.
    It spots you and sprints, front-legs raised.
    \item[Player 2:]
    (Mildrain)
    Are you sure I don't have fireball?
    \item[Player 1:]
    (Sinkmaul)
    I'm running in the opposite direction.
    \item[\gls{gm}:]
    Spend \pgls{ap}!
    You turn and move towards another giant snail as it opens its face-flaps towards you.
    \item[Player 3:]
    (Cleftbarb)
    Let's run back the way we came.
    \dicef{9}
    \item[\gls{gm}:]
    As you run back, the crawler gallops towards you like a horse, but as it rounds the corner, brownish-green liquid sprays onto it.
    It collapses into a ball, and two of its hind legs melt together.
    \item[Player 1:]
    (Sinkmaul)
    Is it dead?
    \item[\gls{gm}:]
    It uncurls, and moves into the forest, as the snail pursues it, belching the acid to clear its path.
    \item[Player 2:]
    (Mildrain)
    So this road\ldots
    \item[Player 3:]
    (Cleftbarb)
    Let's just press on.
  \end{description}
}{
  Explaining rules up-front never works.
  But making a small pause to show how to use something makes it \emph{relevant}.
}



\sidequest[ravencops,oathtower,sunway]{\Glsfmttext{enchantedLands}}

\sqpart[\gls{vlg}]{oathtower}% AREA
{Groaning Sepulchres}% NAME
{The troupe must walk quietly and avoid the groaning sepulchres}% SUMMARY

\histEvent{20}{2}{%
  \Glsentrytext{MindElder} finds the goblins overpopulating the area, and fears the day they run out of food; even the most powerful enchantments cannot withstand goblin hunger.
  He rewards loyal goblins with obscene amounts of food, which lets them grow and grow, into hobgoblins, and eventually into \glsfmtplural{ogre}.
  Trials include service in \glsfmttext{oathtower}, hunting dangerous creatures, and plenty of duels (which really helps reduce the population).
  Once the goblins ascend, \glsentrytext{MindElder} places them in an enchanted sleep inside a stone sepulchre (where the goblins can check on them)}

\begin{exampletext}
  Goblins have no natural height limits, so when they eat too much, they just grow and grow, until one day, without any clear cut-off point, people call them a `hobgoblin', and soon after, `\gls{ogre}'.

  When the goblins become \glspl{ogre}, \gls{MindElder} puts them into an enchanted sleep, and tells the goblins their big brothers will awaken when the time of grand feasting comes.
  The goblins must see the sleeping \glspl{ogre} from time to time, or they will suspect murder and betrayal, and even their oaths will not keep them passive.

  So the forest around \gls{oathtower} has slowly filled up with snoring sepulchres, and everyone must tread quietly, lest they wake and ask for breakfast\ldots
\end{exampletext}

\begin{boxtext}
  Past the trees, an arrow's flight away, a mossy tower stands as tall as a feasting hall turns on its end.
  A low groaning noise, like a distant earthquake, floods through the trees, surrounding you.
\end{boxtext}

\Gls{oathtower} has a lot of sepulchres dotted around it, often hidden by trees, and always with little paths leading towards them.
Each one has three \glspl{ogre}, cramped in together.
They stand as wide as a cottage, but not as tall, and the sound of snoring emanates for a few hours each day.

\paragraph{High-pitched noises near \gls{oathtower}}
have a 1 in 6 chance of waking \pgls{ogre}.
The chances increase by 1 for loud noises, or noises closer to the sepulchres.

In order to avoid waking the \glspl{ogre}, \gls{MindElder} has told the goblins to slay anyone making high-pitched noises, such as whistling or laughing.
Farting at any pitch is also banned, as it makes the goblins giggle, which then wakes the \glspl{ogre}.

\enchantedOgre[\NPC{\M\N}{`The Grave'}%
  {slate-coloured skin, with bright-blue eyes}% DESCRIPTION
  {stretches calves}% MANNERISM
  {deer with cheese}% WANTS
  \npcQuote{only asking, only asking\ldots}]

\enchantedOgre[\NPC{\F\N}{Kerning}%
  {bra made from human faces (it helps with running, not modesty)}% DESCRIPTION
  {chews leaves, then spits them out}% MANNERISM
  {\gls{crawler} soup}% WANTS
  \npcQuote{the road goes ever on, until it doesn't.
  `Dead end', they call it}]

\sqpart[\gls{vlg}]{ravencops}% AREA
{Goblins in the Quarry}% NAME
{\Glsfmttext{romeo} should be working, but needs to complete the perfect poem}% SUMMARY
\label{goblinQuarry}

\Gls{MindElder} has sent his son \gls{romeo} to oversee the goblins, excavating rock at the quarry, and cutting long slabs to construct more sepulchres.
But \gls{romeo} can't think of anything but the poem he needs to write, to tell his beloved how how he feels.
Unfortunately, his father raised him to be a perfectionist, which means he can't write perfect poetry, or good poetry, or bad poetry, or any poetry at all.

\begin{speechtext}
  What rhymes with snail?
  Mail, sail, bail\ldots hay-bail?
  Are hey-bails a thing?
  But `hey' is too informal.
  Better to say `hello'.
  `Hello-bail'\ldots no it sounds non-committal.
\end{speechtext}

So he stands looking at a blank sheet of paper, while twenty goblins ignore him, and bicker about pick-axes and the proper way to use a cart.

\begin{boxtext}
  In the near-distance, around this corner (or possibly two), someone, or something, is hitting metal on rocks.
  The metal sounds strange, butt probably iron.
  The rocks give that satisfying crack that rocks give with a long, clean cut.
\end{boxtext}

\paragraph{If the \glspl{pc} ask about the poem's recipient,}
\gls{romeo} explains he has no idea whom he loves, so he can only describe their mind.

\begin{speechtext}
  A quick wit, and very insightful in material matters -- able to tell the weight of a stone, bird, or an entire tree just by looking at it.
  And a deep critical thinker, not in any malicious sense, but nevertheless with cutting questions, whenever the need arises.
  This someone has wisdom beyond their years, though I don't know how old they might be, but still I'm sure of it\ldots

  I read their writing, and learned so much.
  They taught me how to move, and how things move.
  We write back and forth, we know each other so well.

  \ldots and yet, I cannot describe a face.
  But what's a face?
  Who cares?
  I just want to explain how I feel, and marriage to seal the deal.

  `Seal the deal'\\
  `An oath would make me less morose\ldots'

\end{speechtext}

\gls{romeo} does not know whom it's for, and explains he learned from his teacher by reading, and fell in love utterly.
His father doesn't approve of the `oathless' types, and he feels ashamed of loving such an `air-headed' person, despite all she's taught him.

\paragraph{If the \glspl{pc} help him with the poem,}
then he perks up and quickly finishes it, then asks them if they might try to find the recipient.

\paragraph{If the \glspl{pc} do nothing,}
\gls{romeo} remains at the quarry, thinking of the perfect words.

\paragraph{If the troupe commit crimes,}
the goblins will ignore them as long as they can.
They have taken an oath to dig rocks, and they will continue to dig until something shakes them from their oath.

\romeo

\showStdSpells

\paragraph{As the troupe leave,}
they notice \gls{romeo} using the Force \gls{sphere} to make the goblins' rocks lighter.%
\footnote{This tells the \glspl{pc} that \gls{romeo} understands the Force \gls{sphere}, which indicates a link to \gls{juliet}.
This becomes important later, in \nameref{oathlessLovers}, \vpageref{oathlessLovers}.}

\sqpart[\gls{vlg}]{sunway}% AREA
{Elven Steps}% NAME
{A hidden path leads to \glsfmttext{plateauGardens} above}% SUMMARY
\label{hiddenStairs}

In the causeway between the \gls{ravencops} forest and the \gls{plateauGardens}, a single plateau has a hidden stairway, going up.
\Glspl{crawler} cannot make much use of the narrow stairs, with occasional hand-holds for little fingers.
People who don't know about the stairs cannot usually see them, as every step blends into the tall rock-face from below.
But once someone notices the first step, they see the next, and then the next, and so on.

\paragraph{Spotting the rocks}
requires a \roll{Wits}{Vigilance} roll at \tn[12].
The \gls{tn} increases by~+1 in the rain, and by~+3 at night.

\sqpart[\gls{vlg}]{ravencops}% AREA
{The Icebox House}% NAME
{Underground elves live to guard food packed in ice}% SUMMARY
\label{iceboxHouse}

\histEvent{130}{3}{%
  With the old lich killed, \glsfmttext{MindElder} decided to settle down, build the perfect tower, and raise perfect children in a perfect land.
  Unfortunately, the children stole, fought, and disobeyed his orders to stay at home and keep safe.
  He made them all swear oaths to uphold the law, never harming any elf, nor taking property, nor singing out of key.
  Without the ability to sing out of key, none have learnt to sing, but this turned out to be an improvement, as \glsfmttext{MindElder} always enjoyed silence more than song%
}

Thick, glass tiles, a full step wide, pepper the land; these tiles are the roof-windows of elven houses.%
\exRef{stories}{Stories}{elvenGlades}
Smoke rises from a chimney, which juts out through a tree.
Three tall trees surround and hide a stone stairway, leading down to a little door.
Three short taps permits entry.

\begin{boxtext}
  One elf takes water, and whispers gently until the water sleeps, and turns to ice.
  Another prepares a little food, using a rapier's broken-off tip as a knife.
  The rest of the rapier remains mounted on the wall, above the fire; but these elves have no use for weapons.
  Harming people causes pain, and they have promised not to harm anyone.
\end{boxtext}

A little goblin sleeps in a hammock, muttering in his sleep.
`\textit{Carapace pies, tentacle-fry\ldots}'
The elves will have something cooked for him by the time he wakes, and then he must fetch more water, using the ornate bucket, carved from carapace, with an abstract map of the land chiselled around its side.

\elf

\paragraph{If the \glspl{pc} ignore the smoking chimney,}
that's fine.
This \gls{segment} does not advance any plot, and there is nothing the \glspl{pc} need to do.
This \gls{segment} exists simply because elves live in \gls{enchantedLands}, and they will greet guests who knock on their door in a friendly manner, and start telling long, boring stories.

\paragraph{If the \glspl{pc} inspect the bucket-map}
they receive a +2~Bonus to all \gls{navigation} checks within the surrounding \glspl{area}.

\paragraph{If they ask for help,}
they receive it, as long as they make small, reasonable requests.

\paragraph{If they ask questions,}
the elves answer happily.
They know most of the history of the area (find the summary \vpageref{chronologicalEvents}).

\sqpart[\squash\gls{vlg}]{ravencops}% AREA
{Subtle Sepulchres}% NAME
{Most goblins have forgotten about these \glsfmtplural{ogre}}% SUMMARY

\Gls{MindElder} ordered this sepulchre made before he understood how easily the dreams of \glspl{ogre} break.
It houses three, who snore quietly.

\sqpart[\gls{vlg}]{ravencops}% AREA
{The House of Grand Stories}% NAME
{Underground elves tell painless stories in perfect rhythm}% SUMMARY

Beehives buzz around a flowery garden.
Three large boulders (which look quite out of place) hide stairs down to a long hall, where perfect elves tell perfect stories of perfect people.
The stories rhyme in a precise pattern, and follow the hero's journey exactly.
The characters in the stories do no wrong, and have no fights, because fighting hurts people, and the storytellers never think about hurting people.

The elven home has three chambers, for three elves, and a central area for cooking.
Goblins occasionally visit, bringing supplies of snail-meat and stolen vegetables.
The various cupboards also have $1D6-3$ of the following items:

\begin{itemize}
  \item
  Smoked meats (usable as a day's \glspl{ration}).
  \item
  Stormy moonlight from a storm, captured in a large, glass, phial (usable as a Water \gls{ingredient}).
  \item
  Auroch hooves (usable as an Earth \gls{ingredient}).
\end{itemize}

\elf

\end{multicols}

\sqpart[\gls{vlg}]{ravencops}% AREA
{Lake Rancid}% NAME
{The old lake has become a bog}% SUMMARY

\begin{exampletext}
  \Gls{sunderedForest}'s malformed ground rerouted a river which once fed this lake.
  Since then, it became stagnant, then a basin for giant snails to bathe, shit, and mate.
  Soon after, even the snails stopped coming.
\end{exampletext}

\begin{boxtext}
  Not far ahead, a stench.
  It is the most basic and generic stench, the intersection of every standard stench.
  The pure, basic form of stench.

  The surrounding trees seem healthy, but extra dark.
\end{boxtext}

If the \glspl{pc} instigate \pgls{flood} inside \gls{sunderedForest}, it travels through this swamp, replenishing it, bringing the lake back to life, and throwing the stench through \gls{ravencops} and beyond.

\stopcontents[sq]


\commentary{
  \begin{description}
  \item[\gls{gm}:]
  Some miles on, and the Sun's high.
  You keep walking with that `squelch, squelch, squelch' along the snail-path, and spot a potato on the path ahead, held in someone's arms.
  \item[Player 3:]
  (Cleftbarb)
  Hide!
  \item[\gls{gm}:]
  That's \roll{Wits}{Stealth}, \glsentrylong{tn}~6.
  \item[Player 2:]
  (Mildrain)
  Scared of a potato-wielding marauder?
  Okay then\ldots\dicef{6}.
  \item[\gls{gm}:]
  Hiding in the thick darkness which surrounds the road, you wait until the small figure passes, grumbling to himself in the \gls{tradeTongue}.
  Once silence returns, the march continues, through hours of nearly-identical woodland.
  By the time you approach the tower, it's nearly night, but the dusk's light highlights a stone structure in the forest.
  A long cube, a rumbling stone box.
  \item[Player 2:]
  (Mildrain)
  Ignore!
  \item[Player 1:]
  (Sinkmaul)
  Right, `tower'.
  \item[Player 3:]
  What if it's treasure?
  \item[\gls{gm}:]
  The road opens, revealing a shining lake, with the tower in the centre\ldots
  \end{description}
}{
  The pacing sped up suddenly as the players hopped through three \glspl{segment} within a couple of minutes.
  The players still received information (giant snails reduce the \gls{monster} population), gained questions (though nobody has actively asked about the source of the giant potato), and have a new location on the map which they might return to.

  That last \gls{segment} with the `stone box' (sepulchre) has the `\gls{vlg}' symbol listed next to it.
  This indicates it should go onto the map, with the assumption it's always been there.
}

\section{\Glsfmttext{oathtower}}
\label{oathtower}

\Gls{oathtower} stands six storeys tall in the centre of a shining lake.
People can see the tower's top for miles around wherever the tree coverage is not too thick.

Nobody near \gls{oathtower} can breathe in the local mana, because mana always heads towards the largest vacuum.
And since \gls{MindElder} spends so many \glsentrylongpl{mp} each day, he usually has the most missing by the end; so everything flocks towards him, leaving the air feeling stagnant even in a storm.

\printSideQuestsInRegion{oathtower}

\begin{multicols}{2}

\subsection{The Lake}

\noindent
Whistling cane grows around the South side of the lake, giving it an eerie sound whenever the wind blows.%
\exRef{judgement}{Judgement}{whistlingCane}
The elves use it to make paper, while the hobgoblins use it as an instrument.

\begin{boxtext}
  The \gls{oathtower}'s tall, wooden, door has an identical shade of grim-brown to its stone walls.
  The tower stands a stone's throw away, at the centre of a shining lake.
  A single, short, figure in a green tunic stands in a boat by the tower.
  Her ears are long, her skin maggoty-white, and her eyes full of suspicion.
\end{boxtext}

\enchantedGoblin[\npc{\F\N}{Boat Goblin}]

The door inside leads to a hall with a staircase and three doors.
Each storey is identical, so if the \glspl{pc} enter a door, roll to determine what's inside the first three storeys:

\begin{dlist}
  \item
  $1D6$ hobgoblins with a deck of cards, debating the importance of rules in games.
  \item
  $1D6$ hobgoblins mending a boat for the lake around \Glsfmttext{oathtower}.
  \item
  Hobgoblin weapon storage, with $1D6 \times 2$ shortswords.
  \item
  A kitchen with $1D6$ elves.
  They debate politely, but their words hide fierce insinuations about how the others once burnt an egg, or used the wrong type of wood for cooking snail meat.
  \item
  Light supplies, with $1D6 \times 3$ torches, and $1D6 \times 4$ candles.
  \item
  Meat storage, with $1D6$ meals' of \gls{crawler} eggs in salt, $1D6$ meals of snail-meat, $1D6$ meals' worth of vegetables (half of them gigantic and stolen by goblins from the garden plateaus), and $1D6$ meals of auroch meat.
\end{dlist}

\hobgoblin

The hobgoblin guards have \gls{armour} made from \pgls{basilisk}'s hide \gls{covering} their torso, and a mottled-brown helmet made from the shell of a giant snail.

Storeys 3, 4 and 5 have different contents:

\null
\begin{dlist}
  \item
  A small study, with books on poetry, and fantastical erotica where rabbits ride foxes to battle snakes.
  \item
  A cupboard of \glspl{ingredient}, including $1D6-3$ woodspy beaks (Water \glspl{ingredient}), $1D6-3$ \gls{crawler} spinnerets (Fate \glspl{ingredient}), and $1D6-3$ phials of human blood (also Fate \glspl{ingredient}).
  \item
  A lounge, with $1D6-3$ elves lounging.
  These elves have taken so many oaths concerning good behaviour that they are practically incapable of action, and need goblins to tend to them daily.
  \item
  An in-house outhouse which empties into the lake below.
  \item
  \gls{romeo}'s bedroom (with three books on poetry placed randomly around the room).
  \item
  \gls{MindElder}'s bedroom (with a balcony, a bed, and nothing else).
\end{dlist}

\subsubsection{\Glsfmttext{MindElder}'s Routine}
begins at dawn, as he casts spells to learn about the kinds of minds wander his domain (`\textit{does anyone have hurtful plans?  Does anyone think about theft?}').
Over the day, goblins continuously visit him to renew their oaths (to uphold the laws about theft, and high-pitched noises) so that he can enchant the oaths and keep the goblins honest.
If \gls{MindElder} still has any \glsentrylongpl{mp} left by the evening, he practices new spells.

\MindElder

\showStdSpells

\end{multicols}



\commentary{
  \begin{description}
  \item[Player 1:]
  Okay, so the elf at the quarry asked us to deliver this letter, and showed he knows some kind of levitation spell, now this elf is growing floating plants.
  Is she the one?
  \end{description}
}{
  The eye of the story moves North, and finds a stone sepulchre.
  Then it moves West and finds a quarry, surrounded by giant snails.
  The land seems full of life, because the \glspl{sq} from \autoref{elvenForests} place the living things in front of the players.
}

\subsection{\Glsfmttext{plateauGardens} Walls}

\begin{boxtext}
  \Gls{sunway} comes to a harsh stop at a wall of perfectly vertical, and strangely solid, earth.
  It stands half as tall as \pgls{broch}, and tree-branches sway even higher up, their roots poking a crown around the edges.

  The great wall continues down this Sunlit part of the sparse forest farther than you can see, in both directions.
  But it also has crevices, or cracks, or some darkness dotted along it.
  Through one crack, the silhouette of a half-dead tree droops down the wall half way.
\end{boxtext}

The raised plateaus of the \gls{plateauGardens} form this wall, and the cracks which separate the gardens are \gls{shadepaths}.
Dangling foliage lets bear-light elves clamour up, but heavier people risk breaking the delicate foliage.

\paragraph{When someone climbs,}
they roll $1D6 + \gls{weight}$; a roll of 12 or more breaks the branch (or root, or vine) and they tumble down, and hit the ground four \glspl{step} below, while earth or plants dislodge and falls on them after the fall.

The total Damage is 2 plus the character's Strength%
\exRef{core}{Core}{falling}
and everyone standing close receives the die just rolled as Damage too.

\sidequest[sunway]{Places to See in \glsfmttext{sunway}}

\sqpart[\gls{vlg}]{sunway}% AREA
{Elven Steps}% NAME
{A hidden path leads to \glsfmttext{plateauGardens} above}% SUMMARY
\label{hiddenStairs}

In the causeway between the \gls{ravencops} forest and the \gls{plateauGardens}, a single plateau has a hidden stairway, going up.
\Glspl{crawler} cannot make much use of the narrow stairs, with occasional hand-holds for little fingers.
People who don't know about the stairs cannot usually see them, as every step blends into the tall rock-face from below.
But once someone notices the first step, they see the next, and then the next, and so on.

\paragraph{Spotting the rocks}
requires a \roll{Wits}{Vigilance} roll at \tn[12].
The \gls{tn} increases by~+1 in the rain, and by~+3 at night.



\section{\Glsfmttext{shadepaths}}
\label{shadepaths}

These deep ditches receive few \glspl{mp}, so the atmosphere feels stagnant.
The troupe can only receive 1~\glspl{mp} at the end of each \gls{interval}, or 0~\glspl{mp} if \gls{LifeElder} wanders nearby.

\printSideQuestsInRegion{shadepaths}

\histEvent{100}{5}{%
  To stop the giant snails eating all the vegetables, \gls{LifeElder} cracked the land, sundering the soil and creating raised plateaus, where she and the other elves could live, cultivating plants.
  Meanwhile, the snails remained in the lower regions.
  Unfortunately, the elves could not get from one plateau to the other due the tall, sheer walls%
}




\begin{multicols}{2}
\sidequest[shadepaths,plateauGardens,sunway,oathtower]{The Snail Tamer}

\sqpart{shadepaths}% AREA
{Debating Parsnips}% NAME
{\Glsentrytext{SnailTamer} meanders home, wondering if he should make parsnip or potato soup}% SUMMARY

\Gls{SnailTamer} has missed a turn-off in the maze of canyons, and can't remember which direction he's going.
However, he's more concerned with the question of what kind of soup he should make once he returns home.

\begin{boxtext}
  A mile breeze blows.
  A voice in the distance sounds like it's practising Elvish vocabulary, and you recognize the word for `parsnip'.
  Around the corner, \gls{SnailTamer} appears, as naked as the Sun, then stops and stares, as if trying to recognize you.
\end{boxtext}

\begin{speechtext}
  So, like.  Guys?  Yea, you guys.  Okay then.  So we don't make people do things here, no laws about fabrics over your body -- you just walk about however you like, okay?

  Potatoes are always great, but you can't eat potatoes all day.
  It's been ages since I had parsnip soup actually.
  Once I get home I'll have a large bowl of soup, then take that soup-nap where you're just full of soup and curl up on your hammock.
  You know that one?
  Soup-naps are great.
\end{speechtext}

\Gls{SnailTamer} continues meandering and mumbling until someone snaps him out of it.
If \pgls{pc} makes a clear request help getting out of the canyon and onto the plateau, he agrees, says `\textit{let's go then}', and starts walking while whistling a sad song.
Soon enough, an elf in one of the raised gardens will hear him, and lower a few ripe bean-vines.

The bean-vines can take a total \gls{weight} of 10.
The \glspl{pc} can gauge the weight limit with an \roll{Intelligence}{Survival} roll at \tn[8].

\sqpart{plateauGardens}% AREA
{\glsentrysymbol{vlg}~Shell Escape}% NAME
{Giant potato sprouts vomit from \gls{SnailTamer}'s house, it may fall any moment!}% SUMMARY

If you leave potatoes in a bag with a little Sunlight, they will grow arms and try to find earth to implant themselves into.
Giant, magical, elf-potatoes are entirely normal in this regard, but much bigger.

\begin{boxtext}
  Across the next bean-vine bridge, a giant snail-shell stands, dangerously close to falling into the canyon.
  Sunlight shines through the shell, displaying an upper floor with a desk, hammock, some cloaks, and jars.

  Pale tendrils push out the door, then have a hard twist down into the earth, like a limb trying to grasp the ground.
\end{boxtext}

Anything the \glspl{pc} do may push the house over the edge, and into \gls{shadepaths}.

\paragraph{Get an item from the house:}
\roll{Dexterity}{Stealth} at \tn[12] (+1 per item).

\paragraph{Righting the house, so it will not fall over:}
\roll{Strength}{Cultivation}, \tn[12].

\paragraph{Using the vines to secure the house:}
\roll{Intelligence}{Cultivation}, \tn[10].

\paragraph{If the troupe wait long enough,}
\gls{SnailTamer} will arrive (in the next \gls{segment}), and he will see what has become of his house.
Whether or not his house survived, he will need a `walk-nap', to recover from his walk.

\sqpart{plateauGardens}% AREA
{Choosing a Shell}% NAME
{\glsentrysymbol{day}~\Glsentrytext{SnailTamer} to pick a new snail}% SUMMARY

\Gls{SnailTamer} needs to pick a new snail so he can get about safely, but he can't decide which snail is best.
One seems too unbalanced for him to put the wooden carrot-rod onto, another looks `too old', then the next is pregnant (which will cause problems before long).
So he just sits there, beside his long wooden rod and a small leaf-sack of oversized vegetables (usable as 3 days' \glspl{ration} after cooking).

\begin{boxtext}
  The sky clears, revealing bright blue.
  \Gls{SnailTamer} sits on a ledge not far away, overlooking \gls{shadepaths}.
\end{boxtext}

\paragraph{If \gls{SnailTamer} still suffers from the \gls{disgnome} stings,}
the \glspl{pc} can usher him along with a \roll{Charisma}{Cultivation} roll at \tn[10].
Otherwise, he'll pick a snail within \pgls{interval}.

\paragraph{Once \gls{SnailTamer} has a new mount,}
he attaches the `leading rod' with a cork-screw, then takes another nap.

\begin{speechtext}
  The screw doesn't hurt the snail as long as you don't go too deep.
  It's just like shoeing a horse.

  What a big day, eh?
  Well, I'm going to have a top-back-nap, and dream up a name.
  Snails don't need labels, but they also don't object to them!

  Top-back-naps are the best, because you never know where you'll wake up.
\end{speechtext}

\sqpart{sunway}% AREA
{Urgent Delivery}% NAME
{\glsentrysymbol{day}~\Glsentrytext{SnailTamer} races to deliver a letter on snail-back}% SUMMARY

\Gls{LifeElder} was bound to send `her most reliable messenger', and that's \gls{SnailTamer}.
He's not the fastest, or well-spoken, nor is he routinely sure of where he is.
However, he has delivered two messages, which means his success rate is $\frac{2}{2}$, and 100\%.

So she charged him with delivering the letter to \gls{MindElder}, and he's finally en route.

\begin{boxtext}
  Crashing trees interrupts the patter of rain.
  In the distance, through the sparse forest, a giant snail has stopped to feed on the vegetation.
  The rider wears a giant primrose hat, and sits by a long stick with a potato on the end.
\end{boxtext}

\sqpart{oathtower}% AREA
{The Letter of the Law}% NAME
{\glsentrysymbol{afternoon}~\Glsfmttext{SnailTamer} wants someone else to deliver the letter, so he can avoid any binding enchantments}% SUMMARY

At \gls{oathtower}, \gls{SnailTamer} has changed his mind about entering, and decided to hide behind his new mount.
This plan would have worked well, except for the large wooden pole mounted on the snail's back, indicating that it has a rider.

\begin{boxtext}
  Mist obscures the pasture, and \gls{oathtower}.
  The lake has vanished from view.
  To the left, in the distance, another path through the forest has a giant snail, with a long stick attached to the top of its shell.
\end{boxtext}

\paragraph{If the \glspl{pc} agree to deliver the letter,}
\gls{SnailTamer} hands his satchel, which feels dank and smells musty, as it's full of moss.

\begin{speechtext}
  Thanks for taking the satchel up.
  That big tower -- it just scares me, all that nasty, tight, energy wrapped up in one hideous pile of rocks.

  Time I lie down, gather some energy with a tactical-nap.

  You know what they say, yea?
  There's a nap for everything.
\end{speechtext}

Once opened, the moss is in the shape of the letter `F' (which is the letter \gls{LifeElder} wanted to send).



\end{multicols}

\section{\Glsfmttext{plateauGardens}}
\label{plateauGardens}

Elves live along the plentiful gardens, where \gls{LifeElder} uses her spells to encourage vegetables to grow massive in half the usual time, or even less.
Most of the elves have almost no chance to practice any spells, as the \gls{region} has almost no \glspl{mp}; but they don't care.
They learn to sing, construct and deconstruct new ideas, then gossip about infidelities among local robins and kestrels.

\Gls{LifeElder} casts spells every \gls{interval}, so she consumes most of the \glsentrylongpl{mp} in the \gls{region} before anyone else can.
When she's far away, the troupe can divide 2~\glspl{mp} among themselves, and when she's nearer, they only regain 1~\gls{mp} each \gls{interval}.
And, of course, once the troupe regain 0~\glspl{mp}, they should understand, that \gls{LifeElder} must be close.

\printThreadsInRegion{plateauGardens}

\needspace{10\baselineskip}
\begin{multicols}{2}

\subsection{What's Up There?}

Each of \glspl{plateauGardens} is around 250~\glspl{step} across, and holds an abundance of plants.
As the \glspl{pc} step onto \pgls{plateauGardens}, roll three dice, and apply every result.

\begin{dlist}
  \item
  A vine-bridge connects to another plateau.
  It can hold up to \pgls{weight} of 20 before collapsing.
  
  The next plateau holds a snail-shell house, with tools to make snail-saddles below, and sleeping quarters above.
  \item
  A ripened vine-bridge with plenty of beans stretches to another plateau.
  It holds up to \pgls{weight} of 10 before collapsing.
  \item
  \Glspl{disgnome} hide among the flowers.
  Digging up vegetables results in a sting from its roots which saps $1D3$~\glspl{ep}.
  \item
  Fresh vegetables -- tomatoes, potatoes, and carrots -- grow in abundance.
  The garden holds $1D6 \times 10$ days' \glspl{ration}.
  \item
  Fruit trees galore!
  Figs, plums, hazelnuts, and peaches, all ripe.
  The garden holds $1D6 \times 5$ days' \glspl{ration}.
  \item
  Elven song from the next plateau, where $1D3$ elves sing together.
\end{dlist}

Not every plateau has a vine-bridge, so if the troupe manage to ascend, they will not necessarily be able to journey across \glspl{plateauGardens}.

\thread[shadepaths,plateauGardens]{The Tao Mistress}

\segment{plateauGardens}% AREA
{Lazing Elves}% NAME
{A group of elves discuss Philosophy}% SUMMARY

Helin, Huon, and Lav\"e discuss Philosophy (in Elvish), but slowly because the local \gls{disgnome} has dulled their wits.%
\footnote{Apply a -5~Penalty to the elves' Wits, unless the \glsfmtplural{pc} remove \glsfmttext{disgnome}.}
They take a brief interest in the \glspl{pc}, the return to their conversation.
They answer questions with more questions, then argue both sides of their own question.

To avoid vexing the players (while still vexing the \glspl{pc}) you might want to abstract this process with a \roll{Wits}{Empathy} (\tn[10], or 7 for characters who speak Elvish).
A tie results in one proper answer in exchange for the \glspl{pc} leaving the elves in peace, and each Margin grants another answer.

\begin{itemize}
  \item\it
  Do you live here?
  \item[\adforn{51}]\sl
  Ma quetill\"e firyon lamb\"e?
  \item[\adforn{53}]\bf
  I am here and I live.
  I suppose you live here too, for now.
  \item\it
  Can you tell us\ldots
  \item[\adforn{53}]\bf
  Are you `sea humans'?
  \item[\adforn{52}]\bf\sl
  No, sea humans wear sails as clothes. 
  These are `high humans', who live in stone constructions, and make barrels of rotten auroch milk.
  \item\it
  Right, we live in houses, so can you tell us\ldots
  \item[\adforn{53}]\bf
  You're thinking of dwarves.
  These are `forest humans', look at their hunting tools.
  They use these to stab aurochs.
  \item\it
  Who made the snails?
  \item[\adforn{52}]\bf\sl
  Nobody can make a snail except another snail.
  But it's a bit of a snail-and-egg problem.
  Was there a first snail, or a first egg?
  \item\it
  Okay, but who made the snails \emph{big}?
  \item[\adforn{51}]\sl
  Ma firyar nar orqui?
  \item\it
  Sorry, I don't speak Elvish.
  \item[\adforn{53}]\bf
  Sancossi nar firyar, nan sin\"e firyali umir sancossi.
\end{itemize}

\elf[\npc{\T[2]\M\M\El}{Helin \& Huon}]

\showStdSpells

\elf[\npc{\F\El}{Lav\"e}]

\showStdSpells

\segment{plateauGardens}% AREA
{Interview with the Tao Mistress}% NAME
{\Glsfmttext{LifeElder} wanders \glsfmttext{shadepaths}, and her answers seem strange}% SUMMARY

While the \glspl{pc} are in \gls{plateauGardens}, they see a small, red-haired elf below, wandering naked and humming to herself.
Other elves may identify her as the source of all the change in the landscape, but will not give her a name (except to say `Hi').

\Gls{LifeElder} wanders, and sometimes sings in a way that somehow matches the breeze.
She stops to ponder a flower, then alters a seed so the flower will grow purple.

She speaks quickly, and cryptically, and gives deep thought to every word someone says, but quickly tires of conversation.

\begin{itemize}
  \item\it
  What's your name?
  \item[\adforn{54}]\bf
  I'm not really into labels.
  \item\it
  Did you make the snails?
  \item[\adforn{54}]\bf
  Nobody can make a snail.
  Snails are completely impossible, unless you have a snail.
  Is that a paradox?
  \item\it
  Did you make the snails grow big?
  \item[\adforn{54}]\bf
  No, but I suggested it.
  Making someone do something sounds violent.

  Have you had a ride on a snail yet?
  \item\it
  Could you stop making large snails?
  \item[\adforn{54}]\bf
  I could.
  I've not been doing a lot of things, what's one more thing to not do?

  But how will that affect the \glspl{consumer}?
  \item\it
  Who?
  \item[\adforn{54}]\bf
  The sancossi --- who will they eat?
  \item\it
  `Goblins?'
  I don't care what the goblins eat, could you just stop\ldots wait did you say `who'?
  \item[\adforn{54}]\bf
  No, wrong word.
  I haven't spoken Gnomish in a while.
  It should be `whom', shouldn't it?
  \item\it
  We're human.
  Wait! what do you mean `whom the goblins will eat'?
  \item[\adforn{54}]\bf
  You speak Gnomish well for a human.
  You know so much!
  \item\it
  Thanks.
  Listen.
  I know you're busy, but\ldots
  \item[\adforn{54}]\bf
  Come back when you know nothing, and I will teach you nothing.
\end{itemize}

\LifeElder

\showStdSpells

\segment[\squash]{shadepaths}% AREA
{Misty Way}% NAME
{The troupe must coordinate by sound and deduction}% SUMMARY

The \glspl{pc} can see nothing in the causeway, as mist hangs low.
They may have to make a navigation roll just to move about, and projectiles suffer double the normal range penalties.

\segment{shadepaths}% AREA
{A Voice from on Hi}% NAME
{\Glsfmttext{LifeElder} looks down at the party, ready to converse again}% SUMMARY

While the troupe wander through \gls{shadepaths}, \gls{LifeElder} has been searching \glspl{plateauGardens} for \glspl{ingredient} to make \glspl{boon} so she can make more giant snails.

\begin{boxtext}
  A voice from above floats down into the shadows.
  The grey-haired elf looks down, eyes wide-open, and she says ``\textit{I wonder if you found enough food}''.
\end{boxtext}

As before, she remains distracted and indifferent, but potentially helpful if the \glspl{pc} seem peaceful.

Here are the kinds of things she says:

\begin{itemize}
  \item[\adforn{54}]\bf
  What is a human's favourite type of rain?

  Or do humans like all rain equally?
  \item[\adforn{54}]\bf
  Endings have nothing to do with what happens.
  An ending is ultimately a manifestation of values.
  \item[\adforn{54}]\bf
  If it never rained, the plants wouldn't grow.
  \item[\adforn{54}]\bf
  Possessions are just \emph{things}.
  Don't let your things control you -- be free!
\end{itemize}

\paragraph{She leaves soon after,}
hoping to gather \glspl{ingredient} to make 2~\glspl{boon} so she can make more giant snails.

\spell{Hyalmahta}% Name
  {Detailed, Devious, Duplicated}% Enhancements
  {Wax}% Action
  {Earth, Water}% Spheres
  {available plant quantities}% Resist with
  {The caster thinks about past dreams of salad and \arabic{spellTargets} snails in the vicinity begin to eat and grow.
  Within a few days, they reach the size of a human, and after one \showOnset\ reach the size of a house.
  Each snail has Strength~+5, Speed~-3, \gls{dr}~5 (or 10 on the shell), along with the abilities Viscid and Acid Spray}% Description
  {}


\segment{plateauGardens}% AREA
{Sudden Chills}% NAME
{All the mana vanishes, as \glsfmttext{LifeElder} wanders nearby}% SUMMARY

\Gls{LifeElder} wanders nearby, so all the \glspl{mp} in the area vanish.
For the next two \glspl{interval}, nobody in the \gls{area} receives a single drop.
The \glspl{pc} (or friendly elves) may be able to use this to locate \gls{LifeElder} with spells.%
\exRef{core}{Core Rules}{ManaVoidSpell}

\end{multicols}


\chapter{Loose Threads}
\epigraph{
  Can you name the nameless one?

  Can you shoe a snail?
}

\label{looseThreads}


\section{Meandering Tails}

\begin{multicols}{2}

\sidequest[plateauGardens,ravencops,oathtower,sunway]{Oathless Lovers}
\label{oathlessLovers}

\noindent
\Gls{romeo} and \gls{juliet} have never met, but still managed to fall in love through a series of song-spells carved into trees in \gls{sunway}.
Each one spends months composing a new verse, then commits to the dangerous journey to carve more glyphs in their secret, shared spot.%
\footnote{Find that spot \vpageref{sunwayGlyphs}.}

Unfortunately, \gls{romeo} only understands love as an oath to be kept, while \gls{juliet} only understands oaths as an insult and a violation.
If the \glspl{pc} nudge them together, the two soon form a plan to meld the two elven lands into one, with \pgls{spell} so powerful it will destabilize the land, and bring all other threads to a dramatic and confusing conclusion.

\paragraph{If the couple unite,}
they discuss the grand spell, and request the \glspl{pc} help them gather the necessary \glspl{ingredient}.
They need sixteen in total:

%
\null
\begin{itemize}
  \item
  \Repeat{4}{\sqn}\quad 4 Earth \glspl{ingredient}
  \item
  \Repeat{4}{\sqn}\quad 4 Fire \glspl{ingredient}
  \item
  \Repeat{4}{\sqn}\quad 4 Water \glspl{ingredient}
  \item
  \Repeat{4}{\sqn}\quad 4 Fate \glspl{ingredient}
\end{itemize}

\begin{exampletext}
  We want to unite the lands, making all one, a single people and place.
  This \gls{spell} will reframe everyone's problems into a memory.
\end{exampletext}

The couple won't be able to explain the \gls{spell}, except in abstract terms (`united, entirely', `the grand crossing of perspective').
See `\nameref{grandSpell}' \vpageref{grandSpell} for the complete description.

\sqpart[\gls{vlg}]{oathtower}% AREA
{Groaning \Glsfmtplural{sepulchre}}% NAME
{The troupe must walk quietly and avoid the groaning sepulchres}% SUMMARY

\histEvent{20}{2}{%
  \Glsentrytext{MindElder} finds the goblins overpopulating the area, and fears the day they run out of food; even the most powerful enchantments cannot withstand goblin hunger.
  He rewards loyal goblins with obscene amounts of food, which lets them grow and grow, into hobgoblins, and eventually into \glsfmtplural{ogre}.
  Trials include service in \glsfmttext{oathtower}, hunting dangerous creatures, and plenty of duels (which really helps reduce the population).
  Once the goblins ascend, \glsentrytext{MindElder} places them in an enchanted sleep inside a stone sepulchre (where the goblins can check on them)}

\begin{exampletext}
  Goblins have no natural height limits, so when they eat too much, they just grow and grow, until one day, without any clear cut-off point, people call them a `hobgoblin', and soon after, `\gls{ogre}'.

  When the goblins become \glspl{ogre}, \gls{MindElder} puts them into an enchanted sleep, and tells the goblins their big brothers will awaken when the time of grand feasting comes.
  The goblins must see the sleeping \glspl{ogre} from time to time, or they will suspect murder and betrayal, and even their oaths will not keep them passive.

  So the forest around \gls{oathtower} has slowly filled up with snoring sepulchres, and everyone must tread quietly, lest they wake and ask for breakfast\ldots
\end{exampletext}

\begin{boxtext}
  Past the trees, an arrow's flight away, a mossy tower stands as tall as a feasting hall turns on its end.
  A low groaning noise, like a distant earthquake, floods through the trees, surrounding you.
\end{boxtext}

\Gls{oathtower} has a lot of sepulchres dotted around it, often hidden by trees, and always with little paths leading towards them.
Each one has three \glspl{ogre}, cramped in together.
The sound of snoring emanates for a few hours each day.

\iftoggle{verbose}{
  \gls{vlg}~Once the \glspl{pc} approach \gls{oathtower}, place \pgls{sepulchre} on the map on \vpageref{feylands}.
  It should be inside the forest, not far from the \glspl{pc}.
}{}

\paragraph{High-pitched noises near \gls{oathtower}}
have a 1 in 6 chance of waking \pgls{ogre}.
The chances increase by~1 for loud noises, or noises closer to the \glspl{sepulchre}.

In order to avoid waking the \glspl{ogre}, \gls{MindElder} has told the goblins to slay anyone making high-pitched noises, such as whistling or laughing.
Farting is also banned, as it makes the goblins giggle, which then wakes the \glspl{ogre}.

\enchantedOgre[\NPC{\M\N}{`The Grave'}%
  {slate-coloured skin, with bright-blue eyes}% DESCRIPTION
  {stretches calves}% MANNERISM
  {deer with cheese}% WANTS
  \npcQuote{only asking, only asking\ldots}]

\enchantedOgre[\NPC{\F\N}{Kerning}%
  {bra made from human faces (it helps with running, not modesty)}% DESCRIPTION
  {chews leaves, then spits them out}% MANNERISM
  {\gls{crawler} soup}% WANTS
  \npcQuote{the road goes ever on, unless it doesn't.
  `Dead end', they call it}]


\sqpart[\gls{vlg}]{ravencops}% AREA
{Goblins in the Quarry}% NAME
{\Glsfmttext{romeo} should be working, but needs to complete the perfect poem}% SUMMARY
\label{goblinQuarry}

\Gls{MindElder} has sent his son \gls{romeo} to oversee the goblins, excavating rock at the quarry, and cutting long slabs to construct more sepulchres.
But \gls{romeo} can't think of anything but the poem he needs to write, to tell his beloved how he feels.
Unfortunately, his father raised him to be a perfectionist, which means he can't write perfect poetry, or good poetry, or bad poetry, or any poetry at all.

\begin{speechtext}
  What rhymes with snail?
  Mail, sail, bail\ldots hay-bail?
  Are hey-bails a thing?
  But `hey' is too informal.
  Better to say `hello'.
  `Hello-bail'\ldots no it sounds non-committal.
\end{speechtext}

So he stands looking at a blank sheet of paper, while twenty goblins ignore him, and bicker about pick-axes and the proper way to use a cart.

\begin{boxtext}
  In the near-distance, around this corner (or possibly two), someone, or something, is hitting metal on rocks.
  The metal sounds strange, butt probably iron.
  The rocks give that satisfying crack that rocks give with a long, clean cut.
\end{boxtext}

\paragraph{If the \glspl{pc} ask about the poem's recipient,}
\gls{romeo} explains he has no idea whom he loves, so he can only describe their mind.

\begin{speechtext}
  A quick wit, and very insightful in material matters -- able to tell the weight of a stone, bird, or an entire tree just by looking at it.
  And a deep critical thinker, not in any malicious sense, but nevertheless with cutting questions, whenever the need arises.
  This someone has wisdom beyond their years, though I don't know how old they might be, but still I'm sure of it\ldots

  I read their writing, and learned so much.
  They taught me how to move, and how things move.
  We write back and forth, we know each other so well.

  \ldots and yet, I cannot describe a face.
  But what's a face?
  Who cares?
  I just want to explain how I feel, and marriage to seal the deal.

  `Seal the deal'\\
  `An oath would make me less morose\ldots'

\end{speechtext}

\Gls{romeo} does not know whom it's for, and explains he learned from his teacher by reading, and fell in love utterly.
His father doesn't approve of the `oathless' types, and he feels ashamed of loving such a lawless person, despite all she's taught him.

\paragraph{If the \glspl{pc} help him with the poem,}
then he perks up and quickly finishes it, then asks them if they might try to find the recipient.

\paragraph{If the \glspl{pc} do nothing,}
\gls{romeo} remains at the quarry, thinking of the perfect words.

\paragraph{If the troupe commit crimes,}
the goblins will ignore them as long as they can.
They have taken an oath to dig rocks, and they will continue to dig until something shakes them from their oath.

\romeo

\showStdSpells

\paragraph{As the troupe leave,}
they notice \gls{romeo} using the Force \gls{sphere} to make the goblins' rocks lighter.%
\footnote{This tells the \glspl{pc} that \gls{romeo} understands the Force \gls{sphere}, which indicates a link to \gls{juliet}.}

\sqpart[\gls{vlg}]{sunway}% AREA
{Sunlit Glyphs}% NAME
{Elven glyphs, carved in wood, describe a song of ritual magic}% SUMMARY
\label{sunwayGlyphs}

\begin{exampletext}
  \Gls{juliet} began coming here to practice using the Force \gls{sphere} through ritual songs, and carved the notes into local trees to remember them.
  Years later, \gls{romeo} found the patch of glyphs and felt fascinated.
  So little by little, he taught himself the songs, and the rituals of the Force \gls{sphere}.

  And eventually, he began to carve his own glyphs, adding to her songs, or copying them with variation.
\end{exampletext}

\histEvent{45}{4}{%
  \Glsfmttext{juliet} begins carving song-rituals of the Force \gls{sphere} into a patch of trees in \glsfmttext{sunway}%
}

\histEvent{40}{3}{%
  \Glsfmttext{romeo} finds \glsfmttext{juliet}'s song-ritual glyphs, and slowly learns the Force \glsentrytext{sphere} from them%
}

\begin{boxtext}
  The dusky Sunlight makes little rune-shadows across the glyphs carved into all the surrounding trees.
\end{boxtext}

The elves who pass through here don't notice the glyphs because they normally ride giant snails.
The goblins who pass through here don't notice the glyphs, because they don't care.

\paragraph{Understanding the glyphs}
requires an \roll{Intelligence}{Academics} roll.

\begin{boxtable}

  \textbf{Roll} & \textbf{Result} \\\hline

   6 & These glyphs mean musical notes.  \\

   7 & \ldots and a few words in Elvish.  \\

   8 & Two people carved them.  \\

   9 & The `lyrics' pertain to Fire.  \\

  10 & \ldots and Earth.  \\

  11 & They form a magical ritual.  \\

  12 & Actually, they make many rituals.  \\

  13 & The rituals are of the Force \gls{sphere}.  \\

  14 & The ritual won't work without the missing words.  \\

\end{boxtable}


\sqpart{plateauGardens}% AREA
{\Glsfmttext{juliet}'s Flowers}% NAME
{Once dry, each works as a Force \glsfmttext{talisman}}% SUMMARY

A dozen elves spot clouds, describing what the clouds look like, and telling stories about the exploits of the `cloud-willow', and the `cloud-river'.
They think in terms of moving forests, and complete species, rather than individuals.

\Gls{juliet} takes no interest, as she is busy drying her flowers by pressing them onto a rock, then hanging them from a bush.
These are the `flowers of enlightenment', which she grows to make things float.
Check the details \vpageref{flowerOfEnlightenment}.

\juliet

\paragraph{If the \glspl{pc} look tired,}
\gls{juliet} hands them a flower or two, and just says `\textit{flower of enlightenment, for the weight of the world}' (she doesn't speak the \gls{tradeTongue} well, but anyone who speaks Elvish can clarify).

\paragraph{If the \glspl{pc} identify \gls{juliet}}
(and potentially deliver a poem, written \vpageref{goblinQuarry}) she asks them to deliver a message back to \gls{romeo} (although she does not know his name).

\begin{speechtext}
  You met the author?
  Can you deliver a message back for me?

  Tell them to meet at our special place.
  They'll know what that means.

  What eye-colour do they have?
  Are they a man or woman?
\end{speechtext}

\paragraph{If the troupe mention that \gls{romeo} lives in \gls{ravencops},}
she curls her nose in disgust, and seems sad, but perseveres in trying to meet with him.%
\footnote{The two elven groups have incompatible values.}

\sqpart{oathtower}% AREA
{Wherefore Not?}% NAME
{\Glsfmttext{MindElder} tells \glsentrytext{romeo}, `there is no love without an oath'}% SUMMARY

\Gls{MindElder} has no respect for anyone's privacy, and has been \gls{casting} \textit{Witness Mind} to figure out why his son, \gls{romeo}, seems so detached lately.
Once he discovered that \gls{romeo} has become infatuated with an unknown elf from \gls{plateauGardens}, he felt enraged, and began to argue with \gls{romeo} about the character of the `unruly' elves who live over there.

\begin{boxtext}
  An angry voice echoes from \gls{oathtower}, then another.
  Fast, bitter words in Elvish ring out, then the tower goes silent.
\end{boxtext}

\Gls{romeo} eventually leaves through the tower's front door.
He feels angry, but has accepted his father's idea: ``love means nothing without an oath''.

The \glspl{pc} may try to change his mind, but it won't be easy.

\begin{speechtext}
  Why is she so committed to avoiding an oath of love?

  Why would you refuse an oath of love if you don't plan on violating the oath?

  If I can promise to love her, shouldn't she reciprocate?
  Do I have more duties than she?

  Why does she never wear clothes?
  What's wrong with those people?
\end{speechtext}

\paragraph{If the \glspl{pc} can provide rational answers,}
\gls{romeo} calms down and starts thinking clearly.

\paragraph{If the grand plan has already begun,}
\Gls{romeo} reveals he has collected one \gls{ingredient} of each type required (Earth, Fire, Water, and Fate) from a storage room in \gls{oathtower}.
The plan requires twelve more \glspl{ingredient} in total.

\sqpart{sunway}% AREA
{Insider Knowledge}% NAME
{\Glsfmttext{romeo} gathers \glsfmtplural{ingredient}, while talking about the problems with goblins}% SUMMARY

\Gls{romeo} has found a spot of \glspl{marchingMushroom}, which he puts in his snail-gut satchel.
He has enough for two Earth \glspl{ingredient} so far, and hopes to find more.

\paragraph{If he trusts the troupe,}
he speaks openly to them about the problems in the \gls{enchantedLands}.
Otherwise, he hides.

\begin{speechtext}
  My father has played a dangerous game with all these goblins.
  He controls them only as long as they remain fed, but if they feed, they will breed.
  He sends them to kill dangerous creatures, or bring news from far away, but most survive.
  Every \gls{sepulchre} with sleeping \glspl{ogre} marks another failure to control the population.%
  \footnote{\Gls{romeo} won't say so, but he holds the same view of local humans, and would destroy \gls{coppernut} and everyone inside if they looked threatening.}

  If the goblins suspected that \gls{MindElder} wants to reduce them, they would begin to push back against their oaths, or simply leave the area, past where he can reach them.

  We do not want that.
  We do not want to see the goblins abandoned, in great number, to their own hunger.
\end{speechtext}

\sqpart{sunway}% AREA
{Change of Plan}% NAME
{\Glsfmttext{juliet} has figured out the spell will not work unless the troupe engineer an artificial \gls{flood}}% SUMMARY

\Gls{juliet} enters \gls{sunway}, studying her old notes, and realizes the plan for the great spell to unite everyone's perspectives across the two lands will not work; she needs \pgls{flood}.
The troupe will have to scout out \gls{shadepaths} to find where the water comes from,%
\footnote{Check the river locations \vpagerefrange{shadePool}{shadeDamn}.}
and engineer \pgls{flood}, perhaps with a damn.

How the \glspl{pc} engineer \pgls{flood} depends on them, but \gls{shadepaths}'s high walls, and secret streams allows them more opportunities than most locations.
Whatever their plan, it should be abstracted to \pgls{natural}; and if the roll fails, they will simply have to change their methods and tools to provide bigger bonuses.

\sqpart[\squash]{ravencops}% AREA
{The Goblin Hunting Party}% NAME
{A dozen goblins hunt for \glsfmttext{romeo}}% SUMMARY

\Gls{romeo} left his father's lands, to search among the \gls{plateauGardens} for \gls{juliet}.
\Gls{MindElder} has sent out bands of goblins to find his location, and report back.

Majiscule leads six other goblins through the woods.
Soon they will reach \gls{plateauGardens}.
The goblins speak frankly with the troupe about their mission, and will not deviate from it, even for a moment, unless their lives are in danger.

\paragraph{If the grand plan has begun,}
then \gls{romeo} will have used this time to collect one more \gls{ingredient} of each type required.



\sidequest[oathtower,sunway,shadepaths,plateauGardens,ravencops]{The \Glsfmttext{ranger}}

If the troupe tarry too long, the \gls{jotter} will send \gls{dickhead} out to find out what's happening.

\sqpart{sunway}% AREA
{An Old Acquaintance}% NAME
{\Glsfmttext{dickhead} arrives to scope out the situation}% SUMMARY

The troupe find him in the woods, hunting \pgls{griffin} nest.
He moves towards them quietly, hoping to get the jump on them, just to show off his superior stealth \glspl{skill}.

Once out, \gls{dickhead} speaks haughtily of his ability to survive in the forest, and moves with confidence.
He asks the troupe what they've seen, but does not give their stories much importance.

\begin{speechtext}
  So you still have not found the heart of the problem.
  Well keep searching!
  You may not succeed, but it makes for good practice.
\end{speechtext}

\Gls{dickhead} soon leaves, telling everyone not to follow him, as they'll just make noise.

\dickhead

\sqpart{shadepaths}% AREA
{Peeping Woodsman}% NAME
{\Glsfmttext{dickhead} explains his plan to kill \glsfmttext{LifeElder}}% SUMMARY

\Gls{dickhead} has observed the area for some time, noticed the \gls{disgnome} plants, and believes that the giant snails all stem from a single source: a powerful spellcaster.

\begin{speechtext}
  The plan is simple, I kill the \gls{witch}.
  Elves are always wrapped up in their own thing; they never pay attention, and they're probably drowsy from all the \gls{disgnome} in the area.
  So I'll set \pgls{ambush} then loose an arrow on whoever crafts these giant snails.

  Elves are small.
  I'll just need one arrow.
\end{speechtext}

\Gls{dickhead} will leave the \glspl{pc}, as he does not trust them to stay silent while he plans \pgls{ambush} for \gls{LifeElder}.

\sqpart{plateauGardens}% AREA
{Loose Clothing}% NAME
{\Glsfmttext{dickhead}'s crossbow lies abandoned on the ground}% SUMMARY

\begin{exampletext}
  You don't get to be centuries old without learning how to spot \pgls{ambush}.
  As \gls{LifeElder} performed one of her standard spells to query the living things in the area, she found \gls{dickhead}, and guessed the reason for his hiding.
  Her spell has split his limbs into myriad tentacles, leaving his equipment on the ground.
  He slithered away as the spell took hold, confused and dismayed, dropping pieces of his equipment along the way.
\end{exampletext}

The find his \gls{crossbow} and twelve quarrels on the ground, but carrying it sends a clear signal to the elves that they approve of his methods, and makes them dangerous.
They will suffer a -3~Penalty to social rolls with the elves while the \gls{crossbow} is visible.

If the \glspl{pc} follow the trail, have them roll \roll{Wits}{Survival} (\tn[10]).
Success means you can skip to the next \gls{segment}, below.
A tie means they succeed, but only after \pgls{interval} (and another \gls{segment}).

\sqpart{ravencops}% AREA
{Wandering Hood}% NAME
{\Glsfmttext{dickhead}'s clothes lie discarded on the ground}% SUMMARY

The troupe see the last of the ranger's clothing, discarded just before entering the forest.
Following him further will not be easy; the \gls{tn} rises to~14.

\sqpart{oathtower}% AREA
{\squash~Retirement}% NAME
{\Glsfmttext{dickhead} now works for \glsfmttext{MindElder} as a mutated servant}% SUMMARY

The next time the troupe enter \gls{oathtower}, they find \gls{dickhead} in his new form -- a twisted creature, with limbs replaced by tentacles, and his neck so shrunk that his shoulder-blades wrap around his ears.

After \gls{LifeElder} twisted his body, \gls{MindElder} twisted his mind.
He now accompanies \gls{MindElder} everywhere, passing him pens, and washing his clothes in the lake outside.
When \gls{dickhead} has nothing to carry, he ascends \gls{oathtower} by grabbing a window from outside, and pulling himself up the wall.
Each time he passes a window, he takes a good look inside to check that nobody inside is breaking any laws, and that everything seems as it should.


\dickheadReborn




\sidequest[ravencops,oathtower]{An Ordinary Week in the Enchanted Lands}

\sqpart{ravencops}% AREA
{The Square of Life}% NAME
{Giant snail devours \gls{crawler} in acid attack}% SUMMARY

\begin{exampletext}
  \Gls{MindElder} has taken to twisting the mind of snail to make them crave flesh.
  They wouldn't normally hunt well, but the acidic spray, and disarming look, means they regularly consume \glspl{crawler}.
\end{exampletext}

\begin{boxtext}
  The echo of a distant crow's cry reaches you, just before dying.
  The trees look a kind of uniform-brown, without any mottling or variation.
  And the road feels as smooth as pond-scum.
\end{boxtext}

The troupe should make \pgls{bandAct} action to not fall over while walking on freshly-laid snail tracks.
Failing a \roll{Dexterity}{Athletics} roll (\tn[7]) inflicts 1~\gls{ep}.

\begin{boxtext}
  You find a cross-roads, as the path splits left and right.
  The road to the left looks older, and less slimy, but has \pgls{crawler} running towards you.
  The road to the right also looks old, until the giant snail, approaching silently.
\end{boxtext}

If the \glspl{pc} are near the giant snail when it sprays acid, they can roll \roll{Wits}{Survival} (\tn[7]) to leap into the woods as the snail prepares to spray.

\begin{boxtext}
  As the snail-spittle hits the trees and bushes in a messy gush, they let of a tiny hiss and begin to wilt.
  Leaves wither and bark turns black.
  Behind, the \gls{crawler} turns to flee into the woods with two left-legs melded together.
\end{boxtext}

If the troupe let the situation unfold, the giant snail ignores them, and chases the wounded \gls{crawler} up a tree.

\sqpart{ravencops}% AREA
{Got a Permit, Mate?}% NAME
{Oathkeeper goblin wants to check troupe's weapon licences}% SUMMARY

\begin{exampletext}
  The lack of \glspl{crawler} and plentiful giant snails soon brought a lot of goblins to the area.
  \Gls{MindElder} also turned this problem to his advantage by making the goblins swear to capture or kill anyone disturbing the peace.

  The goblins responded with sarcasm, but the spell worked anyway, soon all the goblins lay dead or agreed to take oaths of good behaviour, which work fine as long as the goblins eat regularly.
\end{exampletext}

\begin{boxtext}
  In the distance, a small person in a long, green cloak walks towards you, carrying a potato so large that it cannot see you.
  A long nose points up, just above the top of the potato, and pasty-white ears flop at shoulder-length.
\end{boxtext}

If the goblin (Abjad) sees the troop, it switches the potato to a one-arm hold and points accusingly at the troupe, then speaks in an unusually deep voice.

\begin{speechtext}
  I hope you got a licence for those weapons!
  Show me!
  Show me the licence, earless scum!
\end{speechtext}

The Abjad (the goblin) observes and insists on the following local laws:


\begin{itemize}
  \item
  No high-pitched noises.
  \item
  No jokes, nor words which move to laughter.
  \item
  No unlicensed weapons.
  \item
  No singing out-of-lock.
\end{itemize}


\sqpart{ravencops}% AREA
{Silencing the Starlings}% NAME
{A goblin aims her bow at a starling for the crime of high-pitched song}% SUMMARY

\histEvent{39}{3}{%
  Soon \glsfmttext{enchantedLands} filled with ziggurats, and every high-pitched noise woke the sleeping \glsfmtplural{ogre} inside.
  \Glsfmttext{MindElder} banned all high-pitched noises, including most birds.
  Of course, the ravens and crows remain, which started the name `\glsfmttext{ravencops}'%
}

\begin{exampletext}
  When \gls{MindElder} banned high-pitched voices, this included most birds, because bird-song can wake the sleeping ogres.
  The ecosystem has become strange since then, as it has very few birds, except ravens, crows, and magpies.
  This is why people call the local forest `\gls{ravencops}'.
\end{exampletext}

Mora has heard a starling sing, and thought to herself `that's illegal!', then readied her little bow.
A moment later, the arrow hits, and she finishes the job with a rock, and sucks out the starling's brains.

Mora will happily speak with the troupe, in a low-pitched voice (to avoid waking any ogres), but if she sees them doing something criminal (like carrying weapons without a licence) she flees to sound the alarm (but quietly).

\sqpart{oathtower}% AREA
{Meat Salad}% NAME
{As a snail approaches the tower, \gls{MindElder} turns its mind towards thoughts of meaty salad}% SUMMARY

\histEvent{95}{3}{%
  When giant snails barged into \glsfmttext{enchantedLands}, \glsfmttext{MindElder} decided to kill two birds with one stone, and twisted their little minds to crave flesh.
  The giant snails stalk the woods, looking for some meat to eat with their leaves, and occasionally find \glsfmtplural{crawler}%
}

A giant snail approaches \gls{oathtower}, along a snail road.
The \glspl{pc} probably won't see the snail, unless they're on the road out, but they will certainly see \gls{MindElder} observing the land from the balcony at the top of \gls{oathtower}, and singing a spell to twist the mind of the snail.

Any characters who understand Elvish
will understand the song relates to a meat-based salad, and that the snail should add meat to its salad.

\sqpart{ravencops}% AREA
{Away You Go!}% NAME
{A goblin leads a snail away, and into human lands}% SUMMARY

Mora shoes a snail into a dead-end road.
She organizes a dozen goblins to spread out, and occasionally stab it (while keeping their distance), in order to hound it to the clearing where six more goblins wait to plant spears on the road.

The scene proceeds just as the game-mechanics suggest.
Goblins throwing javelins deal around 1D6-1 to 1D6+1 Damage, and the giant snails have a lot of \gls{dr}, even in their most vulnerable location.
Some javelins hit the shell and shatter, others stick into the snail's `skin' harmlessly, and a few dig into its body enough to inflict a minor wound (perhaps 1 or 2 Damage).

Bringing down the snail will take the entire \gls{interval}, as the goblins run away, some double back to collect javelins which fell on the ground, and others run ahead to climb trees and throw javelins from above.
The goblins have to constantly stop the snail entering the forest.

Once the snail dies,
the goblins take it apart in three stages.

\begin{enumerate}
  \item
  The barbecue, as every goblin feels famished, and must eat immediately.
  \item
  The dissection, where bloated goblins with pot-bellies cut and cure the snail-meat, then make ropes from its innards.
  \item
  The great smashing, where they use rocks to crack the shell into smaller dishes, then use the dishes to transport remaining meat back to the Icebox House.
\end{enumerate}

Each stage takes another \gls{interval}, so the troupe may see the goblins again if they pass through the area the next day.

\sqpart{ravencops}% AREA
{On the Menu}% NAME
{Another giant snail approaches, but ignores anyone on the road}% SUMMARY

Another great snail approaches, with its mind focussed on a meaty-salad.
It ignores anyone on the road.
It only attacks living things in the bushes where it feeds.

The snail stops here and there, inspecting the bushes, and sometimes vomiting on them as part of its external digestion, but never stays for long.

\sqpart{ravencops}% AREA
{The Unmerry Band}% NAME
{A dozen goblins walk, every statement brings suspicion of humour}% SUMMARY

Grawl, Majiscule, and Brev patrol the land, looking for trouble-makers, mapping new paths the snails have brought, and noting local monsters.
And as they walk they bicker; Grawl accuses Majiscule of jokes (which \gls{MindElder} banned, so that the high-pitched goblin laughter would not wake any \glspl{ogre}).

\begin{speechtext}

  If we find a snail, we'll need to go back to get the others, and shoe it towards a dead-end.

  What do you mean, `shoe a snail'?

  Is that a joke?
  Are you trying to be funny?

  You said it!
  You said we might shoe a snail!
  You were making the joke!

  Your face is a joke.
  It is a crime, your face.
  You have a funny-looking, criminal face.

\end{speechtext}

Majiscule hides his face until nightfall.


\thread[shadepaths,plateauGardens]{Events in \glsfmttext{sunderedForest}}

\segment{shadepaths}% AREA
{Light Ascent}% NAME
{Elves fetch water below, and climb up delicate roots}% SUMMARY
\label{shadeAscent}

\begin{boxtext}
  Round the next squelching corner, a skinny silhouette collects the mucky water in a bucket.
  A second figure is climbing down roots on the tall walls, completely naked, except for the bucket on a rope.
\end{boxtext}

\Glspl{plateauGardens} can become very dry, due to the lack of rivers.
Elves often have to descend with snail-shell buckets, fetch filthy water from the rivulets in \gls{shadepaths}, then purify it with spells (or just use it as fertilizer for the gardens).

\paragraph{When the \glspl{pc} approach}
the elves flee up to their plateau, unless the troupe present themselves as \emph{very} non-threatening (\roll{Charisma}{Empathy} at \tn[10]).
Success means that Fanwa and Henton will chat with them, and invite them to discard their possession and climb up the roots.

\stupifiedElf[\npc{\T[2]\F\M\El}{Fanwa \& Henton}%
  \set{spentMP}{6}%
]

\showStdSpells

\paragraph{Climbing the walls}
means hanging onto the \gls{disgnome} roots, and those spiky roots will inflict a mild poison, removing $1D3$ Wits.
It works just like the walls in the \gls{sunway}%
\footnote{See the climbing note \vpageref{climbingWalls}.}
but the \gls{tn} to climb equals $1D3 +$ the character's Strength Bonus, plus the \gls{weight} of all items.

\segment{plateauGardens}% AREA
{Mist Below}% NAME
{Snail eyestalks poke above the misty sea in \glsfmttext{shadepaths} below}% SUMMARY

\begin{boxtext}
  Mist has risen, but is falling into \gls{shadepaths} beside \gls{plateauGardens}.
  Soon \gls{shadepaths} look like fluffy-white rivers.
  A figure in the distance seems to hover between two plateau gardens, but the walking reveals they must be on a vine-bridge which has fallen just below.
\end{boxtext}

The disorienting environment is quite lethal to newcomers, as people who don't know where the land lies can easily make a misstep, tumbling down the sides.

Fast movements require a \roll{Wits}{Survival} roll (\tn[8]) to avoid falling into the misty canals below.
Longer journeys require an \roll{Intelligence}{Survival} roll (\tn[10]).

The mists fade after \pgls{interval}.

\segment[\squash]{shadepaths}% AREA
{Another Treacherous Ascent}% NAME
{More roots on the wall allow light climbers a way up}% SUMMARY

Further \gls{disgnome} roots allow anyone here to climb up to \pgls{plateauGardens} above, just like the previous \gls{segment} (\vpageref{shadeAscent}).

\thread[shadepaths,sunway,plateauGardens]{Grumbling Gardens}

\segment{shadepaths}% AREA
{Excuse Me!}% NAME
{A giant snail blocks the path}% SUMMARY

`Brownie' the snail grew much bigger than the others, and pregnancy did not help.
Now when she moves through \glspl{shadepaths}, she blocks them entirely.
Anyone walking inside \glspl{shadepaths} must simply turn back and find another route.
This may delay the troupe in whatever they wanted to do by \pgls{interval}.

If the troupe try to squeeze past her, have them roll \roll{Dexterity}{Stealth} (\gls{tn}~8 plus the character's Strength Bonus).
Failure inflicts $2D6$ Damage as her massive body smooshes them against \gls{shadepaths} walls.

\segment{sunway}% AREA
{Stampede}% NAME
{An auroch stampede passes through}% SUMMARY

\label{sunwayEarthIngredients}
Elves watch from a garden plateau as aurochs stampede through the causeway.
Behind them, \pgls{crawler} chases, but it's losing steam, and soon retreats into the forest.

\segment{plateauGardens}% AREA
{Uninvited Guest}% NAME
{A rare \glsfmttext{woodspy} stalks the garden}% SUMMARY

\label{plateauWaterIngredient}
Most large animals have trouble reaching up the tall walls to \gls{plateauGardens}, but \glspl{woodspy} are clever, and sometimes manage to pull themselves up using a snail, or finding a piece of vine hanging down.
This \gls{woodspy} has reached \gls{plateauGardens}, and stays close to a vine-bridge, to wait for someone to pass.
Once it grabs someone, it pulls them down onto \gls{shadepaths}.

\paragraph{If any elves are present,}
and still suffer from \gls{disgnome} then
they fail to spot the \gls{woodspy}, and it grabs one.

If the \gls{woodspy} attacks \pgls{npc}, you can resolve the fight by just comparing each \gls{npc}'s \gls{cr} (highest wins), instead of rolling dice alone.

\woodspy

If the \glspl{pc} have uprooted some of the \glspl{disgnome}, any elves spot the \gls{woodspy}.

\segment[\gls{afternoon}]{sunway}% AREA
{Slow Wander Home}% NAME
{The aurochs are returning\ldots slowly}% SUMMARY

Aurochs move slowly through a mild breeze, grazing at \gls{sunway} grass as they go.
Any sudden movement makes the aurochs flee, creating a stampede.

\paragraph{If the troupe wants to pass quietly,}
have them roll \roll{Dexterity}{Empathy} at \tn[8].

The troupe may prefer to wait, but this will take the rest of the \gls{interval}, as the aurochs move slowly, while grazing.

\segment[\gls{evening}]{sunway}% AREA
{Full Moon \& Sacks}% NAME
{Goblins approach \glsfmttext{plateauGardens}, sniffing loudly for food}% SUMMARY

A guilt of goblins approach \gls{plateauGardens}, two sacks each, looking to steal vegetables.
They spent their time searching the walls before finding roots to climb, slowly up.
Once up they split and search, sniffing so loud it almost sounds like a whistle.
And soon they regroup, as Glottal found some carrots.

\paragraph{If the \glspl{pc} stop them,}
they object.

\begin{speechtext}
  Go fuck a badger, we came for grub!
  It's not `theirs', okay?
  They said the trees belong to the ground, so now they're ours.
  They said all good tea is theft, okay!?
  So we have not broken our oaths.
\end{speechtext}

\enchantedGoblin[\NPC{\F\N}{Glottal}%
  {Long nose, stubby fingers}% DESCRIPTION
  {Fingers her sack}% MANNERISM
  {to find the longest carrot}% WANTS
  \npcQuote{Off and get your own nobody's-carrots}]

\enchantedGoblin[\NPC{\M\N}{Grawl}%
  {Emaciated and agitated}% DESCRIPTION
  {Claws cheeks down till the eye-veins show}% MANNERISM
  {eat or fight, now, now, NOW}% WANTS
  \npcQuote{They said `proper tea', not `good tea', idiot}]

\paragraph{If the \glspl{pc} ask the elves,}
the elves say they don't feel happy with goblins taking their food, but refuse to acknowledge the concept of property.

\segment[\gls{morning}]{shadepaths}% AREA
{The Picnic Choir}% NAME
{Elven songs descend from the plateau above}% SUMMARY

A little group of elves sing a love-song to the clear-blue skies, but don't respond to shouts or calls.
They will, however, respond to someone singing with them.

If someone sings well, they will lower the vines of a broken bridge to help them up.
If they sing poorly, the elves leave silently.

Singing \glspl{pc} should roll \roll{Intelligence}{Performance} (\tn[10]) to understand the elven song, and imitate it.



\stopcontents[sq]

\end{multicols}

%\section{Mound}

So we have a tunnel, which takes you right there, although it did have a problem recently\ldots

- We had a rat problem,
- so we put \pgls{crawler} down there,
- but it didn't come out, so we lured this basilisk to use its poisonous breath to kill anything in the hole,
- but the basilisk crawled in once it finished.

So...do you think you can help?


\section{In Closing}

\begin{multicols}{2}

\subsection{The Controlled Collapse of an Ecosystem}

The grotesque stack of \glspl{spell} and plans infecting the Elven lands have all the balance of a two-legged spider.
And like any unbalanced thing, if it falls wrong, then it can fall on people nearby.
The players will need to understand the ecosystem and all its parts in order to make it fall down in the right way.

\subsubsection{Messing with the Goblins}
proves easy.
Goblins tend not to follow rules and laws smoothly,%
\exRef{judgement}{Judgement}{goblin}
and don't fully grasp the rules of \gls{enchantedLands}.

If the \glspl{pc} steal a goblin's weapon's licence, the goblin drops the weapon.
If they tell a goblin that she just made a joke, she will believe them, and hand herself in to \gls{oathtower}.
If the \glspl{pc} feed the goblins beans from the vine-bridges,
they begin to fart, making others explode in laughter, which prompts arrests, arguments over who dealt it, and wakes \pgls{ogre} or two if the goblins are anywhere near \gls{oathtower}.

With enough in-fighting, the goblin population will decrease, but never by much, and not for long.

\subsubsection{Waking the \Glsfmtplural{ogre}}
means death to everyone in \gls{enchantedLands}, and many beyond.
If \pgls{ogre} wakes, hungry and confused after its long sleep, the goblins will do damage-control by leading it to food (perhaps in \gls{oathtower}, or the Icebox House \vpageref{iceboxHouse}).
But if too many wake, a hungry horde will form.

The \glspl{pc} might try to kill the sleeping \glspl{ogre} quietly, but the sepulchres have only tiny grates at the side, enough for a goblin to hear their breathing but not enough to crawl through, or even extend an arm into comfortably.
The \glspl{pc} also don't know where \gls{MindElder} placed all of them.

In total, eight sepulchres remain in \gls{ravencops}, mostly near \gls{oathtower}.

\subsubsection{\Glsfmtname{MindElder}'s Death}
would be disastrous.
Goblin oaths would start to break, little by little, while the elves in \gls{enchantedLands} would lose their enchantments much more slowly.
The goblins would awaken the \glspl{ogre} with screeches and wails held in for half of their lives.

\begin{itemize}
  \item
  Within a week, they would eat through the giant snails in \gls{ravencops} (while many continue to uphold bits and pieces of their various oaths).
  \item
  Within a week and one day, the goblin horde would eat through every elf in \gls{enchantedLands}.
  \item
  The ninth day would begin with a siege upon \gls{coppernut}.
\end{itemize}

\paragraph{Without the \Glsfmttext{disgnome}}
the elves of \gls{plateauGardens} wake up and act a little (but just a little) more proactively.
If the \glspl{pc} convince them to do something, they may convince \gls{LifeElder}.
However, \gls{LifeElder} will not take suggestions from outsiders.

If the elves remain affected by all the \gls{disgnome}, they will be unable to defend themselves, and any goblin attacks will soon find them, as their desperate fingers claw along the cliffs, and find the stairway (\vpageref{hiddenStairs}).

\subsubsection{Making the Elders Talk}
will take perseverance, but it can work.
If \gls{LifeElder} and \gls{MindElder} work together, they can  reduce the giant snails in the area slowly, while \gls{LifeElder} introduces a few, subtle, `goblin-traps'.

\subsubsection{Elven Death by Human Hands}
will prompt a vicious reaction from \gls{romeo} and \gls{juliet}, along with a couple of other elves from the \gls{plateauGardens} if the \gls{disgnome} has been quelled.
The elves will begin planning to wipe out \gls{coppernut} in order to remove all humans from the area.
The plot will begin by casting strange spells on the road out, so they can ensure complete destruction.%
\footnote{If ten refugees return, a battalion of humans may return.
But if nobody returns from the road for a month, then people will simply decide to not travel along that road rather than investigate.}

\subsubsection{The Grand Spell}
\label{grandSpell}
planned by \gls{romeo} and \gls{juliet} will require 16~\glspl{ingredient} in total, plus \pgls{flood} and \pgls{earthquake}.
The \glspl{pc} will have to engineer the flood inside \gls{shadepaths}, where rivers have been artificially diverted.
The \gls{earthquake} will occur naturally, as it always does, at the end of each \gls{cycle}.
This means the plan must wait until the end of \showCycle, so it can only occur at the end of the session.

Once the couple have all they need, they make each one into \pgls{boon}, and begin a grand ritual of song-magic, harmonizing together.

\spell{Shattered Identity in G Minor}% Name
  {Detailed, Distant, Divergent, Duplicated}% Enhancements
  {Warp}% Action
  {Water, Fate, Earth, Fire}% Spheres
  {\roll{Wits}{Survival}}% Resist with
  {The chorus alone is twenty minutes long, but once the spell really gets going, it binds everyone and everything at \spellRange\ into a self-jealous, psychic lump of non-space.}% Description
  {Everyone involved drops pieces of their memories, desires, and their perception of time into the fractured space, and each space involved loops together.

  Everyone involved gains a +\arabic{spellPlusOne}~Bonus to Empathy tasks, and a -\arabic{spellPlusOne}~Penalty to all other tasks.
  Both the Bonus and the Penalty reduce by one each time someone travels between one of the connected area, and finds another part of themselves.}

\begin{boxtext}
  Elves from \gls{plateauGardens} look down at you, scared and confused.
  Their usual blank-eyed cool has gone and you remember kissing \pgls{ogre} good night, and hoping to become like him one day.

  \Gls{oathtower}'s stairs lead up to the \gls{plateauGardens}, or you can go into the elven home through the trees.
  Two goblins stand on the cooker holding your memory of learning the Air \gls{sphere}.

    ``\textit{Sorry, wrong person!}'', the little goblin says in Elvish.

    Up the stairs, and out of the elvish house, \gls{oathtower}'s library is full of goblins, all of them are \gls{MindElder}, shouting for everyone to leave his home.
    Then they turn and point at you in unison, and begin to shriek, ``\textit{Identify! Identify}''.

    An eyeball reaches in, to return your name, if you want it\ldots ?
    The other eyeball searches the room upstairs.

    The name feels right, but the snail has another, and that one feels right too.
    The snail wonders how you know which name is your name, and wonders if you would share.
    The snail is ashamed of its nakedness, so the eyeball slithers back out of the window.

    The goblin hunger was deep, now infectious.
    You could enjoy the taste of your own arm, and it wouldn't actually be \emph{yours}, so it would be fine, and you would get a whole arm to yourself.

    \Gls{SnailTamer} arrives, telling you he loves all of you, and you're all doing really well.
    He's not really there, as he was taking a nap, and just decided to have a little dream with you guys to help everyone out.

\end{boxtext}

The spell ends before long, or it should if the players aren't into Dada-Taoism.
Or if they're getting the vibe, continue handing out memories until they've pieced their characters together.

After the experience ends, \gls{LifeElder} wanders past, stops to observe a spider-web, then continues.

\paragraph{One the spell wears off}
the two groups of elves and the goblins will begin to work together, planning a route forward, each more fully aware of the others points of view.

\iftoggle{verbose}{
  \subsection{Broken Resolutions}

  If your table runs out of time for the night, and you need to wrap everything up quickly, switch to \gls{downtime} plans, and give each \gls{pc} \pgls{action}.
  If they want to negotiate with \gls{MindElder}, they might roll \roll{Intelligence}{Empathy}, or if they think it's time to kill those \glspl{ogre}, sleeping in their sepulchres.

  For some added drama, ask each player to roll under a cup, napkin, or sleeve.
  Leave the \gls{natural} hidden until everyone has decided what they want to do.
  One all results are in, determine the outcomes based entirely on the rolls.
  If \pgls{pc} spent the entire \gls{downtime} trying to redirect the river through \gls{shadepaths}, it could flood the \gls{sunway} and block snails from accessing \gls{ravencops}.
  Or if they spent their time trying to kill \glspl{ogre}, a failed roll could mean that \pgls{ogre} eats the \gls{pc} alive before marauding around \gls{ravencops}.
}{

  \subsection{No Way Home}

  If you want to run this arc over multiple sessions, or if the troupe find themselves stuck here over \gls{downtime}, the elven lands have plenty of reasons for characters to leave, disappear or get distracted.

  \subsubsection{In \glsfmttext{plateauGardens}}

  Characters leave because,

  \begin{enumerate}
    \item
    they ate the wrong thing, and their limbs shrivelled.
    The elves hope to fix them `soon' (and will, once the player returns).
    \item
    they wandered off with an elf (leading to gossip and laughter), and have not returned.
    The next session resumes once all hope is lost for finding them.
    \item
    a clerical error resulted in an immediate summoning from an overseer.
    A new \gls{pc} arrived to deliver the letter.
  \end{enumerate}

  Characters arrive because,

  \begin{enumerate}
    \item
    the \gls{jotter} wants to know what the hold-up is.
    \item
    the \gls{pc} was lost, some time ago.
    Hunger and monsters killed the rest of their troupe.
  \end{enumerate}

  \subsubsection{In the pale forest}

  Characters leave because,

  \begin{enumerate}
    \item
    they whistled out of key, and the goblins told them it was time to leave.
    \item
    \pgls{ogre} woke up, her hand reached out, and she grabbed the character.
    \gls{MindElder} reacted quickly, putting everyone in enchanted sleep, but she holds the character like a child with a doll; best not to wake them.
    \item
    goblins prepared food for Winter, grabbing all that was edible, and stuffing it in ice.
    The `it' included the character, who is `edible'; but the \gls{MindElder}'s enchanted sleep spell should deep them safe.
  \end{enumerate}

  Characters arrive because,

  \begin{enumerate}
    \item
    twelve oathkeeper goblins caught them in the forest, along with a bear and \pgls{griffin}.
    The goblins take their prizes -- all tied to poles -- back to share with the rest.
    \item
    the local overseer sent the character with a letter for \gls{MindElder}.
    It contains a generic proposal for mutual aid in killing beasts.
  \end{enumerate}

}

\end{multicols}

\printglossary[
  title={Factions},
  type=people,
  style=topicmcols,
  ]




