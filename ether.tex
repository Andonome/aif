\chapter{Portal Realms}
\label{ether}
\label{portalRealms}

\epigraph{But the stars that marked our starting fall away.
We must go deeper into greater pain,
for it is not permitted that we stay.}{Dante}

\begin{multicols}{2}

\noindent
Strange stories exist of fantastic lands, with dangerous and bizarre creatures.  The official stance of the \gls{alchemists} is that these lands rest somewhere far away.
Gnomish scholars map some of these places to other planets in the solar system.
More theological thinkers propose that they are the limbo-lands which souls journey through before going to their final resting places.

Whatever the truth, you can travel to them only through extremely rare magical portals.

\subsection{New Skill: Xenomology}
\label{Xenomology}
\index{Xenomology}

This extremely rare skill deals with the strange Portal Realms.
It focusses mainly on basic survival, such as not eating anything ever, a few words in the strange language of the dwarves who live beyond, and descriptions of well-known spots to stay away from.

Characters cannot purchase this skill during character creation, but can pick up the first level by speaking with someone who has a higher level in the skill than they do, or by journeying across the lands themselves for an extended period of time.

\end{multicols}

\section{The Realm of Bright Rocks}\index{Realm of!Bright Rocks}
\label{brightrocks}
\index{Bright Rocks}

\begin{multicols}{2}

\encDesert

\widePic{Johan_Jaeger/desert_wizard}

\begin{boxtext}

  The rocks form a desert, but they're so large that every step forces your ankle to twist a little.
  The sun forces you to look down and cover your eyes, but you steal a glance ahead and notice a massive four-armed man made of rock tending to a crystal flower with nothing but a stare.
  In the distance behind him sits a massive structure of towering rock.
  At its peak, a ball of shadow sits, as if the entire structure were a reverse light house.
  Far above it, a green and blue daytime moon you have never seen before stares down at you.
  The Ainumar sits at the top of the sky, brighter than ever.

\end{boxtext}

\noindent
Imagine a desert, inflated to such proportions that you cannot see sand, only rocks and boulders.
Even walking from A to Z is a dangerous mission as a broken ankle or a quick trip into a broken face are always possibilities.  The sun is unrelenting, never setting and always exceptionally hot.  All moisture in the environment is quickly removed.  The area is nearly entirely barren, and one can wander for days without seeing a single creature.

Occasionally, one of the rocks in the environment stands up, stretches a few arms outward, then wanders to a new location.
These rock men wander the barren landscape, with no visible purpose, and on rare occasions stop to examine a crystalline flower or speak to one another.
They are typically peaceful and thoughtful creatures, but can become wildly violent if the crystalline flowers are disturbed, or even approached.

\subsection{Features of the Desert}

\subsubsection{Heat}

The omnipresent heat pulls all energy from any normal creature, except elves.
Each scene, whether fighting, walking or anything else, inflicts 4 Fatigue Points on those with some proper desert-wear, and 6 Fatigue Points on those without.
There is no night here, and no complete respite from the Sun.

Any shade, such as provided by the massive stone structures of the realm, allows parties to rest, and accrue only 2 Fatigue Points per scene.

The massive rocks, while they look like sand from a distance, provide the persistent threat of a broken ankle.
If anyone decides to run (or fight) in the desert, they immediately make a Dexterity plus Caving roll at TN 7.
Failure indicates $1D6$ damage from a twisted ankle, and the character becomes immediately \emph{prone}.

\paragraph{Archmages} within the realm always use illusion to cast shadows around themselves for protection from the sun.
They appear as balls of darkness, floating on the horizon.%
\footnote{See page \pageref{archmage} for more on Archmages.}

\subsubsection{Crystal Flowers}

The crystalline flowers hum with potent magical energy.
Anyone can notice it with a Wits + Portal Lore roll at TN 7.
Each one holds $2D6-3$ MP, but unlike a mana stone, this MP can be used to create mana stones or magical energy.
Once disturbed, flowers immediately break and release all their magical energy into the air.
Each round, the ambient magical energy decreases by 1, so mages hoping to use this energy for a magical item must act quickly.

The GM should roll $3D6$ twice each day, and combine any encounters present.

\subsubsection{Stone Structures}

Between vast deserts of beige rocks and boulders are standing stones, apparently natural creations like slow-growing plants which shoot up in simple geometric patterns such as spheres or a series of hexagonal pillars.
Some contain signs of modification for the purposes of housing, though the form of the dwellers is unclear as the tunnels shift in size from the gargantuan to the minuscule constantly.

Many radiate mana constantly.
Roll $1D6-2$ to determine what level of Mana Lake the stone structure counts as (rolling less than 1 means the structure is not a mana stone).

\subsection{Encounters}

Every 5 miles travelled prompts 1 roll on the `Encounters in the Desert' chart.
The inhabitants of the land appear to be rather territorial, so travellers will not find many creatures wandering into them.

Anyone remaining still rolls $1D6$ as normal for the left-hand die,
\exRef{core}{core rules}{encounters}
but rolls $1D6-2$ for the right-hand die.

\end{multicols}

\section{The Realm of Shifting Corridors}\index{Realm of!Shifting Corridors}\label{shiftingcorridors}
\index{Shifting Corridors}

\widePic{Johan_Jaeger/ancient_rocks}

\begin{multicols}{2}

\begin{boxtext}
  The corridor continues farther than you can see in the darkness.
  The flat stone floor shows no sign of the strange dwarves.
  No turns or exits present, and up ahead, the tunnel goes dark as no fireflies dance there.
  You have to wander forward in the darkness, with one hand held tight to the side of the unending corridor.

  Then slowly, you notice the walls are closing in\ldots

\end{boxtext}

\noindent
Perfectly smooth stone walls of different colours -- some bright, others dark -- move around when nobody is looking.
The maze does not inhabit a small area within a larger world -- the maze \emph{is} the world, and it is alive.

Misty dew settles slowly on non-moving walls, allowing the inhabitants of this realm to know how long a certain section has remained where it is.
The very notion of a continuous place where people can remain and call home, of places they can return to, is entirely alien to the creatures which populate the labyrinth.

Nothing attempts to breach the omnipresent mist above -- anything climbing above it doesn't come back.

\encLabyrinth

\subsection{Features of the Labyrinth}

\subsubsection{Fireflies}

Fireflies wander the entire realm, providing dim illumination in almost every area.
They seem to feed on the strange fungus in the realm, and provide the only source of light.

\subsubsection{Shifting Walls}

The slowly-shifting walls mean one cannot simply go back the way one came.
Each new scene has a $\frac{1}{3}$rd chance of shutting the passage behind, and a $\frac{2}{3}$ chance of shutting the passage behind \emph{that} one.

\subsubsection{Mist}

A mist covers the top of the land like a ceiling, obscuring the top of the maze's walls.
Some walls are barely the height of a man, while others are impossibly tall, but it all looks roughly the same when mist covers everywhere.

\subsubsection{The Abyss in the Sky}

High above the mist, and completely unknown to anyone who inhabits the area, a thousand octopus-like creatures float in the air.
\footnote{See `\nameref{archmage}', \autopageref{archmage}.}
What they're doing, nobody knows, but on rare occasions all of them battle.

Each scene spent above the mist incurs a chance that an archmage will throw an offensive spell.
However, if a PC successfully attacks an archmage, the others will then attack that same archmage.

When an archmage becomes injured, it generally dies, but sometimes one will fall from the sky, survive inside the labyrinth.

\subsubsection{Dwarves}

Strange `maze dwarves' march through the labyrinth.
They seem to have grown strange after remaining in this strange landscape for so long.
They speak nothing like the normal language of dwarves, but they have the same work ethic -- every day they hunt for gems, gold or other substances in the shifting corridors, and once they find something good, they dig and dig.
Each one carries $4D6 + 50$ gp worth of precious gems and gold nuggets, but are otherwise identical to any dwarven soldier (\autopageref{dwarven_soldier}).

If they see nura, they fight.
Otherwise, they move towards the nearest gem-seam (they seem to have an instinct for exactly where they lie).

Some people say that this realm is a purgatory for greedy dwarves, where their lust for precious stones drives them to work all day, with little sleep, pointlessly.
The labyrinth holds no shops, or luxuries.
The gems are pointless, but still they dig.

Maze dwarves will point any characters in the labyrinth towards the portal to the Realm of Bright Rocks (cell \mbox{[ 1, 1 ]}).

\subsection{Encounters in the Labyrinth}
\label{labyrinth}

Those entering from another Realm begin at the one of the portals.
\footnote{Either the Realm of Darkness and Fire (6,4), or the Realm of Bright Rocks (1,1).}
From there, they see two passages -- one leads to the cell above, and the other to the right.

\begin{exampletext}
  Assuming someone entered from the Desert Realm, they would see two passages: both leading to creatures.
\end{exampletext}

\paragraph{Creature encounters}
mean rolling $1D6$ on the top row.

\begin{exampletext}
  Rolling a `3', and a `1', the two passages show nura and a fungal garden.
\end{exampletext}

\paragraph{Nura Encounters}
Indicate that nura from the same cell on the Nura Encounter Table (\autopageref{encNura}) occupy that tunnel.

\begin{exampletext}
  The cell above our starting cell is [ 1, 2 ] -- on the Nura Encounter Table, this indicates $1D6+6$ goblins.
  So those in the starting cell see two passages\ldots one with 10 goblins, and the other with a fungal garden.

  Assuming the troupe move for the fungal garden, this shifts them one cell to the right.
  Perhaps the goblins give chase, and they run into a wall where shining gems are exposed up by the slowly shifting labyrinth's walls, then they take the first tunnel they see, and there is a \textit{sudden shift} as the walls start closing in quickly\ldots
\end{exampletext}

The encounter table is not a map, but it does work a little like a map.
Assuming the PCs begin at cell `3, 3', if one PC travels up and right, while another PC travels right them up, both with encounter a fungal garden, but they will not necessarily occupy the same space -- these may be different fungal gardens.

\paragraph{Looping edges}
make the encounter table infinite.
Anyone on cell \mbox{[ 6, 6 ]} who travels right would end up on cell \mbox{[ 1, 6 ]}, and travelling up means going to cell \mbox{[ 6, 1 ]}.

\paragraph{Random locations $\longleftrightarrow\hspace{-1.2em}\updownarrow$}
indicate that this passage leads to a completely different cell.
Roll $2D6$, taking the left and right die as the \textit{X} and \textit{Y} axes respectively.
If this rolls yet another Random location, a new tunnel has opened up -- roll twice more to get two new locations.

\paragraph{Gem-seam walls}
suddenly expose themselves in the labyrinth.
The strange dwarves who inhabit this realm always move towards them unerringly, and remain for as long as they can before the walls close in.

\paragraph{Darkness}
occurs because the fireflies which inhabit most of the labyrinth do not enter some passages.

It is unclear why.

\paragraph{Only right $\Rightarrow$}
indicates that this passage only has one way ahead, rather than the usual two routes.
The troupe must continue right, without making any choices until they encounter the `open top $\uparrow$' passage, which indicates that two tunnels are once again available.

\paragraph{Only up $\Uparrow$}
indicates the same thing, but the troupe must wander only upwards until they reach safety.

\paragraph{Sudden shifts}
occur all too often in the labyrinth.
The walls close in suddenly, sealing the passage the troupe came from shut, and begin closing in on the current room.

A Speed + Athletics roll, (TN $3 + 1D6$)
is required to escape successfully.
Those caught in the closing walls are crushed to death.

\paragraph{Slopes up}
are extremely steep, and require a Dexterity + Athletics roll, TN 10.

\paragraph{Slopes down}
require the same roll, but failure indicates that the character starts falling fast, and receives $1D6$ Damage from the fall, then enters the cell on the right.

\paragraph{Path to top}
means yet another steep slope, with a Dexterity + Athletics roll, but this time anyone who manages the climb can breach the ever-present mist of the labyrinth, and see the world.

At the top is darkness, as the mist-ceiling covers the light of the fireflies, below.
However, someone with a light can make out figures floating high above.

\end{multicols}

\settoggle{bestiarychapter}{true}
\encNura
\settoggle{bestiarychapter}{false}

\section{The Realm of Darkness \& Fire}\label{darknessandfire}\index{Realm of!Darkness and Fire}
\index{Nura Land}
\index{Darkness \& Fire}

\widePic{Johan_Jaeger/a_light_in_the_dark}

\begin{figure*}[t]
\label{nuraEncounters}
\begin{encounters}{Darkness and Fire}
  Citadel & Rocky Plains & Result \\\hline
  \li & $1D3$ Goblin nuramancers. \\
  \li & $3D6$ Armoured hobgoblins (page \pageref{hobgoblin}). \\
  \li \lii Lava Man (see page \pageref{lavaman}). \\
  \li \lii Garden. \\
  \li \lii Exposed magma stream. \\
  \li \lii $4D6$ plus 20 goblins (see page \pageref{goblin}). \\
  \li \lii $3D6$ plus 10 deep hobgoblins (see page \pageref{deep_hobgoblin}). \\
  & \lii Rocky rain. \\
  & \lii $3D6$ Nura Wolves (page \pageref{nura_wolf})\\
  & \lii Mana Sink (level $1D3$). \\
  & \lii $3D6\times 2$ Ghouls, lead by a goblin nuramancer. \\
  & \lii $5D6\times 2$ Ghouls, lead by a goblin nuramancer. \\
  & \lii Nura Spider (page \pageref{nura_spider})\\
  & \lii Massive Cliff \\
\end{encounters}
\end{figure*}

\begin{multicols}{2}

\noindent
Deep below the mountains, below the earth, below the bowels of the ocean, there is a realm of eternal heat.
Magma bubbles up from below.
Backwards waterfalls occasionally form as rock ceilings above burst and water -- fresh or salt -- pours in and then quickly evaporates upon meeting the hot ground, then rises up and joins the cold lakes which sit above before the passage is blocked or filled entirely with water.
Food grows plentifully and quickly -- a plant can grow up to a foot an hour once it lands on a fertile corpse or patch of earth in a steamy room, then spurt out a cloud of noxious spores to replicate again.
All creatures within this realm eat constantly and grow at incredible rates.
Many only live for a few days as a longer natural lifespan would too quickly slow down the rapid evolution required to survive in such an environment.

  This is the realm where some poor creatures are not eaten but changed into monstrous versions of natural things and then spat back out to devour the land above with a hellish hunger.  This is the realm of the nura.  Here goblins and hobgoblins multiply like a swarm of ants.  They expand shoddily built tunnels, hoping to find more warm spots, filling in caverns above, then those caverns inevitably collapse.  Many are camped close to dwarven settlements, though they must be very close in order to attack -- the nura find it difficult to travel the long, bare tunnels where no food grows due to their constant hunger.

  The omnipresent heat in the area is too much for most people and requires normal humans to be constantly hydrated if they do not want to accrue serious fatigue problems.  Much of the food of this area is edible and much is not, but telling which is which can be near impossible, not simply due to their oddness but because the plants of this area are constantly evolving and changing to avoid the ravenous mouths of the inhabitants.

\subsection{Features of the Hellscape}

\subsubsection{The Citadel}

While naked and violent hobgoblin tribes run amuck outside, the centre of the realm contains a grand Citadel.
Rather than a building, a large area of black rocks  -- coal, obsidian, and others -- were dug out to form passageways, houses, tunnels and bridges.
The Citadel holds various well-tended gardens, and hundreds of armoured hobgoblins.
A few goblins inhabit the realm with a basic knowledge of nura magic.

\subsubsection{Mana Sinks}

These grand, black, obsidian obelisks hum with an unnatural magical energy.
Once approached, they act as a reverse mana lake, drawing in the most potent source of nearby mana.
They drain 1 MP per round.

\subsubsection{Gardens}

\begin{boxtext}

  The purple tree has a strange smell, and looks altogether upside down, but you can't stare too long as the heat drains all energy from you, and you need to eat.
Pulling off the strange fruit, it tastes sour, and spicy, then the trees far ahead begin to rustle.
Goblins wander through them, though it's not clear if they're more from the Citadel or a random tribe which wanders the hellish wastelands here.

\end{boxtext}

Mauve, purple and brown plants grow rapidly here, feeding on the ambient heat and nutritious soil of the realm.  They grow so fast that people can almost see them getting larger.  The taste is pleasant enough, but unfortunately for outsiders who come here, they rot as quickly as they grow, and anyone feeding on them immediately gains an extremely painful stomach, suffering half their own HP in Fatigue Points.

These gardens can grow and vanish quickly, leaving an area barren.
The nura constantly hunt for them, and staying in one for any length of time invites danger.

The fauna of the area appear quite strange. Much is a type of mauve bamboo which grows almost an inch per second before claiming all nutrition in the area and eventually rotting.  The only water in the area comes from above: on rare occasions water pours from fissures in the roof, but mostly the water comes from when those fissures hit lava streams, then forms steam.  The steam coalesces across a roof and eventually `rains' down.

Trees here are built to catch water in their roots from above, so they tend to be spiky at the top.  Meanwhile, the only light available is from magma streams, so leaves all turn downwards to capture lights.

\subsubsection{Magma Streams}

These streams of molten rock create intense heat all around them, inflicting 4 Fatigue Points each scene.
Anyone touching one of the streams suffers $2D6+2$ Damage.

\subsubsection{Nura}

The nura here cannot make weapons or armour.
Besides the lack of raw materials, they simply don't know how to.
The best they can do is use sticks from gardens to craft spears or clubs.

Some few within the Citadel own weapons or armour from raids on dwarvish settlements above.

\subsubsection{Rocky Rain}

Everywhere outside the Citadel is unsafe.
Random sections of the roof crumble, letting little flakes of rock float down onto everyone's head.
The crumbling rocks become heavier and heavier, soon inflicting $1D6$ damage on anyone below.
Characters caught in this `rain', can leave the area if they start marching immediately.

The Citadel still experiences rain, but it has sufficient roof cover to remain safe in most places.
Once the rain has gone, hobgoblins come out to clear up the mess, gathering the rocks, and pouring them into a nearby magma stream.

\end{multicols}

