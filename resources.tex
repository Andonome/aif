\chapter{Tools \& Toys}

\epigraph{As far as the laws of Mathematics refer to reality, they are not certain; and as far as they are certain, they do not refer to reality.}{Albert Einstein}

\section{Mana Lakes}\label{mana_lake}\index{Mana Lakes}

\begin{multicols}{2}

\noindent
Throughout Fenestra there are little nodes where mana wells up from the ground like a wellspring.
In these locations, some number of MP is created each round which then raises the current MP of mages or nearby magical items.
Whoever has the most empty mana slots -- whether an item or a person -- gets that mana point.
It pours into the biggest vacuum like water into a hole.
So if a gnome with a current maximum of 6 MP spent 4 MP, that would mean they had a deficit of 4.
Another mage who had only spent 1 MP could only get a point after that gnome.\footnote{If two are tied for having the biggest deficit, ties are broken by the win going to the highest current maximum MP holder, then the highest Wits Attribute holder, then the highest Intelligence Bonus holder.}

Mana lakes also extend the range of all spells cast within to the entire mana lake at least as spells travel easily from one end to the other.  Each has a level, denoting the number of MP it doles out each round, so a mana lake of level 3 would give out 3 MP per round.

Magical items regenerate mana just like people do, so many miracle workers create magical items especially to sit in mana lakes, and recharge quickly.

A selection of such mana lakes with magical items (or some other permanent magical effect) are given below.
If your \glspl{pc} encounter a mana lake on the Random Encounter tables, or you want one for one of your own stories, just pop one of the following into the map.

%! \manalake{King's Hand Valley}{Necromancy 4}{Divinity (Qualme)}{Instant}{Ranged Sentient Talisman}{+4}{9}{2}

There is a valley which was once filled with bandits who would pillage local villages.
To put a stop to it, a priest of Qualm\"{e} cursed it with a mummified hand of former Rex Dalius Quennome.\footnote{See page \pageref{h_dalius}.}
Nobody has any idea how he obtained the hand.

\begin{boxtext}

  Your arrow hits the bandit leader in the face, killing him instantly.
  The other duck down behind the hill, then screaming starts and all of them jump over the top, running towards you, despite your drawn bows.

  As you loose more arrows, you see the leader you shot stand back up with an arrow still protruding from his eye socket and blood smeared from his mouth downwards.
  Another stands up, his throat torn out, and both move quickly towards the bandits running towards you.

\end{boxtext}

If, at any time, a dead humanoid body is in the area, it rises from the dead to eat the souls of the living.  Most of the time such dead creatures like to stay and feast off the ambient magics in the area.  They hunker down under snow, or red, squelching leaves, or simply lie and let the moss grow over them.   At first impatient traders would attempt to sneak across the valley as a short cut.  Later, only a few brave warriors would come to test their strength.  Now none come, and the dead are being eaten away by the moss.  But they will not leave.  They simply remain, and wait to hear human speech again, and feel human warmth.  It has been a long time since their last meal and they have gathered a decade's worth of hunger.

The hand hangs from a tree, and looks very much like a hanging vine or a part of the tree it hangs from.
It holds a total of 7 MP and spends 2 to raise anyone in the area as a ghoul as per the second level of the necromancy sphere Calling the Dead.
It automatically activates instantly for as long as it has MP to use.

%! \manalake{The Petrified Forest}{Illusion 5}{Devotion (Alass\"{e})}{4 scenes}{Massive Sentient Artefact}{+5}{13}{1}

One day during the season of C\'{a}lea, a great flood came and washed a marsh clean out.  Mud trickled out for the entire season and when all was done, a petrified forest was uncovered.  Stone cylinders from one to six feet tall were uncovered throughout this mile long valley.  The valley is currently covered in grass and fungi but is otherwise uninhabited by any vegetation.

Around the centre of the valley, anyone mentioning something will soon see an illusion of that thing.  The illusions each have a total of 6 points to divide between Speed and Dexterity Bonuses, but have no other Traits.  People moving through might imagine what the forest was once like might see the petrified trees turning into real trees.  Those worrying about monsters will see an illusion of a monster.

The illusion holds 15 MP in total and can spend 3 at any point in order to create an illusion of up to 5 steps wide (or tall).
Basic illusions are TN 11 to identify as illusions while larger illusions are only TN 9.

%! \manalake{The Myriad Web Forest}{Aldaron 3}{Devotion (Alass\"{e})}{2 scenes}{Talisman}{+4}{5}{1}

\begin{boxtext}

  You continue towards the sound of the crying deer, and brushing foliage aside you can finally see the great tree, a stone's throw away.
  Half way up the gargantuan tree, a giant, fat spider looks down, clasped to various branches.
  Beneath it, another giant spider is bound to the tree by a great web with two legs sticking out the side.
  Beside it, bears, deer, and one chitincrawler are trapped in more webbing, all around the tree.

  The bushes start to rustle, and man-sized arachnids pour out from every direction.

\end{boxtext}

In the middle of a great forest sits a massive and majestic Shiva tree.
A guardian spell was placed on it long ago by a priestess of Laiqu\"{e} such that if anyone enters the area, marauding animals are summoned to the area.
This is no problem to an initiate of Laiqu\"{e}, as they can typically charm such animals -- but anyone who was not a friend of the forest could be in serious trouble.

Since that time, a massive nura spider has taken up residence in the area.  This spider, grown to gargantuan size and twisted by strange magics, has made its home here.  It perpetually traps summoned animals in its web then eats them.  The summoned animals do not leave until they have been killed or leave the area.  However, the nura spider does not simply kill them but bind them with its web and slowly liquefies them with its poison.  This blurring of the boundaries of when the animal is dead or alive, of where the web ends and the creature begins, means that the animals so summoned remain after the spell has ended.

At any given time, its web is full of up to a half dozen assorted creatures -- griffins, aurochs, a bear and such.
The web must be laid slowly, but once on the ground or stretched between branches and tree trunks it is extremely strong.

\npc{\N}{Gargantuan Nura Spider}

\animal{6}% STRENGTH
{2}% DEXTERITY 
{3}% SPEED
{0}% WITS
{4}% DR
{3}% AGGRESSION
{Athletics 2, Tactics 2}% SKILLS
{\web}% ABILITIES
{}

During her time here, the spider has become incredibly large and fat and has given birth to many children.
Some move away to terrorise nearby farms.
Others remain and usually starve to death after failing to fight her for food.
She has one mate -- one male spider with whom she mates.
When hunger overcomes her she sometimes eats another section of his legs.
Currently, he has two and a half legs left.
He cannot move properly as she she has bound him in webbing and left him on branch near the top of the Shiva tree, hanging like her prey.
Every so often she approaches the top to mate with him, or to stop one of her children attempting to eat him.

\npc{\T[8]\N}{8 Nura Spider Children}

\morphspider

The Shiva tree holds 11 MP in total and can conjure animals as per the spell Forest's Call.

%! \manalake{The Wishing Wellspring}{Conjuration 5}{Alchemy}{6 scenes}{Sentient Artefact}{+6}{14}{1}

Long ago gnomes who lived near this area thought it would be fun to create a functioning wishing well for anyone who needed it.
They found a natural wellspring deep underground and created a layer of polished stone on top, complete with a gazebo.

In the centre of this stone floor placed in the middle of a large plain is the well.
The metal bucket and ropes were reinforced with steel wiring and function to this day, though they could do with being replaced.

\begin{boxtext}

  The wishing spring lies just ahead.
  The roof has deteriorated badly, but the stones laid all around still shine brightly.
  All the rocks are smooth, and as many are gemstones the entire platform glistens in the sunlight.
  Then you notice a foot sticking out the side.
  It belongs to a fat man whose face has been completely burnt off.

\end{boxtext}

Any time someone makes an audible wish, puts the well's bucket down and drops a coin into the well, a summoning spell activates and creates the thing they wished for inside the bucket.
When it is drawn back up, the wisher can grab the item.

Some time after the well's creation, a priest of Qualm\"{e} came and noted how some people were disrespecting the well and were using it too commonly.
Their wishes threatened to break the magic which allowed the well to function.
He decided to protect the well with a spell of his own.
He carved a message in elvish -- the tongue of all academics -- into the side of the well.
It proclaims ``Mine m'{e}re ilyain er'' -- ``Only one wish for everyone''.
The skull of an older priest of Qualm\"{e} was placed at the bottom of the well to guard against misuse.
If anyone attempts a second wish a storm of lightning and fire erupts from the well and spreads across a 5 step diameter, dealing $2D6+2$ Damage to everyone there.

The well holds a total of 16 MP and an use any of the first four levels of the Conjuration sphere.
If the well spends mana to summon an item, the mana must, of course, remain spent, so the well will be depleted by 3 MP for as long as the item remains in use.
If the well runs low on mana because too many people are making wishes, it simply cancels the oldest spell it has cast and the wished-for item disappears, whether it is by the well or a hundred leagues away.

\subsection{The Dwarvish Prophet}

\textit{Level: 3}

\noindent Deep underground there are many natural streams, flowing quickly into the depths.  Some of these run alongside passages which the dwarves carve out between the various realms they inhabit or realms they inhabited.  It is not uncommon for underground travellers to wander along natural caverns only to later find a dwarvish road to take them part of the journey.  On one of these roads is a carving of the face of a great dwarvish prophet.

\begin{boxtext}

  The massive face has no beard, leading almost everyone to believe she was female.
  A waterfall flows down her face and has worn down everything but the nose, leaving a single, perfect nose and part of the lips while the rest of the face looks more and more like bare rock with each passing decade.
  There are various possibilities as to who the face might represent -- there have been many dwarvish prophets and almost half were female, but the exact identity has never been discovered.

  As your guide passes through, runes around the statue's head glow.

  ``Exactly which happens is not well understood.'', he explains.

  ``Sometimes she blesses you, sometimes she curses you, and sometimes she does nothing at all.''

  ``Some say that spells themselves can become insane after long centuries sitting along in the dark.

  The guide wanders past, content with his fate.
  It's time for you to move past too.

\end{boxtext}

Those following the Codes of the Tribe or Experience receive $2D6$ FP (costing 2 MP).
Those following Qualm\"{e} or the Code of Acquisition lose $1D6+4$ FP.
Others are unaffected.
The statue head holds 12 MP in total.

%! \manalake{The Resting Grove}{Illusion 5}{Devotion (Alass\"{e})}{4 scenes}{Massive Sentient Artefact}{+5}{5}{2}

\begin{boxtext}

  Large, ripe fruits litter the forest clearing ahead, and short tufts of grass lay about and look perfect for sitting.
  Looking up, you suddenly notice various tree houses, with men looking down at you.

\end{boxtext}

This group of trees and bushes looks like a lucky find.  There are apples, blueberries, raspberries and tiny strawberries growing all around.  Even better, when one piece of fruit it picked, more grows in its place.  If it rains, the trees naturally knit their branches together  to form an overhead canopy thick enough to keep the rain out.  If people nestle up to their trunks, they can warp and open up, allowing people to sleep inside the newly created hollow.  The entire grove is a luxurious place which caters for the every need of anyone within, including creating simple items like tables or goblets out of wood.

Once people leave the area, any items they have remain as they were, but the trees return to normal.  Any items stored in those temporary shelters is absorbed by the tree, and further mutating magics will not necessarily bring the items back because the tree might not open out in the same way the second time it opens.

While this place was once a welcome resting stop for many a traveller, or at least for those brave enough to go near such strange magics, it has since been taken over by bandits.
They use the trees as a continuous supply of arrows for their bows.
They sit in tree houses at the top and shoot down to any attempting to get them out of the area.
Since the grove is not far from a major road, the local town master has made a high price on the bandits' heads.
None have claimed the bounty yet.

\humanarcher[\npc{\T[12]\E\Hu}{12 Bandits}]

The grove contains 13 MP in total and spends 4 in order to make any alteration or 1 MP to calm any wild animal in the area.
All spells cast count as having an Intelligence Bonus of +3.

\end{multicols}

