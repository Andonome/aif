\chapter{Bestiary}
\label{bestiary}

\settoggle{genExamples}{true}

\section[Nura]{Nura \N}

\begin{multicols}{2}

\subsection[Beasts]{\A\N\ Beasts}

Nura beasts are normal animals which have been captured inside caves leading to nura kidnappers.
They are later taken into the depths and experimented upon by the magics of the Path of Nura, emerging as misshapen creatures.
These creatures all inherit the hunger of the nura and tend not to survive long as they must either starve to death or come to blows with humans.
Their extreme hunger can give them incredible bravery, adding 2 to all Morale checks.

While `standard' nura beasts are presented below for easy, persistent reference, all admit of variation -- stranger creatures than these could be mutated and these creatures can be given any number of additions.

\best[\N]{Nura Cat}\label{nura_cat}

House cats and wild cats turned into nura are capable of pulling down a deer with a single swipe to the jugular.
They move at incredible speeds and can often pounce on people with a drawn sword faster than they can move to defend themselves.

\begin{boxtext}

  The chicken coup shatters as a cat the size of a man explodes out of the hut and into your face.

\end{boxtext}

\paragraph{Abilities:}Nura cats have such vicious claws they grant 1 Damage to all attacks.

\begin{boxtext}

  After your swing, the cat bounds towards the distant forest.

\end{boxtext}

\paragraph{Ecology:} Once released into the wild, these cats universally remember and cling to their old territories.
They often return home and eat their original owners, and if nobody hears the screams, it might be a day before anyone wonders why they haven't seen anyone from that house in a while.

\paragraph{Encounters:}

\begin{itemize}

  \item
  Farmers in the distance panic as sheep scatter -- a nura cat flees with a sheep.
  ``That's the third one today'', one farmer mutters.
  If the players attack, the cat flees.
  If they guard the pen, it jumps between them.
  If they don't manage to kill it, the beast kills all the livestock in the area, bit by bit.
  \item
  While travelling through the forest, the cat jumps out of hiding and kills a horse, then retreats and waits for the party to abandon the horse.
  \item
  A hungry cat tails the party from a distance.
  As they sleep, it stares at them.
  \item
  The party happen upon a nura cat eating a cow it killed and dragged away from a farm.
  It hisses, but does not approach if left alone.

\end{itemize}

\nuracat

\best[\N]{Nura Crab}\label{nura_crab}

These giant, spindly crabs are taller than a man and can be an extreme menace around various islands.
Many have grown tired of fishermen batting them with oars so they have learned to snap weapons with a swipe from their claws.

\begin{boxtext}

  As your ore hits the water, something underneath grabs it, then pulls down hard.

\end{boxtext}

\paragraph{Ecology:} These creatures can be a real menace -- they eat all the fish in a lagoon, leaving the local fishermen with nothing and little way of dealing with the problem except waiting for the crab to starve to death.

\begin{boxtext}

  Looking at the shore, you can guess what's underneath.
  Six crabs, as large as a man, with bright red shells, have pulled out of the water as an old man inspects his boat.
  One grabs his neck, and the rest form a scrum around him.
  He dies with no more sound than the heavy carapaced legs tinkling across the stony shore.

\end{boxtext}

\pic{Alhaz/crab}

\nuracrab[\npc{\A\N}{Nura Crab}]

\best[\N]{Nura Horse}
\label{nura_horse}

These stretched-out and all-too-human horses have large, sharp teeth which they use to grind meat in a horrid defiance of nature.

\begin{boxtext}

  The hobgoblins stop suddenly in the burning street, and dart into the stable, and the horses begin to scream.
  By the time you reach the building, they break out, riding the horses.
  Their mounts are larger than ever, and they immediately gather around a collapsed woman and begin to eat her alive as the riders cheer them on.

\end{boxtext}

\best[\N]{Nura Slug}
\label{nura_slug}

These giant slugs eat constantly -- their corrosive slime can eat up whole fields or areas of arable land overnight.

\nurahorse

\nuraslug[\npc{\A\N}{Nura Slug}]

\best[\A\N]{Nura Spider}
\label{nura_spider}

These terrifying creatures, made from continuously feeding ordinary house spiders, can grow to any size given enough food.

\begin{boxtext}

  You patrol the city's walls while every single citizen stands at their own door, and every farmer from outside gathers in the central square, terrified of the nura army outside.
  Outlasting the nura should be easy, just as long as the city's walls hold strong and the guards patrolling those walls push away any ladders that the hobgoblins put up.
  You have food, and they need it far more than you.

  Suddenly, massive legs lash over the side of the walls.
  They grope about and snatch up the wall-guards, then disappear.
  Once the screaming stops, ladders snap onto the side of the empty city wall.
  The army is climbing up into the city.

\end{boxtext}

\paragraph{Encounters:}

\begin{itemize}

  \item
  A single slug creeps over to eat the party's food as they rest.
  Then three more come, then ten more, then twenty.
  Soon the area is covered in nothing but the slug.
  \item
  Farmers rush to make makeshift armour, as the giant slugs are coming, and they can spit acidic bile.
  The party can help with a Wits + Crafts check, TN 8.
  Each margin creates a serviceable suite to use to confront the slugs.
  Moments later, dozens overrun the village.
  \item
  The party find a caravan of traders fleeing from nura slugs.
  The horses were killed, and the slugs currently sit on top, eating them.
  The party can easily go around, as the slugs just want to eat.
  However, the traders beg them to intervene before the food they were transporting is eaten too.

\end{itemize}

\nuraspider

\paragraph{Abilities:} Such spiders can lay webs along the ground -- spotting them requires a Wits + Vigilance action, TN 9.
Once caught in the web, a resisted Strength roll against the web is required -- each spider's web has a Strength rating equal to its own plus one.
Those successfully damaged by a spider are poisoned, inflicting a number of Fatigue Points equal to the Damage endured.

\best[\N]{Nura Wolf}\label{nura_wolf}

\paragraph{Tactics:} Wolves usually gather around someone, several flank them and one jumps onto their chest, wrestling them to the ground.
The rest immediately attack with a Damage bonus.

Like any other wolf, nura wolves hunt in packs.
Unlike other wolves, they are faster.
They can jump on someone and wrap their powerful jaws around their throat before the target has considered drawing a sword.
They move together at the speed of lightning and fall upon farms and sometimes even towns, stealing babies, feasting on the folk and their cattle and generally consuming everything in front of them.

\paragraph{Encounters:}

\begin{itemize}

  \item
  In the distance, the wolves howl.
  A moment later, aurochs race past the party as a pack chase after them.
  \item
  Nura wolves stalk the party, but instinctively know never to be seen.
  Their stalking movements stop any missile coming towards them -- they attack together and all target a single character, grabbing them, and pulling them back into the shadows.

\end{itemize}

The pack attacks at night.
Wolves are some of the few nura capable of waiting, if only for some hours.
If encountered before that time, they stalk, and wait for their prey to become tired.

\nurawolf[\npc{\A\N}{Nura Wolf}]

\subsection[Nura Humanoids]{\E\N\ Nura Humanoids}

The nura gods, which nobody has seen but must surely exist, cannot create, only twist the creations of others.  They took the various creatures made by other gods and then had them misshapen, poisoned and enhanced before filling them with a ravenous hunger.

Nura humanoids occasionally gain access to magic -- in such cases, bump their Intelligence bonus up to 0 or 1, add some mana points but retain the remainder of the stats.
Nura creatures have a tribal instinct to follow anyone who understands the Path of Nura.
Further, those magic users who gain access to the Saurecanta sphere are often treated reverentially in a matriarchal manner due to their ability to turn mundane creatures into brothers in arms.
Nura humanoids more than the beasts have an instinct which powerfully propels them to eat fellow nura creatures last and non-nura humanoids creatures first.

Hobgoblins who successfully invade a dwarvish stronghold must eat until they devour all food stored, and then either journey back down, or travel farther up in order to look for more food.
The next step upwards is often gnome-territory, where the nura will eat all they need on that day, and transform any remaining gnomes into goblins.
Finally, they break out the top, and begin turning humans into ogres.
This merry-go-round of creatures can often have a single band of nura replacing their number, bit by bit, replenishing their own numbers with a different race at every point.

\paragraph{Society:}
Nura societies cannot last long away from their deep, underground homeland, because no natural environment can sustain their hunger.

Nura naturally love tyrants and in fact respect no other rulers.
Said rulers are never able to tame their intrinsically destructive natures but are expected mainly to select good targets, something measured by the amount of food the gained from a raid.
Goblins, being the most intelligent of the nura, make for natural leaders, but many nura will also follow any powerful miracle worker, or even the occasional dragon.
If said leaders ever fail to deliver the goods, perhaps by selecting raid targets with precious little food, they are generally eaten by the rest.

Nura are generally too disorganized to fashion their own clothes or weaponry, so everything they have is stolen.
Their underground societies, in the Realm of Darkness and Fire,\footnote{See page \pageref{darknessandfire}.} use spears and rocks as weapons almost exclusively.
\best[\N\E]{Goblins}
\label{goblin}

Goblins are twisted and stretched little versions of gnomes.
They come in all shapes and shades, including green, long-eared, grey, fat, and skeletal.
The gnomish intelligence has left them, though they are still capable of creating nasty little underground traps, and occasionally remember how to cast illusion spells.

\begin{boxtext}

  A hand reaches through the open window and grabs the steaming chicken whole.

\end{boxtext}

\goblin[\npc{\F\M\N}{Goblin}]

\paragraph{Tactics:} `Suicidal' hardly begins to cover the goblins' attitude to warfare, but this does not mean they come unprepared.
They favour spears as their weapon of choice, and like to start any fight from a distance.
They are well aware that they cannot run away from many creatures, so their best bet it always to fight to a fast conclusion.

\begin{boxtext}

  Running out the door, a rock descends upon your head.
  You crash to the ground, stunned, and look up to see a dozen goblins on the roof, pulling off large segments of the chimney to throw down on the rest.

\end{boxtext}

\pic{loh/goblin}

\best[\F\M\N]{Goblin Nuramancer}
\label{goblinnuramancer}

Among the nura, none are very bright, but some rare goblins do learn to cast spells.

These creatures often specialize in Necromancy, and raise powerful armies of ghouls.

\goblinnuramancer[\npc{\E\N}{Goblin Nuramancer}]

\best[\N\E]{Hobgoblins}\label{hobgoblin}

Hobgoblins are twisted versions of dwarves, grown as large as a human.
They have overly long ape-like arms and sloping foreheads.
They sprint over the land at an incredible rate, and when armies are raised, little can stand against them.
Often their beards will turn stark white while their skin appears as mottled rocks.
 
\hobgoblin[\npc{\E\N}{Hobgoblin}]

Hobgoblins deep underground fight, hunt for food, switch tribes, and fight some more.  They cannot make the complex equipment dwarves do, so they have to pick up weapons as they go.

\pic{loh/hobgoblin}

\best[\F\M\N]{Ogre}
\label{ogre}

When humans are twisted by the things in the deeps they become not simply tall but monstrously tall.
They become oversized mockeries of people, constantly changing weight and appearance.
When they are well fed they grow incredibly obese and start sprouting hair in new places.
When they starve, they eat away at their own fat, becoming wide-boned skeletal creatures with deep sunken eyes.

\ogre[\npc{\E\N}{Ogre}]

\begin{boxtext}

  The mad priest sits at the other end of the massive room in front of the only source of light -- a burning brazier.
  The side of the room initially seem to contain massive statues, but then one moves.
  They look like ogres, but decked in heavy chainmail, stitched together from the chain of every dead warrior brought back from town.

  They move in unison, like well-trained dogs, and raise their swords.

\end{boxtext}

\paragraph{Encounters:}

\begin{itemize}

  \item
  A single skeletal ogre spots the party in the distance.
  Three days without food has left him emaciated and insane.
  He charges towards them with a horse scream.
  \item
  The party see massive rocks lying on the path ahead.
  When they approach where the rocks are, ogres above launch rocks and boulders at the party.
  \item
  Four ogres spit-roast a horse.
  Leather armour, swords, and other \gls{guard} gear lie nearby.
  The ogres have massacred a \gls{guard} party already, so they have no need to attack.
  Spotting the party, they say `move along and you won't get et'.

\end{itemize}

\end{multicols}

\settoggle{genExamples}{false}
