% !TeX root = ../main.tex

\section*{Acknowledgements and Thanks}

\begin{multicols}{2}

Many thanks to the house Physicist Angus MacDougall, for working out how seasons should work in \autoref{seasons}.

\subsection*{Artists}

\subsubsection{Johan Jaeger}

For:

The Mt Arthur image, \pageref{Johan_Jaeger/mountain_river};
the Realm of Bright Rocks, page \pageref{Johan_Jaeger/desert_wizard};
the Realm of Shifting Corridors, \pageref{Johan_Jaeger/ancient_rocks};
and the Realm of Fire and Darkness, page \pageref{Johan_Jaeger/a_light_in_the_dark}.
\subsubsection{Studio DA}

For:

The chitincrawler, page \pageref{Studio_DA/chitincrawler},

The ogre, page \pageref{da:ogre};
the ooze, page \pageref{Studio_DA/jelly};
and the woodspy, page \pageref{Studio_DA/woodspy}.
(Find them at {\tt www.ko-fi.com/studioda})

\iftoggle{core}{}{%
  \subsubsection{Leonard}
  for the `Next Day' image (\autopageref{Leonard/next_day}).
}

\subsubsection{Tom Prante}

For:

Shale, page \pageref{Tom_Prante/autumn};
Liberty, page \pageref{Tom_Prante/swamp};
the Pebbles, page \pageref{Tom_Prante/ancient_valley};
Quennome, page \pageref{Tom_Prante/the_old_path};
Whiteplains, page \pageref{Tom_Prante/inaok};
and
Eastlake, page \pageref{Tom_Prante/winter}.

\subsubsection{Alhaz}

For the nura crab, page \pageref{Alhaz/crab}.

\subsubsection{Decky}

For the watcher, page \pageref{Decky/watcher}.

\subsubsection{Lady of Hats}

For the 
dragon, page \pageref{loh/dragon};
dryad, page \pageref{loh/dryad};
gnoll, page \pageref{loh/gnoll};
goblin, page \pageref{loh/goblin};
griffin, page \pageref{loh/griffin};
hobgoblin, page \pageref{loh/hobgoblin};
and
the rock man, page \pageref{loh/rockman}.

\end{multicols}

\section*{How to Not Read This Book}

\begin{multicols}{2}

\noindent Who has time to read \pageref{lastpage} pages?  It's not like you can remember all of them.

Well for a start, we don't need to know the entire map of the world.
It's divided into seven regions, so just pick one for your campaign -- there's snowy waste, a deep forest ruled mostly by elves, islands, and a more urban area full of political upheaval.
Each has a unique set of encounters which characterize the location.
Roll up a couple of encounters to get to know the area, look up the creatures in the bestiary chapter, and write down a couple of the encounters you want to pull on the players on your \gls{gm} sheet.
The other creatures don't matter if you're not using them.

Once you have a few antagonists and creatures for your players to encounter, check out the creatures presented.
You'll find a few suggestions for monsters' encounters in the bestiary, \autoref{bestiary}.

You can just skim-read the Games Master Resources section.
It has some magical items, mana-lakes, et c. -- just remember where they are so you can grab them when you need them.

\end{multicols}

\begin{alltt}
COPYRIGHT
       Copyright \copyright 2019 Free Software Foundation, Inc.  License GPLv3+:
  GNU GPL version 3 or later <https://gnu.org/licenses/gpl.html>.
       This is free software: you are free to change and redistribute it.
  There is NO WARRANTY, to the extent permitted by law.

\end{alltt}
