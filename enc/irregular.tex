\documentclass[10pt,twoside]{book}

\usepackage{config/bind}
\usepackage{config/booklet}
\setcounter{bookLevel}{2}
\newglossaryentry{fashionista}{
  name={Weft},
  text={Weft},
  type={people},
  prefix={Warden},
  description={leader of the world of fasion among the \glsfmtplural{warden}}
}

\longnewglossaryentry{susjot}{
  name={Titrate},
  text={Titrate},
  type={people},
  prefix={\glsentrytext{jotter}\space},
  description={hurries around every nearby \glsentrytext{bothy}, and impatiently micromanages every \glsentrytext{guard}}
}

\newcommand\susjot[1][]{
  \Person{\NPC{\glssymbol{sylf}\M\Hu}{\Glsfmttext{susjot}}{greasy}{rotates rings}{to hurry up, and finish}#1}%
    {{2}{1}{2}}% BODY
    {{1}{0}{1}}% MIND
    {
        \set{Athletics}{2}
        \set{Deceit}{1}
        \set{Larceny}{1}
        \set{Survival}{2}
        \Dagger
        \longsword
        \partialchain
        \addtocounter{fp}{5}
        \longsword
    }% SKILLS
    {\adrenalinesurge \charge}% KNACKS
    {2 rings worth 10~\glspl{sp}, wax tablet (for notes)}% EQUIPMENT
    {}% ABILITIES
}


\longnewglossaryentry{keep}{
  name={The Keep},
  text={the Keep},
  category={location},
  prefix={a\space},
  description={sits on a hill}
}

\longnewglossaryentry{smuggler}{
  name={Feliscourt},
  type={people},
  description={has a big keep to keep}
}


\longnewglossaryentry{enchantedLands}{
  name={The Enchanted Lands},
  text={the Enchanted Lands},
  type={people},
  description={have no crime, nor murder, no broken promises, no singing out of keys, and to jokes}
}

\longnewglossaryentry{plateauGardens}{
  name={The Plateau Gardens},
  text={the~Gardens},
  plural={Plateaus},
  type={people},
  description={have smooth, stone edges, as high as a fortress wall.
  The gardens have few monsters to fear, but also very little water}
}

\longnewglossaryentry{SnailTamer}{
  name={Sint\"e},
  type={people},
  parent={plateauGardens},
  description={is a fearless snail-tamer, who needs a nap}
}

\longnewglossaryentry{LifeElder}{
  name={Hi},
  type={people},
  parent={plateauGardens},
  description={she says, happily, but rejects your labels}
}

\longnewglossaryentry{MindElder}{
  name={Ungwe},
  type={people},
  description={keeps his promises, and your promises, and everyone's promises}
}



\declareRegions{broch,village,lonelyRoad,Forest}

\begin{document}

\miniCover{Irregularities}{}

{
  \footnotesize
  \bookletThreads{broch}
  \bookletThreads{lonelyRoad}
  \bookletThreads{Forest}
  \bookletThreads{village}
}

\clearpage

\thread{By Any Other Name}

\pagestyle{minizine}%

\begin{exampletext}
  Farmers have found claw-marks rending the earth.
  A terrible smell comes from the forest, which makes them boke.
  It might be \pgls{basilisk}, but \glspl{basilisk} don't roar\ldots
\end{exampletext}

Whatever it is, the \glspl{pc} must find and kill it.

\segment[\gls{morning}]{broch}% AREA
{Easy Job}% NAME
{\Pgls{jotter} needs the troupe to kill \pgls{basilisk}}% SUMMARY

\Gls{jotter}~Filchvore congratulates the troupe on their mission, and mentions that \gls{basilisk} bodies can fetch a good price, and reminds them that half of the sale goes to the \gls{templeOfBeasts}.
The \gls{village} in question is fifteen miles down the road, so they'd best get moving.

\iftoggle{intro}{
  \paragraph{If \pgls{pc} asks for better \glspl{weapon},}
  they can roll \roll{Charisma}{Empathy} at \tn[10].
  Success means the \emph{player} should interpret how the character might ask for free equipment.

  \paragraph{If \pgls{pc} chats to other \glspl{guard} in the \gls{broch},}
  \gls{gDigger}~\composeHumanName\ laughs at the notion of \pgls{basilisk} roaring.

  \begin{speechtext}
    \Glspl{monster} don't roar.
    That's why people call them `the voiceless'.
  \end{speechtext}
}{}

\segment[\gls{afternoon}]{lonelyRoad}% AREA
{The Dry Valley}% NAME
{A trader asks the troupe to stay the night}% SUMMARY

The dry valley never floods, even during \pgls{storm}.
Caverns down to the \gls{deep} swallow the water before it can rise too high.

\begin{boxtext}
  Two plumes of smoke rise over the tall trees in the distance.
  After a mile, the road descends into the Dry Valley, and the trees open for a blessed moment.
  The distant smoke comes from \pgls{village}, over five miles away.
  The second comes from \pgls{bothy}, a little down the road, where two men push a donkey into a cart.

  Going down the road, the trees crowd around again, obscuring everything.
\end{boxtext}

Ten traders stay in the \gls{bothy}, with three carts.
They have no \glspl{guard} with them, so once they see the \glspl{pc} they begin asking a thousand questions, and speaking in a friendly manner, hoping the \glspl{pc} will stay and watch their animals overnight.
The \gls{bothy} barely has enough room for the ten traders, so they have to take all the vegetables from the \gls{village} out to make room for their horses and donkeys to sleep in some safety.

\iftoggle{intro}{
  If the \glspl{pc} have \pgls{torch}, they can press on and make the \gls{village} that night.
  However, any light they have (or noise they make) may attract nearby \glspl{monster} ($\frac{1}{\dicef{6}}$ chance).
}{}

Traders \composeHumanName\ and \composeHumanName\ have little to sell, but if the troupe need food, they can offer \rations, \rations, or \rations\ for \mkPrice[cp]{21} each.

\currentName\ mourns her old dwarf-friend --- Dent --- who protected her and her caravan the last time she came, five days ago.
He ran into the forest, shouting something about `wyrmsign', and nobody heard of him since.

\segment[\squash~\gls{night}]{lonelyRoad}% AREA
{Dank Night}% NAME
{Something wanders the forest, and it reeks}% SUMMARY

Whether the \glspl{pc} sleep in the \gls{bothy} or not, they hear a sound in the night like bedsheets being torn apart.
Soon after, a stench comes.
The smell upsets the stomach, and inflicts \pgls{ep} on each character.
Nearby animals begin to panic.
The troupe will not sleep well, so they must take another \gls{ep} due to the dreamless night.

If the \glspl{pc} rush out to find the source of the smell, find it on \gls{segment}~\vref{irregularDragon}.

%\thread{The Condition}

\segment{Forest}% AREA
{Wyrmsign}% NAME
{The troupe find patches of torn-up earth}% SUMMARY

Patches of earth have been torn up by massive claws.
An \roll{Intelligence}{Medicine} at \tn[12] tells \pgls{pc} that someone's looking for mushrooms to make \pgls{elixir}.

The troupe can follow the claw-marks to the maker, five miles away.%
\footnote{See \vref{irregularDragon}.}
They roll \roll{Wits}{Survival} at \tn[6], but failure means they become lost.

\segment{Forest}% AREA
{Grounded Dragon}% NAME
{He will not leave until someone makes \pgls{elixir}}% SUMMARY

\label{irregularDragon}

Approaching the dragon, the troupe may see flame-marks on trees --- as the dragon tries to alleviate himself, he often lets out a little flame.

\begin{boxtext}
  Ahead, nestled among the trees, a mass of green scales writhe around its own bloated stomach.
  Claws caress and push into the ballooned abdomen.
  The shoulders and tail twitch.
  The \gls{fiend} stares hatefully, with bright-yellow eyes.
\end{boxtext}

\begin{speechtext}
  Do not flee.
  I will not harm you.
  I have needs.

  There is a cave-mouth, not too far from here.
  Willow trees surround it, but keep their distance, making the shape of an eye.

  I can no longer fit inside, my stomach has become so heavy.
  Go there, to the cave mouth, and descend until the smell of lilacs.
  This smell comes from a tall, green, mushroom.
  Cut that mushroom down, and return with a barrel's worth of the cap.

  I must have \pgls{elixir} from that mushroom's cap, so I can fly again.
\end{speechtext}

\paragraph{Any rude speech}
results in a single shot of flame from the dragon.

\segment[\gls{night}]{village}% AREA
{Hurry Up}% NAME
{The dragon lets the troupe know that it waits for them}% SUMMARY

The dragon approaches the \gls{village} at night, just to let the troupe know it still waits for that fungus.
The troupe will hear its irritated, painful, scratchings across nearby trees.
Then the smell comes as it farts.
Everyone in the \gls{village} gains \pgls{ep}.

\segment{Forest}% AREA
{How Long?}% NAME
{The dragon wants the fungus, fast}% SUMMARY

The dragon has run out of patience.
If they went into the cave (\vpageref{caveMouth}) it waits for them to come back up.
Wherever they are, it will have that fungus from them, or eat them.

If the fungus is found, the dragon creates a fire, and brews \pgls{elixir}, using some water and well-placed rocks.
\Pgls{interval} after, he expels his problem --- an undigested dwarf, in full plate armour, with an undigestible beard.

\subsection{Locations}

\subsubsection{At the \Glsfmttext{village}}\label{wyrmBailey}
the locals don't want a bunch of \glspl{guard} hanging around, and they're not shy about saying so.
They think the troupe should head into the forest as soon as possible, kill whatever's making those noises, and leave.

\begin{speechtext}
  Shame you couldn't get proper equipment, like that dwarf --- what was his name? --- shiny plate armour, head-to-toe.
  %Axe with golden inlay.
  %Said his mum was a librarian.
\end{speechtext}

\paragraph{If the troupe ask about the `\gls{basilisk}',}
they say there's some sound, like a newborn thundercloud, squealing.
Then there's the stench.

\paragraph{If the troupe ask about local caverns,}
the farmers can give them vague directions (which adds +2 to any roll to find them).
Any local guides can guide the \glspl{pc} straight there, without a roll.

\subsubsection{The Cave Mouth}\label{caveMouth}
can be found down-hill with a \roll{Intelligence}{Survival} roll at \tn[12].
It lies five miles from the \gls{village}.

\begin{boxtext}
  A small river trickles down with a little burble.
  Sometimes it spreads across the path, making everything slippery.
  Sometimes it keeps itself politely to one side.
\end{boxtext}

\paragraph{The first two miles}
requires a \roll{Dexterity}{Caving} at \tn[8].
Failure inflicts \dmg{0}~Damage.

\begin{boxtext}
  About a mile down, caverns open up around, staring downward, and letting out their own little rivers.
  The river soon picks up, becoming wider, and stronger.
\end{boxtext}

\paragraph{Wading down further}
requires a \roll{Strength}{Caving} roll at \tn[8].
Failure pulls the character down-river to the mushroom fields below (and inflicts \dmg{4}~Damage).

\paragraph{At the mushroom fields}
the troupe find a grand underground lake, full of islands.


\begin{boxtext}
  The cavern looks like a starry night-sky, flattened onto the ground.
  The pricks of feint light and little sparkling reflections don't look like much.
  They have no obvious distance.

  At your feet, cold water.
\end{boxtext}

If the troupe wade into the water, they find it knee-height.
Roll to find the contents of each island:

{
  \footnotesize
  \begin{dlist}
    \item
    Whatwas Fungus --- the Spore-Folk children.
    \item
    \Glspl{glowshroom} \glsentrydesc{glowshroom}.
    \item
    \Glspl{marchingMushroom} \glsentrydesc{marchingMushroom}.
    \item
    \Glspl{bedshroom} \glsentrydesc{bedshroom}.
    \item
    A spore-folk, which retreats to watch the \glspl{pc} from afar.
    \item
    The water becomes suddenly deep.
    The \gls{pc} cannot wade, but might swim.
  \end{dlist}
}

\randomize

\sporeFolk

\ifnum\value{Charisma}>0
  {\small\showStdSpells}
\fi

The `spore-folk' are sentient fungi.
They cannot speak or hear (but communicate with each other once they grow old and smart enough).
They view the \glspl{pc} with suspicion, but may be helpful if \pgls{pc} does not look incompetent (which they interpret as having bad intentions).

\sporeFolk

\ifnum\value{Charisma}>0
  {\small\showStdSpells}
\fi

If the \glspl{pc} try to trade, the spore-folk offer their own `children' --- Whatwas Fungus --- which function as \pgls{talisman}.
However, the \glspl{pc} really need to have the body of a spore-folk.

\wotWosFungus

\showTalisman

%\sporeFolk

\paragraph{Ascending}
presents a new challenge: the other cave-mouths, with their own rivers present alternative routes.
If the troupe did not mark the correct route, they will have to try to figure out which way they came down with an \roll{Intelligence}{Caving} roll at \tn[10].
Each Failure Margin means they are lost for another \gls{interval}.

\clearpage

\Person{\NPC{\glsentrysymbol{vlg}\E}{Dragon}{Long and slinky}{scratches at ground}{to find a laxative}}%
  {{5}{3}{3}}% BODY
  {{1}{2}{-3}}% MIND
  {
      \set{Brawl}{4}
      \set{Academics}{1}
      \set{Deceit}{2}
      \set{Flight}{2}
      \set{Fire}{2}
      \set{Fate}{2}
  }% SKILLS
  {\lucky}% KNACKS
  {}% EQUIPMENT
  {\flight, \quadruped, \hide{r4t5}}% ABILITIES

\label{theDragon}

{
  \scriptsize
  \showStdSpells[\stepcounter{enc}]
}

\end{document}


## Background
- Each time the dragon tries and fails to burp or fart, the pressure builds in his abdomen, then he belches fire.

### 1) Strange Noises (Bailey)
- People hear noises sounding...
  - like an asthmatic motorbike.
  - like a newborn thundercloud, squealing.
  - like bedsheets being torn open.
- A jotter orders the PCs to get rid of the noises.

### 2) Murmurs in the Deep (Woods)
- Getting in the direction of the noises and moving against the wind inflicts 1 EP per interval, because the PCs need to move through a dense cloud of toxic and malodour cloud of dragon sickness-farts.
- The roars and farts echo through the cavern. Finding the entrance requires good auditory orientation (maybe multiple entrances exist?) 

### 3) Double Shot (Cavern)
- The dragon suffers from constipations. He can barely move and his only way to release some of the pressure is through uncontrolled belches and farts.
- The lair is filled with a mix of stomach-twisting body gases.
- The dragons ask the PCs to gather laxative mushrooms growing further down the cavern.

### 4) Laxative Mushrooms (Cavern)
- The dragon farts enrich flammable methane inside the cavern.
- The mushrooms are sentient (which effects do they have?)
