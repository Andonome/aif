\documentclass[10pt,twoside]{book}

\usepackage{config/bind}
\usepackage{config/booklet}
\setcounter{bookLevel}{2}
\newglossaryentry{fashionista}{
  name={Weft},
  text={Weft},
  type={people},
  prefix={Warden},
  description={leader of the world of fasion among the \glsfmtplural{warden}}
}

\longnewglossaryentry{susjot}{
  name={Titrate},
  text={Titrate},
  type={people},
  prefix={\glsentrytext{jotter}\space},
  description={hurries around every nearby \glsentrytext{bothy}, and impatiently micromanages every \glsentrytext{guard}}
}

\newcommand\susjot[1][]{
  \Person{\NPC{\glssymbol{sylf}\M\Hu}{\Glsfmttext{susjot}}{greasy}{rotates rings}{to hurry up, and finish}#1}%
    {{2}{1}{2}}% BODY
    {{1}{0}{1}}% MIND
    {
        \set{Athletics}{2}
        \set{Deceit}{1}
        \set{Larceny}{1}
        \set{Survival}{2}
        \Dagger
        \longsword
        \partialchain
        \addtocounter{fp}{5}
        \longsword
    }% SKILLS
    {\adrenalinesurge \charge}% KNACKS
    {2 rings worth 10~\glspl{sp}, wax tablet (for notes)}% EQUIPMENT
    {}% ABILITIES
}


\longnewglossaryentry{keep}{
  name={The Keep},
  text={the Keep},
  category={location},
  prefix={a\space},
  description={sits on a hill}
}

\longnewglossaryentry{smuggler}{
  name={Feliscourt},
  type={people},
  description={has a big keep to keep}
}


\longnewglossaryentry{enchantedLands}{
  name={The Enchanted Lands},
  text={the Enchanted Lands},
  type={people},
  description={have no crime, nor murder, no broken promises, no singing out of keys, and to jokes}
}

\longnewglossaryentry{plateauGardens}{
  name={The Plateau Gardens},
  text={the~Gardens},
  plural={Plateaus},
  type={people},
  description={have smooth, stone edges, as high as a fortress wall.
  The gardens have few monsters to fear, but also very little water}
}

\longnewglossaryentry{SnailTamer}{
  name={Sint\"e},
  type={people},
  parent={plateauGardens},
  description={is a fearless snail-tamer, who needs a nap}
}

\longnewglossaryentry{LifeElder}{
  name={Hi},
  type={people},
  parent={plateauGardens},
  description={she says, happily, but rejects your labels}
}

\longnewglossaryentry{MindElder}{
  name={Ungwe},
  type={people},
  description={keeps his promises, and your promises, and everyone's promises}
}


\toggletrue{intro}

\externalReferent{core}

\declareRegions{broch,village,lonelyRoad,Forest}

\begin{document}

\miniCover{Irregularities}{}

{
  \footnotesize
  \bookletThreads{broch}
  \bookletThreads{lonelyRoad}
  \bookletThreads{Forest}
  \bookletThreads{village}
}

\clearpage

\thread{By Any Other Name}

\pagestyle{minizine}%

\begin{exampletext}
  Fifteen farmers carry spears past the \gls{edge}.
  Great claws have torn up the ground and scarred the trees.
  Not far away, a deep roar, and a screech, like a newborn thundercloud, squealing;
  then a stench so bad the farmers buckle, and boke.
  \marketBoatman\ cries ``\textit{\gls{basilisk}, scarper!}''.
  So they scarper.

  Back at the \gls{village}, huddled by a cauldron of soup, the farmers discuss the noises.

  ``\textit{\Glsfmtplural{basilisk} have no voice.
  I don't know what that was}'', says \marketBoatman.
  ``\textit{All the same\ldots we'd best ask \marketFence\ to summon the \gls{guard}.}''
\end{exampletext}

\segment[\gls{night}]{broch}% AREA
{Easy Job}% NAME
{\Pgls{jotter} needs the troupe to kill \pgls{basilisk}}% SUMMARY

Trader \marketFence\ arrives at the \gls{broch} that night, and requests aid from \gls{jotter}~Filchvore.
Filchvore congratulates the troupe on their mission, and mentions that \gls{basilisk} bodies can fetch a good price, then reminds them that half of the sale goes to the \gls{templeOfBeasts}.
The \gls{village} in question is fifteen miles down the road, so they should leave at dawn.

\iftoggle{intro}{
  %!
  \paragraph{If \pgls{pc} asks for better \glspl{weapon},}
  they can roll \roll{Charisma}{Empathy} at \tn[10].
  Success means the \emph{player} should interpret how the character might ask for free equipment.

  \paragraph{If \pgls{pc} chats to other \glspl{guard} in the \gls{broch},}
  \gls{gDigger}~\composeHumanName\ laughs at the notion of \pgls{basilisk} roaring.

  \begin{speechtext}
    \Glspl{monster} don't roar.
    That's why people call them `the voiceless'.
  \end{speechtext}
}{}

\segment[\gls{afternoon}]{lonelyRoad}% AREA
{The Dry Valley}% NAME
{A trader asks the troupe to stay the night}% SUMMARY

The dry valley never floods, even during \pgls{storm}.
Caverns swallow the water before it can rise too high.

\begin{boxtext}
  Two plumes of smoke rise over the tall trees in the distance.
  After a mile, the road descends into the Dry Valley, and the trees open for a blessed moment.
  The distant smoke comes from \pgls{village}, over five miles away.
  The other comes from \pgls{bothy}, a little down the road, where two men push a donkey into a cart.

  Going down the road, the trees crowd around again, obscuring everything.
\end{boxtext}

Ten traders with three carts stay in the \gls{bothy}.
They take their barrels and boxes out the wagons, and place them underneath, to make room for their horses and donkeys to sleep in some safety.
They have no \glspl{guard} with them, so once they see the \glspl{pc} they begin asking a thousand questions, and speaking in a friendly manner, hoping the \glspl{pc} will stay outside and watch their animals overnight.

\composeHumanName\ offers a free meal of \rations\ to each \gls{pc}.
\composeHumanName\ refuses to sell them \glspl{torch} until morning.%
\footnote{Any noise or light has a $\frac{1}{\dicef{6}}$ chance of summoning \pgls{monster}.}

\currentName\ mourns her old dwarf-friend --- Ben --- who protected her and her caravan five days ago, just outside the \gls{village}.
He ran into the forest, shouting something about `wyrm-sign', and nobody heard of him since.

\segment[\gls{night}]{lonelyRoad}% AREA
{Dank Night}% NAME
{Something wanders the forest, and it reeks}% SUMMARY
\label{dankNight}

Whether the \glspl{pc} sleep in the \gls{bothy} or not, they hear a sound in the night like bedsheets being torn apart.
Soon after, the stench!
It can curdle yesterday's dinneer, and inflicts \pgls{ep} on each character.
Nearby animals begin to panic.
The troupe will not sleep well, so they must take another \gls{ep} due to the dreamless night.

\segment{Forest}% AREA
{Wyrmsign}% NAME
{The troupe find patches of torn-up earth}% SUMMARY

Patches of earth have been torn up by massive claws.
The troupe can follow them to their source with a \roll{Wits}{Survival} roll at \tn[6].
Failure means they become lost.

If the troupe chased after the noises immediately, in \gls{segment}~\vpageref{dankNight}, they find soon after.
But if they waited, the claw-mark tracks lead them five miles away.
An \roll{Intelligence}{Medicine} roll at \tn[12] tells them the claws were digging for mushrooms to make \pgls{elixir}.

\segment{Forest}% AREA
{Grounded}% NAME
{The \glsfmttext{fiend} demands a fungal \gls{elixir}}% SUMMARY
\label{irregularDragon}

Approaching closer, the trees show black, charred, patches where the flames have licked them.
Then the growls begin, deep and full of anguish.

\begin{boxtext}
  Ahead, nestled among the trees, a mass of green scales writhe around its own bloated stomach.
  Claws caress and push into the ballooned abdomen.
  The shoulders and tail twitch.
  The \gls{fiend} stares hatefully, with bright-yellow eyes.
\end{boxtext}

\Person{\NPC{\glsentrysymbol{vlg}\E}{the Dragon}{Long and slinky}{scratches at ground}{to find a laxative}%
  \npcQuote{Eat you?  No, I wouldn't.  In fact, couldn't}}%
  {{5}{3}{3}}% BODY
  {{1}{2}{2}}% MIND
  {
      \set{Brawl}{4}
      \set{Academics}{3}
      \set{Deceit}{2}
      \set{Cultivation}{2}
      \set{Flight}{2}
      \set{Fire}{2}
      \set{Fate}{2}
  }% SKILLS
  {\lucky}% KNACKS
  {}% EQUIPMENT
  {\flight \quadruped \hide{r4t5}}% ABILITIES

\label{irregularWyrm}

\begin{speechtext}
  Do not flee.
  I will not harm you.
  I have needs.

  There is a cave-mouth, not too far from here.
  Willow trees surround it, from a distance, making the shape of an eye.

  I can no longer fit inside, my stomach has become so heavy.
  Go there, to the cave mouth, and descend into the fungus lakes.
  Find the tallest one you can, and cut it down.

  I must have \pgls{elixir} from that mushroom's cap, so I can fly again.
\end{speechtext}

Find the cave \vpageref{caveMouth}.

\paragraph{Any rude speech}
results in a single, flaming, belch.

\showStdSpells[\stepcounter{enc}]

\vspace{-2em}
\segment[\gls{night}]{village}% AREA
{Hurry Up}% NAME
{The dragon lets the troupe know that it waits for them}% SUMMARY

The dragon approaches the \gls{village} at night, just to let the troupe know it still waits for that fungus.
The troupe will hear its irritated, painful, scratchings across nearby trees.
Then the smell comes as it farts.
Everyone in the \gls{village} gains \pgls{ep}.

\segment{Forest}% AREA
{How Long?}% NAME
{The dragon wants the fungus, fast}% SUMMARY

The dragon has run out of patience.
If the \glspl{pc} entered the cave, it waits for them at the mouth.
If not, it hounds them down, like a dog herding sheep.

\subsection{Locations}

\subsubsection{At the \Glsfmttext{village}}\label{wyrmBailey}
the locals don't want a bunch of \glspl{guard} hanging around, and they're not shy about saying so.
They think the troupe should head into the forest as soon as possible, kill whatever's making those noises, and leave.

\begin{speechtext}
  Shame you couldn't get proper equipment, like that dwarf --- what was his name? --- shiny plate armour, head-to-toe.
  %Axe with golden inlay.
  %Said his mum was a librarian.
\end{speechtext}

\paragraph{If the troupe ask about local caverns,}
the farmers can give them vague directions (which adds +2 to any roll to find them).
Any local guides can guide the \glspl{pc} straight there, without a roll.

\subsubsection{The Cave Mouth}\label{caveMouth}
lies five miles from the \gls{village}.
The \glspl{pc} can find it with an \roll{Intelligence}{Survival} roll at \tn[12].

%!
% Need to save space.  When 'intro' is false, the boxtext no longer has the
% 'What do you do?' ending.
\togglefalse{intro}

\begin{boxtext}
  A small river trickles through a patch of willow trees, then into the open, and down a hole the size of a portcullis.
\end{boxtext}

The hole begins with a drop, two~\glspl{step} \emph{down}.%
\exRef{core}{Core Rules}{falling}

\begin{boxtext}
  Over the first mile, the river keeps politely to one side of the cavern, but occasionally spreads out, making the ground-rocks slippery.
  About a mile down, caverns open up around, staring downward, and letting out their own little rivers.
  The river soon picks up, becoming wider, and stronger.
\end{boxtext}

\paragraph{Wading down further}
requires a \roll{Strength}{Caving} roll at \tn[8].
Failure pulls the character down-river to the mushroom fields below (and inflicts \dmg{4}~Damage).

\paragraph{At the mushroom fields}
the troupe find a grand underground lake, full of islands.


\begin{boxtext}
  The cavern looks like a starry night-sky, flattened onto the ground.
  The pricks of feint light and little sparkling reflections don't look like much.
  They have no obvious distance.

  At your feet, cold water.
\end{boxtext}

If the troupe wade into the water, they find it knee-height.
Roll to find the contents of each island:

{
  \footnotesize
  \begin{dlist}
    \item
    The water becomes suddenly deep.
    The \gls{pc} cannot wade, but might swim.
    \item
    \Glspl{glowshroom} \glsentrydesc{glowshroom}.
    \item
    \Glspl{marchingMushroom} \glsentrydesc{marchingMushroom}.
    \item
    \Glspl{bedshroom} \glsentrydesc{bedshroom}.
    \item
    A spore-folk, which retreats to watch the \glspl{pc} from afar.
    \item
    Whatwas Fungus --- the spore-folk's children, which function as \pgls{talisman}.
  \end{dlist}
}

\randomize

\sporeFolk

\ifnum\value{Charisma}>0
  {\small\showStdSpells[\stepcounter{enc}]}
\fi

\noindent
The `spore-folk' are sentient fungi.
They cannot speak or hear, but in time they learn to speak with each other.
They view the \glspl{pc} with suspicion, but may be helpful if \pgls{pc} does not look incompetent (which they interpret as having bad intentions).

\sporeFolk

\ifnum\value{Charisma}>0
  {\small\showStdSpells[\stepcounter{enc}]}
\fi

\wotWosFungus

\noindent
If the \glspl{pc} try to trade, the spore-folk offer their own `children' --- Whatwas Fungus.
However, the \glspl{pc} need to have the body of a spore-folk adult to make \pgls{elixir}.
They must murder, or face the dragon.

\showTalisman

%\sporeFolk

\paragraph{Ascending}
presents a new challenge: the other cave-mouths, with their own rivers present alternative routes.
If the troupe did not mark the correct route, they will have to try to figure out which way they came down with an \roll{Intelligence}{Caving} roll at \tn[10].
Each Failure Margin means they are lost for another \gls{interval}.

\paragraph{If the \glspl{pc} retreive the fungus,}
the dragon takes a concave stone to simmer water, creates a delicate fire underneath, and chews nearby herbs.
As the \gls{elixir} brews, it may grant the troupe the blessing of its company, and may teach any spellcaster present \pgls{spell}, if they speak politely.

\Pgls{interval} later, the dragon roars, then its eyes turn silently upwards in relief.
A half-digested dwarf emerges, with full plate armour, golden axe, and beard still intact.

\end{document}

## Background
- Each time the dragon tries and fails to burp or fart, the pressure builds in his abdomen, then he belches fire.

### 1) Strange Noises (Bailey)
- People hear noises sounding...
  - like an asthmatic motorbike.
  - like a newborn thundercloud, squealing.
  - like bedsheets being torn open.
- A jotter orders the PCs to get rid of the noises.

### 2) Murmurs in the Deep (Woods)
- Getting in the direction of the noises and moving against the wind inflicts 1 EP per interval, because the PCs need to move through a dense cloud of toxic and malodour cloud of dragon sickness-farts.
- The roars and farts echo through the cavern. Finding the entrance requires good auditory orientation (maybe multiple entrances exist?) 

### 3) Double Shot (Cavern)
- The dragon suffers from constipations. He can barely move and his only way to release some of the pressure is through uncontrolled belches and farts.
- The lair is filled with a mix of stomach-twisting body gases.
- The dragons ask the PCs to gather laxative mushrooms growing further down the cavern.

### 4) Laxative Mushrooms (Cavern)
- The dragon farts enrich flammable methane inside the cavern.
- The mushrooms are sentient (which effects do they have?)
