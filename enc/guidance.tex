\documentclass[10pt,twoside]{book}

\usepackage{config/bind}
\usepackage{config/booklet}
\setcounter{bookLevel}{2}
\makeglossaries

\makeglossaries

\newglossaryentry{outerKingdom}{
  name={The Western Kingdom},
  text={the Western Kingdom},
  description={is the nearest kingdom}
  }

\newglossaryentry{king}{
  name={Rex Wyatt Fenestra},
  text={Rex Wyatt},
  first={Rex Wyatt Fenestra},
  description={keeps a tight control over Fenestra through the use of magical portals}
  }

\newglossaryentry{nightguard}{
  name={The Night Guard},
  text={the Night Guard},
  description={is the organization, commissioned by \glsentrytext{king}, to look after the realm in lieu of the standing armies which village masters once kept}
  }

\newcommand{\spiderqueen}{
\NPC{\F}{\Glsentrytext{spiderqueen}}{Delicate}{Looks upwards}{Experience}\label{spiderqueen}

\person{0}% STRENGTH
  {3}% DEXTERITY
  {2}% SPEED
  {{2}% INTELLIGENCE
  {3}% WITS
  {-1}}% CHARISMA
  {3}% DR
  {1}% COMBAT
  {Deceit~2, Stealth~3, Tactics~2, Wyldcrafting~3
  \Path{Blood \& Song}{\aldaron~3, \polymorph~4}}% SKILLS
  {None}% EQUIPMENT
  {\addtocounter{strb}{3}\lockedmana{5}\addtocounter{xpbonus}{4}}
}

\newglossaryentry{spiderqueen}{
  name={The Spider Queen},
  text={the Spider Queen},
  description={is an old elf who collects pet chitincrawlers.  She has also started to shapeshift to look like them}
  }

\newglossaryentry{town}{
  name={Greytown},
  text={Greytown},
  description={is the central town where the action takes place}
}

\newcommand{\alemaster}{
  \humandiplomat[\NPC{\M}{Anatole}{Snippy}{Tongue-in-cheek}{Alass\"e}]
}
\newglossaryentry{alemaster}{
  name={Anatole},
  text={Anatole},
  description={is the highest ranking member of the temple of Alass\"{e}, who are synonymous with the Ale Guild in this region},
  first={Ale Master Anatole}
  }

\newcommand{\beardedalemaster}{

\humandiplomat[\NPC{\F}{\glsentrytext{beardedalemaster}}{Inquisitive}{Hands behind her back}{Alass\"e}]
}

\newglossaryentry{beardedalemaster}{
  name={Susan},
  text={Susan},
  description={is the leader of the Ale Guild in the Shale, and became rich selling watered-down dwarvish ale},
  first={Guildmistress Susan}
  }

\newcommand{\woodspyleader}{
\NPC{\M}{\glsentrytext{woodspyleader}}{Suspicious}{Scratches back of head}{Ohta}\label{woodspyleader}

\person{4}% STRENGTH
{2}% DEXTERITY
{-1}% SPEED
{{0}% INTELLIGENCE
{1}% WITS
{1}}% CHARISMA
{0}% DR
{2}% COMBAT
{Academics 1, Deceit 2, Empathy 1, Wyldcrafting 1}% SKILLS
{\longsword, \shield, \completechain, dagger}% EQUIPMENT
{}
}

\newglossaryentry{woodspyleader}{
  name={Brook},
  text={Brook},
  description={is the leader of the Woodspy Bandits, and meets with \glsentrytext{traitor} on a semi-regular basis in \glsentrytext{whitehorse}. He always enters battle with a mask over his face, and puts on a faux lower-class accent},
  first={Woodspy Leader Brook}
  }

\newcommand{\keras}{
  \NPC{\M}{\glsentrytext{keras} the Gnomish Illusionist}{Sly}{Waggles finger}{Experience}

  \person{-1}% STRENGTH
  {2}% DEXTERITY
  {-1}% SPEED
  {{3}% INTELLIGENCE
  {1}% WITS
  {1}}% CHARISMA
  {0}% DR
  {0}% COMBAT
  {Academics~3, Empathy~2, Stealth~2, Vigilance~2
  \Path{Alchemy}{\conjuration~2, \illusion~4}}% SKILLS
  {\Dagger, 20gp, crudely drawn maps}% EQUIPMENT
  {\addtocounter{fp}{10}}
}

\newglossaryentry{keras}{
  name={Keras},
  first={Keras Tri\"{e}l},
  description={is a curious gnomish illusionist, who often gets in trouble studying things he shouldn't}
  }

\newcommand{\elfprince}{
\NPC{\M}{\Glsentrytext{elfprince}}{Precise}{Flat palms}{Experience}

  \person{-1}% STRENGTH
  {2}% DEXTERITY
  {1}% SPEED
  {{1}% INTELLIGENCE
  {3}% WITS
  {2}}% CHARISMA
  {0}% DR
  {1}% COMBAT
  {Projectiles~2, Academics~2, Empathy~2, Performance~3, Seafaring 2, Stealth~1, Vigilance~2, Wyldcrafting~2\Path{Blood}{\aldaron~3, \enchantment~2, \fate~2, \force~1}}% SKILLS
  {\shortsword, shortbow, dagger, earrings worth 20 sp, necklace worth 2 gp, bracelets each worth 2 sp}% EQUIPMENT
  {}
}

\newglossaryentry{elfprince}{
  name={Prince Arinel},
  text={Arinel},
  first={Prince Arinel},
  description={an elf, famed for his songs}
  }

\newcommand{\thenecromancer}{

\NPC{\M\D}{\Glsentrytext{necromancer}}{Arrogant}{Slack jaw}{Qualm\"e}

\person{1}% STRENGTH
{2}% DEXTERITY
{0}% SPEED
{{2}% INTELLIGENCE
{1}% WITS
{-4}}% CHARISMA
{2}% DR
{1}% COMBAT
{Projectiles~3, Academics~2, Crafts~(fletchery)~2, Deceit~1, Performance~1, Vigilance~3, Wyldcrafting~1
\knacks{\snapshot, \mightydraw, \perfectsneakattack}
\Path{Devotion (Qualm\"{e})}{\aldaron~3, \enchantment~3, \necromancy~5}}%SKILLS
{Hunting Bow, dagger}%EQUIPMENT
{\lockedmana{3}}
}

\newglossaryentry{necromancer}{
  name={Antonym},
  first={the Undead Priest Antonym},
  description={is a priest of Qualm\"{e}, from the time of Nolan Shale.
  He has since died, but kept himself alive in order to study Song Magic.
  Unfortunately, death has robbed him of his voice}
  }

\newcommand{\forestpriest}{
\NPC{\M}{\Glsentrytext{forestpriest}}{Precise}{Looks down}{Alass\"e} 

\person{0}% STRENGTH
  {2}% DEXTERITY
  {1}% SPEED
  {{2}% INTELLIGENCE
  {3}% WITS
  {2}}% CHARISMA
  {0}% DR
  {1}% COMBAT
  {Academics~2, Crafts~1, Deceit~2, Seafaring~1, Stealth~2, Vigilance~1, Wyldcrafting~2
  \Path{Blood}{\aldaron~2, \fate~3, \polymorph~4}}% SKILLS
  {\quarterstaff, dagger}% EQUIPMENT
  {\addtocounter{fp}{10}}
}

\newglossaryentry{forestpriest}{
  name={Adrian},
  first={Adrian, high priest of Laiqu\"{e}},
  text={Adrian},
  description={is a half-elvish high priest of Laiqu\"e, from the time of Nolan Shale.  He has become irritated at how far men are encroaching into the forest, and hopes to stop them}
  }

\newcommand{\banditking}{
\NPC{\M}{\glsentrytext{banditking}}{Approachable}{Drums fingers}{Qualm\"e}

\person{3}% STRENGTH
  {3}% DEXTERITY
  {0}% SPEED
  {{0}% INTELLIGENCE
  {-1}% WITS
  {2}}% CHARISMA
  {0}% DR
  {3}% COMBAT
  {Academics~1, Empathy~1, Deceit~3, Tactics~2, Wyldcrafting~3\knacks{\defender, \laststand}}% 
  {\longsword, \partialchain, 50 sp wrapped in cotton wool, knife}% 
  {\addtocounter{fp}{5}}
}

\newglossaryentry{banditking}{
  name={Burke},
  first={Burke Whiteplains},
  description={was the child of a noble in Whiteplains, but his family were killed by the Rex.  He's lived as a criminal ringleader ever since}
  }

\newcommand{\sewerking}{

\NPC{\M}{\Glsentrytext{sewerking}}{Alpha male}{Hands on sword}{Experience}

\person{1}% STRENGTH
  {0}% DEXTERITY
  {1}% SPEED
  {{2}% INTELLIGENCE
  {0}% WITS
  {1}}% CHARISMA
  {0}% DR
  {1}% COMBAT
  {Academics~1, Caving~1, Deceit~3, Seafaring~1, Tactics~2, Vigilance~2
  \Path{Alchemy \& Nura}{\conjuration~3, \illusion~3, \necromancy~3, \saurecanta~2}}% SKILLS
  {\shortsword, \partialleather, dagger, 20sp}% EQUIPMENT
  {\addtocounter{fp}{5}, \lockedmana{8}}
}

\newglossaryentry{sewerking}{
  name={Hare},
  first={Hare Whiteplains},
  description={is a Whiteplains noble, currently living under \glsentrytext{town}}
  }

\newcommand{\sewerthief}{
  \NPC{\M}{David}{Cheeky}{Grins wide}{Tribe}

  \person{1}% STRENGTH
{2}% DEXTERITY
{2}% SPEED
{{-1}% INTELLIGENCE
{1}% WITS
{1}}% CHARISMA
{0}% DR
{2}% COMBAT
{Deceit~2, Larceny~2, Stealth~1, Vigilance~2, Wyldcrafting~1
\knacks{\stunningstrike, \perfectsneakattack}}% SKILLS
{\shortsword, dagger}% EQUIPMENT
{}
}

\newglossaryentry{sewerthief}{
  name={David},
  first={David Whiteplains},
  description={misses the comforts of nobility.  He spends a lot of his time running errands through \glsentrytext{pig} and across \glsentrytext{town}, such as delivering cursed magical items to \glsentrytext{banditking}}
  }

\newcommand{\nurabaron}{
\NPC{\M\N}{\glsentrytext{nurabaron}}{Hungry}{Wide eyed}{Tribe}

\person{5}% STRENGTH
  {0}% DEXTERITY
  {4}% SPEED
  {{-4}% INTELLIGENCE
  {-2}% WITS
  {-4}}% CHARISMA
  {0}% DR
  {1}% COMBAT
  {Academics~1, Deceit~2\knacks{\adrenalinesurge}}% SKILLS
  {\greatsword}% EQUIPMENT
  {}
}

\newglossaryentry{nurabaron}{
  name={Clandon, the Villagemaster},
  text={Master Clandon},
  first={Village Master Clandon of Redfall},
  description={was a jovial and simple Villagemaster until he had an accident with a magical item, and has since turned into an ogre}
  }

\newcommand{\captain}{
\NPC{\M}{\Glsentrytext{captain}}{Solemn}{Pitying glances}{Tribe}

\person{1}%STRENGTH
  {2}% DEXTERITY
  {0}%  SPEED
  {{1}% INTELLIGENCE
  {-1}% WITS
  {1}}% CHARISMA
  {0}% DR
  {2}% COMBAT
  {Academics~1, Empathy~2, Stealth~2, Tactics~2, Wyldcrafting~1\knacks{\adrenalinesurge}}% SKILLS
  {\longsword, dagger, 2sp}% EQUIPMENT
  {\addtocounter{fp}{5}}
}

\newglossaryentry{captain}{
  name={Captain Oscar Quenn},
  text={Captain Oscar},
  first={Captain Oscar of \glsentrytext{nightguard}},
  description={is the captain of \glsentrytext{nightguard} in \glsentrytext{town}}
  }

\newcommand{\traitor}{
\NPC{\M}{\Glsentrytext{traitor}}{Suspicious}{Grins wide}{Acquisition}

\person{2}% STRENGTH
  {1}% DEXTERITY
  {2}% SPEED
  {{1}% INTELLIGENCE
  {0}% WITS
  {1}}% CHARISMA
  {0}% DR
  {2}% COMBAT
  {Athletics~2, Deceit~1, Larceny~1, Wyldcrafting~2\knacks{\adrenalinesurge, \disarm, \defender, \furiousblows}}% SKILLS
  {\longsword, \partialchain}% EQUIPMENT
  {\addtocounter{fp}{5}}
}

\newcommand{\townmaster}{
  \humandiplomat[\NPC{\M}{\glsentrytext{townmaster}}{Enthusiastic}{Snorts, loudly}{Acquisition}]
}

\newglossaryentry{townmaster}{
  name={Catelina},
  text={Lord Catelina},
  first={Townmaster Catelina},
  description={is the local Town Master, currently attempting to rebuild the \glsentrytext{lostcity} by collecting magical items from the nura}
  }

\newglossaryentry{traitor}{
  name={Lieutenant Darren},
  text={Lieutenant Darren},
  first={Lieutenant Darren of \glsentrytext{nightguard}},
  description={is an active and ambitious member of \glsentrytext{nightguard}, currently working with \glsentrytext{townmaster} in secret, and taking information to the Woodspy Bandits},
  }

\newcommand{\pigowner}{
\NPC{\F}{\Glsentrytext{pigowner}}{Threatening}{Chin pointed up}{Acquisition}

\person{1}% STRENGTH
  {2}% DEXTERITY
  {-1}% SPEED
  {{1}% INTELLIGENCE
  {2}% WITS
  {1}}% CHARISMA
  {0}% DR
  {1}% COMBAT
  {Deceit~3, Empathy~1, Tactics~1\knacks{\adrenalinesurge}}% SKILLS
  {\Dagger, a hidden dagger, and an even more hidden dagger.}% EQUIPMENT
  {\addtocounter{fp}{5}}
}

\newglossaryentry{pigowner}{
  name={Wendy, Owner of \glsentrytext{pig}},
  text={Wendy Pig},
  first={Wendy Pig, owner of \glsentrytext{pig}},
  description={a middle-aged, fierce woman who runs the toughest bar in \glsentrytext{town}}
  }

\newcommand{\warningbard}{
  \humanbard[\NPC{\M}{\glsentrytext{warningbard}}{Excited}{Drums Fingertips}{Laiqu\"e}]
}

\newglossaryentry{warningbard}{
  name={Roger},
  text={Roger},
  first={Roger the bard},
  description={is a young and headstrong man who writes better than he sings}
  }

%%%%% Adventures in Fenestra Places

\newglossaryentry{redfall}{
  name={Redfall},
  first={Redfall Village},
  description={is rather out-of-the-way village, famed for its red-earth clay and waterfall}
  }

\newglossaryentry{college}{
  name={The College of Alchemy},
  text={the College of Alchemy},
  first={the College of Alchemy},
  description={is where all alchemists go for training.  They maintain a complete monopoly on alchemical magic, and all spells must be learned through them or one of their outposts}
  }

\newglossaryentry{pig}{
  name={The Mincing Pig},
  text={the Mincing Pig},
  first={the Mincing Pig Tavern},
  description={boasts the reputation as the loudest tavern in town, placed close to the entrance.
  Many people don't get farther than `the Pig' when coming to \glsentrytext{town}}
  }

\newglossaryentry{whitehorse}{
  name={The White Horse},
  text={the White Horse},
  first={the White Horse Inn},
  description={is a serene and esteemed establishment, with a reputation for high-quality clientel}
  }

\newglossaryentry{lostcity}{
  name={The Lost City},
  text={the Lost City},
  first={the Lost City},
  description={is a city where men lived centuries ago.  It is now so destroyed and overgrown that people do not know they have entered until they see some part of an old stone building}
  }

\newglossaryentry{shatteredcastle}{
  name={The Shattered Castle},
  text={the Shattered Castle},
  description={is where \glsentrytext{king} lives.  It has six sections in six different regions, and each one contains a magical gateway to the hub in Whiteplains, so different parts of the  castle exist in different places but the inside of the building functions as a single palace}
  }


\newglossaryentry{greentower}{
  name={The Green Tower},
  text={the Green Tower},
  description={was commissioned by \glsentrytext{townmaster} as an outpost to \glsentrytext{lostcity}, where the woodspy bandits stay, lead by \glsentrytext{traitor}}
  }


\externalReferent{judgement}

\declareRegions{broch,bothy,lonelyRoad}

\assignHumanName[\fjot]

\begin{document}

\miniCover{A Short Guide}{}

\randomize
\Person{%
  \NPC{\N\F}% Symbol
    {Goblin}% Name
    {lanky}% Description
    {eerie stare}% Mannerism
    {meaty food}% Wants
  }%
  {{-1}{1}{1}}% BODY
  {{-1}{2}{-3}}% MIND
  {%
      \set{Projectiles}{2}
    \set{Caving}{2}
    \set{Stealth}{1}
  }% SKILLS
  {}% KNACKS
  {}% EQUIPMENT
  {\ifodd\value{r3}\else\mutation{r4}\fi}% ABILITIES

\randomize
\randomize
\randomize
\Person{%
  \NPC{\N\M}% Symbol
    {Goblin}% Name
    {veiny ears}% Description
    {clicks tongue}% Mannerism
    {sugary food}% Wants
  }%
  {{0}{1}{0}}% BODY
  {{0}{-1}{-2}}% MIND
  {%
    \set{Melee}{1}
    \set{Brawl}{2}
    \set{Deceit}{1}
    \set{Stealth}{2}
  }% SKILLS
  {}% KNACKS
  {}% EQUIPMENT
  {\ifodd\value{r3}\else\mutation{r4}\fi}% ABILITIES

\vfill
{
  \footnotesize\sffamily
  \noindent
  With thanks to Dyson Logos for the maps \vpagerefrange{Dyson_Logos/winding_overview}{Dyson_Logos/winding_cave} (\texttt{CC-BY}).
}

\clearpage

\pagestyle{minizine}%

\subsubsection{The \Glsfmttext{broch}}
stands on a tall hill at the \gls{edge}.
It keeps the local \glspl{village} safe from wandering \glspl{monster} by making noise to attract them, so  archers can shoot them down.
But today the \glspl{ranger} have all left, along with the last piper.
So \gls{jotter}~\fjot\ gives the \gls{broch}'s pipes to the \glspl{pc}, but fails to mention that someone else should stand at the top with \pgls{bow}.

The roll is \roll{Strength}{Performance} at \tn[12] (to shout) or \tn[9] (if they can play the pipes).
On success, begin the next \gls{monster} encounter.
On a tie, the \gls{monster} arrives an hour after Sunset.

Whether or not the pipes succeed, anyone on the top can see a figure before dusk:

\begin{boxtext}
  On the high hill, on top of a tower, feels like being on top of a boat, watching a  sea of green in every direction.
  Small openings let you see patches of the \gls{lonelyRoad}, heading towards open farmland.
  The farmland huddles around a distant town.
  
  Not far below, a bailey's high walls stand defiantly against the \gls{edge}.
  And a man in black approaches the bailey, with two child-sized figures on a leash, both pale and naked.
\end{boxtext}

\paragraph{If the \glspl{pc} report this to the \gls{jotter},}
he sends them out immediately to investigate.

\paragraph{Otherwise,}
everyone in the \gls{village} stands on their rooves, waving their arms at the \gls{broch}.
If the \glspl{pc} don't respond to that, five men with spears approach the \gls{broch} to inform \Gls{jotter}~\fjot\ of the situation, and \fjot\ sends the \glspl{pc}.

\subsubsection{At the \Glsfmttext{village}}
\gls{ranger} Spittlespite lays groaning in bed, festering wound in his leg, loaded crossbow at the ready, and a couple of scared farmers with pitchforks watching over two goblins in the corner.

\begin{speechtext}
  A bloodied servant of the \gls{warden}'s son told us that their caravan has been ambushed by goblins, so naturally, we had to move immediately, or the warden would have our heads if he hears we hesitated\ldots
  Goblins ambushed the Warden's son and his entourage, and took them captive, then ate all their horses.
  We tracked them down, and after a bloody fight, I got away with two goblins on a leash.
  The rest died, or were taken.
  Dead either way\ldots

  But we killed a lot of those \gls{sylf}'s sucklings, too.
  No more than four goblins survived, and ushered bound prisoners in the direction of, what I believe, must be their lair.
  But those will grow hungry soon, and I fear you have no more than a couple of days before all those men are goblin feed.
\end{speechtext}

Pointing towards two goblins, he continues:

\begin{speechtext}
  Take one with you; two are too much trouble, as you can see by my leg.
  They seem to only understand blade and food, but that should be enough for them to take you to where others are.
\end{speechtext}

\subsubsection{Through the Woods}
the goblin's lair is 20 miles away, in the part of the woods seldom traversed.
Only the goblin knows where exactly, but is unable to say.
Dense forest terrain limits the travel to 4~miles per \gls{interval}.
The missing troupe numbers nine men, plus the \gls{warden}'s son, and goblins will eat one person per \gls{interval}, on average.
The \gls{warden}'s son is not above manipulating and sacrificing his men to save his own hide, so he will ensure that he is the last one to go.

The goblin starts with 3~\glspl{ep}.
It requires 3~\glspl{ration} per day, and each \gls{ration} it doesn't eat will gain it 2~\glspl{ep} at the end of the day.

The goblin doesn't speak the same language as the \glspl{pc}, but it understands the offers of food, and threats of violence.
If the goblin's hands are tied, it will be visibly upset, and will try to subtly lead the troupe in a wrong direction, moving one mile away from the lair each \gls{interval}.
\roll{Intelligence}{Survival} at \gls{tn} 10 to notice at the start of each \gls{interval}, or \glspl{pc} realize after two \glspl{interval}.
It might also occasionally trip on purpose, and then struggle to get up, making the troupe move one less mile each \gls{interval}.
At night, it will attempt to eat as much food as possible, and destroy any equipment, if left unsupervised.

It is malicious, cunning, and shameless, and its behaviour will primarily depend on how hungry it is.
Each \gls{interval}, note its \glspl{ep}, then select (or if you prefer, roll 1D6 for) one of appropriate options.
Don't repeat any results, unless you used all of the ones under the same section:

If the goblin is well fed (no \glspl{ep}, and has eaten this \gls{interval}):


\begin{dlist}
  \item
  Lazily drags its feet, and will fall asleep if left undisturbed.
  The troupe moves two miles less this \gls{interval}, unless someone keeps hurrying the goblin, in which case, the penalty reduces to one mile.
  \item
  Needs to take a dump.
  A big one.
  No, it doesn't need to go to a bush, nor does it care if anyone's watching, it'll do it right here, right now.
  In fact, there is a grin across its face as the most horrid smell violates the troupe's nostrils.
  Everybody who stays nearby during the suspiciously lengthy act will stink for \pgls{interval}.
  \item
  Squeals while pointing at the \glspl{pc}~\glspl{ration} and its stomach.
  Throws a tantrum if not fed.
  \item
  Throws a handful of its fresh shit on the campfire, then wipes its hand on the closest \gls{pc}.
  If the campfire is not extinguished, the stench will prevent anyone nearby (except the goblin) from sleeping well that night, resulting in one extra \gls{ep} in the morning.
  \item
  Leads the troupe through mud, then tries to trip the clumsiest \gls{pc}, who resists with \roll{Dexterity}{Athletics} against goblin's \roll{Dexterity}{Brawl}.
  If it succeeds, it cackles at its victim, as they struggle in deep mud.
  \item
  Attempts to steal from \pgls{pc}.
  It will prioritize whatever useful thing it can destroy, or secretly drop in a bush, or anything sharp or easily flammable it can use to cause harm.
\end{dlist}

If the goblin is reasonably fed (at least one \gls{ep}, or hasn't eaten this \gls{interval}):

\begin{dlist}
  \item
  Excitedly pulls in a new direction: on the ground, there are fresh tracks of a large boar.
  The goblin points towards them, and then towards troupe's weapons.
  Going after the boar will send the troupe one mile away from their goal, and successfully tracking it down will require a \roll{Wits}{Survival} at \gls{tn} 12.
  \item
  Picks up a fistful of dirt, and throws it in the face of the closest \gls{pc}, blinding them on a hit.
  Then it tries to escape.
  \item
  Screeches loudly enough to attract an encounter.
  \item
  Falls down, and pretends it sprained an ankle, crying in pain.
  If \pgls{pc} tries to examine it up close, it leaps and tries to bite them.
  If the wound isn't treated with \roll{Intelligence}{Medicine} at \gls{tn}~10, an infection develops, and the \gls{pc} gains one \gls{ep} in the following \gls{interval} (and the treatment \gls{tn} rises by 2), and loses \pgls{hp} one \gls{interval} after that.
  \item
  Spots that stitching is loose on one of the \glspl{pc} bags, then bites into it, tearing a hole.
  All the items in it spill to the ground, and the bag becomes unusable.
  The \gls{pc} without a bag can carry items only in its hands.
  \item
  Feels spiteful, and leads the troupe through a Thorny Thicket.%
\exRef{judgement}{Judgement}{thorny_thickets}
Being almost naked, it wants to be carried through the spiky growths, but if the \gls{pc} carrying it has any food, it will try to steal and immediately eat some.
It has +2 to this roll due to surprise and terrain.


\end{dlist}


If the goblin is hungry (more then one \gls{ep}, but fewer \glspl{ep} than \glspl{hp}):

\begin{dlist}
  \item
  Leads the troupe into a cave, but there is only a stash of dried nuts (worth one \gls{ration}) hidden under a rock there.
  This moves the troupe one mile further from the lair.
  \item
  Drools, and hungrily stares at the weakest \gls{pc}, or the troupe's pack animal.
  \item
  Hungrily screams during night, waking everyone up.
  All \glspl{pc} will have one more \glspl{ep} in the morning, due to a sleepless night.
  \item
  Jumps on a nearby \gls{pc}'s backpack, trying to steal and immediately eat food.
  If the \glspl{pc} manage to tear it off the backpack before it succeeds, it will grab a random item instead, preferably something small, and fling it into woods.
  \roll{Wits}{Vigilance} at \gls{tn}~10 to find the item fast, or the troupe will have to spend more time looking for it, moving one less mile this \gls{interval}.
  \item
  It forages as it goes, and eats everything the moment it picks it up.
  It rolls a \roll{Wits}{Survival} at \gls{tn}~14, and on success, will only find enough for one \gls{ration} during the whole \gls{interval}.
  Stopping, and slightly deviating from the path all the time means that the troupe travels two miles less this \gls{interval}.
  \item
  When the troupe are about to eat, it spits on their food, making sure it's seen.
  A big, nasty ball of spit, with a glob of something foul in it, that the goblin picked up and chewed along the way.
  Is that earwax?
  The goblin hopes no one will want to eat this now, so it can have it all.
  Anyone eating this food has to roll \roll{Strength}{Survival} at \gls{tn}~12, or they develop a random disease%
  \exRef{judgement}{Judgement}{diseases}
  on the start of the next day.
\end{dlist}

If the goblin is starving (more \glspl{ep} than \glspl{hp}):

\begin{dlist}
  \item
  It leads the troupe to a griffin eating a dead deer.
  The goblin will slobber and gesticulate towards the carcass, and if \glspl{pc} try to move on, it will raise a lot of noise, provoking the beast to charge at the troupe, protecting its catch.
  This detour leads the troupe one mile away from the lair, and the remains of the deer provide two~\glspl{ration}.
  \item
  Becomes disoriented, leading the troupe two miles away from the lair.
  \glspl{pc} can realize the direction makes no sense with \roll{Intelligence}{Survival} at \gls{tn}~10, but this only stops them from moving further away, and they make no progress forward this \gls{interval}.
  \item
  Attacks \pgls{pc}, but falls to the ground, tired, after \pgls{round}.
  \item
  Faints, losing consciousness for \pgls{interval}.
  \roll{Intelligence}{Medicine} at \gls{tn}~10 to wake it up, but unless fed, it still takes \pgls{interval} for it to be able to properly lead the troupe.
  \item
  Keeps stopping to check if anything is edible.
  Digs through the dirt, rummages through bushes, chews grass, but finds no food.
  The troupe moves one less mile this \gls{interval}.
  \item
  Goes delirious, bites off one of its fingers, and eats it and loses \pgls{hp}.
\end{dlist}

\subsection{At the Goblin Lair}
The entrance smells of burnt meat, and is covered with a tangle of stacked branches and leaves, masking it from the outside.
If the barrier is disturbed, a human skull, loosely hanging on the inside of it, rolls down the cave mouth, alerting the lone goblin in the lower right cave.
\sideMap[4]{Dyson_Logos/winding_overview}%
  {%
    \ref{lair:right}/75/34,
    \ref{lair:left}/10/54,
    \ref{lair:lower}/27/26,
  }%
{Cavern side view.\par\null}

Carefully examining the branches reveals the skull, otherwise it's \roll{Wits}{Vigilance} at \gls{tn}~8 to notice, for the \gls{pc} who moves them.
Any loud noises will echo through the caves and alert all the goblins.

\goblin

\mapentry[lair:right]{Gnawing Dark}
One goblin gnawing on a leg bone, guarding the captives.

\mapentry[lair:left]{Crack and Smoke}
A small fire dimly lights the floor littered with bones, one goblin eating badly cooked human remains, and another goblin, and a hobgoblin, asleep.

\mapentry[lair:lower]{Gentle Weeping}
Human captives, if any remain at this time.
Their arms and legs are bound with their own torn clothes, and each has 2~\glspl{ep} for every day since they were abducted.

\hobgoblin

\mapPic[\large]{b}{Dyson_Logos/winding_cave}{
  Goblins!/54/58,
  \ref{lair:right}/77/40,
  \ref{lair:left}/30/65,
  \ref{lair:lower}/15/22,
}

\end{document}
