\documentclass[10pt,twoside]{book}

\usepackage{config/bind}
\usepackage{config/booklet}
\setcounter{bookLevel}{2}
\newglossaryentry{fashionista}{
  name={Weft},
  text={Weft},
  type={people},
  prefix={Warden},
  description={leader of the world of fasion among the \glsfmtplural{warden}}
}

\longnewglossaryentry{susjot}{
  name={Titrate},
  text={Titrate},
  type={people},
  prefix={\glsentrytext{jotter}\space},
  description={hurries around every nearby \glsentrytext{bothy}, and impatiently micromanages every \glsentrytext{guard}}
}

\newcommand\susjot[1][]{
  \Person{\NPC{\glssymbol{sylf}\M\Hu}{\Glsfmttext{susjot}}{greasy}{rotates rings}{to hurry up, and finish}#1}%
    {{2}{1}{2}}% BODY
    {{1}{0}{1}}% MIND
    {
        \set{Athletics}{2}
        \set{Deceit}{1}
        \set{Larceny}{1}
        \set{Survival}{2}
        \Dagger
        \longsword
        \partialchain
        \addtocounter{fp}{5}
        \longsword
    }% SKILLS
    {\adrenalinesurge \charge}% KNACKS
    {2 rings worth 10~\glspl{sp}, wax tablet (for notes)}% EQUIPMENT
    {}% ABILITIES
}


\longnewglossaryentry{keep}{
  name={The Keep},
  text={the Keep},
  category={location},
  prefix={a\space},
  description={sits on a hill}
}

\longnewglossaryentry{smuggler}{
  name={Feliscourt},
  type={people},
  description={has a big keep to keep}
}


\longnewglossaryentry{enchantedLands}{
  name={The Enchanted Lands},
  text={the Enchanted Lands},
  type={people},
  description={have no crime, nor murder, no broken promises, no singing out of keys, and to jokes}
}

\longnewglossaryentry{plateauGardens}{
  name={The Plateau Gardens},
  text={the~Gardens},
  plural={Plateaus},
  type={people},
  description={have smooth, stone edges, as high as a fortress wall.
  The gardens have few monsters to fear, but also very little water}
}

\longnewglossaryentry{SnailTamer}{
  name={Sint\"e},
  type={people},
  parent={plateauGardens},
  description={is a fearless snail-tamer, who needs a nap}
}

\longnewglossaryentry{LifeElder}{
  name={Hi},
  type={people},
  parent={plateauGardens},
  description={she says, happily, but rejects your labels}
}

\longnewglossaryentry{MindElder}{
  name={Ungwe},
  type={people},
  description={keeps his promises, and your promises, and everyone's promises}
}


\toggletrue{intro}

\declareRegions{bothy,Valley,lonelyRoad,Mountain}

\assignHumanName[\bothyOne]
\setcounter{humanNameNo}{8}
\assignHumanName[\bothyTwo]
\newcommand\brochOne{Blowtop}
\newcommand\brochTwo{Mossdown}

\begin{document}

\miniCover{Gong Bandits}{
  \par
  \begin{tikzpicture}
      \coordinate (start) at (-0.9,-0.9);
      \coordinate (tofirst) at (-0.3,0.7);
      \coordinate (tosecond) at (1.0,2.5);
      \coordinate (tomountains) at (3.5,3.0);
      \coordinate (tobailey) at (3.0,4.5);
      \coordinate (fork) at (0,0.3);
      \coordinate (broch) at (0.7,0);
      \coordinate (bothy1) at (0.1,2);
      \coordinate (bothy2) at (2.3,2.2);
      \coordinate (mid) at (2.9,3);
      \coordinate (griffins) at (3.5,4);
      \coordinate (bandits) at (3.3,4);
      \coordinate (bailey) at (1.5,4.5);
      \coordinate (broch2) at (0.03,4.6);
      \coordinate (westroad) at (-1.2,5.3);
      \coordinate (out) at (4,5);
      \foreach \x in {0,...,2}{
        \draw (-1+\x*2,3) -- ++(1,1) -- ++ (1,-1);
      }
      \node at (tofirst) {\rotatebox{82}{6 miles}} ;
      \node at (tosecond) {\rotatebox{7}{6 miles}} ;
      \node at (tomountains) {\rotatebox{57}{7 miles}} ;
      \node at (tobailey) {\rotatebox{-12}{7 miles}} ;
      \draw [gray, dashed] (fork) -- (broch) ;
      \draw [gray, dashed] (start) -- (fork) -- (bothy1) -- (bothy2) -- (mid) -- (bandits) -- (bailey) -- (out) ;
      \draw [gray, dashed] (bailey) -- (broch2) -- (westroad) ;
      \node at (griffins) {\A} ;
      \draw [dotted] (broch2) circle (1.6em) ;
      \draw [dotted] (broch2) circle (2.4em) ;
      \draw [dotted] (broch2) circle (3.2em) ;
      \draw [dotted, fill=white] (broch2) circle (0.8em)  node { \glsentrysymbol{sylf} } ;
      \draw [dotted] (broch) circle (1.6em) ;
      \draw [dotted] (broch) circle (2.4em) ;
      \draw [dotted] (broch) circle (3.2em) ;
      \draw [dotted, fill=white] (broch) circle (0.8em) node { \glsentrysymbol{sylf} } ;
      \draw [dotted, fill=white] (bothy1) circle (0.7em) node { \gls{evening} } ;
      \draw [dotted, fill=white] (bothy2) circle (0.7em) node { \gls{evening} } ;
      \node[anchor=north east] at (bothy1) {\scshape\bothyOne};
      \draw [dotted, fill=white] (bailey) circle (1.2em) node {\normalsize \Hu } ;
      \node[anchor=north] at (bothy2) {\scshape\bothyTwo};
      \node[anchor=north] at (broch) {\scshape\brochOne};
      \node[anchor=north] at (broch2) {\scshape\brochTwo};
  \end{tikzpicture}
}

\appto{\bothyOne}{\space\Glsentrytext{bothy}}
\appto{\bothyTwo}{\space\Glsentrytext{bothy}}
\appto{\brochOne}{~\Glsentrytext{broch}}

{
  \noindent
  The \glspl{pc} must guard a trader over the mountains, past \glspl{monster}, bandits, and an honest man who wants them to pay their taxes.

  \bookletThreads{lonelyRoad}

  \bookletThreads{bothy}

  \bookletThreads{Mountain}

  \bookletThreads{Valley}
}

\clearpage

\pagestyle{minizine}

\thread{\marketFence's Proposal}

\begin{speechtext}
  I have books on \emph{history}, \emph{preservation}, and \emph{secretive cures} for breathrot and meat grippes.
  \Glspl{boon} made from sentient mushrooms in the \gls{deep}, and signet rings forged by dwarves who haven't taken a holiday in \emph{twenty years}.

  Buyers in \gls{ftown} who would sell their own children's teeth for a \underline{chance} to see my wares.
  I have paid five \glsentrylongpl{gp} \underline{up-front} to the \gls{templeOfBeasts} for guardianship to the mountain-pass, and down the open plains which lead to \gls{ftown}, to sell my goods.

  Day one: the \glspl{ranger} disappear on an `urgent mission'.
  Am I not `urgent'?
  Do I not seem `\underline{urgent}' to you?

  Join me from here until the first \gls{village}.
  Keep my cargo and donkeys safe, and I will pay three \glsentrylongpl{gp},%
  \footnote{If the \glspl{pc} ask for more, she adds concessions.}
  and will sell any \gls{monster}-corpse after \gls{harvesting}, completely \underline{tax-free} (and the \gls{templeOfBeasts} can eat \gls{sylf}-shit).
  So your journey will be \emph{easy}, or \emph{profitable}.
\end{speechtext}

\begin{table}
  \begin{tabular}{rl}
    \setcounter{freeHP}{7}%
    \scshape Donkeys & \\
    \footnotesize Charming: & \hpStat \\
    \footnotesize Candid: & \hpStat \\
    \footnotesize Bard: & \hpStat \\
    \hline
    \Glspl{ration}: &
    \setcounter{weight}{0}%
    \setcounter{freeHP}{6}%
    \hpStat \\
    & \hpStat \\
    & \hpStat \\
    \hline
    \Glspl{torch}: &
    \setcounter{freeHP}{7}%
    \hpStat \\
  \end{tabular}
  \label{donkeyTracker}
  \footnotesize
  \begin{minipage}{.3\linewidth}
    Mark off \pgls{ration} per head each day (including the donkeys).

    \Pgls{interval} of travel adds 3~\glspl{ep} to each donkey.
    \Pgls{interval} of rest removes \pgls{ep}.
  \end{minipage}
\end{table}

\humantrader[\NPC{\Hu\F}% Symbol
  {\marketFence}% Name
  {pear-shaped, with short hair}% Description
  {long drag from long pipe}% Mannerism
  {a consistent, reliable life}% Wants
  \setcounter{fp}{5}
]

\marketFence's wagon is a well-stocked donkey troika.
Track the \glspl{torch}, \glspl{ration}, and donkeys' \glspl{ep} \vpageref{donkeyTracker}.

\segment[\gls{afternoon}]{lonelyRoad}% AREA
{An Honest Man}% NAME
{\Pgls{guard} joins, determined to do everything by the book}% SUMMARY

\Gls{soldier} Perspic does things by the book.
He pays the full 50\% tax to the \gls{templeOfBeasts} every time he kills \pgls{monster}, and believes everyone else should do the same, or he will report them to the nearest \gls{jotter}.

\begin{boxtext}
  Ahead, four people in black leather armour sit around a stinking mass of tentacles.
  One stands, waving enthusiastically, then rushes over to introduce himself as `\Gls{soldier} Perspic'.
  He explains that he and the three \glspl{gDigger} with him have killed \pgls{woodspy}, and they've almost finished \gls{harvesting} the corpse.
  He asks you to help him carry it to \brochOne, just a mile out of your way.
\end{boxtext}

\humansoldier[%
\NPC{\M\Hu\glsentrysymbol{sylf}}{\Glsfmttext{soldier}~Perspic}%
  {clean-shaven, braided hair}% DESCRIPTION
  {hands on hips}% MANNERISM
  {to make you do the right thing}% WANTS
  \npcQuote{If nobody pays into the system, new \glsfmtplural{guard} won't get weapons, and the whole system falls apart}%
  \renewcommand\rations{6 days' hard biscuit rations}
  ]

\paragraph{Once Perspic delivers the \gls{woodspy},}
the \gls{broch}'s \gls{jotter} tells him to join the \glspl{pc}, as they `might need help'.
He salutes with an extra hand-twirl, and chases after the cart.

Whether or not the \glspl{pc} invite or help Perspic, he will join them.
And he will make sure that nothing dubious or `un-guildly' happens.
And every night he shares his biscuits and tells stories, like the time he killed \pgls{crawler}, or the time he killed two of the `gong bandits' by \gls{ftown},%
\footnote{If anyone asks why they're called `the gong bandits', he says `I don't remember, it doesn't matter'.}
or the time he helped \Gls{warden}~Fell host a socially tenuous dinner-party with \pgls{chef} and \pgls{seneschal}.

The troupe will arrive at \bothyOne\ before Sundown.
\marketFence\ uses the time to ask each \gls{pc} about their background, and how they joined the \gls{guard}.

\vspace{-1\baselineskip}
\segment[\gls{morning}]{lonelyRoad}% AREA
{Uncivilized Smoke}% NAME
{Someone left their possessions unguarded in the forest}% SUMMARY
\label{rangerCamp}

\begin{exampletext}
  The \glspl{ranger} who once helped \marketFence\ went to raid a bandit camp.
  They made camp in the forest last night, and left some items to avoid being weighed down during the fight.
  They also left the fire smouldering, just enough to give off smoke until a few minutes after Sunrise.
\end{exampletext}

\begin{boxtext}
  Out the \gls{bothy} door, by the dawn's red light, you see a single plume of smoke rising from the forest, half a mile away.
  By the time \marketFence\ wakes up and prepares, the smoke has gone.

  ``\textit{Let's get moving}'', she says.
\end{boxtext}

\paragraph{If the \glspl{pc} investigate,}
they find the \glspl{ranger} left \lootMedium, two sleeping bags, three \glspl{torch}, rope (15~\glspl{step} long), \rations, and \rations.

\vspace{-1\baselineskip}

\segment[\gls{afternoon}]{bothy}% AREA
{Above the Chamberpot}% NAME
{\Pgls{woodspy} waits in the \gls{bothy}}% SUMMARY

Unlike most \glspl{bothy}, \bothyTwo\ has a separate room with a chamberpot and a hatch to empty it.
(an `in-door out-house').
And unlike most out-houses, this one has \pgls{woodspy} on the ceiling.

\begin{boxtext}[How does everyone rest?]
  Slidy-tracks lead up to \bothyTwo\ --- one of the largest in the land.
  The donkeys stop by the door, and wait to be unbridled.
\end{boxtext}

If the troupe don't notice the `slidy' (\gls{woodspy}) tracks, the \gls{woodspy} ambushes the first to use the indoor-outhouse (pick the \glspl{pc} with the highest Speed \gls{attribute}).
Once the door shuts, the \gls{woodspy} descends from above;
tentacles wrap around their throat, trying to silence them.%
\footnote{At this point, ask the other players, `if someone doesn't come out of the toilet for a long time, what would you do?'.
Remember their answer, then resolve the attack.}
Once the \gls{woodspy} kills, it holds the door shut and begins to feed, then leaves through the hatch.

\woodspy

\ifnum\thepage<6%
  If the troupe kill the \gls{woodspy}, the body has \pgls{weight} of \arabic{hp} after \gls{harvesting}, and a well-preserved beak has \pgls{weight} of 1.%
\else%
  \Pgls{woodspy} corpse is valuable, but it has \pgls{weight} of \arabic{hp}.
\fi%
\ifcase\thetemperature\relax\or
  The parts rot after \arabic{r2t4}~days.
\else%
  The remains rot after \arabic{r2t3}~days.
\fi

\segment[\gls{afternoon}]{Mountain}% AREA
{Two-Faced Wagon}% NAME
{\Glspl{guard} ahead carry a dangerous prisoner}% SUMMARY

\begin{exampletext}
  The \glspl{ranger} rased the bandit camp and tied the bandits' leader to a wagon, along with sacks of plunder (mostly longswords).
  The surviving bandits fled across a secret mountain pass, and the surviving \glspl{ranger} still stalk them.
\end{exampletext}

The troupe find their path blocked by the two \glspl{ranger} carrying the bandit leader --- `Guilter' --- on a wagon.

\begin{boxtext}[]
  As the forest thins, mountains rise ahead.
  A small wagon
  \ifnum\thetemperature=0%
    crunches through the mountain's snow-cloak,
  \else%
    trundles along the little path that winds up the mountain,
  \fi%
  pulled by two men and a horse.
\end{boxtext}

The \glspl{ranger} travel slowly, as they feel tired, and don't have to worry about wandering \glspl{monster} this high up the infertile mountains.
The troop will catch up to them soon.

\humansoldier[\NPC{\T[2]\M\Hu\glsentrysymbol{sylf}}{Rotbulge \& Wiltcull}%
  {wide shoulders, hanging gut}% DESCRIPTION
  {pleating beard}% MANNERISM
  {to take Guilter to town}% WANTS
  \npcQuote{hey relax, guys!}%
  \label{vicRangers}]

\begin{boxtext}[]
  Approaching the wagon, you see it holds a pile of sacks, all full, and a man with a long beard and bald head.
  The man's wrists are tied to the side, pulling his arms wide open.

  The path has barely enough room for one wagon, and you clearly cannot pass with the donkey troika.
\end{boxtext}

The sacks have fourteen longswords, taken from the bandits, and Rotbulge wants to sell them tax-free.
Guilter uses this to set everyone against each other:

\begin{description}
  \item[Rotbulge]\it
  Look, I'm sorry about slowing you down, but it's great having yous back there.
  We know old Guilter's going nowhere with you watching him.
  Could you maybe give us a push from the back, and we'll get up faster?
  \item[Guilter]\sc
  Come closer.
  {\small
    I have a secret to tell you\ldots
    The \glspl{ranger} have a wealth of swords in these sacks, and they plan to keep it all.
  }
  \item[Perspic]\rm
  They can't do that.
  Bandit weapons become the property of the \gls{templeOfJustice}.
  \item[Rotbulge]\it
  So the \gls{guard} get to kill bandits, and the \gls{warden}'s job is `selling bandit weapons and keeping the money'.
  How is that fair?
  \item[Guilter]\sc
  Fools!
  {\small
    Now the \glspl{ranger} have no choice but to cut your sleeping throats.
  }
\end{description}

\humanalchemist[\NPC{\M\Hu}{Guilter}%
  {eyes burning with hatred}% DESCRIPTION
  {always mouthing words, mostly silent}% MANNERISM
  {to find the perfect moment}% WANTS
  \npcQuote{mind your step}%
  \setcounter{wounds}{2}%
  \setcounter{weight}{3}%
  ]

{\footnotesize
  \setcounter{diceNo}{1}
  \showStdSpells
}

\paragraph{If the troupe remain behind Guilter,}
he looks for the worst possible moments to cast \glspl{spell} on them, and escape.
\ifnum\thepage<10%
  If he escapes, he grabs his backpack (sitting next to him on the wagon) and threatens to stab a donkey unless someone throws him food.
\fi

\segment{Mountain}% AREA
{Passing Place}% NAME
{The troupe decide whether or not to overtake}% SUMMARY

The path ahead becomes wide enough for two wagons, but Rotbulge and Wiltcull don't want the troupe to pass, so they speed up to avoid the troupe overtaking them.
The troupe can out-pace them
\ifnum\thetemperature=0
  with a \roll{Strength}{Empathy} (\tn[8]) roll to push the cart with the donkies.
  Failure means the cart slips on the ice, and plummets, and smashes into $1D6$ pieces.
\else%
  if they try.
\fi

If the troupe overtake, they can reach the top after \pgls{interval}.
If not, they need two.

\segment{Mountain}% AREA
{Eirie}% NAME
{Any noise summons \glspl{griffin}}% SUMMARY

\griffin[\npc{\T[2]\A}{\Glsfmtplural{griffin}}]

Any loud noises or lights in the dark attracts two \glspl{griffin}.
They find a distant, craggy perch and wait for someone to pass by a precarious spot.
Then they swoop-and-grab, and try to pull their prey off the cliff-side.

\segment{Valley}% AREA
{The Gong Bandits}% NAME
{A wooden \gls{crawler} has a gong hidden inside}% SUMMARY

\humanarcher[%
  \npc{\ifnum\thepage>12\T[8]\else\T[10]\fi\Hu}%
  {\arabic{noAppearing}~Gong Bandits}%
]

The troupe spot \pgls{crawler}, hanging in the forest.
A \roll{Wits}{Survival} roll at \tn[12] will tell them that someone made this `\gls{crawler}' out of wood.
And if anyone says that it's made out of wood to the \glspl{ranger}, then the \glspl{ranger} can tell the \glspl{pc} that the facsimile has a gong hidden inside it, because that is how the gong bandits --- who live on this side of the mountain --- find out that \glspl{guard} have arrived.

\paragraph{If an arrow hits,}
the gong sounds, and the bandits prepare \pgls{ambush}.
Two miles down the road, the bandits wait silently with their \glspl{bow} strung, 20~\glspl{step} into the forest's shadows.

\ifnum\thepage<14
  \paragraph{If the \glspl{pc} left the \glspl{ranger} camp untouched}
  (\gls{segment}~\vref{rangerCamp})
  then two \glspl{ranger} have made it across the mountains, via the secret bandit-path.
  They hear the gong, and stalk through the forest to investigate.
  They join the fight after \pgls{round}, behind the gong bandits.
  Use the same \gls{statblock} as the others, \vpageref{vicRangers}.
\fi

\bigLine

\begin{boxtext}[]
  Over a hill, you see a vast, 
  \ifcase\thetemperature%
    snowy-white plain.
  \or%
    squelching-wet plain.
  \or%
    verdant, plain.
  \else%
    dry plain, ripe for \pgls{burn}.
  \fi%
  At the far edge, dozens of aurochs graze.
  Closer, \pgls{village} waits for you at the end of the \gls{lonelyRoad}.
  And half a mile below, the trees begin to thin, as the forest melts into the open land.
\end{boxtext}

After five more miles, the troupe arrive at that first \gls{village}, where \marketFence\ leaves them to sell her wares in \gls{ftown}.

\end{document}
