% !TeX root = ../main.tex
%%%% This is a document for chunks of text that I'm not sure I want any more.

\section{Random Dungeons}

\subsection{How It Began}

\begin{enumerate}

\item{Magma Flows}
\item{Massive, underground wyrms}
\item{A human settlement just kept building deeper underground}
\item{Natural tunnels formed with the movement of the earth}
\item{Hard-ass dwarvish tunnelling for gold}
\item{Hard-ass dwarvish tunnelling for gems}
\end{enumerate}

\subsection{What Happened Next}
\begin{enumerate}

\item{Dwarves mined for gold}
\item{Dwarves building around an underground stream}
\item{Dwarves building around an underground lake}
\item{Humans mining for coal}
\item{Gnomes moved in to study a mana lake}
\item{Nura tunelled upwards}
\end{enumerate}

\subsection{How It All Ended}

\begin{enumerate}

\item{A necromancer sealed the exits and brought in an army of the dead}
\item{A mad gnome built a conjuration shrine in the centre, filling the area with lethal monsters}
\item{Nura tunnelled upwards, breached the earth's surface, and began raiding villages}
\item{The population mostly died and bandits moved into the safer areas}
\item{A gang of rogue mages enslaved everyone present, then began to experiment with portals}
\item{A power-hungery elvish enchanter now holds everyone in her sway}

\end{enumerate}

\section*{Afterword}

This project has taken a lot of time and money (more than it will ever return) but it's also been a labour of love.  A big part of the motivation is just making a system that my friends and I will find fun, but a greater part is spreading the joy to the rest of the world.  Unfortunately that's difficult when the giants of the RPG industry are so solidly lodged in their place.  If you've played a couple of games with this system and you enjoyed it, consider taking a moment to write up a quick review on-line, either on a blog of your own or just anywhere that accepts that kind of thing.  We can't know ahead of time what kinds of games we like and the opportunity cost of getting into a new RPG is at least as high as the time it takes to read the core book, so RPG players more than most hobbyists rely on the e-word-of-mouth to find the kinds of games they like.  If I say my game's fun then that's cheap, but if you tell people what kinds of things you think it does well and what it does badly, that's valuable information.

	Also, I should explain the languages.  You might have noticed I love them.  Obviously I had to restrain myself because knowing the gnomish word for 'shoe' simply isn't helpful when actually playing an RPG.  However, I will make a couple of notes on actual use:

	The elvish language here is Tolkien's Quenya.  I have no idea why people would create (and have created) a new language for elves when such a tasty one exists already.  HOWEVER -- little to no genuine Elvish is presented in this book.  Rather, various human words are derived from Elvish.  In Quenya, 'Quennome' would actually pronounced with those two 'nn' sounds just like we do when we have a 'fine nap' but your game doesn't need to worry about 'actual pronunciation' because the humans of Fenestra mangle Elvish just as much as you (presumably) do.

	The Gnollish language should never be mistaken for faux-yoda speech.  It's a light-hearted reductio ad absurdum of theories of language which state that language is not just inherent in humans, but that a particular form of language is inherent.  I'm not a fan of the idea so the gnolls are here to show what a race who had an actual genetic grammar would look like -- they innately recognise the categories of verbs, nouns and adjectives and equally innately place verbs at the start of sentences.  Of course this is a tendency, not a rule, so it should never be allowed to interfere with play.  Further, the sounds in the gnoll language are my guess at what a language might sound like given the shape of gnolls' mouths.  If the 'thsh' consonant cluster is difficult to pronounce fluidly with a tongue under six inches long, it's because it was never mean for us. I should also note that the 'gh' consonant is basically the 'ch' in 'loch' but voiced (the same move you make when going from 't' to 'd'), and the 'dh' sound is the voiced 'th' in words like 'this'.

	The Gnomish language is not as far fetched as it may sound.  Various languages have whistling variants and while Iam simply guessing about how well a high-pitched whistle travels underground they can genuinely travel far farther over mountains than simply hollering, and without expending nearly as much energy.  Imagine the kind of high-pitched irritating whistle that you hear at some gigs, then amplify it, then add in some pleasing, highly skilled, variations in tone and pitch and such so that a full conversation can emerge.  Imagine being in a deep, dark cave and hearing whistling coming from both sides as two gnome groups have a conversation over your heads which can pierce even the loudest battle-cries.  Imagine that, with listeners so skilled that they can hear not only how far away another speaker is but roughly how many tunnel-bends the whistle has taken to get where it's going.

	Finally, human languages; the Trade Tongue can be represented with English easily enough.  If you want more detail on the various regions which each have their own languages, I recommend adding in your regional accents to make things distinctive.  Personally, I might use Doric for Eastlake, Edinburghian for the Bearded Mountains region, Glaswegian for Mt Arthur and so on.  Just as Europe's upper classes often showed their status by throwing in Latin or French words, I imagine anyone in Fenestra might use little, mangled, Elvish phrases when trying to be pompous.  In all cases, the human language is written with Gregg's Shorthand, because Gregg's looks like some kind of awesome fantasy language already and it only takes a few hours to learn enough to use.

	Whether you're blogging your findings or not, I'd love to hear about any use of the game -- how things went, what became a problem, what players loved and especially about any areas outside of Fenestra you've created.  You can reach me by writing to troikamoira@gmail.com.  Happy gaming.

