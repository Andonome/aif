% !TeX root = ../main.tex

\section*{Acknowledgements and Thanks}

\subsection*{Artists}

\subsubsection{Dyson Logos}

Forgotten city, page \pageref{lost_city_map}.

The Green Tower, page \pageref{green_tower_map}.

Redfall Keep, page \pageref{redfall_keep_map}.

The town map, page \pageref{town_map}.

The \glsentrytext{pig}, \pageref{mincing_pig_map}.

The sewers beneath the town, page \pageref{sewer_map}.

The ruined village, page \pageref{ruined_village_map}.


\chapter*{How to Not Read This Book}

Who has time to read \pageref{lastpage} pages?  It's not like you can remember all of them.

Well for a start, you can just skim-read the Games Master Resources section.  It has some magical items, mana-lakes, blah, blah -- just remember where they are so you can grab them when you need them.  Go over the random encounter mechanic so you can pull up some random enemies when needs be.

Next up, we don't need to know the entire map of the world -- the world's divided into six sections, so just pick one for your campaign -- there's snowy waste, a deep forest ruled mostly by elves, islands, and a fairly standard fantasy region.  Each has a unique set of encounters which characterize the location.  Roll up a couple of encounters to get to know the area, look up the creatures in the bestiary chapter, and write down a couple of the encounters you want to pull on the players on your GM's character sheet.\footnote{Well, more like a `campaign sheet'.  It's at the back.}  The other creatures don't matter if you're not using them.

The Side Quests section is a little different.  If you're going to use them then you need to be familiar with the overall plot they weave.  A helpful character glossary is provided at the back of the book so you don't have to remember all of the characters in the Side Quests.


