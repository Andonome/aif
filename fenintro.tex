\begin{multicols}{2}

\subsection*{The Primordial Forest}

\begin{exampletext}

  The forest wants to eat you, so pay attention.

  You've had an easy life, in your secluded, little hole, but here we live above ground, where giant creatures crawl everywhere, and I can assure you I'm the gentlest one out here.
  They're probably looking at you and licking their lips.
Well, not `lips'.
  I don't think a single one has lips.
  Some have beaks, others have `mandibles', but you get my point.

  The chitincrawlers lay webs, but don't think they need to wait for you.
  If they get hungry enough, they'll run straight at you, grab you with their claws, and just start eating!

  And remember to be on the lookout for moving trees.
  If you see something shift that's meant to stay still, it could be a woodspy -- like an octopus, but\ldots do they have octopuses where you're from?
  No I didn't think so.
  But I'm glad to have you with us.
  I hear your people can see better in the dark than we can.

  Most of the world sits in darkness, just like this.
  Most of the world lacks roads, beer, beds, and everything that makes life worth living.
  For this reason, we exist, to push back the darkness, and make way for more civilization.

  \subsubsection*{Bandits}

  Notice the trees.
  \ifnum\value{temperature}<1
    They just sitting there in the \seasonDesc n wind right now, but once the warmer seasons hit, they'll be full of fruits, and you'll see things growing all around.
  \else
    There's good eating up there if you can climb.
    The forest is laying a trap for us, but it's a tasty one!
  \fi
  Humans could live out here like some kind of paradise, never working, just taking food from the trees -- at least over C\'alea and Laiqea, and the other warmer seasons.
  Even Toldea has plenty to eat if you know where to look!
  And all this means the forest has laid another trap for us.
  Thieves, cut-throats, and black-alchemists who want to escape the law come into the forest, and she treats them well.
  They live here, tax-free, robbing villages over the cold seasons.

  And obviously it's \emph{our} job to come out here and route them out, by fire and sword.


  \subsubsection*{And worse\ldots}

  Magic's a horrid thing.
  Once someone knows enough of it, they can destroy a city.
  And you can never spot someone who knows it.
  Well sometimes I think you can -- a shifty look in the eye, especially if they've been away from the \gls{alchemists} too long.
  They may as well stay out there, as far as I'm concerned.

  Sometimes we lose whole cities, to curses which bring beasts out of the forests, or maybe they'll just turn a town's walls to ice on a warm C\'alean day and watch those walls melt while the beasts comes in to eat the village.

  You don't know any magic do you?

  Shame.
  We could probably use some out here.
  At least a little blessing or something would be nice.
\end{exampletext}

\subsection*{The \Glsentrytext{edge}}

\begin{exampletext}
  We're coming up to civilization at last!
  See that patch of lawful land?
  No trees, vines, bushes, or anything?
  That means the town must be close.

  I was in a proper \textit{big burn} once.
  We covered the area in oils last Laiquea \ifnum\value{temperature}=3 on a day as warm as this one \else on a scorcher of a day \fi and it burnt so high I swear it reached \gls{ainumar} and made the gods stink for a week of woodsmoke!

  \subsubsection*{Walled Villages}

  The great clear areas around the \gls{edge} gives us a buffer.
  Anything that wants to skitter over here in the daylight gets filled with arrows.
  Of course that won't always stop a basilisk, but it can drive them off.

  We'll see some sheep inside, maybe even cows, but there's never much meat here.
  You can take them out grazing only a short way, where it's safe, and the sight of so many animals always tempts something out of the forest sooner or later.
  Mostly, meat comes from the inside, while the outer circles send back wood, or forest fruits, or anything from the fields outside their walls.

  Most humans on the \gls{edge} learn the bow, or at least how to use a crossbow.
  Anyone who doesn't puts everyone else in danger.
  I grew up in a village like this one -- and we had a lot of sleepless nights, telling each other stories of famous adventurers from back when that sort of thing was still legal.

  Simpler times.
  But also dangerous -- we're much safer now with \gls{king} in charge of everything, and don't let anyone hear you say otherwise.
  Any time these places need help, \gls{king} sends us out to help out.

  We can't stay long, so get some rations, and we'll be on the road soon, headed inwards.

  \subsubsection*{Lonely Roads}

  That's another one of your duties, recruit -- maintain the roads.
  Sometimes bandits slip past the outer villages and camp at the side.
  Sometimes critters in the forest do the same.
  A lot of them are smart enough to know where we go, so they'll sit at the side of the road, picking off traders who carry meat, or just any trader.
  Traders \emph{are} meat as far as the forest is concerned.

  I wanted to be a trader once, but honestly couldn't summon up the courage.
  I survived a couple of trips, but I knew I wouldn't make it for long.
  So if you're ever travelling out to the \gls{edge}, remember to let the word around town, just like I did there.
  The more people who travel together, the safer.
  And if you end up getting eaten by something, maybe the beast will leave the trader alone, and let him get to market.
  Then you'll die a hero!

  Everyone dies a hero in the \gls{guard}.

  See that crossroads ahead?
  That's a good sign.
  We passed the \gls{edge}, now we have two roads, meaning at least two villages around us.
  They'll have walls of their own too, but the farther inwards we get, the safer.

  Sometimes these outer roads break.
  At first you notice nobody is coming to visit the town from that direction.
  Then you notice that nobody who went that way returns\ldots
  A couple of weeks later, and people hold the wakes for anyone who went that way, and hopefully they have a different road out.

  When a village just has one road, then no roads, they just sit there like an island.
  Hopefully someone notices, and \glspl{guard} get to them; but until then they live on alone, without iron, coal, or any other help from the outside.
  If that goes on too long, it's another win for the forest, and a loss for civilization.
  We'll keep on pushing out, but when we reach too far out, the forest eats our fingers.

  \subsubsection*{Quiet Hamlets}

  We're getting closer.
  See that little hamlet?
  No walls, or nothing -- just stone houses for emergencies.
  Very little makes it in this far.

  Whether it's beasts or bandits, they get tempted by the smells along the road, and end up in an altercation with one of the settlements further out.
  Even if something nasty made it in here, they eat the sheep before the people.
  Mostly.

  These inner lands provide most of the meat of Fenestra.
  I bet you've even had some back home.
  No?
  Well lets go up and say `hello'!
  Villagers always give hospitality to the \gls{guard} when they see us.

  \subsubsection*{Little Masters}

  Each area has its own master.
  It's not true what they say about humans -- we don't need leaders telling us what to do, but we have them anyway.
  They don't really do much, but I guess they look nice and fancy.
  Village masters own a few villages, and town masters own a town.
  Then they send their taxes back and have a nice dinner.

  Must be nice.
  Pointless, but nice.
\end{exampletext}

\subsection*{Hungry Towns}

\begin{exampletext}

  I suppose you never seen a big city like Arthur's Wing before.
  No monsters live here, so everyone can rest easy, aside from the cutthroats and thieves, who of course have to worry about the likes of \emph{me} dragging them into our merry little crew and our glorious missions.
  Look at that pathetic beggar over there, asking for food.
  He can clearly walk, but refuses to sign up with us and fight for the crown.
  Even if he got eaten by something, it'll slow that something down while everyone else kills it or gets to safety.
  It's a good deed.
  He could be a hero.
  Everyone dies a hero in the \gls{guard}.

  \subsubsection*{Guilded Temples}

  The various guild-temples obtained their monopolies long before the current Rex.
  The priests of V\'er\"e have always taken care of the court houses, and the priests of Ohta have always dealt with weapons.
  And of course, nobody else is allowed to.

  They all support \gls{king}, and he supports them.
  Of course in the olden days -- back when we still had adventurers -- people would rise through divine gifts to be a top priest, and give blessings to all the people.

  Nowadays is different.
  The guild higher-ups just want to make money.
  Some of them still hear the real call, get gifts from the gods, but then their superiors send them out on a grand mission.
  We can't have useful people doing paperwork all day, so I suppose it's for the best.
  And it's good to have them out with us in the \gls{guard}.

  Let's get some rest.
  You've got a mission already.
  Nobody's come from Greenwell in a week, and someone needs to find out why.
  I've found a few other new recruits, so you won't be lonely.

  Time to be a hero.

\end{exampletext}

\end{multicols}


