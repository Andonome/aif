\chapter{Nura}

\epigraph{A full stomach cannot imagine an empty one.}

\index{Nura}\label{nura}

\section{The Cycle}

\begin{multicols}{2}

\pic{Studio_DA/ogre}{\label{da:ogre}}

\columnbreak

\noindent
The nura are twisted versions of natural creatures.
As nura tunnel up from the depths, they can devour dwarven settlements and turn any uneaten dwarves into more hobgoblins.
Farther up, gnomish warrens can be invaded and all turned into little goblins.
Finally, breaking into the Sunlight above, they attack villages and devour humans, turning the leftovers into ogres.
The twisting magic often comes in the form of magical items which the nura carry with them, and all such items stem from nura spellcasters.
These spellcasters usually stay in the depths, but occasionally some surface to directly turn people into monsters.

Nura reproduce at an alarming rate, growing to adulthood within just a few years.
While those are are turned into nura can be healed through starvation, but any creature born a nura remains a nura forever.

When magic is available, but no humanoids are present, the nura transform local animals.
Spiders, cats, or even horses can be turned into giant monstrosities which tear across the landscape in a desperate search for food.

The nura always want food, but never feel satisfied.
The majority of nura beasts die of starvation, or in fights for food with a local human populace.
As a result, nura gain a +2 bonus to all Morale checks or attempts to resist any mind control which will stop them eating.

\subsection{New Arrivals}

Those recently transformed into nura begin with an intense feeling of hunger, which is generally enough to drive them to murder and cannibalism.
They are always shocked by their new, hideous body and their own actions, but the shock subsides soon.
The sudden loss of intelligence makes people stop questioning their own actions rather rapidly.
Just as rats can eat their own children when hungry and think nothing of it, kind people can turn nasty when their mind is stripped away.

\subsection{Blossoming Hordes}

Once a portal has opened and the nura on the other side have organized themselves, they typically start to transform local creatures, and bring their own animals from the depths.
Goblin wolf-riders start by scouting the area.
They often limit their raids to the bare minimum in order to make sure they can return to everyone else with information.
If a large population nearby cannot be defeated, nura will often invade close by, then retreat from an enemy army while transforming and eating everything in their path.

Even with a dedicated goblin nuramancer leading battle-plans, nura are rarely very organized, so enemy scouts can traverse the dangerous roads just so long as they have fast horses and don't enter the villages.

\subsection{Lockdown \& Death}

While nura hordes can decimate an area, stripping the villages of their livestock, then their lives, and even destroying full cities; no siege can last forever. 
Nura encroachment ends in one of two ways -- either the \gls{guard} eradicate the nura threat, or the nura eat everything available in the area, before starving to death.

Once an area dies, the area lies still for months or years before daring or desperate people attempt to reclaim the land, pulling out the bodies of the dead, and rebuilding what was lost.

\subsection{\Glsfmtplural{blight}}
\label{blight}

If the nura have arrived through a magical portal to their own realm then the danger can never entirely vanish.
They will retreat once food has become scarce, but raiding parties will return through the portal to check regularly.

Every \gls{blight} has some garrison permanently stationed, looking out for the well-being of the portal.
Goblin riders will occasionally emerge, hunt the area clean of game, then disappear, only to re-emerge later.
Such areas will never be safe until someone can close the portal to the nura realm, so they become a perpetual \gls{blight}, where no-one can live safely.

\Glspl{blight} pose less danger the farther one travels.
The nura's constant need for food stops them marching very far, even if they bring supplies.
This creates a `starvation radius', beyond which civilization remains safe.
Nobody has any precise measurements about exactly how far this is, but around 40 miles seems safe enough.

The \gls{guard} dispose of most \glspl{blight} before long, but some stubborn threats stubbornly remain.
Some of these portals rest behind a tiny opening in a cavern, big enough only for goblins to crawl through.
Others began as fortified towers, once home to illegal alchemists, and have nura guarding the perimeter constantly.
The worst of these long-term \glspl{blight} have some intelligent leader, such as a nuramancer, who organizes food-chains from the nura lands and maintains a perpetual garrison on the lookout for anyone attacking their stronghold.

\end{multicols}

\section{Nura Encounters}\index{Nura!Encounters}

\begin{multicols}{2}

\noindent
A nura invasion might begin with a portal to the nura realm.%
\footnote{See \autopageref{darknessandfire}},
where goblins, ogres, and other `natural' nura creep through, and begin eating everything for miles around.
Others begin when a nuramancer learns how to cast the magics which turn people into nura.

However trouble starts, the nura rating infects the normal encounter results, replacing villagers, then deer, and eventually \glspl{guard} and creatures in the forest.

However it starts, it usually starts slowly.
People notice a few strange creatures -- nura slugs, or perhaps a gnomish warren becomes twisted into goblins, and begin raiding nearby villages.

If nura enter an area, the nura rating becomes `1'.
This means that if the left hand die (which covers an encounter chart's X-axis) lands on a `1', and the right hand die (which covers the Y-axis), also lands on a `1', then the normal encounter is replaced with the result on the nura chart.
As the nura rating creeps higher, further encounters on the right-hand roll change place with the nura encounters.
When the nura rating goes up to `5`, then any time the left-hand roll is a `1', and the right-hand roll is `5' or less, consult the nura encounter table.

These numbers are never adjusted like normal encounters tables, so if the troupe enter a region's \gls{edge}, and gain +1 on their left-hand roll, they would no longer encounter any nura, because the first die cannot produce a `1'.
Nura always like to remain close to villages, where they can get food.
Similarly, encounters with a snowstorm or hurricane do not stop, simply because nura are running around an area, so if the right-hand die rolls a total of `0' during a cold season, or `7' during a warm season, there cannot be any nura encounter.

If the local \gls{guard} don't put an end to the growing threat soon, the land becomes very dangerous indeed.
The roads become too dangerous for traders, and even the local wildlife -- from boars to griffins -- fall into nura stomachs.

Once the nura rating rises above `6', it consumes another number on the second encounter die!
If the nura rating reaches `7', any roll of a `1' on the left-hand die, and a `1' or `2' on the right-hand die means the nura replace the normal encounter.

With enough time unchecked, the entire area becomes a hellish \gls{blight}.
Once the nura rating reaches `12', almost all encounters become nura encounters.
All civilized people leave the area, farming ceases, natural creatures flee much of the area, and anyone with half a nugget of sense will never return.

\subsection{Weather}

No matter how bad an area becomes, weather encounters do not change.
Rain falls from the sky, regardless of the calamities on the ground, to these encounters will block any nura in the area, no matter how bad a \gls{blight} may become.

\subsection{Side Quests Involving Nura}

Some Side Quest scenes involve raising the nura threat level.
These are marked with the `\N' symbol.
This slow increase allows the campaign to increase in danger bit by bit.
\iftoggle{core}{%
  The \gls{gm} sheet has a section to keep track of the local `Nura Rating'.
}{}

\subsubsection{Healing the Land}

The \glspl{pc} can lower the general nura rating by closing portals to the nura realm, or killing nuramancers.
Each portal closed or caster killed lowers the nura rating for that area by `3'.

\end{multicols}

