\chapter{Nura}

\epigraph{A full stomach cannot imagine an empty one.}

\index{Nura}\label{nura}

\section{The Cycle}

\begin{multicols}{2}

\pic{Studio_DA/ogre}{\label{da:ogre}}

\columnbreak

\noindent
The nura are twisted versions of natural creatures.
As nura tunnel up from the depths, they can devour dwarven settlements and turn any uneaten dwarves into more hobgoblins.
Farther up, gnomish warrens can be invaded and all turned into little goblins.
Finally, breaking into the Sunlight above, they attack villages and devour humans, turning the leftovers into ogres.
The twisting magic often comes in the form of magical items which the nura carry with them, and all such items stem from nura spellcasters.
These spellcasters usually stay in the depths, but occasionally some surface to directly turn people into monsters.

Nura reproduce at an alarming rate, growing to adulthood within just a few years.
While those are are turned into nura can be healed through starvation, but any creature born a nura remains a nura forever.

When magic is available, but no humanoids are present, the nura transform local animals.
Spiders, cats, or even horses can be turned into giant monstrosities which tear across the landscape in a desperate search for food.

The nura always want food, but never feel satisfied.
The majority of nura beasts die of starvation, or in fights for food with a local human populace.
As a result, nura gain a +2 bonus to all Morale checks or attempts to resist any mind control which will stop them eating.

\subsection{New Arrivals}

Those recently transformed into nura begin with an intense feeling of hunger, which is generally enough to drive them to murder and cannibalism.
They are always shocked by their new, hideous body and their own actions, but the shock subsides soon.
The sudden loss of intelligence makes people stop questioning their own actions rather rapidly.
Just as rats can eat their own children when hungry and think nothing of it, kind people can turn nasty when their mind is stripped away.

\subsection{Blossoming Hordes}

Once a portal has opened and the nura on the other side have organized themselves, they typically start to transform local creatures, and bring their own animals from the depths.
Goblin wolf-riders start by scouting the area.
They often limit their raids to the bare minimum in order to make sure they can return to everyone else with information.
If a large population nearby cannot be defeated, nura will often invade close by, then retreat from an enemy army while transforming and eating everything in their path.

Even with a dedicated goblin nuramancer leading battle-plans, nura are rarely very organized, so enemy scouts can traverse the dangerous roads just so long as they have fast horses and don't enter the villages.

\subsection{Lockdown \& Death}

While nura hordes can decimate an area, stripping the villages of their livestock, then their lives, and even destroying full cities; no siege can last forever. 
Nura encroachment ends in one of two ways -- either the \gls{guard} eradicate the nura threat, or the nura eat everything available in the area, before starving to death.

Once an area dies, the area lies still for months or years before daring or desperate people attempt to reclaim the land, pulling out the bodies of the dead, and rebuilding what was lost.

\subsection{Blights}
\label{blight}
\index{Blights}

If the nura have arrived through a magical portal to their own realm, this changes everything.
The danger can never entirely vanish.

In this case, goblin riders will occasionally emerge, hunt the area clean of game, then disappear, only to re-emerge later.
Such areas will never be safe until someone can close the portal to the nura realm, so they become a perpetual `blight' on the land, where no-one can live safely.
These holes, of course, only pose a danger for some distance -- nura must eat constantly, so they cannot march for many days.
Outside this `starvation radius', people can feel safe, although misjudging this radius means becoming a quick snack.

While most blights meet a short end once the \gls{guard} arrive, some remain long-term, and of course any which have persisted for so long do not admit an easy entrance.
Some portals rest behind a tiny opening in a cavern, big enough only for goblins to crawl through.
Others began as fortified towers, once home to illegal alchemists.
The worst of these long-term blights have some intelligent leader, such as a nuramancer, who organizes food-chains from the nura lands and maintains a perpetual garrison on the lookout for anyone attacking their stronghold.

\end{multicols}

\section{Nura Encounters}\index{Nura!Encounters}

\begin{multicols}{2}

\begin{encounters}{Nura Lands}

  \setcounter{enc}{18}
  \li & $1D6\times 2$ Goblins riding nura spiders. \\
  \li & Lava Man (page \pageref{lavaman}). \\
  \li & $1D3^2$ Ghasts (page \pageref{ghast}). \\
  \li & $1D6\times 2$ Nura Horses (page \pageref{nura_horse}) \\
  \li & $(2D6)^{2}$ ghouls (page \pageref{ghoul}).\\
  \li & $1D6$ Goblins riding ogres. \\
  \li & $2D6$ Hobgoblins riding nura horses. \\
  \li & $2D6-1$ Goblins riding nura wolves. \\
  \li & Nura Cat (page \pageref{nura_cat}) \\
  \li & $1D6$ Nura Spiders (page \pageref{nura_spider}) \\
  \li & $2D6 + 3$ Nura Wolves (page \pageref{nura_spider}) \\
  \li & $1D3^3$ Ogres (page \pageref{hobgoblin}) \\
  \li & $2D6$ Hobgoblins (page \pageref{hobgoblin}) \\
  \li & $1D6^2$ Goblins (page \pageref{goblin}) \\
  \li & $1D6^3$ Nura Slugs (page \pageref{nura_slug}) \\

\end{encounters}

\noindent
The nura are ever present, but with differing degrees.
Sometimes they are running amok across an area, while at other times they have been mostly killed and the land is quiet.
The number of Nura in the area can be given a rating -- anywhere from 1 upwards, where `1' represents almost no nura creatures in the environment and `3' represents a few but not many.
A nura rating of `8' would represent an area in serious danger as tradesmen could not wander the roads without fear of being jumped, and every village would be in danger of a siege at any moment.
Higher numbers represents a hellish landscape where nobody can wander free, lone hamlets are destroyed, and walled villages come under constant siege.
Whenever you roll on the encounter tables, if you roll equal or lower than the Nura Rating, the nura encounter occurs.

For example, if the nura rating is `5' and the PCs are wandering in a forest with an encounter on the roll of an `8', rolling 3-5 will mean a nura encounter, rolling 6-7 will mean no encounter and rolling 8 or more means a normal forest encounter.

You can set the nura rating to fit your current story, but for a standard `background' level, set the nura level to 3, and then raise it by 1 every time the PCs encounter the nura.
Some Side Quests also involve raising the nura threat level.  These are marked with a `\N'.
This slow increase allows the campaign to increase in danger bit by bit.

The players can lower the general nura rating by `plugging holes'.
Nura come from below the ground or through magical portals.
Once an entrance to Fenestra has been sealed off, the nura represent less of a threat.

As nura emerge from magical portals, they find more and more opportunity to take creatures back with them underground, where those creatures too turn into nura.
Once people become less common, nura start taking any animals they can back underground with them.
Eventually, nuramancer emerge from underground and begin to raise the dead.
Entire villages are sometimes killed and pulled back from death, just to roam the landscape and consume souls.

If the nura rating is high enough that it coincides with a regular encounter, drop the encounter and just put nura there -- they eat everything in the landscape, so it makes sense that regular creatures would be seen less often.

\subsection{Characters as Nura}

In dire situations, the PCs may themselves transform into nura.
You can brush over this by skipping to a scene where they `come to', and slowly understand what they did during their fugue, whether this involved killing people or eating them.
Alternatively, characters can make a Wits + Academics roll, TN 12, to avoid doing something stupid and horrifying.

As usual, anyone may spend 5 FP to specify that the spell fails.

\end{multicols}

\section{Nuramancy}

\label{saurecanta}
\index{Nura!Magic}
\index{Saurecanta}
\index{Nuramancy}

\begin{multicols}{2}

\subsection{New Path: The Path of Nura}

\textit{Spheres: Conjuration, Invocation, Necromancy, Saurecanta}

\noindent Occasionally, the strange creatures of the deeps emerge with apparently inborn magical abilities fuelled by the corruption in their bodies.
The Nura humanoids such as goblins and ogres occasionally learn such magics, though it can be difficult as they are never very intelligent, and while the Path of Nura is a strange Path of magic, it is still based upon one's Intelligence.
It is also possible to learn such magics through memorization of corrupt thoughts alone -- books uttering extreme and surreal crimes are known to exist which can teach anyone how to step onto the Path of Nura.

\paragraph{Signs:} When cast, inky black mist, speckled with violent red appears around wherever the spell brings something into existence.

\paragraph{Mana Stones:} Unorthodox books, cutting knives, revolutionary art -- anything which can promote or symbolize drastic change.

\end{multicols}

\sphere{Saurecanta}

\begin{multicols}{2}

\noindent
This new sphere of magic comes from the foul realm under the earth where strange creatures breed and eat at a dizzying pace.
It bears a passing similarity to Polymorph but with fewer restrictions on form and without any ability to disguise oneself as a natural creature.
While a creature is affected by this sphere, they must eat a minimum of thrice the normal amount; this need not mean constant intake of food -- a single massive meal will suffice.
Failure to eat inflicts the usual Fatigue Points.

Each level of Saurecanta is a double-edged sword, allowing targets extra abilities at a cost.
While those affected can gain a lot of power, they are also afflicted with unending hunger.
Any scene in which they do not eat, the characters heal no Fatigue Points, and gain 1.
Meanwhile, the character can heal a number of Fatigue Points each scene equal to their maximum HP, simply by gorging on food.

Saurecanta spell effects never stack with each other, or with Polymorph -- only the highest bonus counts.

\spelllevel \label{saurecantaone}

\spell{Hunger Pains}{Continuous}{Deceit}{The target must eat or suffer 1 HP damage per 2 Fatigue Points, vs Wits + Empathy}

This spell afflicts the target with a ravenous hunger and extreme stomach pains.
The target resists with Wits + Empathy roll to avoid giving into the hunger.

If the target succeeds in resisting the hunger (or simply cannot eat), they suffer 1 HP Damage per 2 Fatigue Points they currently have -- FP cannot be spent to mitigate this.

If the target does eat, every full meal eaten heals 1 Fatigue Point.

Targets who give into their hunger but have no rations or other proper food to hand will attempt to eat \emph{anyone} around them, including companions, via the fastest route possible.

The target can spend 5 FP in order to ignore this spell's effects.

\spelllevel

\spell{Brawn-Form}{Instant \& Continuous}{Medicine}{The target grows massive, gaining $Lv$ points in the highest of their Strength or Speed}

The caster pulls out the target's inner beast, polluting their soul and improving their body.
It has exactly the same effect as Hunger Pains, while also making the target stronger.

The target slowly gains a number of points added to their Speed or Strength (whichever is higher) equal to the spell's level.
Each point so gained reduces the target's Intelligence \emph{and} Charisma.
This Charisma deficit also reduces the target's Fate Points.

These points do not add immediately.
Instead, the target must each a full meal for every point added (each one further reducing both Intelligence and Charisma).
Targets who reach 0 Charisma begin to look obviously unnatural, and have trouble speaking.

Intelligence and Charisma penalties cannot go below -5 each, so anyone who cannot `pay' for the Attribute increases by lowering the other Attributes simply stops gaining Attributes.

Characters who gain more than a single point of Strength break out of their armour, taking 1 point of Damage for each level of DR the armour provided.

This spell does not stack Attribute Bonuses -- casting the spell to increase a target's Strength multiple times will do nothing.
Only the highest bonus counts.

\paragraph{Duration}
is a finicky thing with nuramancy.
While the effects of Hunger Pains are continuous, once the target begins eating, the effects become permanent.

\paragraph{To cure the target,}
they must be made to starve for the same time the spell has been in effect.
\label{nura_recovery}

\spelllevel

\spell{Ultra-Form}{Instant}{Caving}{The target gains Lv bonus to the lower of Strength/ Speed}

The nura ultra-form functions exactly the same as Brawn-Form (above), but raises the \emph{lower} of the target's Strength or Speed.
As before, this increase must be paid for in \emph{both} Charisma and Intelligence, neither of which can reduce to less than -5.

\spelllevel

\spell{Hell-Hound}{Instant}{Caving/ Seafaring/ Wyldcrafting}{An animal becomes monstrous, and has Lv + Int points divided between Strength and Speed}

This spell targets any animal, making it grow monstrously large.
Wolves turn into hyperactive hell-hounds, horses become strange, twisted things with muscular torsos, and long, spider-like legs, and even common slugs can gain grow to monstrous size.

Land-creatures (such as deer) require the Wyldcrafting Skill, sea-creatures use the Seafaring Skill, et c.

Firstly, if the animal's Strength or Speed were below 0, they raise to 0.
Secondly, the animal gains a number of points to divide between Strength and Speed equal to the spell's level plus the caster's Intelligence Bonus.

\end{multicols}

\section{Cursed Items}
\index{Saurecanta Items}

\begin{multicols}{2}

\magicitem{Birthday Cake}{Ultra-Form}{Saurecanta}{Instant}{Pocket Spell}{3}{3}
\label{birthdayCake}

This massive cake gifts the target with +2 to Speed or Strength (whichever is lower), and reduces their Intelligence and Charisma by -2.
Once they start eating, they can't stop!
(or at last require a Wits + Empathy roll, TN 10)

\magicitem{Ogre Dust}{Brawn-Form}{Saurecanta}{1 Scene}{Pocket Spell}{3}{3}\label{ogredust}

The dust, made from ground poppy-seeds, transforms the target into a nura creature.

They are instantly afflicted with both \textit{Hunger Pains} and \textit{Brawn-Form}.
The target rolls Wits + Empathy, TN 10.
If they eat, they lose 3 points in Intelligence and Charisma, while also gaining 3 points in Strength or Speed (whichever is higher), over the course of their meals.

\magicitem{Spider Skull}{Brawn-Form}{Saurecanta}{Instant}{Talisman}{4}{6}\label{spiderskull}

This sheep's skull makes an inviting home for any spider, but any spider which inhabits it grows to become a nura spider within three rounds.
\footnote{See \autopageref{nura_spider}.}
The item could work on any animal small enough to fit inside and make a home.

\end{multicols}
