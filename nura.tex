\chapter{Nura}

\epigraph{A full stomach cannot imagine an empty one.}

\index{Nura}\label{nura}

\section{The Cycle}

\begin{multicols}{2}

\pic{Studio_DA/ogre}{\label{da:ogre}}

\columnbreak

\noindent
The nura are twisted versions of natural creatures.
As nura tunnel up from the depths, they can devour dwarven settlements and turn any uneaten dwarves into more hobgoblins.
Farther up, gnomish warrens can be invaded and all turned into little goblins.
Finally, breaking into the Sunlight above, they attack villages and devour humans, turning the leftovers into ogres.
The twisting magic often comes in the form of magical items which the nura carry with them, and all such items stem from nura spellcasters.
These spellcasters usually stay in the depths, but occasionally some surface to directly turn people into monsters.

Nura reproduce at an alarming rate, growing to adulthood within just a few years.
While those are are turned into nura can be healed through starvation, but any creature born a nura remains a nura forever.

When magic is available, but no humanoids are present, the nura transform local animals.
Spiders, cats, or even horses can be turned into giant monstrosities which tear across the landscape in a desperate search for food.

The nura always want food, but never feel satisfied.
The majority of nura beasts die of starvation, or in fights for food with a local human populace.
As a result, nura gain a +2 bonus to all Morale checks or attempts to resist any mind control which will stop them eating.

\subsection{New Arrivals}

Those recently transformed into nura begin with an intense feeling of hunger, which is generally enough to drive them to murder and cannibalism.
They are always shocked by their new, hideous body and their own actions, but the shock subsides soon.
The sudden loss of intelligence makes people stop questioning their own actions rather rapidly.
Just as rats can eat their own children when hungry and think nothing of it, kind people can turn nasty when their mind is stripped away.

\subsection{Blossoming Hordes}

Once a portal has opened and the nura on the other side have organized themselves, they typically start to transform local creatures, and bring their own animals from the depths.
Goblin wolf-riders start by scouting the area.
They often limit their raids to the bare minimum in order to make sure they can return to everyone else with information.
If a large population nearby cannot be defeated, nura will often invade close by, then retreat from an enemy army while transforming and eating everything in their path.

Even with a dedicated goblin nuramancer leading battle-plans, nura are rarely very organized, so enemy scouts can traverse the dangerous roads just so long as they have fast horses and don't enter the villages.

\subsection{Lockdown \& Death}

While nura hordes can decimate an area, stripping the villages of their livestock, then their lives, and even destroying full cities; no siege can last forever. 
Nura encroachment ends in one of two ways -- either the \gls{guard} eradicate the nura threat, or the nura eat everything available in the area, before starving to death.

Once an area dies, the area lies still for months or years before daring or desperate people attempt to reclaim the land, pulling out the bodies of the dead, and rebuilding what was lost.

\subsection{\Glsfmtplural{blight}}
\label{blight}

If the nura have arrived through a magical portal to their own realm then the danger can never entirely vanish.
They will retreat once food has become scarce, but raiding parties will return through the portal to check regularly.

Every \gls{blight} has some garrison permanently stationed, looking out for the well-being of the portal.
Goblin riders will occasionally emerge, hunt the area clean of game, then disappear, only to re-emerge later.
Such areas will never be safe until someone can close the portal to the nura realm, so they become a perpetual \gls{blight}, where no-one can live safely.

\Glspl{blight} pose less danger the farther one travels.
The nura's constant need for food stops them marching very far, even if they bring supplies.
This creates a `starvation radius', beyond which civilization remains safe.
Nobody has any precise measurements about exactly how far this is, but around 40 miles seems safe enough.
\Gls{king} has banned anyone approaching \glspl{blight}, as anyone entering may become food for the nura, or find themselves turned into one, which will then endanger all the lands around them.

The \gls{guard} dispose of most \glspl{blight} before long, but some stubborn threats stubbornly remain.
Some of these portals rest behind a tiny opening in a cavern, big enough only for goblins to crawl through.
Others began as fortified towers, once home to illegal alchemists, and have nura guarding the perimeter constantly.
The worst of these long-term \glspl{blight} have some intelligent leader, such as a nuramancer, who organizes food-chains from the nura lands and maintains a perpetual garrison on the lookout for anyone attacking their stronghold.

\end{multicols}

\settoggle{bestiarychapter}{false}
\encNura
\settoggle{bestiarychapter}{true}

\section{Nura Encounters}\index{Nura!Encounters}

\begin{multicols}{2}

\noindent
A nura invasion might begin with a portal to the nura realm.%
\footnote{See \autopageref{darknessandfire}},
where goblins, ogres, and other `natural' nura creep through, and begin eating everything for miles around.
Others begin when a nuramancer learns how to cast the magics which turn people into nura.

However trouble starts, the nura rating infects the normal encounter results, replacing villagers, then deer, and eventually \glspl{guard} and creatures in the forest.

However it starts, it usually starts slowly.
People notice a few strange creatures -- nura slugs, or perhaps a gnomish warren becomes twisted into goblins, and begin raiding nearby villages.

If nura enter an area, the nura rating becomes `1'.
This means that if the left hand die (which covers an encounter chart's X-axis) lands on a `1', and the right hand die (which covers the Y-axis), also lands on a `1', then the normal encounter is replaced with the result on the nura chart.
As the nura rating creeps higher, further encounters on the right-hand roll change place with the nura encounters.
When the nura rating goes up to `5`, then any time the left-hand roll is a `1', and the right-hand roll is `5' or less, consult the nura encounter table.

These numbers are never adjusted like normal encounters tables, so if the troupe enter a region's \gls{edge}, and gain +1 on their left-hand roll, they would no longer encounter any nura, because the first die cannot produce a `1'.
Nura always like to remain close to villages, where they can get food.
Similarly, encounters with a snowstorm or hurricane do not stop, simply because nura are running around an area, so if the right-hand die rolls a total of `0' during a cold season, or `7' during a warm season, there cannot be any nura encounter.

If the local \gls{guard} don't put an end to the growing threat soon, the land becomes very dangerous indeed.
The roads become too dangerous for traders, and even the local wildlife -- from boars to griffins -- fall into nura stomachs.

Once the nura rating rises above `6', it consumes another number on the second encounter die!
If the nura rating reaches `7', any roll of a `1' on the left-hand die, and a `1' or `2' on the right-hand die means the nura replace the normal encounter.

With enough time unchecked, the entire area becomes a hellish \gls{blight}.
Once the nura rating reaches `12', almost all encounters become nura encounters.
All civilized people leave the area, farming ceases, natural creatures flee much of the area, and anyone with half a nugget of sense will never return.

\subsection{Weather}

No matter how bad an area becomes, weather encounters do not change.
Rain falls from the sky, regardless of the calamities on the ground, to these encounters will block any nura in the area, no matter how bad a \gls{blight} may become.

\subsection{Side Quests Involving Nura}

Some Side Quest scenes involve raising the nura threat level.
These are marked with the `\N' symbol.
This slow increase allows the campaign to increase in danger bit by bit.
\iftoggle{core}{%
  The \gls{gm} sheet has a section to keep track of the local `Nura Rating'.
}{}

\subsubsection{Healing the Land}

The \glspl{pc} can lower the general nura rating by closing portals to the nura realm, or killing nuramancers.
Each portal closed or caster killed lowers the nura rating for that area by `3'.

\end{multicols}

\section{Nuramancy}

\label{saurecanta}
\index{Nura!Magic}
\index{Saurecanta}
\index{Nuramancy}

\begin{multicols}{2}

\subsection{New Path: The Path of Nura}

\textit{Spheres: Conjuration, Invocation, Necromancy, Saurecanta}

\noindent Occasionally, the strange creatures of the deeps emerge with apparently inborn magical abilities fuelled by the corruption in their bodies.
The Nura humanoids such as goblins and ogres occasionally learn such magics, though it can be difficult as they are never very intelligent, and while the Path of Nura is a strange Path of magic, it is still based upon one's Intelligence.
It is also possible to learn such magics through memorization of corrupt thoughts alone -- books uttering extreme and surreal crimes are known to exist which can teach anyone how to step onto the Path of Nura.

\paragraph{Mana Stones:} Unorthodox books, cutting knives, revolutionary art -- anything which can promote or symbolize drastic change.

\subsection{Characters as Nura}

In dire situations, the \glspl{pc} may themselves transform into nura.
You can brush over this by skipping to a scene where they `come to', and slowly understand what they did during their fugue, whether this involved killing people or eating them.
Alternatively, characters can make a Wits + Academics roll, TN 12, to avoid doing something stupid and horrifying.

As usual, anyone may spend 5 FP to specify that the spell fails.

\end{multicols}

\sphere{Saurecanta}

\begin{multicols}{2}

\noindent
This new sphere of magic comes from the foul realm under the earth where strange creatures breed and eat at a dizzying pace.
It bears a passing similarity to Polymorph but with fewer restrictions on form and without any ability to disguise oneself as a natural creature.
While a creature is affected by this sphere, they must eat a minimum of thrice the normal amount; this need not mean constant intake of food -- a single massive meal will suffice.
Failure to eat inflicts the usual Fatigue Points.

Each level of Saurecanta is a double-edged sword, allowing targets extra abilities at a cost.
While those affected can gain a lot of power, they are also afflicted with unending hunger.
Any scene in which they do not eat, the characters heal no Fatigue Points, and gain 1.
Meanwhile, the character can heal a number of Fatigue Points each scene equal to their maximum HP, simply by gorging on food.

Saurecanta spell effects never stack with each other, or with Polymorph -- only the highest bonus counts.

\spelllevel \label{saurecantaone}

\spell{Hunger Pains}{Continuous}{Deceit}{The target must eat or suffer 1 HP damage per 2 Fatigue Points, vs Wits + Empathy}

This spell afflicts the target with a ravenous hunger and extreme stomach pains.
The target resists with Wits + Empathy roll to avoid giving into the hunger.

If the target succeeds in resisting the hunger (or simply cannot eat), they suffer 1 HP Damage per 2 Fatigue Points they currently have -- FP cannot be spent to mitigate this.

If the target does eat, every full meal eaten heals 1 Fatigue Point.

Targets who give into their hunger but have no rations or other proper food to hand will attempt to eat \emph{anyone} around them, including companions, via the fastest route possible.

The target can spend 5 FP in order to ignore this spell's effects.

\spelllevel

\spell{Brawn-Form}{Instant \& Continuous}{Medicine}{The target grows massive, gaining $Lv$ points in the highest of their Strength or Speed}

The caster pulls out the target's inner beast, polluting their soul and improving their body.
It has exactly the same effect as Hunger Pains, while also making the target stronger.

The target slowly gains a number of points added to their Speed or Strength (whichever is higher) equal to the spell's level.
Each point so gained reduces the target's Intelligence \emph{and} Charisma.
This Charisma deficit also reduces the target's Fate Points.

These points do not add immediately.
Instead, the target must each a full meal for every point added (each one further reducing both Intelligence and Charisma).
Targets who reach 0 Charisma begin to look obviously unnatural, and have trouble speaking.

Intelligence and Charisma penalties cannot go below -5 each, so anyone who cannot `pay' for the Attribute increases by lowering the other Attributes simply stops gaining Attributes.

Characters who gain more than a single point of Strength break out of their armour, taking 1 point of Damage for each level of DR the armour provided.

This spell does not stack Attribute Bonuses -- casting the spell to increase a target's Strength multiple times will do nothing.
Only the highest bonus counts.

\paragraph{Duration}
is a finicky thing with nuramancy.
While the effects of Hunger Pains are continuous, once the target begins eating, the effects become permanent.

\paragraph{To cure the target,}
they must be made to starve for the same time the spell has been in effect.
\label{nura_recovery}

\spelllevel

\spell{Ultra-Form}{Instant}{Caving}{The target gains Lv bonus to the lower of Strength/ Speed}

The nura ultra-form functions exactly the same as Brawn-Form (above), but raises the \emph{lower} of the target's Strength or Speed.
As before, this increase must be paid for in \emph{both} Charisma and Intelligence, neither of which can reduce to less than -5.

\spelllevel

\spell{Hell-Hound}{Instant}{Caving/ Seafaring/ Wyldcrafting}{An animal becomes monstrous, and has Lv + Int points divided between Strength and Speed}

This spell targets any animal, making it grow monstrously large.
Wolves turn into hyperactive hell-hounds, horses become strange, twisted things with muscular torsos, and long, spider-like legs, and even common slugs can gain grow to monstrous size.

Land-creatures (such as deer) require the Wyldcrafting Skill, sea-creatures use the Seafaring Skill, et c.

Firstly, if the animal's Strength or Speed were below 0, they raise to 0.
Secondly, the animal gains a number of points to divide between Strength and Speed equal to the spell's level plus the caster's Intelligence Bonus.

\end{multicols}

\section{Cursed Items}
\index{Saurecanta Items}

\begin{multicols}{2}

\magicitem{Birthday Cake}{Ultra-Form}{Saurecanta}{Instant}{Pocket Spell}{3}{3}
\label{birthdayCake}

This massive cake gifts the target with +2 to Speed or Strength (whichever is lower), and reduces their Intelligence and Charisma by -2.
Once they start eating, they can't stop!
(or at last require a Wits + Empathy roll, TN 10)

\magicitem{Ogre Dust}{Brawn-Form}{Saurecanta}{1 Scene}{Pocket Spell}{3}{3}\label{ogredust}

The dust, made from ground poppy-seeds, transforms the target into a nura creature.

They are instantly afflicted with both \textit{Hunger Pains} and \textit{Brawn-Form}.
The target rolls Wits + Empathy, TN 10.
If they eat, they lose 3 points in Intelligence and Charisma, while also gaining 3 points in Strength or Speed (whichever is higher), over the course of their meals.

\magicitem{Spider Skull}{Brawn-Form}{Saurecanta}{Instant}{Talisman}{4}{6}\label{spiderskull}

This sheep's skull makes an inviting home for any spider, but any spider which inhabits it grows to become a nura spider within three rounds.
\footnote{See \autopageref{nura_spider}.}
The item could work on any animal small enough to fit inside and make a home.

\end{multicols}
